\section{Iteration to the stationary point}
\label{sc:stationaryIteration}
This section illustrates convergence of the procedure of \eqref{eq:Sdefinition}. Recall that the requirement is to compute $\parm$ from
\begin{equation}
 \xang = \parm\gfunc(\parm),
\end{equation}
and to invert the above, we suggested that
\begin{equation}
\begin{split}
 \parm &= \frac{\xang}{\gfunc(\parm)} = \frac{\xang}{\gfunc\left(\frac{\xang}{\gfunc(\parm)}\right)} = \frac{\xang}{\gfunc\left(\frac{\xang}{\gfunc\left(\frac{\xang}{\gfunc(\parm)}\right)}\right)} = ...\\
 \gfunc(\parm) &= \gfunc\left(\frac{\xang}{\gfunc(\parm)}\right) = \gfunc\left(\frac{\xang}{\gfunc\left(\frac{\xang}{\gfunc(\parm)}\right)}\right) = ...
\end{split}
\end{equation}
Re-write the expression to invert as
\begin{equation}
 \frac{\xang}{\gfunc(\parm)} - \parm = 0
\end{equation}
and define the two functions
\begin{align}
 f_1(\parm) &= \parm,\\
 f_2(\parm) &= \frac{\xang}{\gfunc(\parm)}.
\end{align}
The task is then to compute the point where these two functions intersect.
\begin{figure}[ht!]
    \begin{center}
    \resizebox{0.5\textwidth}{!}{\input{iteration.pdf_tex}}
	\caption{Simple iterative procedure showing convergence to the stationary point}
	\label{fg:stationaryiteration}
	\end{center}
\end{figure}
From \fgref{fg:stationaryiteration}, one sees that, starting from the point $\parm_0$
\begin{equation}
 \parm_1 = f_2(\parm_0) = \frac{\xang}{\gfunc(\parm_0)},
\end{equation}
and, in general
\begin{equation}
 \parm_{n+1} = f_2(\parm_n) = \frac{\xang}{\gfunc(\parm_n)}.
\end{equation}
If 
\begin{equation} 
    \left\lvert \dtv{f_2(\parm)}{\parm}\right\rvert < 1,
\end{equation}
then convergence is assured.
