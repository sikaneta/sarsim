\section{Differential geometry of curves in 3D}
\label{sc:diffgeoreview}
This section reviews material abundant in published literature; for instance, see \cite{WoodDiffGeo} for a highly recommended introduction. While the material presented is complete for understanding the application to SAR imaging, it represents only a small tip of the theory and is by no means original - even common differential geometry notation has been maintained.
\par
In this short section all the required tools to develop the SAR satellite signal model are presented.
\par
For a curve in three dimensions given by $\sats(\parm)$, the tangent vector is given by $\vct{T}(\parm) = \sats'(\parm)$, where $\cdot'$ denotes derivative with respect to $\parm$. Because of the arclength parameterisation, $\vct{T}(\parm)$ is a unit vector and $\vct{T}(\parm)\cdot\vct{T}'(\parm)=0$. The \textbf{curvature} is defined as $\kappa(\parm) =\lvert\vct{T}'(\parm)\rvert$. Define the unit vector 
\begin{equation}
\vct{N}(\parm) = \vct{T}'(\parm)/\kappa(\parm). 
\end{equation}
The binormal vector $\vct{B}(\parm)$ is defined as $\vct{B}(\parm) = \vct{T}(\parm)\times\vct{N}(\parm)$. The vectors $\vct{T}(\parm), \vct{N}(\parm), \vct{B}(\parm)$ form an orthonormal basis, see \fgref{fg:diffvectors}
\begin{figure}[ht!]
    \begin{center}
    \resizebox{0.8\textwidth}{!}{\input{diffvectors.pdf_tex}}
	\caption{Illustration of differential geometry vectors}
	\label{fg:diffvectors}
	\end{center}
\end{figure}
\par
Since $\vct{T}(\parm)\cdot\vct{N}(\parm)=0$,
\begin{equation}
\begin{split}
 \vct{T}(\parm)\cdot\vct{N}'(\parm)&=-\vct{T}'(\parm)\cdot\vct{N}(\parm)\\
 &=-\kappa(\parm)
\end{split}
\end{equation}
Define the \textbf{torsion} as $\tau(\parm) = \vct{N}'(\parm)\cdot\vct{B}(\parm)$. Then, by multiplying $\vct{N}'(\parm)=a_T\vct{T}(\parm) + a_N\vct{N}(\parm) + a_B\vct{B}(\parm)$\footnote{since $\vct{T}(\parm), \vct{N}(\parm), \vct{B}(\parm)$ form an orthonormal basis, we can write $\vct{N}'(\parm)$ as a linear combination with parameters $a_T, a_N, a_B$} successively by $\vct{T}(\parm),\vct{N}(\parm),\vct{B}(\parm)$, one finds that
\begin{equation}
 \vct{N}'(\parm) = -\kappa(\parm)\vct{T}(\parm)+\tau(\parm)\vct{B}(\parm)
\end{equation}
Finally, one finds that
\begin{equation}
\begin{split}
 \vct{B}'(\parm) &= \vct{T}'(\parm)\times\vct{N}(\parm) + \vct{T}(\parm)\times\vct{N}'(\parm)\\
 &=\kappa(\parm)\vct{N}(\parm)\times\vct{N}(\parm) + \vct{T}(\parm)\times[-\kappa(\parm)\vct{T}(\parm)+\tau(\parm)\vct{B}(\parm)]\\
 &=-\tau(\parm)\vct{N}(\parm)
 \end{split}
\end{equation}
The previous few expressions lead to the Frenet-Serret equation
\begin{equation}
 \begin{bmatrix}\vct{T}'(\parm)\\\vct{N}'(\parm)\\\vct{B}'(\parm)\end{bmatrix}
 =
 \begin{bmatrix}
  0 &\kappa(\parm)& 0\\
  -\kappa(\parm)& 0 &\tau(\parm)\\
  0 &-\tau(\parm)& 0
 \end{bmatrix}
 \begin{bmatrix}\vct{T}(\parm)\\\vct{N}(\parm)\\\vct{B}(\parm)\end{bmatrix}.
 \label{eq:FrenetSerret}
\end{equation}
The Frenet-Serret equations will be used in numerous places throughout this document to derive quantities related to the SAR signal model.
\par
One observes that
\begin{equation}
\begin{split}
 \sats'(\parm) &= \vct{T}(\parm)\\
 \sats''(\parm) &= \vct{T}'(\parm) = \kappa(\parm)\vct{N}(\parm)\\
 \sats'''(\parm) &= \kappa'(\parm)\vct{N}(\parm) + \kappa(\parm)\vct{N}'(\parm)\\
 &= \kappa'(\parm)\vct{N}(\parm) -\kappa^2(\parm)\vct{T}(\parm)+\kappa(\parm)\tau(\parm)\vct{B}(\parm)
\end{split}
\end{equation}
Thus, a third order polynomial approximation to the satellite orbit around the point $\parm_0$, which we denote as $\satp(\parm)$, can be expressed as
\begin{equation}
\begin{split}
 \sats(\parm)&\approx\satp(\parm) = \sats(\parm_0) + (s-\parm_0)\vct{T}(\parm_0) + \frac{(s-\parm_0)^2}{2!}\kappa(\parm_0)\vct{N}(\parm_0)\\
 &+ \frac{(s-\parm_0)^3}{3!}[\kappa'(\parm_0)\vct{N}(\parm_0) -\kappa^2(\parm_0)\vct{T}(\parm_0)+\kappa(\parm_0)\tau(\parm_0)\vct{B}(\parm_0)]
\end{split}
\end{equation}
To make the notation more compact, we introduce $\vct{T}_0$, $\vct{N}_0$, $\vct{B}_0$, $\kappa_0$, ${\kappa'}_0$, $\tau_0$ as all the respective functions evaluated at some suitably chosen $\parm_0$.
\begin{equation}
\begin{split}
 \satp(\parm) &= \sats(\parm_0) + (\parm-\parm_0)\vct{T}_0 +  \frac{(\parm-\parm_0)^2}{2}\kappa_0\vct{N}_0\\
 &+ \frac{(\parm-\parm_0)^3}{6}[-\kappa^2_0\vct{T}_0 + {\kappa'}_0\vct{N}_0 + \kappa_0\tau_0\vct{B}_0]
 \end{split}
 \label{eq:diffgeoeqn}
\end{equation}
