\section{Stationary phase}
This section presents the results of applying the stationary phase approximation to the fast-time, arclength parameterized signal model. The final result is seen to be a generalization of the classical hyperbolic model with additional terms of the higher order $\rcoeff_3$ and $\rcoeff_4$  appearing in the range equation and the Stolz interpolation. 
\par
We shall find at the end of the section that the presence of the higher order terms results in a Stolz interpolation scheme where there is no closed form for the interpolation points. Instead, interpolation points are given as the roots of a mathematical expression and are assumed (and demonstrated) to be numerically computable.
\par
Recall from \eqref{eq:bestRangeEquation} that
\begin{equation}
\begin{split}
\targetrange(\parm, \vct\target) &= \sqrt{\lvert\satp(\parm)-\vct\target\rvert^2} = \sqrt{\sum_{k=0}^4\rcoeff_k(\parm-\targetxparm)^k}\\
&=\sqrt{\targetrange^2 + \rcoeff_2(\parm-\targetxparm)^2 + \rcoeff_3(\parm-\targetxparm)^3 + \rcoeff_4(\parm-\targetxparm)^4}
\end{split}
\end{equation}
The phase of the function in \eqref{eq:sparmft} is then given by
\begin{equation}
 \Phi(\parm) = -\kr\sqrt{\targetrange^2 + \rcoeff_2(\parm-\targetxparm)^2 + \rcoeff_3(\parm-\targetxparm)^3 + \rcoeff_4(\parm-\targetxparm)^4} - \kparm\parm
 \label{eg:rangeEq1}
\end{equation}
In order to simplify the calculation, let
\begin{equation}
\begin{split}
 -\targetrange\tan\thetas(\parm) &= \sqrt{\rcoeff_2\parm^2 + \rcoeff_3\parm^3 + \rcoeff_4\parm^4}\\
 &=\parm\sqrt{\rcoeff_2 + \rcoeff_3\parm + \rcoeff_4\parm^2}\\
 &=\parm\gfunc(\parm)
\end{split}
\label{eq:gfunc}
\end{equation}
The above relation is really only a mathematical convenience; however, in the simple linear trajectory case (i.e. airborne SAR), the angle $\thetas(\parm)$ corresponds to the look angle. By substituting this relation in \eqref{eg:rangeEq1}, one finds that
\begin{equation}
\begin{split}
 \Phi(\parm) &= -\targetrange\kr\sec\thetas(\parm-\targetxparm) - \kparm\parm\\
 &= -\targetrange\kr\sec\thetas(\parm-\targetxparm) - \kparm(\parm-\targetxparm) - \kparm\targetxparm
 \end{split}
 \label{eq:stationaryPhasePhase}
\end{equation}
Now
\begin{equation}
\begin{split}
 \frac{\d\Phi(\parm)}{\d\parm} &= \frac{\d\Phi(\parm)}{\d(\parm-\targetxparm)}\frac{\d(\parm-\targetxparm)}{\d\parm}\\
 &=-\targetrange\kr\tan\thetas(\parm-\targetxparm)\sec\thetas(\parm-\targetxparm)\frac{\d\thetas(\parm-\targetxparm)}{\d(\parm-\targetxparm)} -\kparm,
 \end{split}
\end{equation}
and the stationary phase prescription instructs that this should be equated to zero, thus
\begin{equation}
\begin{split}
 -\targetrange\kr\tan\thetas(\parm-\targetxparm)\sec\thetas(\parm-\targetxparm)\frac{\d\thetas(\parm-\targetxparm)}{\d(\parm-\targetxparm)} = \kparm.
 \end{split}
\end{equation}
On the other hand, by taking the derivative of \eqref{eq:gfunc}, one calculates that
\begin{equation}
 -\targetrange\sec^2\thetas(\parm-\targetxparm)\frac{\d\thetas(\parm-\targetxparm)}{\d(\parm-\targetxparm)} = \gfunc(\parm-\targetxparm) + (\parm-\targetxparm)\frac{\d\gfunc(\parm-\targetxparm)}{\d(\parm-\targetxparm)},
\end{equation}
which means that the stationary phase value $\stationaryparm$ is the value which satisfies
\begin{equation}
 \kr\sin\thetas(\stationaryparm-\targetxparm)\left[\gfunc(\stationaryparm-\targetxparm) + (\stationaryparm-\targetxparm)\frac{\d\gfunc(\stationaryparm-\targetxparm)}{\d(\stationaryparm-\targetxparm)}\right] = \kparm.
\end{equation}
By letting $\myparm = \stationaryparm - \targetxparm$, the preceding becomes
\begin{equation}
 \kr\sin\thetas(\myparm)\left[\gfunc(\myparm) + \myparm\frac{\d\gfunc(\myparm)}{\d\myparm}\right] = \kparm.
 \label{eq:stationaryAngleRelation}
\end{equation}
\subsection{Relation between arclength and wavenumber}
The remainder of this section seeks to present a means to invert \eqref{eq:stationaryAngleRelation}, that is, to compute some form of $\stationaryparm$ or $\thetas$ given wavenumbers $\kparm$ and $\kr$.
\par
This subsection develops one of the key approximations used to derive the Stolz interpolation points. We begin by recalling from \eqref{eq:gfunc} that 
\begin{equation}
 -\targetrange\tan\thetas(\parm) = \parm\gfunc(\parm)
\end{equation}
Let $\xang=-\targetrange\tan\thetas(\parm)$ so that
\begin{equation}
 \xang = \parm\gfunc(\parm)
\end{equation}
To invert the above, we note that
\begin{equation}
\begin{split}
 \parm &= \frac{\xang}{\gfunc(\parm)} = \frac{\xang}{\gfunc\left(\frac{\xang}{\gfunc(\parm)}\right)} = \frac{\xang}{\gfunc\left(\frac{\xang}{\gfunc\left(\frac{\xang}{\gfunc(\parm)}\right)}\right)} = ...\\
 \gfunc(\parm) &= \gfunc\left(\frac{\xang}{\gfunc(\parm)}\right) = \gfunc\left(\frac{\xang}{\gfunc\left(\frac{\xang}{\gfunc(\parm)}\right)}\right) = ...
\end{split}
\label{eq:Sdefinition}
\end{equation}
where the two expressions above are terminated after some number of compositions with $\parm=0$. The above two equations are the key expressions used in the approximation. \Scref{sc:stationaryIteration} provides a mathematical argument explaining why the above iterative approach converges. 
\par
From the definition of $\gfunc(\parm)$, one computes that
\begin{equation}
 \gfunc(\parm) + \parm\frac{\d\gfunc(\parm)}{\d\parm}    = \gfunc(\parm) + \frac{\rcoeff_3\parm + 2\rcoeff_4\parm^2}{2\gfunc(\parm)}.% + \frac{\rcoeff_3\parm}{2\gfunc^2(\parm)} + \frac{\rcoeff_4\parm^2}{\gfunc^3(\parm)} 
\end{equation}
Thus,
\begin{equation}
\begin{split}
\gfunc(\parm) + \parm\frac{\d\gfunc(\parm)}{\d\parm} = \gfunc(\parm) + \frac{\rcoeff_3\xang}{2\gfunc^2(\parm)} + \frac{\rcoeff_4\xang^2}{\gfunc^3(\parm)}\\ %+ \frac{\xang}{\gfunc(\parm)}+ \frac{\rcoeff_3\parm}{2\gfunc(\parm)} + \frac{\rcoeff_4\parm^2}{\gfunc(\parm)}
% &= \gfunc\left(\frac{-\targetrange\tan\thetas(\parm)}{\sqrt\rcoeff_2}\right) - \frac{\rcoeff_3\targetrange\tan\thetas(\parm)}{2\gfunc^2\left(\frac{-\targetrange\tan\thetas(\parm)}{\sqrt\rcoeff_2}\right)} + \frac{\rcoeff_4\targetrange^2\tan^2\thetas(\parm)}{\gfunc^3\left(\frac{-\targetrange\tan\thetas(\parm)}{\sqrt\rcoeff_2}\right)}
\end{split}
\end{equation}
The above can be inserted into \eqref{eq:stationaryAngleRelation} to yield an expression that relates, on the left, a function that depends on $\sin\thetas(\myparm)$ (which is written as $\sin\thetas$, with $\tan\thetas=\sin\thetas/\sqrt{1-\sin^2\thetas}$) to, on the right, a function of $\kr, \kparm$.
\begin{equation}
 -\frac{\xang}{\sqrt{\targetrange^2+\xang^2}}\left[\gfunc[\myparm(\xang)] + \frac{\rcoeff_3\xang}{2\gfunc^2[\myparm(\xang)]} + \frac{\rcoeff_4\xang^2}{\gfunc^3[\myparm(\xang)]}\right] = \frac{\kparm}{\kr}.
 \label{eq:angleRelation}
\end{equation}
To be specific, with $\gfunc[\myparm(\xang)]$ evaluated with \eqref{eq:Sdefinition}, the above can be considered as
\begin{equation}
 f(\xang; r) = \frac{\kparm}{\kr}
 \label{eq:wavenumberangle}
\end{equation}
and one observes that only in the case that $\rcoeff_3=\rcoeff_4=0$ is the above independent of $\targetrange$.
\par
Equation \eqref{eq:Sdefinition} permits us to write the stationary phase expression, \eqref{eq:stationaryPhasePhase}, as
% \begin{equation}
%  \myparm = \frac{\xang}{\gfunc(\myparm)}%\approx\frac{-\targetrange\tan\thetas}{\gfunc\left(\frac{-\targetrange\tan\thetas}{\sqrt\rcoeff_2}\right)}
% \end{equation}
% The above allows the phase expression at the stationary point to be written as
\begin{equation}
\begin{split}
 \Phi(\stationaryparm) &= -\targetrange\kr\sec\thetas - \myparm\kparm - \kparm\targetxparm\\
 &= -\targetrange\kr\sec\thetas - \kparm\frac{\xang}{\gfunc[\myparm(\xang)]} - \kparm\targetxparm\\
 %&= -\targetrange\kr\sec\thetas + \targetrange\kparm\frac{\tan\thetas}{\gfunc(s)} - \kparm\targetxparm\\
 &= -\targetrange\left[\kr\sec\thetas - \kparm\frac{\tan\thetas}{\gfunc[\myparm(\xang)]}\right] - \kparm\targetxparm.
 \label{eq:newStolz}
 \end{split}
\end{equation}
By inverting \eqref{eq:wavenumberangle}, one can compute $\xang$ as a function of $\kr, \kparm$. From $\xang=-\targetrange\tan\thetas(\parm)$, the value of $\tan\thetas(\parm)$ can be computed and this can be substituted into the above to yield a phase function dependent only upon the spatial wavenumbers (and the range). The first term (the term multiplied by $\targetrange$) in the last line of \eqref{eq:newStolz} thus represents a modified Stolz interpolation function which can easily be computed numerically. \Anref{an:stolz} outlines the particular implementation of the numerical approach adopted in this work. 
\par
Note that in the case $\rcoeff_3=\rcoeff_4=0$, the above reduces to
\begin{equation}
\begin{split}
 \Phi(\stationaryparm) &= -\targetrange\sqrt{\kr^2 - \frac{\kparm^2}{\rcoeff_2}} - \kparm\targetxparm.
 \label{eq:oldStolz}
 \end{split}
\end{equation}
