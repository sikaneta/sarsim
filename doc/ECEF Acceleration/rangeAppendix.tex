\section{Derivation of the arclength-parameterized range function}
\label{an:rangeComputation}
The range between the satellite and a target plays a critical role in SAR processing. This section presents an arclength parameterized expression for this range function. The final approximation of the section, \eqref{eq:rangeInvariant}, yields a version of the function that only weakly depends on the expansion point used to describe the satellite orbit.
\par
For every scatterer at point $\vct\target$, there is an arclength value, $\targetxparm$, such that $\satp(\targetxparm) - \vct\target$ is perpendicular to $\satp'(\targetxparm)$. Further, $\vct\target$ can be completely determined by 
\begin{equation}
 \vct\target = \satp(\targetxparm) + \targetrange\cos\targetdepression\vct{N}(\targetxparm) + \targetrange\sin\targetdepression\vct{B}(\targetxparm),
\end{equation}
where $\targetrange=\lvert\satp(\targetxparm)-\vct\target\rvert$ and $\targetrange\cos\targetdepression = [\vct\target - \satp(\targetxparm)]\cdot\vct{N}(\targetxparm)$.
\par
One computes that
\begin{equation}
\begin{split}
 \satp(\parm)-\vct\target &= \satp(\parm) - \satp(\targetxparm) - \targetrange\cos\targetdepression\vct{N}(\targetxparm) - \targetrange\sin\targetdepression\vct{B}(\targetxparm)\\
 &= (\parm-\targetxparm)\vct{T}_0 + \frac{(\parm-\parm_0)^2 - (\targetxparm - \parm_0)^2}{2}\kappa_0\vct{N}_0\\
 &+\frac{(\parm-\parm_0)^3 - (\targetxparm - \parm_0)^3}{6}[-\kappa^2_0\vct{T}_0 + {\kappa'}_0\vct{N}_0 + \kappa_0\tau_0\vct{B}_0]\\
 &- \targetrange\cos\targetdepression\vct{N}(\targetxparm) - \targetrange\sin\targetdepression\vct{B}(\targetxparm)\\
 &= (a-b)\biggl[\vct{T}_0 + \frac{a+b}{2}\kappa_0\vct{N}_0 +\frac{a^2+ab+b^2}{6}[-\kappa^2_0\vct{T}_0 + {\kappa'}_0\vct{N}_0 + \kappa_0\tau_0\vct{B}_0]\biggr]\\
 &- \targetrange\cos\targetdepression\vct{N}(\targetxparm) - \targetrange\sin\targetdepression\vct{B}(\targetxparm)
 \end{split},
\end{equation}
where
\begin{align}
 a &= (\parm-\parm_0)\\
 b &= (\targetxparm - \parm_0)
\end{align}
More succinctly, one can write
\begin{equation}
 \satp(\parm)-\vct\target = (a-b)(\alpha_T\vct{T}_0 + \alpha_N\vct{N}_0 + \alpha_B\vct{B}_0)- \targetrange\cos\targetdepression\vct{N}(\targetxparm) - \targetrange\sin\targetdepression\vct{B}(\targetxparm)
\end{equation}
where
\begin{align}
 \alpha_T &= 1 - \kappa^2_0\frac{a^2+ab+b^2}{6}\\
 \alpha_N &= \kappa_0\frac{a+b}{2} + {\kappa'}_0\frac{a^2+ab+b^2}{6}\\
 \alpha_B &= \kappa_0\tau_0\frac{a^2+ab+b^2}{6}
\end{align}
From the Frenet-Serret equations, \eqref{eq:FrenetSerret}, one can make the approximation that
\begin{align}
 \targetrange\cos\targetdepression\vct{N}(\targetxparm)&\approx\targetrange\cos\targetdepression\vct{N}_0 + (\targetxparm-\parm_0)\targetrange\cos\targetdepression(-\kappa_0\vct{T}_0 + \tau_0\vct{B}_0)\\
 \targetrange\sin\targetdepression\vct{B}(\targetxparm)&\approx\targetrange\sin\targetdepression\vct{B}_0 - (\targetxparm-\parm_0)\targetrange\sin\targetdepression\tau_0\vct{N}_0
\end{align}
and since $\targetxparm-\parm_0 = b$, the range expression can be written as
\begin{equation}
\begin{split}
 \satp(\parm)-\vct\target &= [(a-b)\alpha_T + b\targetrange\kappa_0\cos\targetdepression]\vct{T}_0\\
 &+ [(a-b)\alpha_N - \targetrange\cos\targetdepression + b\targetrange\tau_0\sin\targetdepression]\vct{N}_0\\
 &+ [(a-b)\alpha_B- \targetrange\sin\targetdepression - b\targetrange\tau_0\cos\targetdepression]\vct{B}_0
 \end{split}
\end{equation}
The computation of square of the above expression yields a polynomial in $(a-b) = (\parm-\targetxparm)$ (after using sagemath),
\begin{equation}
 \lvert\satp(\parm)-\vct\target\rvert^2 = \sum_{k=0}^5\rcoeff_k(\parm-\targetxparm)^k
\end{equation}
with the following coefficients
\begin{equation}
 \begin{split}
 \rcoeff_0 &= b^{2}\kappa_0^2\targetrange^2\cos^2\targetdepression + b^2\targetrange^{2}\tau_0^{2} + \targetrange^{2}\\
 \rcoeff_1 &= \frac{2b^2\tau_0\targetrange}{3}(b{\kappa'}_0 + 2\kappa_0)\sin\targetdepression - \frac{2b^2\targetrange}{3}(b\kappa_0\tau_0^2 + b\kappa_0^3 + {\kappa'}_0)\cos\targetdepression\\
 \rcoeff_2 &= 1 - [\kappa_0+b{\kappa'}_0 + b^2(\kappa_0^3 + \kappa_0\tau_0^2)]\targetrange\cos\targetdepression + b^2{\kappa'}_0\tau_0\targetrange\sin\targetdepression\\ 
 &+ \frac{b^4}{9}(\kappa_0^4+\kappa_0^2\tau_0^2 + {\kappa'}_0^2) + \frac{b^2\kappa_0}{3}(\kappa_0 + 2b{\kappa'}_0)\\
 \rcoeff_3 &= -\frac{\targetrange}{3}(\kappa_0\tau_0\sin\targetdepression + {\kappa'}_0\cos\targetdepression) + \frac{b^3}{3}(\kappa_0^4+\kappa_0^2\tau_0^2 + {\kappa'}_0^2) + \frac{4b^2}{3}\kappa_0{\kappa'}_0\\ 
 &+ \frac{b\targetrange}{3}[{\kappa'}_0\tau_0\sin\targetdepression - (\kappa_0\tau_0^2+\kappa_0^3)\cos\targetdepression]\\
 \rcoeff_4 &= -\frac{\kappa_0^2}{12} + \frac{13b^2}{36}(\kappa_0^4+\kappa_0^2\tau_0^2 + {\kappa'}_0^2) + \frac{5b\kappa_0{\kappa'}_0}{6}\\
 \rcoeff_5 &= \frac{b}{6}(\kappa_0^4+\kappa_0^2\tau_0^2 + {\kappa'}_0^2 + \kappa{\kappa'}_0)\\
 \rcoeff_6 &= \frac{1}{36}(\kappa_0^4+\kappa_0^2\tau_0^2 + {\kappa'}_0^2)
 \end{split}
 \label{eq:rangeVariant}
\end{equation}
In the region of the chosen point of expansion, $\parm_0$, $b$ evaluates to a relatively small number, thus, one can make the approximation that
\begin{equation}
 \begin{split}
 \rcoeff_0 &= \targetrange^{2}\\
 \rcoeff_1 &= 0\\
 \rcoeff_2 &= 1 - \kappa_0\targetrange\cos\targetdepression\\
 \rcoeff_3 &= -\frac{\targetrange}{3}(\kappa_0\tau_0\sin\targetdepression + {\kappa'}_0\cos\targetdepression)\\
 \rcoeff_4 &= -\frac{\kappa_0^2}{12}\\
 \rcoeff_5 &= 0\\
 \rcoeff_6 &= 0
 \end{split}
 \label{eq:rangeInvariant}
\end{equation}
