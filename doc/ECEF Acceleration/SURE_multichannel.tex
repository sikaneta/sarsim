\section{Multi-channel design and configuration}
\label{sc:multigeometry}
Instead of beam spoiling or spotlighting, we propose an operating configuration that time-multiplexes a sequence of beams using an $\channelM + 1$ multichannel design, as illustrated in \ref{fg:fivechan} (with a five-channel system). This design can be realised with a phased-array that has the ability to change transmit and receive beam tables on a pulse by pulse basis \cite{CalabreseDiego2014, SikanetaGierullTGRS2015}. Additionally, $\channelM + 1$ digitizers and a switching mechanism to route and combine the measurements from each phased-array element (i.e. form subarrays) are required to realise this multichannel system. Although we shall consider a general $\channelM + 1$ channel system, we shall occasionally choose specific values of $\channelM$ for illustrative purposes. The proposed system is also a uniform antenna array.
\begin{figure}[h!]
\begin{center}
 \resizebox{0.8\columnwidth}{!}{\input{fivechan.pdf_tex}}
 \caption{Five channel schematic for design. Circles denote the phase-centre location while the angle denotes the direction of the transmit and receive beams.}
 \label{fg:fivechan}
 \end{center}
\end{figure}
\par
As can be seen from the figure, if viewed vertically, at each sampling point, the system is configured to make $5$ measurements with $5$ different antenna patterns. If each of these antenna patterns has a beamwidth given by $\threeDB/(\channelM + 1)$, then at each sampling point, the system scans over a total azimuth beamwidth of $\threeDB$. The reduced beamwidth at each sampling instant corresponds to a reduced required PRF for each channel according to
\begin{equation}
 \prfEffective \geq \aziBW/(\channelM + 1).
\end{equation}
\subsection{System design size}
Consider the requirement for an azimuth resolution of $\resxDesired$. From fundamental SAR theory, for a classical stripmap mode, this corresponds to an antenna length given by \cite{Cumming2005}
\begin{equation}
 \antennaLengthDesired = 2\resxDesired,
\end{equation}
which, in turn, corresponds to a required azimuth beamwidth of
\begin{equation}
 \threeDBDesired = \frac{\wavelength}{2\resxDesired}.
\end{equation}
If this desired beamwidth is divided into $\channelM + 1$ parts of width
\begin{equation}
 \threeDBEffective = \frac{\threeDBDesired}{\channelM + 1}=\frac{\wavelength}{2(\channelM + 1)\resxDesired},
\end{equation}
then each channel requires an antenna of length
\begin{equation}
 \antennaLengthEffective = 2(\channelM + 1)\resxDesired.
\end{equation}
The required PRF is given by
\begin{equation}
 \prfEffective = \frac{2\satv}{\wavelength}\threeDBEffective = \frac{2\satv}{\antennaLengthEffective} = \frac{\satv}{(\channelM + 1)\resxDesired},
 \label{eq:requiredPRF}
\end{equation}
which corresponds to a required two-way phase-centre separation of
\begin{equation}
 \phaseSep = (\channelM + 1)\resxDesired.
\end{equation}
Now, with a transmit antenna of length $\antennaLengthEffective = 2(\channelM + 1)\resxDesired$ and a receive antenna of the same length, the effective phase centre positions are given by  multiples of $\phaseSep = (\channelM + 1)\resxDesired$. The total antenna length, as illustrated in figure \ref{fg:antennaLenghts}, will be given by
\begin{equation}
 \antennaLength = (\channelM + 1)\antennaLengthEffective = 2(\channelM + 1)^2\resxDesired.
\end{equation}
\begin{figure}[h!]
\begin{center}
 \resizebox{0.5\columnwidth}{!}{\input{antennaLengths.pdf_tex}}
 \caption{Antenna Lengths to achieve desired resolution for an example 11 channel system for a desired resolution of $\resxDesired$.}
 \label{fg:antennaLenghts}
 \end{center}
\end{figure}
Let us examine what this means for a specific case of $\resxDesired = 0.1$. As listed in table \ref{tb:Simulation}, a traditional stripmap SAR would have to be $0.2$ m in azimuth length to achieve this resolution. Additionally the required PRF would be $\prf = 75$ KHz for a satellite travelling at 7500 m/s which corresponds to a rather limited swath. On the other hand, with $\channelM=10$, the required PRF is $\prfEffective = 6.818$ KHz which corresponds to a range-swath width of approximately 22 Km (minus any time needed for chirp transmission) which would be even larger in ground range.
\par
The choice of $\resxDesired=0.1, \channelM=10$ leads to an antenna of length $24.2$ m with each subaperture having a length of $2.2$ m. This antenna length of $24.2$ m is only about $60$\% longer than RADARSAT-2.
\begin{table}[h!]
\begin{center}
\caption{System parameters for $\resxDesired=0.1 \text{m}$ and $\satv=7500 \text{m/s}$.}
\label{tb:Simulation}
 \begin{tabular}{r|c|c|c|c}\\\hline
  {\bf $\channelM$} & {\bf $\antennaLengthEffective$ m} & {\bf $\antennaLength$ m} & {\bf $\prfEffective$ Hz} & {\bf Swath (slant-range Km)}\\\hline 
0 & 0.20 & 0.20 & 75000 & 2.00\\\hline
2 & 0.60 & 1.80 & 25000 & 6.00\\\hline
4 & 1.00 & 5.00 & 15000 & 10.00\\\hline
6 & 1.40 & 9.80 & 10710 & 14.00\\\hline
8 & 1.80 & 16.20 & 8330 & 18.00\\\hline
{\bf 10} & {\bf 2.20} & {\bf 24.20} & {\bf 6810} & {\bf 22.00}\\\hline
12 & 2.60 & 33.80 & 5760 & 26.00\\\hline
14 & 3.00 & 45.00 & 5000 & 30.00\\\hline
 \end{tabular}
 \end{center}
\end{table}
\subsection{Traditional HRWS configuration and design}
It is useful to examine the implication of utilising a traditional HRWS configuration that does not use a sequence of beams as proposed in this paper. With this design, $\channelM + 1$ channels transmit a wide beam that covers the desired range of angles corresponding to the desired resolution; see \fgref{fg:equivHRWS}. The spatial distribution of two-way phase-centres at each pulse again compensates for a lower PRF according to \eqref{eq:requiredPRF} \cite{GebertPHD}.
\par
To satisfy the spatial sampling requirement, the two-way phase-centre separation must be the same as with the proposed multi-beam design. This means that the receive antenna elements must be spaced by $2\resxDesired$, giving a total receive antenna length of $2(\channelM + 1)\resxDesired$ which is $(\channelM + 1)$ times shorter than the length proposed by the multi-beam design. This means that, on a pulse-by-pulse basis, the total receive area to capture reflected flux is reduced by a factor of $\channelM + 1$ resulting in a corresponding loss in SNR. For this reason, the multi-beam design is recommended.
\begin{figure}[h!]
\begin{center}
 \resizebox{0.7\columnwidth}{!}{\input{equivalentHRWS.pdf_tex}}
 \caption{Equivalent HRWS system.}
 \label{fg:equivHRWS}
 \end{center}
\end{figure}
