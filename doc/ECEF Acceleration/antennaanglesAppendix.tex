\section{Antenna pattern angles}
\label{an:Angles}
Let us examine $\uRangeVectorParm\cdot\vct{N}(\parm)$ and $\uRangeVectorParm\cdot\vct{B}(\parm)$ by exanding these terms around $\parm_0, \targetnoparmzero$. A first order expansion of the look vector yields
\begin{equation}
 \uRangeVectorParm \approx \uRangeVectorParmZero + (\parm-\parm_0)\frac{\mtx{P}_{\rangeVectorParmZero}\vct{T}_0}{\targetrangez} + \frac{\mtx{P}_{\rangeVectorParmZero}(\targetnoparm-\targetnoparmzero)}{\targetrangez},
\end{equation}
where $\rangeVectorParmZero=\defrangeVectorParmZero$.
\par
Since $\vct{T}_0$ is perpendicular to $\rangeVectorParmZero$, one immediately finds that $\mtx{P}_{\rangeVectorParmZero}\vct{T}_0 = \vct{T}_0$. Furthermore
\begin{equation}
 \targetnoparmzero = \sat(\parm_0) + \targetrangez\cos\targetdepressionz\vct{N}_0 + \targetrangez\sin\targetdepressionz\vct{B}_0,
 \label{eq:pzero}
\end{equation}
and
\begin{equation}
\begin{split}
 \targetnoparm &= \sat(\parm_0) + (\targetxparm - \parm_0)\vct{T}_0 + \frac{(\targetxparm - \parm_0)^2}{2}\kappa_0\vct{N}_0 + \frac{(\targetxparm - \parm_0)^3}{6}[-\kappa^2_0\vct{T}_0 + {\kappa'}_0\vct{N}_0 + \kappa_0\tau_0\vct{B}_0]\\ 
 &+ \targetrange\cos\targetdepression\left[\vct{N}_0 + (\targetxparm - \parm_0)\biggl(-\kappa_0\vct{T}_0 + \tau_0\vct{B}_0\biggr)\right] + \targetrange\sin\targetdepression\left[\vct{B}_0 + (\targetxparm - \parm_0)\biggl(- \tau_0\vct{N}_0\biggr)\right]
 \label{eq:ptarget}
 \end{split}
\end{equation}
In the above, we have used the Frenet-Serret equations of \eqref{eq:FrenetSerret} to make the approximation that
\begin{align}
 \mtx{N}(\parm)&\approx\vct{N}_0 + (\parm - \parm_0)\biggl[-\kappa_0\vct{T}_0 + \tau_0\vct{B}_0\biggr]\\
 \mtx{B}(\parm)&\approx\vct{B}_0 + (\parm - \parm_0)\biggl[- \tau_0\vct{N}_0\biggr]
\end{align}
Now with
\begin{equation}
 \uRangeVectorParmZero = \cos\targetdepressionz\vct{N}_0 + \sin\targetdepressionz\vct{B}_0
\end{equation}
one can calculate that
\begin{equation}
 \uRangeVectorParmZeroPerp = \sin\targetdepressionz\vct{N}_0 - \cos\targetdepressionz\vct{B}_0.
 \label{eq:rperp}
\end{equation}
By combining equations \eqref{eq:pzero}, \eqref{eq:ptarget} and \eqref{eq:rperp}, one arrives at the following expressions
\begin{equation}
  \mtx{P}_{\rangeVectorParmZero}(\targetnoparm-\targetnoparmzero) = [\uRangeVectorParmZeroPerp\cdot(\targetnoparm-\targetnoparmzero)]\uRangeVectorParmZeroPerp
\end{equation}
and
\begin{equation}
\begin{split}
  \uRangeVectorParmZeroPerp\cdot(\targetnoparm-\targetnoparmzero)&\approx\frac{1}{6}(\targetxparm-\parm_0)^3[\kappa_0\tau_0\cos\targetdepressionz - {\kappa'}_0\sin\targetdepressionz]+\frac{1}{2}\kappa(\targetxparm-\parm_0)^2\sin\targetdepressionz\\
  &- (\targetxparm-\parm_0)\targetrange\tau_0\cos(\targetdepression-\targetdepressionz) -\targetrange\sin(\targetdepression-\targetdepressionz)
 \end{split}
\end{equation}
The only term that makes a difference in the above is $\targetrange\sin(\targetdepression-\targetdepressionz)$, thus
\begin{equation}
 \frac{\mtx{P}_{\rangeVectorParmZero}(\targetnoparm-\targetnoparmzero)}{\targetrangez}\approx-\frac{\targetrange}{\targetrangez}\sin(\targetdepression-\targetdepressionz)\uRangeVectorParmZeroPerp
\end{equation}
Thus, one finally arrives at
\begin{equation}
\begin{split}
 \uRangeVectorParm &\approx \uRangeVectorParmZero + (\parm-\parm_0)\frac{\vct{T}_0}{\targetrangez} + \frac{\targetrange}{\targetrangez}\sin(\targetdepression-\targetdepressionz)\uRangeVectorParmZeroPerp\\
 &=\cos\targetdepressionz\vct{N}_0 + \sin\targetdepressionz\vct{B}_0 + (\parm-\parm_0)\frac{\vct{T}_0}{\targetrangez}\\ 
 &+ \frac{\targetrange}{\targetrangez}\sin(\targetdepression-\targetdepressionz)[\sin\targetdepressionz\vct{N}_0 - \cos\targetdepressionz\vct{B}_0]\\
 &= [\cos\targetdepressionz + \frac{\targetrange}{\targetrangez}\sin\targetdepressionz\sin(\targetdepression-\targetdepressionz)]\vct{N}_0\\
 &+ [\sin\targetdepressionz - \frac{\targetrange}{\targetrangez}\cos\targetdepressionz\sin(\targetdepression-\targetdepressionz)]\vct{B}_0\\
 &+ (\parm-\parm_0)\frac{\vct{T}_0}{\targetrangez}
 \end{split}
\end{equation}
Note that in the neighbourhood of $\targetnoparmzero$, $\sin(\targetdepression-\targetdepressionz)\approx\targetdepression-\targetdepressionz$ and $\frac{\targetrange}{\targetrangez}\approx 1$ one arrives at the approximation that
\begin{equation}
\begin{split}
 \uRangeVectorParm &\approx \cos\targetdepression\vct{N}_0 + \sin\targetdepression\vct{B}_0 + (\parm-\parm_0)\frac{\vct{T}_0}{\targetrangez}
 \end{split}
\end{equation}
Substitution of the approximations for $\vct{N}(\parm)$ and $\vct{B}(\parm)$ yields
\begin{equation}
\begin{split}
 \uRangeVectorParm\cdot\vct{N}(\parm)&\approx\cos\targetdepressionz + \frac{\targetrange}{\targetrangez}\sin\targetdepressionz\sin(\targetdepression-\targetdepressionz)\\
 &+ \tau_0(\parm-\parm_0)[\sin\targetdepressionz - \frac{\targetrange}{\targetrangez}\cos\targetdepressionz\sin(\targetdepression-\targetdepressionz)]\\
 &-\kappa_0\frac{(\parm-\parm_0)^2}{\targetrangez}
\end{split}
 \end{equation}
and
\begin{equation}
\begin{split}
 \uRangeVectorParm\cdot\vct{B}(\parm)&\approx\sin\targetdepressionz - \frac{\targetrange}{\targetrangez}\cos\targetdepressionz\sin(\targetdepression-\targetdepressionz)\\
 &- \tau_0(\parm-\parm_0)[\cos\targetdepressionz + \frac{\targetrange}{\targetrangez}\sin\targetdepressionz\sin(\targetdepression-\targetdepressionz)]
\end{split}
\end{equation} 
In summary, and with only the non-negligible terms retained (for the Sentinel-1 orbit examined in this document, $\kappa_0$ and $\tau_0$ take on values of approximately $10^{-7}$, $10^{-8}$, repectively which means that they can be considered negligible over over a scene of several hundred kilometers in $\parm$), one finds that
\begin{align}
 \vhat{\range}(\stationaryparm)\cdot\vct{T}(\stationaryparm) &= -\kparm/\kr\label{eq:ender}\\
 \vhat{\range}(\stationaryparm)\cdot\vct{N}(\stationaryparm) &= \cos\targetdepressionz + \frac{\targetrange}{\targetrangez}\sin\targetdepressionz\sin(\targetdepression-\targetdepressionz)\label{eq:cos}\\
 \vhat{\range}(\stationaryparm)\cdot\vct{B}(\stationaryparm) &= \sin\targetdepressionz - \frac{\targetrange}{\targetrangez}\cos\targetdepressionz\sin(\targetdepression-\targetdepressionz),\label{eq:sin}
\end{align}
which leads the following
\begin{equation}
 \begin{split}
  &\alongtrack\kr\vhat{\range}(\stationaryparm)\cdot\vct{T}(\stationaryparm) + \kr\acrosstrack\cdot[\vhat{\range}(\stationaryparm)\cdot\vct{N}(\stationaryparm), \vhat{\range}(\stationaryparm)\cdot\vct{B}(\stationaryparm)]\\
  &=-\alongtrack\kparm+\kr\acrosstrack\cdot\uRangeVectorParmZero+\kr\targetrange\frac{\acrosstrack\cdot\uRangeVectorParmZeroPerp}{\targetrangez}\sin(\targetdepression-\targetdepressionz)
 \end{split}
\end{equation}
Note that in the neighbourhood of $\targetrangez, \targetdepressionz$, \eqref{eq:cos} and \eqref{eq:sin} closely approximate the first order expansion of the cosine and sine functions; thus, in this neighbourhood, one may write
\begin{align}
 \vhat{\range}(\stationaryparm)\cdot\vct{T}(\stationaryparm) &= -\kparm/\kr\\
 \vhat{\range}(\stationaryparm)\cdot\vct{N}(\stationaryparm) &= \cos\targetdepression\label{eq:cossimp}\\
 \vhat{\range}(\stationaryparm)\cdot\vct{B}(\stationaryparm) &= \sin\targetdepression\label{eq:sinsimp}
\end{align}
% By omitting terms proportional to $\kappa_0$ and $\tau_0$, the previous two expressions combine to yield the approximation:
% \begin{equation}
%  \begin{bmatrix}
%  \uRangeVectorParm\cdot\mtx{N}(\parm)\\
%  \uRangeVectorParm\cdot\mtx{B}(\parm)
%  \end{bmatrix}
% \approx
% \uRangeVectorParmZero + \uRangeVectorParmZeroPerp\frac{(\targetnoparm-\targetnoparmzero)\cdot\uRangeVectorParmZeroPerp}{\range_0}
% \end{equation}
% where the unit vector $\uRangeVectorParmZeroPerp$ is perpendicular to $\uRangeVectorParmZero$ and
% % \begin{align}
%  \uRangeVectorParm\cdot\mtx{N}(\parm)&\approx\cos\depression_0 + \frac{\sin\depression_0}{\range(\parm_0,\targetnoparmzero)}(\targetnoparm - \targetnoparmzero)\cdot[\sin\depression_0\vct{N}_0 - \cos\depression_0\vct{B}_0]\\
%  \uRangeVectorParm\cdot\mtx{B}(\parm)&\approx\sin\depression_0 - \frac{\cos\depression_0}{\range(\parm_0,\targetnoparmzero)}(\targetnoparm - \targetnoparmzero)\cdot[\sin\depression_0\vct{N}_0 - \cos\depression_0\vct{B}_0]
% \end{align}
% where we have defined
% \begin{equation}
%  \uRangeVectorParmZero = \cos\depression_0\vct{N}_0 + \sin\depression_0\vct{B}_0,
% \end{equation}
%Because $\uRangeVectorParmZero$ has been chosen to be perpendicular to $\vct{T}_0$ and by omitting terms proportional to $\kappa_0$ and $\tau_0$, 
% This can be written as
% 
% For instance,
% \begin{equation}
% \begin{split}
%  \frac{\d\uRangeVectorParm\cdot\vct{N}(\parm)}{\d\parm} &= \left\{\frac{\d\uRangeVectorParm}{\d\parm}\right\}\cdot\vct{N}(\parm) + \uRangeVectorParm\cdot\vct{N}'(\parm)\\
%  &= \frac{\mtx{P}_{\rangeVectorParm}\vct{r}'(\parm, \targetnoparm)}{\lvert\rangeVectorParm\rvert}\cdot\vct{N}(\parm) + \uRangeVectorParm\cdot\left[-\kappa(\parm)\vct{T}(\parm) + \tau(\parm)\vct{B}(\parm)\right]\\
%  &= \frac{\mtx{P}_{\rangeVectorParm}\mtx{T}(\parm)}{\lvert\rangeVectorParm\rvert}\cdot\vct{N}(\parm) + \uRangeVectorParm\cdot\left[-\kappa(\parm)\vct{T}(\parm) + \tau(\parm)\vct{B}(\parm)\right],
%  \end{split}
% \end{equation}
% where the Frenet-Serret equations of \eqref{eq:FrenetSerret} have been used. Similarly
% \begin{equation}
%  \frac{\d\uRangeVectorParm\cdot\vct{B}(\parm)}{\d\parm} = \frac{\mtx{P}_{\rangeVectorParm}\mtx{T}(\parm)}{\lvert\rangeVectorParm\rvert}\cdot\vct{B}(\parm) + -\tau(\parm)\uRangeVectorParm\cdot\vct{N}(\parm)
% \end{equation}
% If one chooses $\parm_0$ such that $\uRangeVectorParmZero$ is perpendicular to $\vct{T}_0$, then the expressions evaluate to
% \begin{align}
%  \frac{\d\uRangeVectorParm\cdot\vct{N}(\parm)}{\d\parm}\Biggr|_{\parm_0} &= \tau_0\uRangeVectorParmZero\cdot\vct{B}_0\\
%  \frac{\d\uRangeVectorParm\cdot\vct{B}(\parm)}{\d\parm}\Biggr|_{\parm_0} &= -\tau_0\uRangeVectorParmZero\cdot\vct{N}_0
% \end{align}
% Thus
% \begin{align}
%  \uRangeVectorParm\cdot\vct{N}(\parm) &\approx\uRangeVectorParmZero\cdot\vct{N}_0 + (\parm-\parm_0)\tau_0\uRangeVectorParmZero\cdot\vct{B}_0\\
%  \uRangeVectorParm\cdot\vct{B}(\parm) &\approx\uRangeVectorParmZero\cdot\vct{B}_0 + -(\parm-\parm_0)\tau_0\uRangeVectorParmZero\cdot\vct{N}_0
% \end{align}
% which, for small values of the torsion, $\tau(\parm)$, can be considered constant. 
% \begin{align}
%  \vhat{\range}(\stationaryparm)\cdot\vct{T}(\stationaryparm) &= -\kparm/\kr\\
%  \vhat{\range}(\stationaryparm)\cdot\vct{N}(\stationaryparm) &= \left[\uRangeVectorParmZero + \frac{\mtx{P}_{\rangeVectorParmZero}(\targetnoparm-\targetnoparmzero)}{\lvert\rangeVectorParmZero\rvert}\right]\cdot\vct{N}_0\\
%  \vhat{\range}(\stationaryparm)\cdot\vct{B}(\stationaryparm) &= \left[\uRangeVectorParmZero + \frac{\mtx{P}_{\rangeVectorParmZero}(\targetnoparm-\targetnoparmzero)}{\lvert\rangeVectorParmZero\rvert}\right]\cdot\vct{B}_0
% \end{align}
\par
In summary, \eqref{eq:ender}, \eqref{eq:cos}, \eqref{eq:sin}, \eqref{eq:cossimp}, \eqref{eq:sinsimp} and \eqref{eq:kparmangle} lead to the following expression for the signal in $k$-space
% 
% The above expression has been expanded around $\parm_0$ under the condition that $\uRangeVectorParmZero$ is perpendicular to $\vct{T}_0$. This condition holds for a set of points $\targetnoparm$, but not all. An expression for $\uRangeVectorParmZero = \vct{c}(\parm_0)-(\targetnoparm-\targetnoparmzero)/\lvert\vct{c}(\parm_0)-(\targetnoparm-\targetnoparmzero)\rvert$ allows for a generalization of the above equations
% \begin{equation}
%  \hat{\range}(\parm_0, \targetnoparm) \approx \hat{\range}(\parm_0, \targetnoparmzero) + \frac{\mtx{P}_{\vct{r}(\parm_0, \targetnoparmzero) }[\targetnoparm - \targetnoparmzero]}{\lvert\vct{r}(\parm_0, \targetnoparmzero)\rvert} 
% \end{equation}
% \par
%The expression for the signal in $k$-space is then given by
\begin{equation}
\begin{split}
 &\SkSf{\channelIndex}(\kr,\kparm)=\Snvelope(\krc)\eex{\im\alongtrack\kparm-\im\kr\acrosstrack\cdot\uRangeVectorParmZero}\\
 &\int\frac{\reflectivity(\targetnoparm)\eex{-\im\kr\targetrange\frac{\acrosstrack\cdot\uRangeVectorParmZeroPerp}{\targetrangez}\sin(\targetdepression-\targetdepressionz)}\dPattern{\channelIndex}[\kr, -\kparm/\kr, \cos\targetdepression, \sin\targetdepression]}{\amplitude\range^2(\stationaryparm[\kr,\kparm])}\eex{-\im\kr\amplitude\range(\stationaryparm[\kr,\kparm])}\d\targetnoparm.
 \end{split}
\end{equation}
