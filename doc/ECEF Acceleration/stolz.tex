\section{Numerical implementation of the Stolz interpolation}
\label{an:stolz}
This section outlines an approach to support numerical implementation of the proposed, generalized Stolz interpolation. Because the generalized Stolz interpolation points do not have a closed form as they do in, for instance, \cite{Cumming2005, Bamler1992}, this section presents a numerical approach. There is nothing particularly insightful about the proposed algorithm; in fact, this section aims to present an approach that avoids confusion and facilitates numerical implementation.
\par
So, to begin, recall that
\begin{equation}
 \xang = -\targetrange\tan\thetas,
\end{equation}
so that
\begin{align}
 \sin\thetas &= -\frac{\xang}{\sqrt{\targetrange^2+\xang^2}}\label{eq:angsin}\\
 \cos\thetas &= \frac{\targetrange}{\sqrt{\targetrange^2+\xang^2}}\label{eq:angcos}.
\end{align}
Also recall that
\begin{equation}
 -\frac{\xang}{\sqrt{\targetrange^2+\xang^2}}\biggl[\gfunc(\parm)+\frac{\rcoeff_3\xang}{2\gfunc^2(\parm)}+\frac{\rcoeff_4\xang^2}{\gfunc^3(\parm)}\biggr] = \frac{\kparm}{\kr}.
\end{equation}
From \eqref{eq:newStolz}, when one makes the (Stolt interpolation) change of variables,
\begin{equation}
 \krstolt = \kr\sec\thetas - \kparm\frac{\tan\thetas}{\gfunc(\parm)},
\end{equation}
which leads to the relation that
\begin{equation}
 \kr = \frac{\targetrange\krstolt}{\sqrt{\targetrange^2+\xang^2}} - \frac{\kparm\xang}{\gfunc(\parm)\sqrt{\targetrange^2+\xang^2}}.
\end{equation}
\subsection{Simplification for iterative root-finding}
In the opinion of the authors, implementation of the Stolz interpolation (into computer code) benefits from defining the following function
\begin{equation}
 \shelp{l}{m}{n} = \frac{\xang^l}{(\targetrange^2+\xang^2)^m\gfunc^n(\parm)}
\end{equation}
The Newton iterative root finding method requires computation of the derivative of the above function, which, in turn, requires caculation of the derivative of $\gfunc(\parm)$ with respect to $\xang$. Specifically,
\begin{equation}
\begin{split}
 \frac{\d\gfunc(\parm)}{\d\xang} &= \frac{\d\gfunc(\parm)}{\d\parm}\frac{\d\parm}{\d\xang}\\
 &=\frac{\rcoeff_3+2\rcoeff_4\parm}{2\gfunc(\parm)}\frac{\d\parm}{\d\xang}\\
 &=\left(\frac{\rcoeff_3}{2\gfunc(\parm)}+\frac{\rcoeff_4\xang}{\gfunc^2(\parm)}\right)\frac{\d\parm}{\d\xang}
\end{split}
\label{eq:dgpart1}
\end{equation}
Because $\xang=\parm\gfunc(\parm)$, one derives
\begin{equation}
\begin{split}
 1 &= \frac{\d\parm}{\d\xang}\gfunc(\xparm) + \parm\frac{\d\gfunc(\parm)}{\d\parm}\frac{\d\parm}{\d\xang}\\
 & = \frac{\d\parm}{\d\xang}\left[\gfunc(\xparm) + \parm\frac{\rcoeff_3+2\rcoeff_4\parm}{2\gfunc(\parm)}\right]\\
 & = \frac{\d\parm}{\d\xang}\left[\gfunc(\xparm) + \frac{\xang}{\gfunc(\parm)}\left(\frac{\rcoeff_3}{2\gfunc(\parm)} + \frac{\rcoeff_4\xang}{\gfunc^2(\parm)}\right)\right]\\
\end{split}
\label{eq:dgpart2}
\end{equation}
By substituting \eqref{eq:dgpart2} into \eqref{eq:dgpart1}, one arrives at
\begin{equation}
 \frac{\d\gfunc(\parm)}{\d\xang} = \frac{\rcoeff_3\gfunc^2(\xparm) + 2\rcoeff_4\xang\gfunc(\xparm)}{2\gfunc^4(\xparm) + \rcoeff_3\xang\gfunc(\xparm) + 2\rcoeff_4\xang^2}
\end{equation}
The derivative of $\shelp{l}{m}{n}$ is thus given by\footnote{For use in a Newton numerical, iterative, root-finding procedure},
\begin{multline}
 \frac{\d\shelp{l}{m}{n}}{\d\xang} = \shelp{l-1}{m}{n}\\
 \left(l-m\frac{2\xang^2}{\targetrange^2+\xang^2} - n\frac{\rcoeff_3\xang\gfunc(\xparm) + 2\rcoeff_4\xang^2}{2\gfunc^4(\xparm) + \rcoeff_3\xang\gfunc(\xparm) + 2\rcoeff_4\xang^2}\right)
\end{multline}
\par
With the aforementioned definition,
\begin{equation}
 \kr = \targetrange\krstolt\shelp{0}{1/2}{0} - \kparm\shelp{1}{1/2}{1},
 \label{eq:stoltRelation}
\end{equation}
and, when this is substituted into \eqref{eq:angleRelation}, along with the realtions in \eqref{eq:angsin} and \eqref{eq:angcos}, one obtains
\begin{equation}
\begin{split}
 &\targetrange\krstolt\shelp{1}{1}{-1} - \kparm\shelp{2}{1}{0}
 +\frac{\targetrange\krstolt\rcoeff_3}{2}\shelp{2}{1}{2}\\
 &+\left(\targetrange\krstolt\rcoeff_4-\frac{\rcoeff_3\kparm}{2}\right)\shelp{3}{1}{3}
 - \rcoeff_4\kparm\shelp{4}{1}{4} + \kparm = 0
\end{split}
\label{eq:stoltInversionX}
\end{equation}
Equation \ref{eq:stoltInversionX} permits numerical computation of $\xang$ given values for $\kparm$ and $\krstolt$. This computed value for $\xang$ can be inserted into \ref{eq:stoltRelation} to provide the particular value of $\kr$ at which the data need to be estimated (through interpolation).
