\section{Multichannel signal processing for high resolution}
%This section presents the promised third component of the improved SAR imaging capability; namely, the fundamentals of a suitable signal processing method. As we have already seen, the desired configuration impacts the design. The signal processing method presented in this section, on the other hand, is one of potentially several others, and its adoption does not place any restrictions upon the configuration or the design.  
%\par
The material in \scref{sc:multigeometry} depends on a strict timing regime. A practical system may only operate with near-ideal timing conditions and this raises the questions of impact and how best to process the data under these conditions. 
\par
Even if timing conditions are perfect, i.e. the \gls{dpca} condition has been met, \cite{Sikaneta2014}, the best approach to data processing needs further investigation. A simple approach first concatenates the measurements from each beam into a uniformly-sampled time series and then transforms the data from the fast-time, slow-time domain into the fast-time Doppler domain. Given that, for each beam direction, the data correspond to different Doppler centroids, one could assign each response to different portions of the Doppler spectrum (or to different Doppler frequency bands). The union of these Doppler frequency bands corresponds to a wider Doppler spectrum and thereby to higher overall azimuth resolution. If the Doppler bands are non-overlapping, the concept of the union of the frequency bands is straight-forward. Optimal processing of Doppler bands that do overlap, however, requires a more rigorous approach. Optimal processing of data collected under non-ideal timing further calls for a flexible yet robust processing approach.
\par
Given that processing should apply to very high-resolution systems, it is best that the approach be suitable for a wide-band system. 
%\subsection{Orbit considerations}
%Satellite trajectory models for space-based SAR systems, such as the parabolic or hyperbolic range models, need to be improved to handle high-resolution SAR imaging modes with long synthetic apertures. For instance, such is the case for the staring spotlight mode on TerraSAR/TanDEM-X. Several authors, such as those in \cite{PratsIraola2014, Meng2018, Wu2016}, have derived methods to achieve improved SAR processing methods for very high-resolution systems. Reference \cite{PratsIraola2014} provides an excellent description of the approach most likely taken with 0.19cm resolution (azimuth) of the staring spotlight mode on TerraSAR/TanDEM-X.
%\par
%In addition to the new high-resolution spotlight modes, developments in multi-channel SAR systems promise the ability to also create high-resolution imagery, \cite{NovelRadar}. This document presents a first-principles-based model for the multi-channel SAR mode that extends the work of, for instance, \cite{NovelRadar, Sikaneta2014}, thereby developing the theory to process very high-resolution SAR imagery from multi-channel configurations. 
% \subsection{Signal processing components}
% The document is structured as follows: the first section outlines limitations inherent with a circular orbit assumption (even locally) demonstrating that not only is the curvature of the orbit an issue for high resolution SAR, but so also the torsion (curvature and torsion to be defined in \scref{sc:diffgeoreview}). The section shows that uniform slow-time sampling does not, in general, equate to uniform azimuth spatial sampling. These results are demonstrated with real Sentinel-1 orbit data.
% \par
% The next section reviews some introductory concepts from differential geometry and applies the thoery to the curve of a SAR satellite in an Earth-Centred, Earth-Fixed coordinate system. The adoption of the differential geometry approach introduces an arclength parameterization of the SAR orbit which provides a natural setting for SAR imaging. This section is followed by a description of how physical parameters such as the state vectors or the accelerations (time based) transform into differential geometry parameters such as the curvature and torsion.
% \par
% Sections \ref{sc:rangeHistory} and \ref{sc:sararc} extend the work reported in \cite{NovelRadar, Sikaneta2014} by applying the differential geometry approach to SAR imaging, specifically to a SAR antenna that contains a spatial offset baseline. The inclusion of the baseline sets the stage for multi-channel processing. The next section transforms the signal into the wavenumber domain leading to \Scref{sc:antennaAngles} which examines the transformation of the antenna pattern in the wavenumber domain. \Scref{sc:antennaAngles} shows, importantly, that the individual data signals from each channel can be described by a commmon componenent that does not depend on the channel baseline or channel azimuth antenna pattern, and a varying component that depends only on the arclength parameter (there is only a negligeable range dependence on the multi-channel signal processing). This greatly simplifies and accelerates the multi-channel signal processing.
% \par
% The next section applies an already published multi-channel signal processing approach to the developed signal. This is followed by a demonstration of the entire approach to the super resolution approach.
% \par
% The authors know no better language than mathematics to describe the new parameterizations of the SAR signal and to describe this signal in different domains. That stated, we repeatedly strive to minimize the amount of mathematics and to phrase the derivations and interpretations of the results using physical quantities and concepts. In an attempt to make the document more readable, mathematical derivations have been placed into an appendix with only the results of the mathematical investigations reported in the main text.
