\section{MIMO configuration}
\label{an:mimo}
This section demonstrates how subarrays of a uniform phased array antenna can be combined to yield signals of the form of \eqref{eq:antennaArgument2}. In the configuration described below, Transmit/Receive T/R modules form the basic element of the phased array. 
\par
As illustrated in \fgref{fg:mimo}, 
\begin{figure}[ht!]
    \resizebox{\textwidth}{!}{\input{mimoconfig.pdf_tex}}
	\caption{Configuration of transmit and receive. Each transmitted signal is received by all receivers}
	\label{fg:mimo}
\end{figure}
assume that every transmission, from every element, is received and summed independently (i.e. assume superpostiion applies). Assume further that signals are transmitted at time $\fasttime$, but with different delays on transmission (programmed delays) for each T/R module
and that the echoes of these signals are received at time $\fasttime + 2\targetrange(\parm)/c$, but with different Rx delays for each T/R module (programmed delays)
\par
Now, the time delay to irradiate a target depends on the position of the transmitter and the location of the target. In the model, we denote the target position as $\targetxparm$ and the position of each transmitter and receiver as $\anTx{\txn}$, $\txn\in\{0, \txN-1\}$ and $\anRx{\txm}$, $\txm\in\{0, \txM-1\}$, respectively. We also denote the transmitted waveform as 
\begin{equation}
 \pulse(\fasttime)=\envelope(\fasttime)\eex{\im\omega_0\fasttime},
\end{equation}
where $\envelope(\fasttime)$ is some baseband waveform. With this notation, and with transmit and receive delays given by $\txDelay{\txn}$ and $\rxDelay{\txm}$, respectively, the far field delay for a transmit/receive pair is given by
\begin{equation}  \mnDelay(\parm) = \frac{2\amplitude\range(\parm, \targetnoparm)}{c} + \txDelay{\txn} + \rxDelay{\txn} 
+ \frac{\anTx{\txn}\cdot\uRangeVectorParm}{c} + \frac{\anRx{\txm}\cdot\uRangeVectorParm}{c}
\end{equation}
and the return signal is given by 
\begin{equation}
\pulse_{mn}(\fasttime, \parm) = \txAmp{\txn}\rxAmp{\txm}\pulse[\fasttime - \mnDelay(\parm)].
\end{equation}
Through the principle of superposition, the system measured signal is given by
\begin{equation}
\begin{split}
\pulse_{s}(\fasttime, \parm) &= \sum_{\txm,\txn}\pulse_{\txm\txn}(\fasttime, \parm)\\
&=\sum_{\txm,\txn}\txAmp{\txn}\rxAmp{\txm}\pulse[\fasttime - \mnDelay(\parm)].
\end{split}
\end{equation}
In the fast-time frequency domain, this signal can be written as
\begin{equation}
\begin{split}
\Pulse_{s}(\omega', \parm)&=\sum_{\txm,\txn}\txAmp{\txn}\rxAmp{\txm}\Pulse(\omega')\eex{-\im\omega'\mnDelay(\parm)}\\
&=\Pulse(\omega')\eex{-\im\omega'\frac{2\amplitude\range(\parm, \targetnoparm)}{c}}\sum_{\txn}\txAmp{\txn}\eex{-\im\omega'[\txDelay{\txn}+\frac{\anTx{\txn}\cdot\uRangeVectorParm}{c}]}\\
&\cdot\sum_{\txm}\rxAmp{\txm}\eex{-\im\omega'[\rxDelay{\txm}+\frac{\anRx{\txm}\cdot\uRangeVectorParm}{c}]},
\end{split}
\end{equation}
where $\omega'=\omega+\omega_0$.
\par
Suppose that $\anTx{\txn} = \txn\bline$ and that $\anRx{\txm} = \txm\bline$, i.e. a uniformly spaced array, and that the timing delays are chosen such that $\txDelay{\txn} = -\txn\bline\cdot\uzero/c$ and $\rxDelay{\txm} = -\txm\bline\cdot\uzero/c$ for some given look vector $\uzero$. In this case, the signal becomes
 \begin{equation}
 \begin{split}
  \Pulse_{s}(\omega', \parm)&=\Pulse(\omega')\eex{-\im\omega'\frac{2\amplitude\range(\parm, \targetnoparm)}{c}}\sum_{\txn}\txAmp{\txn}\eex{-\im\omega'\frac{\txn}{c}\bline\cdot[-\uzero+\uRangeVectorParm]}\\
  &\cdot\sum_{\txm}\rxAmp{\txm}\eex{-\im\omega'\frac{\txm}{c}\bline\cdot[-\uzero+\uRangeVectorParm]}
 \end{split}
 \end{equation}
Further, suppose that the weights, $\txAmp{\txn}$ and $\rxAmp{\txm}$, are such that $\exists \txn_\channelIndex, \txm_\channelIndex$ with the properties that both
 \begin{equation}
  \sum_{\txn}\txAmp{\txn}\eex{-\im\omega'\frac{\txn-\txn_\channelIndex}{c}\bline\cdot[-\uzero+\uRangeVectorParm]}
 \end{equation}
 and
 \begin{equation}
  \sum_{\txm}\rxAmp{\txm}\eex{-\im\omega'\frac{\txm-\txm_\channelIndex}{c}\bline\cdot[-\uzero+\uRangeVectorParm]}
 \end{equation}
 are real $\forall\omega'$. As a particular example, if $\txAmp{\txn}=A_{T_x}$, a constant, for $\txn\in\{\txn_0, \txn_0+1, \txn_0+2, \ldots, \txn_0+\txN-1\}$, then $\txn_\channelIndex=\txn_0+\txN/2$. With this condition,
 \begin{equation}
 \begin{split}
  \Pulse_{s}(\omega', \parm)&=\Pulse(\omega')\eex{-\im\omega'\frac{2\amplitude\range(\parm, \targetnoparm)}{c}}\eex{-\im\omega'\frac{\txn_\channelIndex}{c}\bline\cdot[-\uzero+\uRangeVectorParm]}\eex{-\im\omega'\frac{\txm_\channelIndex}{c}\bline\cdot[-\uzero+\uRangeVectorParm]}\\&\sum_{\txn}\txAmp{\txn}\eex{-\im\omega'\frac{\txn-\txn_\channelIndex}{c}\bline\cdot[-\uzero+\uRangeVectorParm]}
  \sum_{\txm}\rxAmp{\txm}\eex{-\im\omega'\frac{\txm-\txm_\channelIndex}{c}\bline\cdot[-\uzero+\uRangeVectorParm]}
 \end{split}
 \end{equation}
With $\omega'/c=\kr/2$, let
 \begin{equation}
  \pAntennaParm{\channelIndex} = \frac{\txn_\channelIndex\bline + \txm_\channelIndex\bline}{2},
 \end{equation}
 and define
 \begin{equation}
 \begin{split}
  \dPattern{\channelIndex}[\kr, \uRangeVectorParm] = 
  &\sum_{\txn}\txAmp{\txn}\eex{-\im\kr\frac{\txn-\txn_\channelIndex}{2}\bline\cdot[-\uzero+\uRangeVectorParm]}\\&
  \sum_{\txm}\rxAmp{\txm}\eex{-\im\kr\frac{\txm-\txm_\channelIndex}{2}\bline\cdot[-\uzero+\uRangeVectorParm]}.
  \end{split}
 \end{equation}
 Then,
 \begin{equation}
 \begin{split}
  \Pulse_{s}(\kr, \parm)&=\Pulse(\kr)\eex{-\im\kr\amplitude\range(\parm, \targetnoparm)}\eex{-\im\kr\pAntennaParm{\channelIndex}\cdot\uRangeVectorParm}\eex{\im\kr\pAntennaParm{\channelIndex}\cdot\uzero}\dPattern{\channelIndex}[\kr, \uRangeVectorParm]
 \end{split}
 \end{equation}
Finally, compute $\eex{-\im\kr\pAntennaParm{\channelIndex}\cdot\uzero}\Pulse_{s}(\kr, \parm)$ to obtain the expression in \eqref{eq:Skst1}.
