\section{Signal processing algorithm}
This section presents the promised third component of the improved SAR imaging capability; namely, the fundamentals of a suitable signal processing method. As we have already seen, the desired configuration impacts the design. The signal processing method presented in this section, on the other hand, is one of potentially several others, and its adoption does not place any restrictions upon the configuration or the design.  
\par
The previous material depends on a strict timing regime. A practical system may only operate with near-ideal timing conditions and this raises the questions of impact and how best to process the data under these conditions. 
\par
Even if timing conditions are perfect, the best approach to data processing is not entirely clear. A simple approach first concatenates the measurements from each beam into a uniformly-sampled time series and then transforms the data from the fast-time, slow-time domain into the fast-time Doppler domain. Given that, for each beam direction, the data correspond to different Doppler centroids, one could assign each response to different portions of the Doppler spectrum (or to different Doppler frequency bands). The union of these Doppler frequency bands corresponds to a wider Doppler spectrum and thereby to higher overall azimuth resolution. If the Doppler bands are non-overlapping, the concept of the union of the frequency bands is straight-forward. Optimal processing of Doppler bands that do overlap, however, requires a more rigorous approach. Optimal processing of data collected under non-ideal timing further calls for a flexible yet robust processing approach.
\par
Given that processing should apply to very high-resolution systems, it is best that the approach be suitable for a wide-band system. 
\par
The derivation of an optimal processing approach would require more space than available for a letter publication. This section, therefore, only presents the final algorithm along with references to the derivation.
\subsection{The measured signal}
In continuous time, the signal from antenna $m$, beam $n$ at fast time $\fasttime$ and slow time $\parm$ is represented as
\begin{equation}
 \ztzt{m,n}(\fasttime, \parm)=\stst{m,n}(\fasttime, \parm) + \ntnt{m,n}(\fasttime, \parm)
\end{equation}
where $\ntnt{m,n}(\fasttime, \parm)$ represents white noise and $\stst{m,n}(\fasttime, \parm)$ the desired clutter signal. We introduce the vector signal
\begin{equation}
 \vecsigTime(\fasttime, \parm) = 
 \begin{bmatrix}
 \begin{bmatrix}
 \stst{0,0}(\fasttime, \parm)\\
 \stst{1,0}(\fasttime, \parm)\\
 \vdots\\
 \stst{M,0}(\fasttime, \parm)
 \end{bmatrix}\\
 \begin{bmatrix}
 \stst{0,1}(\fasttime, \parm)\\
 \stst{1,1}(\fasttime, \parm)\\
 \vdots\\
 \stst{M,1}(\fasttime, \parm)
 \end{bmatrix}\\
 \vdots\\
 \begin{bmatrix}
 \stst{0,N}(\fasttime, \parm)\\
 \stst{1,N}(\fasttime, \parm)\\
 \vdots\\
 \stst{M, N}(\fasttime, \parm)
 \end{bmatrix}
 \end{bmatrix}
 +
 \boldsymbol{\nu}(\fasttime, \parm)
\end{equation}
where $\boldsymbol{\nu}(\fasttime, \parm)$ is a vector of white noise values.
\par
Due to the relationship between Doppler frequency and azimuth angle, the measured vector signal is, perhaps, most readily processed in the frequency domain. As a simplification, this letter assumes that the two-way phase-centre positions of each channel are evenly aligned along the direction of motion so that each two-way phase centre is located at along-track position $m\phaseSep\uSatVelocity{\parm}$, where $\uSatVelocity{\parm}$ denotes the unit vector in the direction of the satellite velocity vector over slow-time. Without going into the derivation, we present the unambiguous frequency domain representation of the signal as
\begin{equation}
 \mathcal{F}\{\stst{m,n}(\fasttime, \parm)\} = \SkSk{m,n}(\kr, \kparm)
\end{equation}
where,
\begin{equation}
 \SkSk{m,n}(\kveczero) = \eex{-\im [m\phaseSep\kappa+n\effrelvel/\prf]\kparm}\GkGk{n}(\kveczero).
 \label{eq:signalModel}
\end{equation}
The phase term at the front corresponds to spatial sampling delays from one channel to the next and temporal sampling delays from one beam to the next. The signal from each beam contains many terms that are conveniently separated into terms that correspond to the antenna pattern (or beam) and terms that correspond to the imaging geometry.
\begin{equation}
\begin{split}
 &\GkGk{n}(\kveczero) = \int\dPattern{n}[\kr,\uRangeVector{\kveczero, \targetdepression}]\vkern(\kr,\kparm,\depression)\d\depression,
 \end{split}
 \label{eq:stripSAR3d}
\end{equation}
where the amplitude of the antenna pattern is given by $\dPattern{n}[\kr,\uRangeVector{\kveczero, \targetdepression}]$ which depends on the direction of the beam through $n$, does {\bf not} depend on the two-way phase-centre location (no $m$ dependence) and also depends on the wavelength through $\kr$. The term that corresponds to the geometry and the reflectivity is given by
\begin{equation}
\begin{split}
 &\vkern(\kveczero, \depression) = C_0\Snvelope(\krc)\frac{(\kr^2-\kparm^2)^{\frac{1}{4}}}{v_\sat\kr}\\
 &\int\frac{\reflectivity[\targetrange, \targetxparm, \depression]}{\sqrt{\targetrange}}\eex{-\im \targetxparm\kparm-\im \targetrange\sqrt{\kr^2-\kparm^2}}\d\targetrange\d\targetxparm
 \end{split},
 \label{eq:vparm}
\end{equation} 
where, $\kparm=2\pi\fparm/\effrelvel$, $\effrelvel$ represents the effective satellite velocity (relative) and $\fparm$ represents the slow-time Doppler frequency. Also $\kappa = \effrelvel/\amplitude\velocity_s$, where $\amplitude\velocity_s$ is the amplitude of the satellite velocity vector, $\satVelocity{\parm}$, and is assumed constant over $\parm$.  The function, $\reflectivity[\targetrange, \targetxparm, \depression]$, represents the terrain reflectivity as a function of range, $\targetrange$ from the radar, along-track position, $\targetxparm$ and depression angle, $\targetdepression$. Note that the hyperbolic phase term, $\sqrt{\kr^2-\kparm^2}}$, may be insufficiently accurate for extremely high azimuth resolution \cite{Mittermayer}. In this case, a more accurate expression can be computed from the material in \cite{NovelRadar}. The look-direction vector in the 2D frequency domain is given by
\begin{equation}
 \uRangeVector{\kveczero, \targetdepression} = \frac{1}{\kr}
 \begin{bmatrix}
  -\kappa\kparm\\
  \cos\targetdepression\sqrt{\kr^2 - \kparm^2}\\
  \sqrt{\sin^2\targetdepression[\kr^2-\kparm^2] + \kparm^2(1-\kappa^2)}
 \end{bmatrix}.
\end{equation}
Finally, $\Snvelope(\krc)$ denotes the baseband frequency ($\krc$) response of the pulsed waveform after processing for pulse compression, and $C_0$ is some constant. Derivation of the above material can be found in \cite{NovelRadar}.
\subsection{Multi-channel processing}
Let
\begin{equation}
 \vct{h}(\kr, \kparm, \depression) = 
 \begin{bmatrix}
 \eex{-\im 0\temporalbaseline\kparm}\dPattern{0}[\kr,\uRangeVector{\kveczero, \targetdepression}]
 \begin{bmatrix}
 \eex{-\im 0\spatialbaseline\kparm}\\
 \eex{-\im 1\spatialbaseline\kparm}\\
 \vdots\\
 \eex{-\im M\spatialbaseline\kparm}
 \end{bmatrix}\\
 \eex{-\im 1\temporalbaseline\kparm}\dPattern{1}[\kr,\uRangeVector{\kveczero, \targetdepression}]
 \begin{bmatrix}
 \eex{-\im 0\spatialbaseline\kparm}\\
 \eex{-\im 1\spatialbaseline\kparm}\\
 \vdots\\
 \eex{-\im M\spatialbaseline\kparm}
 \end{bmatrix}\\
 \vdots\\
 \eex{-\im N\temporalbaseline\kparm}\dPattern{N}[\kr,\uRangeVector{\kveczero, \targetdepression}]
 \begin{bmatrix}
 \eex{-\im 0\spatialbaseline\kparm}\\
 \eex{-\im 1\spatialbaseline\kparm}\\
 \vdots\\
 \eex{-\im M\spatialbaseline\kparm}
 \end{bmatrix}
 \end{bmatrix},
\end{equation}
where, $\temporalbaseline=\effrelvel/\prf$ and $\spatialbaseline=\phaseSep\kappa$.
\par
The sampled signal is gven by
\begin{equation}
\begin{split}
\vecsigFreqSampled(\kr, \kparm) &= \sum_l\int\vct{h}(\kr, \kparm+l\kparmPRF)\vkern(\kr,\kparm+l\kparmPRF,\depression)\d\depression\\ 
&+ \boldsymbol{\nu}(\kr, \kparm)\\
&=\int\antennaMatrix(\kargs)\vkernvect(\kr, \kparm, \depression)\d\depression+ \boldsymbol{\nu}(\kr, \kparm),
\end{split}
\end{equation}
where
\begin{equation}
	\antennaMatrix(\kargs)=
    \begin{bmatrix}
    	\hdots & \vct{h}(\kr, \kparm - \kparmPRF, \depression) & \vct{h}(\kr, \kparm, \depression) & \hdots
    \end{bmatrix},
\end{equation}
and
\begin{equation}
	\vkernvect(\kr, \kparm, \depression) = 
    \begin{bmatrix}
    	\vdots\\
        \vkern(\kr,\kparm-\kparmPRF,\depression)\\
        \vkern(\kr,\kparm,\depression)\\
        \vdots
    \end{bmatrix},
\end{equation}
and, where the spatial sampling frequency is given by
\begin{equation}
 \kparmPRF = \frac{2\pi\prf}{(M+1)(N+1)\effrelvel}.
\end{equation}
Each component of $\vecsigFreqSampled(\kr, \kparm)$ is undersampled in $\kparm$; however, by applying linear filters an unambiguous scalar spectrum of the signal may be reconstructed. Let us consider a single value of $\depression$ and apply a filtering matrix via $\hrwsFilterMatrix(\kargs)\antennaMatrix(\kargs)=\diagAntennaMatrix(\kargs)$. If the result, $\diagAntennaMatrix(\kargs)$, is diagonal, then we have constructed a mechanism to extract unambiguous signal components - we need only assign the elements of the vector
\begin{equation}
	\hrwsFilterMatrix(\kargs)\antennaMatrix(\kargs)\vkernvect(\kr, \kparm, \depression) + \hrwsFilterMatrix(\kargs)\boldsymbol{\nu}(\kr, \kparm)
\end{equation}
to each band. As seen in the above, however, the noise component has also been filtered by this matrix and can possibly lead to degraded SNR. In addition to desiring a diagonal matrix (or as close as possible) for $\diagAntennaMatrix(\kargs)$, it is also fitting to select this matrix such that it has diagonal elements given by the average of the antenna patterns, i.e. $D_{ll}(\kparm) = \dPatternDesired[\kr,\uRangeVector{\kr, \kparm-l\kparmPRF, \targetdepression}]$
%\par
%Filters can be chosen to best approximate the following desired signal
%\begin{equation}
%\begin{split}
% &\GkGkD(\kveczero) = C_0\Snvelope(\krc)\frac{(\kr^2-\kparm^2)^{\frac{1}{4}}}{v_\sat\kr}\\
% &\int\dPatternDesired[\kr,\uRangeVector{\kveczero, \targetdepression}]\frac{\reflectivity[\targetrange, \targetxparm, \depression]}{\sqrt{\targetrange}}\eex{-\im \targetxparm\kparm-\im \targetrange\sqrt{\kr^2-\kparm^2}}\d\targetrange\d\targetxparm\d\depression
% \end{split}
% \label{eq:stripSAR3dDesired}
%\end{equation}
where
\begin{equation}
 \dPatternDesired[\kr,\uRangeVector{\kveczero, \targetdepression}]=\sqrt{\frac{1}{N+1}\sum_{n=0}^{N}\biggl\lvert\dPattern{n}[\kr,\uRangeVector{\kveczero, \targetdepression}]\biggr\rvert^2}.
\end{equation}
This desired antenna pattern has support over a wide range of azimuth angles and hence leads to a high-resolution signal.
\par
As can be shown, the optimal filter matrix, in a mean-squared error sense, is given by
\begin{equation}
\begin{split}
 \hrwsFilterMatrix(&\kargs)=\diagAntennaMatrix(\kargs)\htr{\antennaMatrix}(\kargs)\\
 &\biggl[\antennaMatrix(\kargs)\htr{\antennaMatrix}(\kargs)+\frac{1-\cost}{\cost}\mtx{R}_n(\kargs)\biggr]^{-1},
\end{split}
\end{equation}
where $\mtx{R}_n(\kargs)$ is the noise covariance matrix.
\subsection{Simulated Point Spread Function}
To demonstrate the signal processing approach, this section generates a simulated signal using the design parameters of an 11-channel system. The simulation, with parameters listed in \Tbref{tb:simulation}, has been computed in Python with plots generated with Matplotlib.
\begin{table}[h!]
\begin{center}
 \caption{Simulation parameters}
 \label{tb:simulation}
 \begin{tabular}{r|l}
  $\prf$ & $6810$ Hz\\\hline
  $\antennaLengthEffective$ & $2.2$ m\\\hline
  $\antennaLength$ & $22$ m\\\hline
  $M$ & $11$\\\hline
  $\wavelength$ & $0.031067$ m\\\hline
  $\Delta\threeDBEffective$ & $0.809$ deg
 \end{tabular}
 \end{center}
\end{table}
The simulator computes the back-folded signal for each of the 121 channels of data, computes the processing filters, applies the filters to the back-folded data and presents the amplitude of the reconstructed signal in the Doppler domain in \fgref{fg:reconstructed}. The response in the Doppler domain shows the desired response across a region spanning 75 KHz with each local maximum in this region highlighting the response of each sub-beam.
\begin{figure}[h!]
\begin{center}
 \resizebox{\columnwidth}{!}{%% Creator: Matplotlib, PGF backend
%%
%% To include the figure in your LaTeX document, write
%%   \input{<filename>.pgf}
%%
%% Make sure the required packages are loaded in your preamble
%%   \usepackage{pgf}
%%
%% Figures using additional raster images can only be included by \input if
%% they are in the same directory as the main LaTeX file. For loading figures
%% from other directories you can use the `import` package
%%   \usepackage{import}
%% and then include the figures with
%%   \import{<path to file>}{<filename>.pgf}
%%
%% Matplotlib used the following preamble
%%   \usepackage{fontspec}
%%   \setmainfont{Bitstream Vera Serif}
%%   \setsansfont{Bitstream Vera Sans}
%%   \setmonofont{Bitstream Vera Sans Mono}
%%
\begingroup%
\makeatletter%
\begin{pgfpicture}%
\pgfpathrectangle{\pgfpointorigin}{\pgfqpoint{8.000000in}{6.000000in}}%
\pgfusepath{use as bounding box, clip}%
\begin{pgfscope}%
\pgfsetbuttcap%
\pgfsetmiterjoin%
\definecolor{currentfill}{rgb}{1.000000,1.000000,1.000000}%
\pgfsetfillcolor{currentfill}%
\pgfsetlinewidth{0.000000pt}%
\definecolor{currentstroke}{rgb}{1.000000,1.000000,1.000000}%
\pgfsetstrokecolor{currentstroke}%
\pgfsetdash{}{0pt}%
\pgfpathmoveto{\pgfqpoint{0.000000in}{0.000000in}}%
\pgfpathlineto{\pgfqpoint{8.000000in}{0.000000in}}%
\pgfpathlineto{\pgfqpoint{8.000000in}{6.000000in}}%
\pgfpathlineto{\pgfqpoint{0.000000in}{6.000000in}}%
\pgfpathclose%
\pgfusepath{fill}%
\end{pgfscope}%
\begin{pgfscope}%
\pgfsetbuttcap%
\pgfsetmiterjoin%
\definecolor{currentfill}{rgb}{1.000000,1.000000,1.000000}%
\pgfsetfillcolor{currentfill}%
\pgfsetlinewidth{0.000000pt}%
\definecolor{currentstroke}{rgb}{0.000000,0.000000,0.000000}%
\pgfsetstrokecolor{currentstroke}%
\pgfsetstrokeopacity{0.000000}%
\pgfsetdash{}{0pt}%
\pgfpathmoveto{\pgfqpoint{1.000000in}{0.600000in}}%
\pgfpathlineto{\pgfqpoint{7.200000in}{0.600000in}}%
\pgfpathlineto{\pgfqpoint{7.200000in}{5.400000in}}%
\pgfpathlineto{\pgfqpoint{1.000000in}{5.400000in}}%
\pgfpathclose%
\pgfusepath{fill}%
\end{pgfscope}%
\begin{pgfscope}%
\pgfpathrectangle{\pgfqpoint{1.000000in}{0.600000in}}{\pgfqpoint{6.200000in}{4.800000in}} %
\pgfusepath{clip}%
\pgfsetrectcap%
\pgfsetroundjoin%
\pgfsetlinewidth{1.003750pt}%
\definecolor{currentstroke}{rgb}{0.000000,0.000000,1.000000}%
\pgfsetstrokecolor{currentstroke}%
\pgfsetdash{}{0pt}%
\pgfpathmoveto{\pgfqpoint{1.082503in}{0.590000in}}%
\pgfpathlineto{\pgfqpoint{1.082600in}{0.616502in}}%
\pgfpathlineto{\pgfqpoint{1.082600in}{0.616502in}}%
\pgfpathlineto{\pgfqpoint{1.082600in}{0.616502in}}%
\pgfpathlineto{\pgfqpoint{1.082621in}{0.590000in}}%
\pgfpathmoveto{\pgfqpoint{1.083606in}{0.590000in}}%
\pgfpathlineto{\pgfqpoint{1.083673in}{0.681866in}}%
\pgfpathlineto{\pgfqpoint{1.083673in}{0.681866in}}%
\pgfpathlineto{\pgfqpoint{1.083673in}{0.681866in}}%
\pgfpathlineto{\pgfqpoint{1.084315in}{0.590000in}}%
\pgfpathmoveto{\pgfqpoint{1.085674in}{0.590000in}}%
\pgfpathlineto{\pgfqpoint{1.085819in}{0.741107in}}%
\pgfpathlineto{\pgfqpoint{1.086355in}{0.825737in}}%
\pgfpathlineto{\pgfqpoint{1.086355in}{0.825737in}}%
\pgfpathlineto{\pgfqpoint{1.086557in}{0.590000in}}%
\pgfpathmoveto{\pgfqpoint{1.087882in}{0.590000in}}%
\pgfpathlineto{\pgfqpoint{1.087965in}{0.676478in}}%
\pgfpathlineto{\pgfqpoint{1.087965in}{0.676478in}}%
\pgfpathlineto{\pgfqpoint{1.087965in}{0.676478in}}%
\pgfpathlineto{\pgfqpoint{1.088106in}{0.590000in}}%
\pgfpathmoveto{\pgfqpoint{1.089195in}{0.590000in}}%
\pgfpathlineto{\pgfqpoint{1.089574in}{0.743430in}}%
\pgfpathlineto{\pgfqpoint{1.089574in}{0.743430in}}%
\pgfpathlineto{\pgfqpoint{1.089574in}{0.743430in}}%
\pgfpathlineto{\pgfqpoint{1.089649in}{0.590000in}}%
\pgfpathmoveto{\pgfqpoint{1.091088in}{0.590000in}}%
\pgfpathlineto{\pgfqpoint{1.091720in}{0.753489in}}%
\pgfpathlineto{\pgfqpoint{1.092256in}{0.679314in}}%
\pgfpathlineto{\pgfqpoint{1.092366in}{0.590000in}}%
\pgfpathmoveto{\pgfqpoint{1.093176in}{0.590000in}}%
\pgfpathlineto{\pgfqpoint{1.093865in}{0.963146in}}%
\pgfpathlineto{\pgfqpoint{1.094390in}{0.590000in}}%
\pgfpathmoveto{\pgfqpoint{1.094487in}{0.590000in}}%
\pgfpathlineto{\pgfqpoint{1.094938in}{0.636646in}}%
\pgfpathlineto{\pgfqpoint{1.095475in}{0.843687in}}%
\pgfpathlineto{\pgfqpoint{1.096011in}{0.775620in}}%
\pgfpathlineto{\pgfqpoint{1.096548in}{0.774302in}}%
\pgfpathlineto{\pgfqpoint{1.097084in}{0.802907in}}%
\pgfpathlineto{\pgfqpoint{1.097290in}{0.590000in}}%
\pgfpathmoveto{\pgfqpoint{1.098378in}{0.590000in}}%
\pgfpathlineto{\pgfqpoint{1.098693in}{1.018291in}}%
\pgfpathlineto{\pgfqpoint{1.098693in}{1.018291in}}%
\pgfpathlineto{\pgfqpoint{1.098693in}{1.018291in}}%
\pgfpathlineto{\pgfqpoint{1.099346in}{0.590000in}}%
\pgfpathmoveto{\pgfqpoint{1.100211in}{0.590000in}}%
\pgfpathlineto{\pgfqpoint{1.100303in}{0.789294in}}%
\pgfpathlineto{\pgfqpoint{1.100839in}{0.976510in}}%
\pgfpathlineto{\pgfqpoint{1.101376in}{0.636631in}}%
\pgfpathlineto{\pgfqpoint{1.101912in}{0.857462in}}%
\pgfpathlineto{\pgfqpoint{1.102448in}{0.925805in}}%
\pgfpathlineto{\pgfqpoint{1.103144in}{0.590000in}}%
\pgfpathmoveto{\pgfqpoint{1.103696in}{0.590000in}}%
\pgfpathlineto{\pgfqpoint{1.104058in}{1.041736in}}%
\pgfpathlineto{\pgfqpoint{1.104594in}{1.107337in}}%
\pgfpathlineto{\pgfqpoint{1.105131in}{0.597311in}}%
\pgfpathlineto{\pgfqpoint{1.105667in}{0.694466in}}%
\pgfpathlineto{\pgfqpoint{1.106740in}{1.085540in}}%
\pgfpathlineto{\pgfqpoint{1.107276in}{0.949646in}}%
\pgfpathlineto{\pgfqpoint{1.107813in}{0.734419in}}%
\pgfpathlineto{\pgfqpoint{1.108349in}{0.762395in}}%
\pgfpathlineto{\pgfqpoint{1.109422in}{1.209395in}}%
\pgfpathlineto{\pgfqpoint{1.109959in}{1.162525in}}%
\pgfpathlineto{\pgfqpoint{1.111031in}{0.810068in}}%
\pgfpathlineto{\pgfqpoint{1.111568in}{1.321599in}}%
\pgfpathlineto{\pgfqpoint{1.112104in}{1.062165in}}%
\pgfpathlineto{\pgfqpoint{1.112343in}{0.590000in}}%
\pgfpathmoveto{\pgfqpoint{1.112984in}{0.590000in}}%
\pgfpathlineto{\pgfqpoint{1.113177in}{0.919885in}}%
\pgfpathlineto{\pgfqpoint{1.113714in}{1.248480in}}%
\pgfpathlineto{\pgfqpoint{1.114786in}{0.667826in}}%
\pgfpathlineto{\pgfqpoint{1.115323in}{1.085074in}}%
\pgfpathlineto{\pgfqpoint{1.115859in}{0.831266in}}%
\pgfpathlineto{\pgfqpoint{1.116396in}{0.868076in}}%
\pgfpathlineto{\pgfqpoint{1.116932in}{1.295660in}}%
\pgfpathlineto{\pgfqpoint{1.117469in}{1.266685in}}%
\pgfpathlineto{\pgfqpoint{1.117890in}{0.590000in}}%
\pgfpathmoveto{\pgfqpoint{1.118129in}{0.590000in}}%
\pgfpathlineto{\pgfqpoint{1.118542in}{1.210847in}}%
\pgfpathlineto{\pgfqpoint{1.119614in}{1.315374in}}%
\pgfpathlineto{\pgfqpoint{1.120151in}{0.836388in}}%
\pgfpathlineto{\pgfqpoint{1.120687in}{1.212610in}}%
\pgfpathlineto{\pgfqpoint{1.121224in}{1.031850in}}%
\pgfpathlineto{\pgfqpoint{1.121760in}{1.104622in}}%
\pgfpathlineto{\pgfqpoint{1.122297in}{1.434248in}}%
\pgfpathlineto{\pgfqpoint{1.123369in}{0.772863in}}%
\pgfpathlineto{\pgfqpoint{1.124442in}{1.318395in}}%
\pgfpathlineto{\pgfqpoint{1.125515in}{0.813541in}}%
\pgfpathlineto{\pgfqpoint{1.126052in}{1.022494in}}%
\pgfpathlineto{\pgfqpoint{1.126588in}{1.352139in}}%
\pgfpathlineto{\pgfqpoint{1.127124in}{0.794049in}}%
\pgfpathlineto{\pgfqpoint{1.127661in}{1.258884in}}%
\pgfpathlineto{\pgfqpoint{1.128197in}{1.488142in}}%
\pgfpathlineto{\pgfqpoint{1.129164in}{0.590000in}}%
\pgfpathmoveto{\pgfqpoint{1.129325in}{0.590000in}}%
\pgfpathlineto{\pgfqpoint{1.129807in}{1.500767in}}%
\pgfpathlineto{\pgfqpoint{1.129807in}{1.500767in}}%
\pgfpathlineto{\pgfqpoint{1.129807in}{1.500767in}}%
\pgfpathlineto{\pgfqpoint{1.130343in}{1.455190in}}%
\pgfpathlineto{\pgfqpoint{1.130865in}{0.590000in}}%
\pgfpathmoveto{\pgfqpoint{1.130895in}{0.590000in}}%
\pgfpathlineto{\pgfqpoint{1.131416in}{1.357513in}}%
\pgfpathlineto{\pgfqpoint{1.131416in}{1.357513in}}%
\pgfpathlineto{\pgfqpoint{1.131416in}{1.357513in}}%
\pgfpathlineto{\pgfqpoint{1.131952in}{1.257438in}}%
\pgfpathlineto{\pgfqpoint{1.132489in}{1.443143in}}%
\pgfpathlineto{\pgfqpoint{1.133025in}{1.233645in}}%
\pgfpathlineto{\pgfqpoint{1.133562in}{1.509351in}}%
\pgfpathlineto{\pgfqpoint{1.134098in}{1.328234in}}%
\pgfpathlineto{\pgfqpoint{1.135171in}{1.351153in}}%
\pgfpathlineto{\pgfqpoint{1.136244in}{0.764565in}}%
\pgfpathlineto{\pgfqpoint{1.137317in}{1.592515in}}%
\pgfpathlineto{\pgfqpoint{1.138390in}{0.928000in}}%
\pgfpathlineto{\pgfqpoint{1.138926in}{1.595033in}}%
\pgfpathlineto{\pgfqpoint{1.139463in}{1.520950in}}%
\pgfpathlineto{\pgfqpoint{1.140535in}{1.316850in}}%
\pgfpathlineto{\pgfqpoint{1.141072in}{1.602376in}}%
\pgfpathlineto{\pgfqpoint{1.141608in}{1.256381in}}%
\pgfpathlineto{\pgfqpoint{1.142145in}{1.511466in}}%
\pgfpathlineto{\pgfqpoint{1.142681in}{1.714272in}}%
\pgfpathlineto{\pgfqpoint{1.143218in}{1.622372in}}%
\pgfpathlineto{\pgfqpoint{1.143754in}{1.076380in}}%
\pgfpathlineto{\pgfqpoint{1.144290in}{1.471911in}}%
\pgfpathlineto{\pgfqpoint{1.144827in}{1.551989in}}%
\pgfpathlineto{\pgfqpoint{1.145363in}{1.316879in}}%
\pgfpathlineto{\pgfqpoint{1.145900in}{1.415540in}}%
\pgfpathlineto{\pgfqpoint{1.146436in}{1.347755in}}%
\pgfpathlineto{\pgfqpoint{1.148046in}{1.617826in}}%
\pgfpathlineto{\pgfqpoint{1.148582in}{1.012671in}}%
\pgfpathlineto{\pgfqpoint{1.149118in}{1.337126in}}%
\pgfpathlineto{\pgfqpoint{1.150728in}{1.740267in}}%
\pgfpathlineto{\pgfqpoint{1.151264in}{1.325159in}}%
\pgfpathlineto{\pgfqpoint{1.151801in}{1.930877in}}%
\pgfpathlineto{\pgfqpoint{1.152337in}{1.549796in}}%
\pgfpathlineto{\pgfqpoint{1.152873in}{0.752817in}}%
\pgfpathlineto{\pgfqpoint{1.153946in}{1.840268in}}%
\pgfpathlineto{\pgfqpoint{1.154333in}{0.590000in}}%
\pgfpathmoveto{\pgfqpoint{1.154660in}{0.590000in}}%
\pgfpathlineto{\pgfqpoint{1.155019in}{1.576008in}}%
\pgfpathlineto{\pgfqpoint{1.156092in}{1.766477in}}%
\pgfpathlineto{\pgfqpoint{1.156628in}{1.253791in}}%
\pgfpathlineto{\pgfqpoint{1.157165in}{1.294685in}}%
\pgfpathlineto{\pgfqpoint{1.157701in}{1.323071in}}%
\pgfpathlineto{\pgfqpoint{1.158238in}{1.655667in}}%
\pgfpathlineto{\pgfqpoint{1.158774in}{1.440188in}}%
\pgfpathlineto{\pgfqpoint{1.159311in}{1.451322in}}%
\pgfpathlineto{\pgfqpoint{1.160920in}{1.918467in}}%
\pgfpathlineto{\pgfqpoint{1.161993in}{1.140114in}}%
\pgfpathlineto{\pgfqpoint{1.162529in}{2.005615in}}%
\pgfpathlineto{\pgfqpoint{1.163066in}{1.699251in}}%
\pgfpathlineto{\pgfqpoint{1.163602in}{1.402866in}}%
\pgfpathlineto{\pgfqpoint{1.164139in}{1.588167in}}%
\pgfpathlineto{\pgfqpoint{1.164675in}{1.995781in}}%
\pgfpathlineto{\pgfqpoint{1.165211in}{1.762249in}}%
\pgfpathlineto{\pgfqpoint{1.165694in}{0.590000in}}%
\pgfpathmoveto{\pgfqpoint{1.165796in}{0.590000in}}%
\pgfpathlineto{\pgfqpoint{1.166284in}{1.898104in}}%
\pgfpathlineto{\pgfqpoint{1.166284in}{1.898104in}}%
\pgfpathlineto{\pgfqpoint{1.166284in}{1.898104in}}%
\pgfpathlineto{\pgfqpoint{1.166821in}{1.878363in}}%
\pgfpathlineto{\pgfqpoint{1.167894in}{1.424824in}}%
\pgfpathlineto{\pgfqpoint{1.168967in}{1.871962in}}%
\pgfpathlineto{\pgfqpoint{1.170576in}{1.304160in}}%
\pgfpathlineto{\pgfqpoint{1.171112in}{1.337445in}}%
\pgfpathlineto{\pgfqpoint{1.171649in}{1.851607in}}%
\pgfpathlineto{\pgfqpoint{1.172185in}{1.633571in}}%
\pgfpathlineto{\pgfqpoint{1.173794in}{2.119213in}}%
\pgfpathlineto{\pgfqpoint{1.174331in}{1.556011in}}%
\pgfpathlineto{\pgfqpoint{1.174867in}{1.705272in}}%
\pgfpathlineto{\pgfqpoint{1.175404in}{2.078884in}}%
\pgfpathlineto{\pgfqpoint{1.175940in}{1.846883in}}%
\pgfpathlineto{\pgfqpoint{1.176477in}{1.569159in}}%
\pgfpathlineto{\pgfqpoint{1.177013in}{1.759246in}}%
\pgfpathlineto{\pgfqpoint{1.178086in}{1.917320in}}%
\pgfpathlineto{\pgfqpoint{1.178415in}{0.590000in}}%
\pgfpathmoveto{\pgfqpoint{1.178839in}{0.590000in}}%
\pgfpathlineto{\pgfqpoint{1.179159in}{1.825733in}}%
\pgfpathlineto{\pgfqpoint{1.179159in}{1.825733in}}%
\pgfpathlineto{\pgfqpoint{1.179159in}{1.825733in}}%
\pgfpathlineto{\pgfqpoint{1.180232in}{1.541080in}}%
\pgfpathlineto{\pgfqpoint{1.180768in}{1.572942in}}%
\pgfpathlineto{\pgfqpoint{1.181305in}{1.568191in}}%
\pgfpathlineto{\pgfqpoint{1.181841in}{1.938430in}}%
\pgfpathlineto{\pgfqpoint{1.182377in}{1.873978in}}%
\pgfpathlineto{\pgfqpoint{1.183450in}{1.820742in}}%
\pgfpathlineto{\pgfqpoint{1.183987in}{1.828607in}}%
\pgfpathlineto{\pgfqpoint{1.184523in}{2.123309in}}%
\pgfpathlineto{\pgfqpoint{1.185596in}{1.706956in}}%
\pgfpathlineto{\pgfqpoint{1.186669in}{2.169482in}}%
\pgfpathlineto{\pgfqpoint{1.187205in}{1.466327in}}%
\pgfpathlineto{\pgfqpoint{1.187742in}{1.767256in}}%
\pgfpathlineto{\pgfqpoint{1.188278in}{2.002417in}}%
\pgfpathlineto{\pgfqpoint{1.189888in}{1.074426in}}%
\pgfpathlineto{\pgfqpoint{1.190960in}{2.111021in}}%
\pgfpathlineto{\pgfqpoint{1.191497in}{1.561180in}}%
\pgfpathlineto{\pgfqpoint{1.192033in}{1.749319in}}%
\pgfpathlineto{\pgfqpoint{1.192570in}{2.013419in}}%
\pgfpathlineto{\pgfqpoint{1.194179in}{1.471311in}}%
\pgfpathlineto{\pgfqpoint{1.195252in}{2.257028in}}%
\pgfpathlineto{\pgfqpoint{1.195788in}{1.991167in}}%
\pgfpathlineto{\pgfqpoint{1.196325in}{1.210116in}}%
\pgfpathlineto{\pgfqpoint{1.197398in}{2.201547in}}%
\pgfpathlineto{\pgfqpoint{1.198471in}{1.563021in}}%
\pgfpathlineto{\pgfqpoint{1.199543in}{2.078270in}}%
\pgfpathlineto{\pgfqpoint{1.200616in}{0.905391in}}%
\pgfpathlineto{\pgfqpoint{1.201689in}{2.066918in}}%
\pgfpathlineto{\pgfqpoint{1.202226in}{1.683459in}}%
\pgfpathlineto{\pgfqpoint{1.202735in}{0.590000in}}%
\pgfpathmoveto{\pgfqpoint{1.202789in}{0.590000in}}%
\pgfpathlineto{\pgfqpoint{1.203298in}{1.664091in}}%
\pgfpathlineto{\pgfqpoint{1.204371in}{2.013343in}}%
\pgfpathlineto{\pgfqpoint{1.204908in}{1.864326in}}%
\pgfpathlineto{\pgfqpoint{1.205444in}{1.946953in}}%
\pgfpathlineto{\pgfqpoint{1.206517in}{2.320422in}}%
\pgfpathlineto{\pgfqpoint{1.207053in}{1.540497in}}%
\pgfpathlineto{\pgfqpoint{1.207590in}{2.042391in}}%
\pgfpathlineto{\pgfqpoint{1.208126in}{2.171039in}}%
\pgfpathlineto{\pgfqpoint{1.209199in}{1.452883in}}%
\pgfpathlineto{\pgfqpoint{1.209736in}{1.888772in}}%
\pgfpathlineto{\pgfqpoint{1.210272in}{1.867504in}}%
\pgfpathlineto{\pgfqpoint{1.210809in}{2.025848in}}%
\pgfpathlineto{\pgfqpoint{1.211881in}{1.077940in}}%
\pgfpathlineto{\pgfqpoint{1.212954in}{2.119975in}}%
\pgfpathlineto{\pgfqpoint{1.214027in}{1.654725in}}%
\pgfpathlineto{\pgfqpoint{1.215100in}{2.116989in}}%
\pgfpathlineto{\pgfqpoint{1.215636in}{2.050240in}}%
\pgfpathlineto{\pgfqpoint{1.216173in}{1.387349in}}%
\pgfpathlineto{\pgfqpoint{1.217246in}{2.219386in}}%
\pgfpathlineto{\pgfqpoint{1.217782in}{2.098374in}}%
\pgfpathlineto{\pgfqpoint{1.218319in}{1.390362in}}%
\pgfpathlineto{\pgfqpoint{1.218319in}{1.390362in}}%
\pgfpathlineto{\pgfqpoint{1.218319in}{1.390362in}}%
\pgfpathlineto{\pgfqpoint{1.219392in}{2.173488in}}%
\pgfpathlineto{\pgfqpoint{1.219928in}{1.748847in}}%
\pgfpathlineto{\pgfqpoint{1.221001in}{1.758240in}}%
\pgfpathlineto{\pgfqpoint{1.221537in}{2.172033in}}%
\pgfpathlineto{\pgfqpoint{1.222074in}{1.929699in}}%
\pgfpathlineto{\pgfqpoint{1.222610in}{1.251336in}}%
\pgfpathlineto{\pgfqpoint{1.223147in}{1.601191in}}%
\pgfpathlineto{\pgfqpoint{1.223683in}{2.077894in}}%
\pgfpathlineto{\pgfqpoint{1.224219in}{1.978737in}}%
\pgfpathlineto{\pgfqpoint{1.225291in}{0.590000in}}%
\pgfpathmoveto{\pgfqpoint{1.225293in}{0.590000in}}%
\pgfpathlineto{\pgfqpoint{1.225829in}{2.315699in}}%
\pgfpathlineto{\pgfqpoint{1.225829in}{2.315699in}}%
\pgfpathlineto{\pgfqpoint{1.225829in}{2.315699in}}%
\pgfpathlineto{\pgfqpoint{1.226902in}{1.931850in}}%
\pgfpathlineto{\pgfqpoint{1.227974in}{2.194470in}}%
\pgfpathlineto{\pgfqpoint{1.228511in}{2.135174in}}%
\pgfpathlineto{\pgfqpoint{1.229047in}{1.530110in}}%
\pgfpathlineto{\pgfqpoint{1.229584in}{2.107641in}}%
\pgfpathlineto{\pgfqpoint{1.230120in}{1.994771in}}%
\pgfpathlineto{\pgfqpoint{1.230657in}{2.187316in}}%
\pgfpathlineto{\pgfqpoint{1.231730in}{1.855318in}}%
\pgfpathlineto{\pgfqpoint{1.232802in}{2.160529in}}%
\pgfpathlineto{\pgfqpoint{1.233875in}{1.356104in}}%
\pgfpathlineto{\pgfqpoint{1.234412in}{1.845282in}}%
\pgfpathlineto{\pgfqpoint{1.234948in}{2.335479in}}%
\pgfpathlineto{\pgfqpoint{1.235485in}{2.104927in}}%
\pgfpathlineto{\pgfqpoint{1.236021in}{1.307231in}}%
\pgfpathlineto{\pgfqpoint{1.236557in}{2.013465in}}%
\pgfpathlineto{\pgfqpoint{1.237094in}{2.196332in}}%
\pgfpathlineto{\pgfqpoint{1.237630in}{2.152720in}}%
\pgfpathlineto{\pgfqpoint{1.238167in}{1.933163in}}%
\pgfpathlineto{\pgfqpoint{1.238703in}{2.152357in}}%
\pgfpathlineto{\pgfqpoint{1.240313in}{1.474231in}}%
\pgfpathlineto{\pgfqpoint{1.241922in}{2.113933in}}%
\pgfpathlineto{\pgfqpoint{1.242995in}{1.231069in}}%
\pgfpathlineto{\pgfqpoint{1.244068in}{2.266380in}}%
\pgfpathlineto{\pgfqpoint{1.245140in}{1.102215in}}%
\pgfpathlineto{\pgfqpoint{1.245677in}{2.125471in}}%
\pgfpathlineto{\pgfqpoint{1.246213in}{2.106087in}}%
\pgfpathlineto{\pgfqpoint{1.246750in}{1.700590in}}%
\pgfpathlineto{\pgfqpoint{1.247286in}{1.849461in}}%
\pgfpathlineto{\pgfqpoint{1.247823in}{2.240637in}}%
\pgfpathlineto{\pgfqpoint{1.248359in}{1.808242in}}%
\pgfpathlineto{\pgfqpoint{1.248896in}{2.114295in}}%
\pgfpathlineto{\pgfqpoint{1.249968in}{2.061937in}}%
\pgfpathlineto{\pgfqpoint{1.250505in}{2.062242in}}%
\pgfpathlineto{\pgfqpoint{1.251041in}{2.122604in}}%
\pgfpathlineto{\pgfqpoint{1.252114in}{1.745303in}}%
\pgfpathlineto{\pgfqpoint{1.253187in}{2.332622in}}%
\pgfpathlineto{\pgfqpoint{1.254260in}{1.066227in}}%
\pgfpathlineto{\pgfqpoint{1.254796in}{2.169192in}}%
\pgfpathlineto{\pgfqpoint{1.255333in}{1.974339in}}%
\pgfpathlineto{\pgfqpoint{1.255869in}{1.386425in}}%
\pgfpathlineto{\pgfqpoint{1.256406in}{1.650421in}}%
\pgfpathlineto{\pgfqpoint{1.256942in}{2.241992in}}%
\pgfpathlineto{\pgfqpoint{1.257478in}{1.835543in}}%
\pgfpathlineto{\pgfqpoint{1.259624in}{2.100886in}}%
\pgfpathlineto{\pgfqpoint{1.260697in}{1.824556in}}%
\pgfpathlineto{\pgfqpoint{1.261234in}{1.913024in}}%
\pgfpathlineto{\pgfqpoint{1.262306in}{2.149934in}}%
\pgfpathlineto{\pgfqpoint{1.262843in}{1.484260in}}%
\pgfpathlineto{\pgfqpoint{1.263379in}{1.623351in}}%
\pgfpathlineto{\pgfqpoint{1.263916in}{2.050535in}}%
\pgfpathlineto{\pgfqpoint{1.264452in}{1.927889in}}%
\pgfpathlineto{\pgfqpoint{1.264989in}{1.563674in}}%
\pgfpathlineto{\pgfqpoint{1.265525in}{1.697567in}}%
\pgfpathlineto{\pgfqpoint{1.266061in}{2.086319in}}%
\pgfpathlineto{\pgfqpoint{1.266598in}{1.816795in}}%
\pgfpathlineto{\pgfqpoint{1.267671in}{1.959816in}}%
\pgfpathlineto{\pgfqpoint{1.268207in}{2.155388in}}%
\pgfpathlineto{\pgfqpoint{1.268744in}{2.091297in}}%
\pgfpathlineto{\pgfqpoint{1.269280in}{2.060127in}}%
\pgfpathlineto{\pgfqpoint{1.269817in}{1.564486in}}%
\pgfpathlineto{\pgfqpoint{1.270353in}{1.839461in}}%
\pgfpathlineto{\pgfqpoint{1.270889in}{2.197269in}}%
\pgfpathlineto{\pgfqpoint{1.271426in}{2.017047in}}%
\pgfpathlineto{\pgfqpoint{1.272499in}{1.681236in}}%
\pgfpathlineto{\pgfqpoint{1.273035in}{2.149769in}}%
\pgfpathlineto{\pgfqpoint{1.273572in}{1.749568in}}%
\pgfpathlineto{\pgfqpoint{1.274108in}{1.541780in}}%
\pgfpathlineto{\pgfqpoint{1.275181in}{2.033418in}}%
\pgfpathlineto{\pgfqpoint{1.275717in}{1.947512in}}%
\pgfpathlineto{\pgfqpoint{1.276254in}{1.714679in}}%
\pgfpathlineto{\pgfqpoint{1.276790in}{2.015264in}}%
\pgfpathlineto{\pgfqpoint{1.277327in}{1.816742in}}%
\pgfpathlineto{\pgfqpoint{1.278399in}{2.069094in}}%
\pgfpathlineto{\pgfqpoint{1.278936in}{1.635273in}}%
\pgfpathlineto{\pgfqpoint{1.279472in}{1.826105in}}%
\pgfpathlineto{\pgfqpoint{1.280009in}{2.111806in}}%
\pgfpathlineto{\pgfqpoint{1.280545in}{1.930294in}}%
\pgfpathlineto{\pgfqpoint{1.281082in}{1.318157in}}%
\pgfpathlineto{\pgfqpoint{1.281618in}{1.780139in}}%
\pgfpathlineto{\pgfqpoint{1.282155in}{2.055656in}}%
\pgfpathlineto{\pgfqpoint{1.282691in}{1.831890in}}%
\pgfpathlineto{\pgfqpoint{1.283764in}{1.606847in}}%
\pgfpathlineto{\pgfqpoint{1.284300in}{2.025719in}}%
\pgfpathlineto{\pgfqpoint{1.284837in}{1.617078in}}%
\pgfpathlineto{\pgfqpoint{1.285373in}{1.680506in}}%
\pgfpathlineto{\pgfqpoint{1.285910in}{1.883734in}}%
\pgfpathlineto{\pgfqpoint{1.286446in}{1.790686in}}%
\pgfpathlineto{\pgfqpoint{1.286982in}{1.828505in}}%
\pgfpathlineto{\pgfqpoint{1.287519in}{1.945039in}}%
\pgfpathlineto{\pgfqpoint{1.288055in}{1.516667in}}%
\pgfpathlineto{\pgfqpoint{1.288592in}{1.722506in}}%
\pgfpathlineto{\pgfqpoint{1.289128in}{2.075603in}}%
\pgfpathlineto{\pgfqpoint{1.289665in}{1.791440in}}%
\pgfpathlineto{\pgfqpoint{1.290201in}{1.638816in}}%
\pgfpathlineto{\pgfqpoint{1.290738in}{1.773540in}}%
\pgfpathlineto{\pgfqpoint{1.291274in}{1.928711in}}%
\pgfpathlineto{\pgfqpoint{1.292347in}{1.629905in}}%
\pgfpathlineto{\pgfqpoint{1.292883in}{1.687599in}}%
\pgfpathlineto{\pgfqpoint{1.293420in}{1.949680in}}%
\pgfpathlineto{\pgfqpoint{1.293420in}{1.949680in}}%
\pgfpathlineto{\pgfqpoint{1.293420in}{1.949680in}}%
\pgfpathlineto{\pgfqpoint{1.293956in}{1.673957in}}%
\pgfpathlineto{\pgfqpoint{1.295029in}{1.676982in}}%
\pgfpathlineto{\pgfqpoint{1.295565in}{1.600695in}}%
\pgfpathlineto{\pgfqpoint{1.296638in}{1.901445in}}%
\pgfpathlineto{\pgfqpoint{1.297175in}{1.314945in}}%
\pgfpathlineto{\pgfqpoint{1.297711in}{1.797540in}}%
\pgfpathlineto{\pgfqpoint{1.298248in}{2.042382in}}%
\pgfpathlineto{\pgfqpoint{1.299321in}{1.217728in}}%
\pgfpathlineto{\pgfqpoint{1.300393in}{1.907693in}}%
\pgfpathlineto{\pgfqpoint{1.300930in}{1.707098in}}%
\pgfpathlineto{\pgfqpoint{1.301466in}{1.651104in}}%
\pgfpathlineto{\pgfqpoint{1.302003in}{1.741496in}}%
\pgfpathlineto{\pgfqpoint{1.302539in}{1.723695in}}%
\pgfpathlineto{\pgfqpoint{1.304148in}{1.555928in}}%
\pgfpathlineto{\pgfqpoint{1.304685in}{1.402500in}}%
\pgfpathlineto{\pgfqpoint{1.305758in}{1.927052in}}%
\pgfpathlineto{\pgfqpoint{1.306294in}{0.861031in}}%
\pgfpathlineto{\pgfqpoint{1.306831in}{1.698201in}}%
\pgfpathlineto{\pgfqpoint{1.307367in}{1.821663in}}%
\pgfpathlineto{\pgfqpoint{1.308440in}{1.307739in}}%
\pgfpathlineto{\pgfqpoint{1.308976in}{1.555306in}}%
\pgfpathlineto{\pgfqpoint{1.309513in}{1.769631in}}%
\pgfpathlineto{\pgfqpoint{1.310049in}{1.491383in}}%
\pgfpathlineto{\pgfqpoint{1.310586in}{1.580747in}}%
\pgfpathlineto{\pgfqpoint{1.311122in}{1.525807in}}%
\pgfpathlineto{\pgfqpoint{1.312195in}{1.724424in}}%
\pgfpathlineto{\pgfqpoint{1.313804in}{1.105662in}}%
\pgfpathlineto{\pgfqpoint{1.314877in}{1.830330in}}%
\pgfpathlineto{\pgfqpoint{1.315414in}{0.813433in}}%
\pgfpathlineto{\pgfqpoint{1.315950in}{1.646718in}}%
\pgfpathlineto{\pgfqpoint{1.316486in}{1.654226in}}%
\pgfpathlineto{\pgfqpoint{1.317559in}{1.102319in}}%
\pgfpathlineto{\pgfqpoint{1.318632in}{1.614146in}}%
\pgfpathlineto{\pgfqpoint{1.319705in}{1.309671in}}%
\pgfpathlineto{\pgfqpoint{1.321851in}{1.574025in}}%
\pgfpathlineto{\pgfqpoint{1.322924in}{1.103735in}}%
\pgfpathlineto{\pgfqpoint{1.323997in}{1.632024in}}%
\pgfpathlineto{\pgfqpoint{1.324533in}{0.996717in}}%
\pgfpathlineto{\pgfqpoint{1.325069in}{1.226367in}}%
\pgfpathlineto{\pgfqpoint{1.325606in}{1.602048in}}%
\pgfpathlineto{\pgfqpoint{1.326142in}{1.388751in}}%
\pgfpathlineto{\pgfqpoint{1.326679in}{1.199502in}}%
\pgfpathlineto{\pgfqpoint{1.327215in}{1.460942in}}%
\pgfpathlineto{\pgfqpoint{1.327752in}{1.342844in}}%
\pgfpathlineto{\pgfqpoint{1.328824in}{1.277830in}}%
\pgfpathlineto{\pgfqpoint{1.329361in}{1.427253in}}%
\pgfpathlineto{\pgfqpoint{1.329897in}{0.986893in}}%
\pgfpathlineto{\pgfqpoint{1.330434in}{1.390707in}}%
\pgfpathlineto{\pgfqpoint{1.330970in}{1.466198in}}%
\pgfpathlineto{\pgfqpoint{1.331507in}{1.036674in}}%
\pgfpathlineto{\pgfqpoint{1.332043in}{1.061427in}}%
\pgfpathlineto{\pgfqpoint{1.333116in}{1.532972in}}%
\pgfpathlineto{\pgfqpoint{1.333652in}{0.815427in}}%
\pgfpathlineto{\pgfqpoint{1.334189in}{1.096462in}}%
\pgfpathlineto{\pgfqpoint{1.334725in}{1.482755in}}%
\pgfpathlineto{\pgfqpoint{1.335262in}{1.263695in}}%
\pgfpathlineto{\pgfqpoint{1.336335in}{1.368216in}}%
\pgfpathlineto{\pgfqpoint{1.337944in}{1.109200in}}%
\pgfpathlineto{\pgfqpoint{1.338480in}{1.392517in}}%
\pgfpathlineto{\pgfqpoint{1.339002in}{0.590000in}}%
\pgfpathmoveto{\pgfqpoint{1.339033in}{0.590000in}}%
\pgfpathlineto{\pgfqpoint{1.339553in}{1.302170in}}%
\pgfpathlineto{\pgfqpoint{1.340090in}{1.359464in}}%
\pgfpathlineto{\pgfqpoint{1.341163in}{0.851643in}}%
\pgfpathlineto{\pgfqpoint{1.342235in}{1.380362in}}%
\pgfpathlineto{\pgfqpoint{1.342772in}{0.870697in}}%
\pgfpathlineto{\pgfqpoint{1.343308in}{0.923524in}}%
\pgfpathlineto{\pgfqpoint{1.343845in}{1.312408in}}%
\pgfpathlineto{\pgfqpoint{1.344381in}{1.117309in}}%
\pgfpathlineto{\pgfqpoint{1.344918in}{1.261310in}}%
\pgfpathlineto{\pgfqpoint{1.346527in}{0.820304in}}%
\pgfpathlineto{\pgfqpoint{1.347600in}{1.229828in}}%
\pgfpathlineto{\pgfqpoint{1.348046in}{0.590000in}}%
\pgfpathmoveto{\pgfqpoint{1.348248in}{0.590000in}}%
\pgfpathlineto{\pgfqpoint{1.348673in}{1.082772in}}%
\pgfpathlineto{\pgfqpoint{1.349209in}{1.193472in}}%
\pgfpathlineto{\pgfqpoint{1.350282in}{0.629999in}}%
\pgfpathlineto{\pgfqpoint{1.350818in}{0.882026in}}%
\pgfpathlineto{\pgfqpoint{1.351355in}{1.109874in}}%
\pgfpathlineto{\pgfqpoint{1.351891in}{0.842421in}}%
\pgfpathlineto{\pgfqpoint{1.352428in}{0.845543in}}%
\pgfpathlineto{\pgfqpoint{1.352964in}{1.123928in}}%
\pgfpathlineto{\pgfqpoint{1.353501in}{0.875572in}}%
\pgfpathlineto{\pgfqpoint{1.354037in}{1.029091in}}%
\pgfpathlineto{\pgfqpoint{1.354573in}{0.926596in}}%
\pgfpathlineto{\pgfqpoint{1.355646in}{0.609129in}}%
\pgfpathlineto{\pgfqpoint{1.356719in}{1.036894in}}%
\pgfpathlineto{\pgfqpoint{1.357035in}{0.590000in}}%
\pgfpathmoveto{\pgfqpoint{1.357687in}{0.590000in}}%
\pgfpathlineto{\pgfqpoint{1.358328in}{1.058678in}}%
\pgfpathlineto{\pgfqpoint{1.358865in}{0.854508in}}%
\pgfpathlineto{\pgfqpoint{1.359938in}{0.716906in}}%
\pgfpathlineto{\pgfqpoint{1.360474in}{0.838424in}}%
\pgfpathlineto{\pgfqpoint{1.360839in}{0.590000in}}%
\pgfpathmoveto{\pgfqpoint{1.361420in}{0.590000in}}%
\pgfpathlineto{\pgfqpoint{1.362084in}{0.731472in}}%
\pgfpathlineto{\pgfqpoint{1.362084in}{0.731472in}}%
\pgfpathlineto{\pgfqpoint{1.362084in}{0.731472in}}%
\pgfpathlineto{\pgfqpoint{1.362392in}{0.590000in}}%
\pgfpathmoveto{\pgfqpoint{1.362777in}{0.590000in}}%
\pgfpathlineto{\pgfqpoint{1.363156in}{0.841619in}}%
\pgfpathlineto{\pgfqpoint{1.363693in}{0.864901in}}%
\pgfpathlineto{\pgfqpoint{1.364074in}{0.590000in}}%
\pgfpathmoveto{\pgfqpoint{1.365279in}{0.590000in}}%
\pgfpathlineto{\pgfqpoint{1.365839in}{0.678845in}}%
\pgfpathlineto{\pgfqpoint{1.365839in}{0.678845in}}%
\pgfpathlineto{\pgfqpoint{1.365839in}{0.678845in}}%
\pgfpathlineto{\pgfqpoint{1.366001in}{0.590000in}}%
\pgfpathmoveto{\pgfqpoint{1.367178in}{0.590000in}}%
\pgfpathlineto{\pgfqpoint{1.367448in}{0.796879in}}%
\pgfpathlineto{\pgfqpoint{1.367448in}{0.796879in}}%
\pgfpathlineto{\pgfqpoint{1.367448in}{0.796879in}}%
\pgfpathlineto{\pgfqpoint{1.367884in}{0.590000in}}%
\pgfpathmoveto{\pgfqpoint{1.372684in}{0.590000in}}%
\pgfpathlineto{\pgfqpoint{1.372812in}{0.657938in}}%
\pgfpathlineto{\pgfqpoint{1.372812in}{0.657938in}}%
\pgfpathlineto{\pgfqpoint{1.372812in}{0.657938in}}%
\pgfpathlineto{\pgfqpoint{1.372924in}{0.590000in}}%
\pgfpathmoveto{\pgfqpoint{1.374922in}{0.590000in}}%
\pgfpathlineto{\pgfqpoint{1.374958in}{0.609283in}}%
\pgfpathlineto{\pgfqpoint{1.374958in}{0.609283in}}%
\pgfpathlineto{\pgfqpoint{1.374958in}{0.609283in}}%
\pgfpathlineto{\pgfqpoint{1.374984in}{0.590000in}}%
\pgfpathmoveto{\pgfqpoint{1.481421in}{0.590000in}}%
\pgfpathlineto{\pgfqpoint{1.481709in}{0.761179in}}%
\pgfpathlineto{\pgfqpoint{1.481709in}{0.761179in}}%
\pgfpathlineto{\pgfqpoint{1.481709in}{0.761179in}}%
\pgfpathlineto{\pgfqpoint{1.482245in}{0.733732in}}%
\pgfpathlineto{\pgfqpoint{1.482501in}{0.590000in}}%
\pgfpathmoveto{\pgfqpoint{1.483288in}{0.590000in}}%
\pgfpathlineto{\pgfqpoint{1.483855in}{0.816150in}}%
\pgfpathlineto{\pgfqpoint{1.484927in}{0.758197in}}%
\pgfpathlineto{\pgfqpoint{1.485464in}{0.661236in}}%
\pgfpathlineto{\pgfqpoint{1.486000in}{0.864643in}}%
\pgfpathlineto{\pgfqpoint{1.486000in}{0.864643in}}%
\pgfpathlineto{\pgfqpoint{1.486000in}{0.864643in}}%
\pgfpathlineto{\pgfqpoint{1.486537in}{0.647433in}}%
\pgfpathlineto{\pgfqpoint{1.487073in}{0.848327in}}%
\pgfpathlineto{\pgfqpoint{1.487610in}{0.785266in}}%
\pgfpathlineto{\pgfqpoint{1.488063in}{0.590000in}}%
\pgfpathmoveto{\pgfqpoint{1.488215in}{0.590000in}}%
\pgfpathlineto{\pgfqpoint{1.488682in}{0.832349in}}%
\pgfpathlineto{\pgfqpoint{1.489755in}{0.961748in}}%
\pgfpathlineto{\pgfqpoint{1.490292in}{0.865627in}}%
\pgfpathlineto{\pgfqpoint{1.490828in}{0.928991in}}%
\pgfpathlineto{\pgfqpoint{1.491365in}{1.276562in}}%
\pgfpathlineto{\pgfqpoint{1.491901in}{0.953364in}}%
\pgfpathlineto{\pgfqpoint{1.492438in}{0.829994in}}%
\pgfpathlineto{\pgfqpoint{1.492974in}{0.964377in}}%
\pgfpathlineto{\pgfqpoint{1.493438in}{0.590000in}}%
\pgfpathmoveto{\pgfqpoint{1.493566in}{0.590000in}}%
\pgfpathlineto{\pgfqpoint{1.494047in}{1.100384in}}%
\pgfpathlineto{\pgfqpoint{1.494583in}{1.288008in}}%
\pgfpathlineto{\pgfqpoint{1.494583in}{1.288008in}}%
\pgfpathlineto{\pgfqpoint{1.495656in}{0.924686in}}%
\pgfpathlineto{\pgfqpoint{1.496729in}{1.497052in}}%
\pgfpathlineto{\pgfqpoint{1.497265in}{1.254935in}}%
\pgfpathlineto{\pgfqpoint{1.497802in}{1.028408in}}%
\pgfpathlineto{\pgfqpoint{1.499411in}{1.384003in}}%
\pgfpathlineto{\pgfqpoint{1.500484in}{1.432857in}}%
\pgfpathlineto{\pgfqpoint{1.501020in}{0.916090in}}%
\pgfpathlineto{\pgfqpoint{1.501557in}{1.476569in}}%
\pgfpathlineto{\pgfqpoint{1.502093in}{1.431415in}}%
\pgfpathlineto{\pgfqpoint{1.503166in}{1.165862in}}%
\pgfpathlineto{\pgfqpoint{1.503703in}{1.682795in}}%
\pgfpathlineto{\pgfqpoint{1.504239in}{1.564321in}}%
\pgfpathlineto{\pgfqpoint{1.504776in}{1.551021in}}%
\pgfpathlineto{\pgfqpoint{1.505312in}{1.456570in}}%
\pgfpathlineto{\pgfqpoint{1.505848in}{1.662214in}}%
\pgfpathlineto{\pgfqpoint{1.506385in}{1.434038in}}%
\pgfpathlineto{\pgfqpoint{1.506921in}{1.527001in}}%
\pgfpathlineto{\pgfqpoint{1.507458in}{1.467481in}}%
\pgfpathlineto{\pgfqpoint{1.507994in}{1.501583in}}%
\pgfpathlineto{\pgfqpoint{1.508531in}{1.507996in}}%
\pgfpathlineto{\pgfqpoint{1.509603in}{1.836049in}}%
\pgfpathlineto{\pgfqpoint{1.510140in}{1.451911in}}%
\pgfpathlineto{\pgfqpoint{1.510676in}{1.577064in}}%
\pgfpathlineto{\pgfqpoint{1.511749in}{1.832465in}}%
\pgfpathlineto{\pgfqpoint{1.512286in}{1.648122in}}%
\pgfpathlineto{\pgfqpoint{1.512822in}{1.831425in}}%
\pgfpathlineto{\pgfqpoint{1.513359in}{1.885913in}}%
\pgfpathlineto{\pgfqpoint{1.513895in}{1.647482in}}%
\pgfpathlineto{\pgfqpoint{1.514968in}{1.944146in}}%
\pgfpathlineto{\pgfqpoint{1.515504in}{1.605838in}}%
\pgfpathlineto{\pgfqpoint{1.516041in}{1.719664in}}%
\pgfpathlineto{\pgfqpoint{1.517650in}{1.970908in}}%
\pgfpathlineto{\pgfqpoint{1.518186in}{1.946995in}}%
\pgfpathlineto{\pgfqpoint{1.518723in}{2.114271in}}%
\pgfpathlineto{\pgfqpoint{1.519259in}{1.777132in}}%
\pgfpathlineto{\pgfqpoint{1.519796in}{2.041937in}}%
\pgfpathlineto{\pgfqpoint{1.520332in}{2.039442in}}%
\pgfpathlineto{\pgfqpoint{1.520869in}{1.850750in}}%
\pgfpathlineto{\pgfqpoint{1.521405in}{1.916086in}}%
\pgfpathlineto{\pgfqpoint{1.521942in}{2.261426in}}%
\pgfpathlineto{\pgfqpoint{1.522478in}{2.137545in}}%
\pgfpathlineto{\pgfqpoint{1.523551in}{2.001084in}}%
\pgfpathlineto{\pgfqpoint{1.524087in}{2.314229in}}%
\pgfpathlineto{\pgfqpoint{1.524624in}{2.037634in}}%
\pgfpathlineto{\pgfqpoint{1.525160in}{1.878318in}}%
\pgfpathlineto{\pgfqpoint{1.527306in}{2.323160in}}%
\pgfpathlineto{\pgfqpoint{1.528379in}{2.101250in}}%
\pgfpathlineto{\pgfqpoint{1.528915in}{2.241646in}}%
\pgfpathlineto{\pgfqpoint{1.529452in}{2.338365in}}%
\pgfpathlineto{\pgfqpoint{1.529988in}{2.298911in}}%
\pgfpathlineto{\pgfqpoint{1.530524in}{2.162895in}}%
\pgfpathlineto{\pgfqpoint{1.531597in}{2.422780in}}%
\pgfpathlineto{\pgfqpoint{1.532134in}{2.256157in}}%
\pgfpathlineto{\pgfqpoint{1.532670in}{2.446090in}}%
\pgfpathlineto{\pgfqpoint{1.533207in}{2.432615in}}%
\pgfpathlineto{\pgfqpoint{1.533743in}{1.992517in}}%
\pgfpathlineto{\pgfqpoint{1.534280in}{2.270514in}}%
\pgfpathlineto{\pgfqpoint{1.534816in}{2.526635in}}%
\pgfpathlineto{\pgfqpoint{1.535352in}{2.374299in}}%
\pgfpathlineto{\pgfqpoint{1.535889in}{2.177867in}}%
\pgfpathlineto{\pgfqpoint{1.536962in}{2.642741in}}%
\pgfpathlineto{\pgfqpoint{1.537498in}{2.373749in}}%
\pgfpathlineto{\pgfqpoint{1.538035in}{2.414723in}}%
\pgfpathlineto{\pgfqpoint{1.539107in}{2.547469in}}%
\pgfpathlineto{\pgfqpoint{1.539644in}{2.361904in}}%
\pgfpathlineto{\pgfqpoint{1.540180in}{2.493434in}}%
\pgfpathlineto{\pgfqpoint{1.542326in}{2.666011in}}%
\pgfpathlineto{\pgfqpoint{1.542863in}{2.574314in}}%
\pgfpathlineto{\pgfqpoint{1.543399in}{2.585013in}}%
\pgfpathlineto{\pgfqpoint{1.545545in}{2.781040in}}%
\pgfpathlineto{\pgfqpoint{1.547154in}{2.591972in}}%
\pgfpathlineto{\pgfqpoint{1.547690in}{2.916172in}}%
\pgfpathlineto{\pgfqpoint{1.547690in}{2.916172in}}%
\pgfpathlineto{\pgfqpoint{1.547690in}{2.916172in}}%
\pgfpathlineto{\pgfqpoint{1.548763in}{2.424943in}}%
\pgfpathlineto{\pgfqpoint{1.549300in}{2.857504in}}%
\pgfpathlineto{\pgfqpoint{1.549836in}{2.831008in}}%
\pgfpathlineto{\pgfqpoint{1.550373in}{2.555672in}}%
\pgfpathlineto{\pgfqpoint{1.550373in}{2.555672in}}%
\pgfpathlineto{\pgfqpoint{1.550373in}{2.555672in}}%
\pgfpathlineto{\pgfqpoint{1.551445in}{2.885731in}}%
\pgfpathlineto{\pgfqpoint{1.551982in}{2.684197in}}%
\pgfpathlineto{\pgfqpoint{1.552518in}{2.743899in}}%
\pgfpathlineto{\pgfqpoint{1.553591in}{2.990077in}}%
\pgfpathlineto{\pgfqpoint{1.554128in}{2.800661in}}%
\pgfpathlineto{\pgfqpoint{1.554664in}{2.882085in}}%
\pgfpathlineto{\pgfqpoint{1.555201in}{2.982877in}}%
\pgfpathlineto{\pgfqpoint{1.555737in}{2.894856in}}%
\pgfpathlineto{\pgfqpoint{1.556273in}{2.826941in}}%
\pgfpathlineto{\pgfqpoint{1.556810in}{2.939882in}}%
\pgfpathlineto{\pgfqpoint{1.557346in}{2.902653in}}%
\pgfpathlineto{\pgfqpoint{1.557883in}{2.889247in}}%
\pgfpathlineto{\pgfqpoint{1.558419in}{3.028828in}}%
\pgfpathlineto{\pgfqpoint{1.558956in}{2.845110in}}%
\pgfpathlineto{\pgfqpoint{1.559492in}{2.890793in}}%
\pgfpathlineto{\pgfqpoint{1.560028in}{3.116086in}}%
\pgfpathlineto{\pgfqpoint{1.560565in}{3.094266in}}%
\pgfpathlineto{\pgfqpoint{1.561101in}{2.847812in}}%
\pgfpathlineto{\pgfqpoint{1.561638in}{2.874506in}}%
\pgfpathlineto{\pgfqpoint{1.562174in}{3.145850in}}%
\pgfpathlineto{\pgfqpoint{1.562711in}{3.125237in}}%
\pgfpathlineto{\pgfqpoint{1.563247in}{2.766407in}}%
\pgfpathlineto{\pgfqpoint{1.563247in}{2.766407in}}%
\pgfpathlineto{\pgfqpoint{1.563247in}{2.766407in}}%
\pgfpathlineto{\pgfqpoint{1.564320in}{3.229687in}}%
\pgfpathlineto{\pgfqpoint{1.565393in}{2.858856in}}%
\pgfpathlineto{\pgfqpoint{1.565929in}{3.256539in}}%
\pgfpathlineto{\pgfqpoint{1.566466in}{3.162788in}}%
\pgfpathlineto{\pgfqpoint{1.567002in}{2.942510in}}%
\pgfpathlineto{\pgfqpoint{1.567539in}{3.064990in}}%
\pgfpathlineto{\pgfqpoint{1.568611in}{3.256845in}}%
\pgfpathlineto{\pgfqpoint{1.569148in}{3.168305in}}%
\pgfpathlineto{\pgfqpoint{1.570757in}{3.212720in}}%
\pgfpathlineto{\pgfqpoint{1.571294in}{3.290884in}}%
\pgfpathlineto{\pgfqpoint{1.572367in}{3.036524in}}%
\pgfpathlineto{\pgfqpoint{1.572903in}{3.408533in}}%
\pgfpathlineto{\pgfqpoint{1.573439in}{3.300733in}}%
\pgfpathlineto{\pgfqpoint{1.573976in}{3.082757in}}%
\pgfpathlineto{\pgfqpoint{1.574512in}{3.245690in}}%
\pgfpathlineto{\pgfqpoint{1.575049in}{3.386733in}}%
\pgfpathlineto{\pgfqpoint{1.576122in}{3.091919in}}%
\pgfpathlineto{\pgfqpoint{1.577194in}{3.482781in}}%
\pgfpathlineto{\pgfqpoint{1.578267in}{3.194032in}}%
\pgfpathlineto{\pgfqpoint{1.578804in}{3.557107in}}%
\pgfpathlineto{\pgfqpoint{1.579340in}{3.495543in}}%
\pgfpathlineto{\pgfqpoint{1.579877in}{3.113973in}}%
\pgfpathlineto{\pgfqpoint{1.580413in}{3.400509in}}%
\pgfpathlineto{\pgfqpoint{1.580949in}{3.424437in}}%
\pgfpathlineto{\pgfqpoint{1.581486in}{3.367003in}}%
\pgfpathlineto{\pgfqpoint{1.582022in}{3.453725in}}%
\pgfpathlineto{\pgfqpoint{1.582559in}{3.262928in}}%
\pgfpathlineto{\pgfqpoint{1.584168in}{3.479717in}}%
\pgfpathlineto{\pgfqpoint{1.584705in}{3.269019in}}%
\pgfpathlineto{\pgfqpoint{1.585241in}{3.470661in}}%
\pgfpathlineto{\pgfqpoint{1.585777in}{3.610520in}}%
\pgfpathlineto{\pgfqpoint{1.586850in}{3.406360in}}%
\pgfpathlineto{\pgfqpoint{1.587923in}{3.796694in}}%
\pgfpathlineto{\pgfqpoint{1.588996in}{3.146201in}}%
\pgfpathlineto{\pgfqpoint{1.590069in}{3.789871in}}%
\pgfpathlineto{\pgfqpoint{1.590605in}{3.166378in}}%
\pgfpathlineto{\pgfqpoint{1.591142in}{3.422740in}}%
\pgfpathlineto{\pgfqpoint{1.592215in}{3.692333in}}%
\pgfpathlineto{\pgfqpoint{1.592751in}{3.391705in}}%
\pgfpathlineto{\pgfqpoint{1.593288in}{3.527316in}}%
\pgfpathlineto{\pgfqpoint{1.594360in}{3.717027in}}%
\pgfpathlineto{\pgfqpoint{1.594897in}{3.636080in}}%
\pgfpathlineto{\pgfqpoint{1.595433in}{3.473430in}}%
\pgfpathlineto{\pgfqpoint{1.597043in}{3.809000in}}%
\pgfpathlineto{\pgfqpoint{1.597579in}{3.441901in}}%
\pgfpathlineto{\pgfqpoint{1.598115in}{3.594662in}}%
\pgfpathlineto{\pgfqpoint{1.598652in}{3.911321in}}%
\pgfpathlineto{\pgfqpoint{1.599188in}{3.636092in}}%
\pgfpathlineto{\pgfqpoint{1.599725in}{3.463622in}}%
\pgfpathlineto{\pgfqpoint{1.600798in}{3.971647in}}%
\pgfpathlineto{\pgfqpoint{1.601870in}{3.289878in}}%
\pgfpathlineto{\pgfqpoint{1.602943in}{3.980997in}}%
\pgfpathlineto{\pgfqpoint{1.604016in}{3.359031in}}%
\pgfpathlineto{\pgfqpoint{1.605089in}{3.925240in}}%
\pgfpathlineto{\pgfqpoint{1.605626in}{3.809301in}}%
\pgfpathlineto{\pgfqpoint{1.606162in}{3.662526in}}%
\pgfpathlineto{\pgfqpoint{1.606698in}{3.688888in}}%
\pgfpathlineto{\pgfqpoint{1.607771in}{3.942364in}}%
\pgfpathlineto{\pgfqpoint{1.608308in}{3.617416in}}%
\pgfpathlineto{\pgfqpoint{1.608844in}{3.925356in}}%
\pgfpathlineto{\pgfqpoint{1.609917in}{3.981367in}}%
\pgfpathlineto{\pgfqpoint{1.610453in}{3.678305in}}%
\pgfpathlineto{\pgfqpoint{1.610990in}{3.874289in}}%
\pgfpathlineto{\pgfqpoint{1.611526in}{4.140471in}}%
\pgfpathlineto{\pgfqpoint{1.612599in}{3.571709in}}%
\pgfpathlineto{\pgfqpoint{1.613672in}{4.175792in}}%
\pgfpathlineto{\pgfqpoint{1.614745in}{3.062781in}}%
\pgfpathlineto{\pgfqpoint{1.615818in}{4.110380in}}%
\pgfpathlineto{\pgfqpoint{1.616354in}{3.891948in}}%
\pgfpathlineto{\pgfqpoint{1.616891in}{3.014787in}}%
\pgfpathlineto{\pgfqpoint{1.617427in}{3.804323in}}%
\pgfpathlineto{\pgfqpoint{1.618500in}{4.149952in}}%
\pgfpathlineto{\pgfqpoint{1.619036in}{3.724616in}}%
\pgfpathlineto{\pgfqpoint{1.619573in}{3.996535in}}%
\pgfpathlineto{\pgfqpoint{1.620109in}{4.051000in}}%
\pgfpathlineto{\pgfqpoint{1.620646in}{4.226759in}}%
\pgfpathlineto{\pgfqpoint{1.621182in}{3.846476in}}%
\pgfpathlineto{\pgfqpoint{1.621719in}{4.080508in}}%
\pgfpathlineto{\pgfqpoint{1.622255in}{4.196871in}}%
\pgfpathlineto{\pgfqpoint{1.622792in}{4.143510in}}%
\pgfpathlineto{\pgfqpoint{1.623328in}{3.640657in}}%
\pgfpathlineto{\pgfqpoint{1.623864in}{4.134856in}}%
\pgfpathlineto{\pgfqpoint{1.624401in}{4.377356in}}%
\pgfpathlineto{\pgfqpoint{1.625474in}{3.733669in}}%
\pgfpathlineto{\pgfqpoint{1.626547in}{4.271034in}}%
\pgfpathlineto{\pgfqpoint{1.627083in}{4.252495in}}%
\pgfpathlineto{\pgfqpoint{1.627619in}{3.279533in}}%
\pgfpathlineto{\pgfqpoint{1.628156in}{3.487609in}}%
\pgfpathlineto{\pgfqpoint{1.628692in}{4.289520in}}%
\pgfpathlineto{\pgfqpoint{1.629229in}{4.238678in}}%
\pgfpathlineto{\pgfqpoint{1.630302in}{3.695105in}}%
\pgfpathlineto{\pgfqpoint{1.631374in}{4.504069in}}%
\pgfpathlineto{\pgfqpoint{1.631911in}{4.114520in}}%
\pgfpathlineto{\pgfqpoint{1.632447in}{3.924582in}}%
\pgfpathlineto{\pgfqpoint{1.633520in}{4.427768in}}%
\pgfpathlineto{\pgfqpoint{1.634593in}{3.987587in}}%
\pgfpathlineto{\pgfqpoint{1.635130in}{4.356598in}}%
\pgfpathlineto{\pgfqpoint{1.635666in}{4.249929in}}%
\pgfpathlineto{\pgfqpoint{1.636739in}{3.485425in}}%
\pgfpathlineto{\pgfqpoint{1.637812in}{4.444992in}}%
\pgfpathlineto{\pgfqpoint{1.638885in}{3.284091in}}%
\pgfpathlineto{\pgfqpoint{1.639957in}{4.564790in}}%
\pgfpathlineto{\pgfqpoint{1.641030in}{3.498367in}}%
\pgfpathlineto{\pgfqpoint{1.642103in}{4.589220in}}%
\pgfpathlineto{\pgfqpoint{1.642640in}{4.373315in}}%
\pgfpathlineto{\pgfqpoint{1.643176in}{3.347822in}}%
\pgfpathlineto{\pgfqpoint{1.643713in}{4.275800in}}%
\pgfpathlineto{\pgfqpoint{1.644249in}{4.576352in}}%
\pgfpathlineto{\pgfqpoint{1.644785in}{4.505244in}}%
\pgfpathlineto{\pgfqpoint{1.645322in}{2.987749in}}%
\pgfpathlineto{\pgfqpoint{1.645858in}{4.453575in}}%
\pgfpathlineto{\pgfqpoint{1.646931in}{4.397351in}}%
\pgfpathlineto{\pgfqpoint{1.647468in}{3.634151in}}%
\pgfpathlineto{\pgfqpoint{1.647468in}{3.634151in}}%
\pgfpathlineto{\pgfqpoint{1.647468in}{3.634151in}}%
\pgfpathlineto{\pgfqpoint{1.648540in}{4.453330in}}%
\pgfpathlineto{\pgfqpoint{1.650150in}{4.088518in}}%
\pgfpathlineto{\pgfqpoint{1.650686in}{4.679997in}}%
\pgfpathlineto{\pgfqpoint{1.651223in}{4.465072in}}%
\pgfpathlineto{\pgfqpoint{1.652295in}{4.002850in}}%
\pgfpathlineto{\pgfqpoint{1.653368in}{4.646667in}}%
\pgfpathlineto{\pgfqpoint{1.653905in}{4.027556in}}%
\pgfpathlineto{\pgfqpoint{1.654441in}{4.202697in}}%
\pgfpathlineto{\pgfqpoint{1.655514in}{4.667835in}}%
\pgfpathlineto{\pgfqpoint{1.656051in}{4.260556in}}%
\pgfpathlineto{\pgfqpoint{1.656587in}{4.384674in}}%
\pgfpathlineto{\pgfqpoint{1.657123in}{4.451470in}}%
\pgfpathlineto{\pgfqpoint{1.657660in}{4.732712in}}%
\pgfpathlineto{\pgfqpoint{1.658733in}{4.006268in}}%
\pgfpathlineto{\pgfqpoint{1.660342in}{4.584365in}}%
\pgfpathlineto{\pgfqpoint{1.660878in}{2.782150in}}%
\pgfpathlineto{\pgfqpoint{1.661415in}{4.553188in}}%
\pgfpathlineto{\pgfqpoint{1.661951in}{4.711408in}}%
\pgfpathlineto{\pgfqpoint{1.663024in}{4.269841in}}%
\pgfpathlineto{\pgfqpoint{1.664097in}{4.773638in}}%
\pgfpathlineto{\pgfqpoint{1.664634in}{4.557450in}}%
\pgfpathlineto{\pgfqpoint{1.665170in}{4.360124in}}%
\pgfpathlineto{\pgfqpoint{1.665706in}{4.499511in}}%
\pgfpathlineto{\pgfqpoint{1.666243in}{4.785177in}}%
\pgfpathlineto{\pgfqpoint{1.666779in}{4.602754in}}%
\pgfpathlineto{\pgfqpoint{1.667316in}{4.312751in}}%
\pgfpathlineto{\pgfqpoint{1.667852in}{4.500516in}}%
\pgfpathlineto{\pgfqpoint{1.668925in}{4.726273in}}%
\pgfpathlineto{\pgfqpoint{1.669461in}{4.313599in}}%
\pgfpathlineto{\pgfqpoint{1.669998in}{4.373425in}}%
\pgfpathlineto{\pgfqpoint{1.671071in}{4.732062in}}%
\pgfpathlineto{\pgfqpoint{1.672144in}{4.313998in}}%
\pgfpathlineto{\pgfqpoint{1.672680in}{4.773762in}}%
\pgfpathlineto{\pgfqpoint{1.673217in}{4.651503in}}%
\pgfpathlineto{\pgfqpoint{1.673753in}{4.611310in}}%
\pgfpathlineto{\pgfqpoint{1.674289in}{4.329989in}}%
\pgfpathlineto{\pgfqpoint{1.675362in}{4.776870in}}%
\pgfpathlineto{\pgfqpoint{1.676435in}{4.510224in}}%
\pgfpathlineto{\pgfqpoint{1.677508in}{4.793780in}}%
\pgfpathlineto{\pgfqpoint{1.678044in}{4.551188in}}%
\pgfpathlineto{\pgfqpoint{1.678581in}{4.614121in}}%
\pgfpathlineto{\pgfqpoint{1.679117in}{4.598366in}}%
\pgfpathlineto{\pgfqpoint{1.679654in}{4.786659in}}%
\pgfpathlineto{\pgfqpoint{1.680190in}{4.741781in}}%
\pgfpathlineto{\pgfqpoint{1.680727in}{4.455068in}}%
\pgfpathlineto{\pgfqpoint{1.681263in}{4.599496in}}%
\pgfpathlineto{\pgfqpoint{1.682336in}{4.801954in}}%
\pgfpathlineto{\pgfqpoint{1.682872in}{4.469154in}}%
\pgfpathlineto{\pgfqpoint{1.683409in}{4.587622in}}%
\pgfpathlineto{\pgfqpoint{1.683945in}{4.791280in}}%
\pgfpathlineto{\pgfqpoint{1.684482in}{4.783205in}}%
\pgfpathlineto{\pgfqpoint{1.685018in}{4.634576in}}%
\pgfpathlineto{\pgfqpoint{1.685555in}{4.644589in}}%
\pgfpathlineto{\pgfqpoint{1.686091in}{4.835985in}}%
\pgfpathlineto{\pgfqpoint{1.686627in}{4.805459in}}%
\pgfpathlineto{\pgfqpoint{1.687164in}{4.676820in}}%
\pgfpathlineto{\pgfqpoint{1.687700in}{4.683965in}}%
\pgfpathlineto{\pgfqpoint{1.688237in}{4.829652in}}%
\pgfpathlineto{\pgfqpoint{1.688773in}{4.755158in}}%
\pgfpathlineto{\pgfqpoint{1.689310in}{4.770799in}}%
\pgfpathlineto{\pgfqpoint{1.689846in}{4.678782in}}%
\pgfpathlineto{\pgfqpoint{1.690382in}{4.739783in}}%
\pgfpathlineto{\pgfqpoint{1.690919in}{4.797712in}}%
\pgfpathlineto{\pgfqpoint{1.691455in}{4.741791in}}%
\pgfpathlineto{\pgfqpoint{1.692528in}{4.594677in}}%
\pgfpathlineto{\pgfqpoint{1.693065in}{4.880116in}}%
\pgfpathlineto{\pgfqpoint{1.693601in}{4.842418in}}%
\pgfpathlineto{\pgfqpoint{1.694138in}{4.615352in}}%
\pgfpathlineto{\pgfqpoint{1.694674in}{4.802414in}}%
\pgfpathlineto{\pgfqpoint{1.695747in}{4.875138in}}%
\pgfpathlineto{\pgfqpoint{1.696283in}{4.730146in}}%
\pgfpathlineto{\pgfqpoint{1.696820in}{4.762677in}}%
\pgfpathlineto{\pgfqpoint{1.697356in}{4.920119in}}%
\pgfpathlineto{\pgfqpoint{1.697893in}{4.847554in}}%
\pgfpathlineto{\pgfqpoint{1.698965in}{4.770287in}}%
\pgfpathlineto{\pgfqpoint{1.699502in}{4.850608in}}%
\pgfpathlineto{\pgfqpoint{1.700038in}{4.817407in}}%
\pgfpathlineto{\pgfqpoint{1.700575in}{4.822380in}}%
\pgfpathlineto{\pgfqpoint{1.701111in}{4.799188in}}%
\pgfpathlineto{\pgfqpoint{1.701648in}{4.713378in}}%
\pgfpathlineto{\pgfqpoint{1.702184in}{4.883406in}}%
\pgfpathlineto{\pgfqpoint{1.702720in}{4.850226in}}%
\pgfpathlineto{\pgfqpoint{1.703257in}{4.698053in}}%
\pgfpathlineto{\pgfqpoint{1.703793in}{4.831464in}}%
\pgfpathlineto{\pgfqpoint{1.704866in}{4.921003in}}%
\pgfpathlineto{\pgfqpoint{1.705403in}{4.789965in}}%
\pgfpathlineto{\pgfqpoint{1.705939in}{4.793463in}}%
\pgfpathlineto{\pgfqpoint{1.706476in}{4.963982in}}%
\pgfpathlineto{\pgfqpoint{1.707012in}{4.884779in}}%
\pgfpathlineto{\pgfqpoint{1.707548in}{4.766174in}}%
\pgfpathlineto{\pgfqpoint{1.708085in}{4.916798in}}%
\pgfpathlineto{\pgfqpoint{1.708621in}{4.891788in}}%
\pgfpathlineto{\pgfqpoint{1.709158in}{4.882225in}}%
\pgfpathlineto{\pgfqpoint{1.709694in}{4.893023in}}%
\pgfpathlineto{\pgfqpoint{1.710767in}{4.821568in}}%
\pgfpathlineto{\pgfqpoint{1.711840in}{4.918554in}}%
\pgfpathlineto{\pgfqpoint{1.712376in}{4.770031in}}%
\pgfpathlineto{\pgfqpoint{1.712913in}{4.877079in}}%
\pgfpathlineto{\pgfqpoint{1.713986in}{4.945634in}}%
\pgfpathlineto{\pgfqpoint{1.715059in}{4.848756in}}%
\pgfpathlineto{\pgfqpoint{1.715595in}{5.002384in}}%
\pgfpathlineto{\pgfqpoint{1.716131in}{4.918965in}}%
\pgfpathlineto{\pgfqpoint{1.716668in}{4.829867in}}%
\pgfpathlineto{\pgfqpoint{1.717204in}{4.917451in}}%
\pgfpathlineto{\pgfqpoint{1.717741in}{4.960624in}}%
\pgfpathlineto{\pgfqpoint{1.718277in}{4.938881in}}%
\pgfpathlineto{\pgfqpoint{1.718814in}{4.901447in}}%
\pgfpathlineto{\pgfqpoint{1.718814in}{4.901447in}}%
\pgfpathlineto{\pgfqpoint{1.718814in}{4.901447in}}%
\pgfpathlineto{\pgfqpoint{1.719350in}{4.939016in}}%
\pgfpathlineto{\pgfqpoint{1.719886in}{4.913340in}}%
\pgfpathlineto{\pgfqpoint{1.720423in}{4.907534in}}%
\pgfpathlineto{\pgfqpoint{1.720959in}{4.955011in}}%
\pgfpathlineto{\pgfqpoint{1.721496in}{4.858172in}}%
\pgfpathlineto{\pgfqpoint{1.722032in}{4.908131in}}%
\pgfpathlineto{\pgfqpoint{1.723105in}{4.965090in}}%
\pgfpathlineto{\pgfqpoint{1.724178in}{4.907206in}}%
\pgfpathlineto{\pgfqpoint{1.725251in}{5.002551in}}%
\pgfpathlineto{\pgfqpoint{1.725787in}{4.881756in}}%
\pgfpathlineto{\pgfqpoint{1.726324in}{4.913613in}}%
\pgfpathlineto{\pgfqpoint{1.726860in}{4.994441in}}%
\pgfpathlineto{\pgfqpoint{1.727397in}{4.977775in}}%
\pgfpathlineto{\pgfqpoint{1.727933in}{4.916675in}}%
\pgfpathlineto{\pgfqpoint{1.728469in}{4.973301in}}%
\pgfpathlineto{\pgfqpoint{1.729006in}{4.973751in}}%
\pgfpathlineto{\pgfqpoint{1.730079in}{4.945859in}}%
\pgfpathlineto{\pgfqpoint{1.730615in}{4.936849in}}%
\pgfpathlineto{\pgfqpoint{1.731152in}{4.967498in}}%
\pgfpathlineto{\pgfqpoint{1.731688in}{4.961566in}}%
\pgfpathlineto{\pgfqpoint{1.732224in}{4.958040in}}%
\pgfpathlineto{\pgfqpoint{1.732761in}{5.013452in}}%
\pgfpathlineto{\pgfqpoint{1.733297in}{4.960178in}}%
\pgfpathlineto{\pgfqpoint{1.734370in}{5.053046in}}%
\pgfpathlineto{\pgfqpoint{1.734907in}{4.959324in}}%
\pgfpathlineto{\pgfqpoint{1.735443in}{4.964509in}}%
\pgfpathlineto{\pgfqpoint{1.735980in}{5.022653in}}%
\pgfpathlineto{\pgfqpoint{1.736516in}{5.021376in}}%
\pgfpathlineto{\pgfqpoint{1.737589in}{4.961527in}}%
\pgfpathlineto{\pgfqpoint{1.738662in}{5.037924in}}%
\pgfpathlineto{\pgfqpoint{1.739198in}{4.977361in}}%
\pgfpathlineto{\pgfqpoint{1.739735in}{4.988685in}}%
\pgfpathlineto{\pgfqpoint{1.740271in}{5.016324in}}%
\pgfpathlineto{\pgfqpoint{1.740807in}{4.993284in}}%
\pgfpathlineto{\pgfqpoint{1.741344in}{4.976117in}}%
\pgfpathlineto{\pgfqpoint{1.741880in}{5.033178in}}%
\pgfpathlineto{\pgfqpoint{1.742417in}{5.030860in}}%
\pgfpathlineto{\pgfqpoint{1.742953in}{5.029076in}}%
\pgfpathlineto{\pgfqpoint{1.744026in}{5.052577in}}%
\pgfpathlineto{\pgfqpoint{1.744563in}{5.009404in}}%
\pgfpathlineto{\pgfqpoint{1.745099in}{5.015665in}}%
\pgfpathlineto{\pgfqpoint{1.745635in}{5.055941in}}%
\pgfpathlineto{\pgfqpoint{1.746172in}{5.035446in}}%
\pgfpathlineto{\pgfqpoint{1.746708in}{4.993537in}}%
\pgfpathlineto{\pgfqpoint{1.747781in}{5.072418in}}%
\pgfpathlineto{\pgfqpoint{1.748854in}{5.007524in}}%
\pgfpathlineto{\pgfqpoint{1.749927in}{5.064691in}}%
\pgfpathlineto{\pgfqpoint{1.751000in}{5.018758in}}%
\pgfpathlineto{\pgfqpoint{1.751536in}{5.076774in}}%
\pgfpathlineto{\pgfqpoint{1.752073in}{5.026970in}}%
\pgfpathlineto{\pgfqpoint{1.753145in}{5.071417in}}%
\pgfpathlineto{\pgfqpoint{1.753682in}{5.069724in}}%
\pgfpathlineto{\pgfqpoint{1.754218in}{5.041034in}}%
\pgfpathlineto{\pgfqpoint{1.754755in}{5.041807in}}%
\pgfpathlineto{\pgfqpoint{1.755291in}{5.066430in}}%
\pgfpathlineto{\pgfqpoint{1.755828in}{5.042339in}}%
\pgfpathlineto{\pgfqpoint{1.756364in}{5.030879in}}%
\pgfpathlineto{\pgfqpoint{1.756901in}{5.088156in}}%
\pgfpathlineto{\pgfqpoint{1.757437in}{5.075488in}}%
\pgfpathlineto{\pgfqpoint{1.757973in}{5.038868in}}%
\pgfpathlineto{\pgfqpoint{1.758510in}{5.048332in}}%
\pgfpathlineto{\pgfqpoint{1.759046in}{5.083501in}}%
\pgfpathlineto{\pgfqpoint{1.759583in}{5.081218in}}%
\pgfpathlineto{\pgfqpoint{1.760119in}{5.041537in}}%
\pgfpathlineto{\pgfqpoint{1.760656in}{5.071343in}}%
\pgfpathlineto{\pgfqpoint{1.761192in}{5.096755in}}%
\pgfpathlineto{\pgfqpoint{1.761728in}{5.068092in}}%
\pgfpathlineto{\pgfqpoint{1.762265in}{5.077085in}}%
\pgfpathlineto{\pgfqpoint{1.762801in}{5.101029in}}%
\pgfpathlineto{\pgfqpoint{1.763338in}{5.094609in}}%
\pgfpathlineto{\pgfqpoint{1.763874in}{5.063673in}}%
\pgfpathlineto{\pgfqpoint{1.764411in}{5.101693in}}%
\pgfpathlineto{\pgfqpoint{1.764947in}{5.097112in}}%
\pgfpathlineto{\pgfqpoint{1.766556in}{5.072734in}}%
\pgfpathlineto{\pgfqpoint{1.767093in}{5.116931in}}%
\pgfpathlineto{\pgfqpoint{1.767629in}{5.060590in}}%
\pgfpathlineto{\pgfqpoint{1.768166in}{5.097865in}}%
\pgfpathlineto{\pgfqpoint{1.768702in}{5.106949in}}%
\pgfpathlineto{\pgfqpoint{1.769239in}{5.074061in}}%
\pgfpathlineto{\pgfqpoint{1.769775in}{5.078070in}}%
\pgfpathlineto{\pgfqpoint{1.770848in}{5.115875in}}%
\pgfpathlineto{\pgfqpoint{1.771921in}{5.093687in}}%
\pgfpathlineto{\pgfqpoint{1.772457in}{5.127134in}}%
\pgfpathlineto{\pgfqpoint{1.772994in}{5.108252in}}%
\pgfpathlineto{\pgfqpoint{1.774066in}{5.105613in}}%
\pgfpathlineto{\pgfqpoint{1.774603in}{5.107315in}}%
\pgfpathlineto{\pgfqpoint{1.775676in}{5.084044in}}%
\pgfpathlineto{\pgfqpoint{1.776212in}{5.111269in}}%
\pgfpathlineto{\pgfqpoint{1.776749in}{5.106274in}}%
\pgfpathlineto{\pgfqpoint{1.777285in}{5.100877in}}%
\pgfpathlineto{\pgfqpoint{1.778358in}{5.143190in}}%
\pgfpathlineto{\pgfqpoint{1.778894in}{5.112103in}}%
\pgfpathlineto{\pgfqpoint{1.779431in}{5.115380in}}%
\pgfpathlineto{\pgfqpoint{1.780504in}{5.135991in}}%
\pgfpathlineto{\pgfqpoint{1.781040in}{5.112459in}}%
\pgfpathlineto{\pgfqpoint{1.781577in}{5.133057in}}%
\pgfpathlineto{\pgfqpoint{1.782649in}{5.141372in}}%
\pgfpathlineto{\pgfqpoint{1.783186in}{5.105424in}}%
\pgfpathlineto{\pgfqpoint{1.783722in}{5.135373in}}%
\pgfpathlineto{\pgfqpoint{1.784259in}{5.132236in}}%
\pgfpathlineto{\pgfqpoint{1.784795in}{5.119866in}}%
\pgfpathlineto{\pgfqpoint{1.785332in}{5.128272in}}%
\pgfpathlineto{\pgfqpoint{1.785868in}{5.132448in}}%
\pgfpathlineto{\pgfqpoint{1.786405in}{5.131971in}}%
\pgfpathlineto{\pgfqpoint{1.786941in}{5.123907in}}%
\pgfpathlineto{\pgfqpoint{1.788014in}{5.165403in}}%
\pgfpathlineto{\pgfqpoint{1.789087in}{5.118557in}}%
\pgfpathlineto{\pgfqpoint{1.790160in}{5.157996in}}%
\pgfpathlineto{\pgfqpoint{1.790696in}{5.133730in}}%
\pgfpathlineto{\pgfqpoint{1.791232in}{5.138026in}}%
\pgfpathlineto{\pgfqpoint{1.791769in}{5.156837in}}%
\pgfpathlineto{\pgfqpoint{1.792305in}{5.153003in}}%
\pgfpathlineto{\pgfqpoint{1.792842in}{5.148581in}}%
\pgfpathlineto{\pgfqpoint{1.793378in}{5.151778in}}%
\pgfpathlineto{\pgfqpoint{1.793915in}{5.150071in}}%
\pgfpathlineto{\pgfqpoint{1.794451in}{5.126683in}}%
\pgfpathlineto{\pgfqpoint{1.794988in}{5.147311in}}%
\pgfpathlineto{\pgfqpoint{1.795524in}{5.148915in}}%
\pgfpathlineto{\pgfqpoint{1.796060in}{5.166398in}}%
\pgfpathlineto{\pgfqpoint{1.796060in}{5.166398in}}%
\pgfpathlineto{\pgfqpoint{1.796060in}{5.166398in}}%
\pgfpathlineto{\pgfqpoint{1.796597in}{5.148783in}}%
\pgfpathlineto{\pgfqpoint{1.797133in}{5.162347in}}%
\pgfpathlineto{\pgfqpoint{1.797670in}{5.175106in}}%
\pgfpathlineto{\pgfqpoint{1.798743in}{5.144710in}}%
\pgfpathlineto{\pgfqpoint{1.799279in}{5.182414in}}%
\pgfpathlineto{\pgfqpoint{1.799815in}{5.168647in}}%
\pgfpathlineto{\pgfqpoint{1.800888in}{5.146773in}}%
\pgfpathlineto{\pgfqpoint{1.801961in}{5.178027in}}%
\pgfpathlineto{\pgfqpoint{1.803034in}{5.148073in}}%
\pgfpathlineto{\pgfqpoint{1.803570in}{5.148980in}}%
\pgfpathlineto{\pgfqpoint{1.805180in}{5.184621in}}%
\pgfpathlineto{\pgfqpoint{1.805716in}{5.180227in}}%
\pgfpathlineto{\pgfqpoint{1.806253in}{5.162475in}}%
\pgfpathlineto{\pgfqpoint{1.806789in}{5.176644in}}%
\pgfpathlineto{\pgfqpoint{1.807326in}{5.185278in}}%
\pgfpathlineto{\pgfqpoint{1.807862in}{5.181903in}}%
\pgfpathlineto{\pgfqpoint{1.808398in}{5.170510in}}%
\pgfpathlineto{\pgfqpoint{1.808935in}{5.173513in}}%
\pgfpathlineto{\pgfqpoint{1.809471in}{5.175147in}}%
\pgfpathlineto{\pgfqpoint{1.810008in}{5.164540in}}%
\pgfpathlineto{\pgfqpoint{1.811617in}{5.189086in}}%
\pgfpathlineto{\pgfqpoint{1.812153in}{5.151716in}}%
\pgfpathlineto{\pgfqpoint{1.812690in}{5.184623in}}%
\pgfpathlineto{\pgfqpoint{1.813226in}{5.184472in}}%
\pgfpathlineto{\pgfqpoint{1.813763in}{5.181082in}}%
\pgfpathlineto{\pgfqpoint{1.814299in}{5.181668in}}%
\pgfpathlineto{\pgfqpoint{1.814836in}{5.194038in}}%
\pgfpathlineto{\pgfqpoint{1.815372in}{5.178793in}}%
\pgfpathlineto{\pgfqpoint{1.815909in}{5.186085in}}%
\pgfpathlineto{\pgfqpoint{1.816445in}{5.185019in}}%
\pgfpathlineto{\pgfqpoint{1.816981in}{5.206583in}}%
\pgfpathlineto{\pgfqpoint{1.817518in}{5.191919in}}%
\pgfpathlineto{\pgfqpoint{1.818054in}{5.182465in}}%
\pgfpathlineto{\pgfqpoint{1.818591in}{5.185503in}}%
\pgfpathlineto{\pgfqpoint{1.819127in}{5.203867in}}%
\pgfpathlineto{\pgfqpoint{1.819664in}{5.196108in}}%
\pgfpathlineto{\pgfqpoint{1.820736in}{5.192629in}}%
\pgfpathlineto{\pgfqpoint{1.821273in}{5.172299in}}%
\pgfpathlineto{\pgfqpoint{1.821809in}{5.189229in}}%
\pgfpathlineto{\pgfqpoint{1.822882in}{5.203230in}}%
\pgfpathlineto{\pgfqpoint{1.823955in}{5.178405in}}%
\pgfpathlineto{\pgfqpoint{1.825028in}{5.214998in}}%
\pgfpathlineto{\pgfqpoint{1.825564in}{5.190389in}}%
\pgfpathlineto{\pgfqpoint{1.826101in}{5.202024in}}%
\pgfpathlineto{\pgfqpoint{1.826637in}{5.203703in}}%
\pgfpathlineto{\pgfqpoint{1.827174in}{5.220550in}}%
\pgfpathlineto{\pgfqpoint{1.827710in}{5.183605in}}%
\pgfpathlineto{\pgfqpoint{1.828247in}{5.200299in}}%
\pgfpathlineto{\pgfqpoint{1.828783in}{5.199319in}}%
\pgfpathlineto{\pgfqpoint{1.829319in}{5.216834in}}%
\pgfpathlineto{\pgfqpoint{1.829856in}{5.183007in}}%
\pgfpathlineto{\pgfqpoint{1.830392in}{5.190044in}}%
\pgfpathlineto{\pgfqpoint{1.830929in}{5.194593in}}%
\pgfpathlineto{\pgfqpoint{1.831465in}{5.223856in}}%
\pgfpathlineto{\pgfqpoint{1.832002in}{5.208339in}}%
\pgfpathlineto{\pgfqpoint{1.832538in}{5.189072in}}%
\pgfpathlineto{\pgfqpoint{1.833074in}{5.195447in}}%
\pgfpathlineto{\pgfqpoint{1.833611in}{5.198633in}}%
\pgfpathlineto{\pgfqpoint{1.834147in}{5.225494in}}%
\pgfpathlineto{\pgfqpoint{1.834684in}{5.220568in}}%
\pgfpathlineto{\pgfqpoint{1.835757in}{5.205185in}}%
\pgfpathlineto{\pgfqpoint{1.836293in}{5.222540in}}%
\pgfpathlineto{\pgfqpoint{1.836293in}{5.222540in}}%
\pgfpathlineto{\pgfqpoint{1.836293in}{5.222540in}}%
\pgfpathlineto{\pgfqpoint{1.837902in}{5.187076in}}%
\pgfpathlineto{\pgfqpoint{1.838439in}{5.223787in}}%
\pgfpathlineto{\pgfqpoint{1.838975in}{5.202766in}}%
\pgfpathlineto{\pgfqpoint{1.839512in}{5.207775in}}%
\pgfpathlineto{\pgfqpoint{1.840048in}{5.201018in}}%
\pgfpathlineto{\pgfqpoint{1.840585in}{5.229516in}}%
\pgfpathlineto{\pgfqpoint{1.841121in}{5.209939in}}%
\pgfpathlineto{\pgfqpoint{1.841657in}{5.215945in}}%
\pgfpathlineto{\pgfqpoint{1.842194in}{5.196523in}}%
\pgfpathlineto{\pgfqpoint{1.842730in}{5.208613in}}%
\pgfpathlineto{\pgfqpoint{1.843803in}{5.217499in}}%
\pgfpathlineto{\pgfqpoint{1.844876in}{5.207780in}}%
\pgfpathlineto{\pgfqpoint{1.846485in}{5.225580in}}%
\pgfpathlineto{\pgfqpoint{1.848095in}{5.214457in}}%
\pgfpathlineto{\pgfqpoint{1.848631in}{5.231874in}}%
\pgfpathlineto{\pgfqpoint{1.850240in}{5.210013in}}%
\pgfpathlineto{\pgfqpoint{1.850777in}{5.221631in}}%
\pgfpathlineto{\pgfqpoint{1.851313in}{5.205546in}}%
\pgfpathlineto{\pgfqpoint{1.852386in}{5.205611in}}%
\pgfpathlineto{\pgfqpoint{1.852923in}{5.228208in}}%
\pgfpathlineto{\pgfqpoint{1.853459in}{5.220360in}}%
\pgfpathlineto{\pgfqpoint{1.855068in}{5.225996in}}%
\pgfpathlineto{\pgfqpoint{1.855605in}{5.228719in}}%
\pgfpathlineto{\pgfqpoint{1.856141in}{5.227345in}}%
\pgfpathlineto{\pgfqpoint{1.857214in}{5.208023in}}%
\pgfpathlineto{\pgfqpoint{1.857751in}{5.218757in}}%
\pgfpathlineto{\pgfqpoint{1.858287in}{5.219230in}}%
\pgfpathlineto{\pgfqpoint{1.859360in}{5.215419in}}%
\pgfpathlineto{\pgfqpoint{1.860969in}{5.228130in}}%
\pgfpathlineto{\pgfqpoint{1.861506in}{5.214064in}}%
\pgfpathlineto{\pgfqpoint{1.862042in}{5.221551in}}%
\pgfpathlineto{\pgfqpoint{1.863115in}{5.243496in}}%
\pgfpathlineto{\pgfqpoint{1.864188in}{5.220456in}}%
\pgfpathlineto{\pgfqpoint{1.864724in}{5.223627in}}%
\pgfpathlineto{\pgfqpoint{1.865261in}{5.227170in}}%
\pgfpathlineto{\pgfqpoint{1.865797in}{5.224892in}}%
\pgfpathlineto{\pgfqpoint{1.866870in}{5.210099in}}%
\pgfpathlineto{\pgfqpoint{1.867943in}{5.241807in}}%
\pgfpathlineto{\pgfqpoint{1.869016in}{5.222292in}}%
\pgfpathlineto{\pgfqpoint{1.869552in}{5.239236in}}%
\pgfpathlineto{\pgfqpoint{1.870089in}{5.229750in}}%
\pgfpathlineto{\pgfqpoint{1.871161in}{5.212553in}}%
\pgfpathlineto{\pgfqpoint{1.871698in}{5.219146in}}%
\pgfpathlineto{\pgfqpoint{1.872771in}{5.233697in}}%
\pgfpathlineto{\pgfqpoint{1.873844in}{5.211012in}}%
\pgfpathlineto{\pgfqpoint{1.874916in}{5.235805in}}%
\pgfpathlineto{\pgfqpoint{1.875453in}{5.221109in}}%
\pgfpathlineto{\pgfqpoint{1.875989in}{5.226886in}}%
\pgfpathlineto{\pgfqpoint{1.877599in}{5.242873in}}%
\pgfpathlineto{\pgfqpoint{1.878672in}{5.216199in}}%
\pgfpathlineto{\pgfqpoint{1.879208in}{5.227844in}}%
\pgfpathlineto{\pgfqpoint{1.879744in}{5.219131in}}%
\pgfpathlineto{\pgfqpoint{1.880281in}{5.202248in}}%
\pgfpathlineto{\pgfqpoint{1.881890in}{5.232263in}}%
\pgfpathlineto{\pgfqpoint{1.882427in}{5.221182in}}%
\pgfpathlineto{\pgfqpoint{1.883499in}{5.221804in}}%
\pgfpathlineto{\pgfqpoint{1.884036in}{5.245441in}}%
\pgfpathlineto{\pgfqpoint{1.884572in}{5.227552in}}%
\pgfpathlineto{\pgfqpoint{1.885645in}{5.221717in}}%
\pgfpathlineto{\pgfqpoint{1.886718in}{5.232934in}}%
\pgfpathlineto{\pgfqpoint{1.888327in}{5.207115in}}%
\pgfpathlineto{\pgfqpoint{1.889400in}{5.228350in}}%
\pgfpathlineto{\pgfqpoint{1.889937in}{5.222999in}}%
\pgfpathlineto{\pgfqpoint{1.890473in}{5.225290in}}%
\pgfpathlineto{\pgfqpoint{1.891010in}{5.247933in}}%
\pgfpathlineto{\pgfqpoint{1.891546in}{5.214710in}}%
\pgfpathlineto{\pgfqpoint{1.892082in}{5.240619in}}%
\pgfpathlineto{\pgfqpoint{1.894765in}{5.204821in}}%
\pgfpathlineto{\pgfqpoint{1.896374in}{5.220376in}}%
\pgfpathlineto{\pgfqpoint{1.896910in}{5.226296in}}%
\pgfpathlineto{\pgfqpoint{1.897447in}{5.204239in}}%
\pgfpathlineto{\pgfqpoint{1.897983in}{5.217231in}}%
\pgfpathlineto{\pgfqpoint{1.899056in}{5.235817in}}%
\pgfpathlineto{\pgfqpoint{1.899593in}{5.219609in}}%
\pgfpathlineto{\pgfqpoint{1.900665in}{5.220012in}}%
\pgfpathlineto{\pgfqpoint{1.901202in}{5.230534in}}%
\pgfpathlineto{\pgfqpoint{1.902811in}{5.206333in}}%
\pgfpathlineto{\pgfqpoint{1.903348in}{5.205423in}}%
\pgfpathlineto{\pgfqpoint{1.905493in}{5.236415in}}%
\pgfpathlineto{\pgfqpoint{1.906030in}{5.210013in}}%
\pgfpathlineto{\pgfqpoint{1.906566in}{5.216077in}}%
\pgfpathlineto{\pgfqpoint{1.907103in}{5.222413in}}%
\pgfpathlineto{\pgfqpoint{1.907639in}{5.208756in}}%
\pgfpathlineto{\pgfqpoint{1.908176in}{5.228791in}}%
\pgfpathlineto{\pgfqpoint{1.908712in}{5.215801in}}%
\pgfpathlineto{\pgfqpoint{1.909785in}{5.200080in}}%
\pgfpathlineto{\pgfqpoint{1.911394in}{5.214179in}}%
\pgfpathlineto{\pgfqpoint{1.911931in}{5.188708in}}%
\pgfpathlineto{\pgfqpoint{1.912467in}{5.206788in}}%
\pgfpathlineto{\pgfqpoint{1.913003in}{5.227919in}}%
\pgfpathlineto{\pgfqpoint{1.913540in}{5.212490in}}%
\pgfpathlineto{\pgfqpoint{1.914613in}{5.226235in}}%
\pgfpathlineto{\pgfqpoint{1.915149in}{5.224355in}}%
\pgfpathlineto{\pgfqpoint{1.916222in}{5.193829in}}%
\pgfpathlineto{\pgfqpoint{1.916759in}{5.213063in}}%
\pgfpathlineto{\pgfqpoint{1.917295in}{5.224979in}}%
\pgfpathlineto{\pgfqpoint{1.917831in}{5.197479in}}%
\pgfpathlineto{\pgfqpoint{1.918368in}{5.213954in}}%
\pgfpathlineto{\pgfqpoint{1.918904in}{5.201900in}}%
\pgfpathlineto{\pgfqpoint{1.919441in}{5.210030in}}%
\pgfpathlineto{\pgfqpoint{1.919977in}{5.215032in}}%
\pgfpathlineto{\pgfqpoint{1.920514in}{5.196010in}}%
\pgfpathlineto{\pgfqpoint{1.921050in}{5.213777in}}%
\pgfpathlineto{\pgfqpoint{1.922123in}{5.190193in}}%
\pgfpathlineto{\pgfqpoint{1.923196in}{5.220097in}}%
\pgfpathlineto{\pgfqpoint{1.923732in}{5.188075in}}%
\pgfpathlineto{\pgfqpoint{1.924269in}{5.192382in}}%
\pgfpathlineto{\pgfqpoint{1.924805in}{5.190943in}}%
\pgfpathlineto{\pgfqpoint{1.925341in}{5.185028in}}%
\pgfpathlineto{\pgfqpoint{1.925878in}{5.194420in}}%
\pgfpathlineto{\pgfqpoint{1.926414in}{5.166839in}}%
\pgfpathlineto{\pgfqpoint{1.927487in}{5.219075in}}%
\pgfpathlineto{\pgfqpoint{1.928024in}{5.177004in}}%
\pgfpathlineto{\pgfqpoint{1.928560in}{5.188608in}}%
\pgfpathlineto{\pgfqpoint{1.929097in}{5.218475in}}%
\pgfpathlineto{\pgfqpoint{1.929633in}{5.209360in}}%
\pgfpathlineto{\pgfqpoint{1.930706in}{5.176796in}}%
\pgfpathlineto{\pgfqpoint{1.931242in}{5.216353in}}%
\pgfpathlineto{\pgfqpoint{1.931779in}{5.211959in}}%
\pgfpathlineto{\pgfqpoint{1.932852in}{5.192779in}}%
\pgfpathlineto{\pgfqpoint{1.933388in}{5.207571in}}%
\pgfpathlineto{\pgfqpoint{1.934461in}{5.174621in}}%
\pgfpathlineto{\pgfqpoint{1.935534in}{5.201973in}}%
\pgfpathlineto{\pgfqpoint{1.936070in}{5.201221in}}%
\pgfpathlineto{\pgfqpoint{1.936607in}{5.167465in}}%
\pgfpathlineto{\pgfqpoint{1.937143in}{5.198912in}}%
\pgfpathlineto{\pgfqpoint{1.937680in}{5.200080in}}%
\pgfpathlineto{\pgfqpoint{1.938216in}{5.168315in}}%
\pgfpathlineto{\pgfqpoint{1.938752in}{5.173117in}}%
\pgfpathlineto{\pgfqpoint{1.939289in}{5.184250in}}%
\pgfpathlineto{\pgfqpoint{1.939825in}{5.179420in}}%
\pgfpathlineto{\pgfqpoint{1.940898in}{5.147679in}}%
\pgfpathlineto{\pgfqpoint{1.941971in}{5.196030in}}%
\pgfpathlineto{\pgfqpoint{1.942507in}{5.150697in}}%
\pgfpathlineto{\pgfqpoint{1.943044in}{5.157836in}}%
\pgfpathlineto{\pgfqpoint{1.943580in}{5.197123in}}%
\pgfpathlineto{\pgfqpoint{1.944117in}{5.186831in}}%
\pgfpathlineto{\pgfqpoint{1.944653in}{5.164648in}}%
\pgfpathlineto{\pgfqpoint{1.945726in}{5.197248in}}%
\pgfpathlineto{\pgfqpoint{1.946263in}{5.194408in}}%
\pgfpathlineto{\pgfqpoint{1.946799in}{5.149890in}}%
\pgfpathlineto{\pgfqpoint{1.947335in}{5.166475in}}%
\pgfpathlineto{\pgfqpoint{1.947872in}{5.195981in}}%
\pgfpathlineto{\pgfqpoint{1.948408in}{5.169128in}}%
\pgfpathlineto{\pgfqpoint{1.948945in}{5.141052in}}%
\pgfpathlineto{\pgfqpoint{1.950018in}{5.192811in}}%
\pgfpathlineto{\pgfqpoint{1.951090in}{5.150198in}}%
\pgfpathlineto{\pgfqpoint{1.952163in}{5.183845in}}%
\pgfpathlineto{\pgfqpoint{1.953236in}{5.135409in}}%
\pgfpathlineto{\pgfqpoint{1.954309in}{5.176806in}}%
\pgfpathlineto{\pgfqpoint{1.955382in}{5.115194in}}%
\pgfpathlineto{\pgfqpoint{1.955918in}{5.151056in}}%
\pgfpathlineto{\pgfqpoint{1.956455in}{5.173653in}}%
\pgfpathlineto{\pgfqpoint{1.956991in}{5.116201in}}%
\pgfpathlineto{\pgfqpoint{1.957528in}{5.140674in}}%
\pgfpathlineto{\pgfqpoint{1.958601in}{5.188190in}}%
\pgfpathlineto{\pgfqpoint{1.959137in}{5.130379in}}%
\pgfpathlineto{\pgfqpoint{1.959673in}{5.156829in}}%
\pgfpathlineto{\pgfqpoint{1.960210in}{5.182353in}}%
\pgfpathlineto{\pgfqpoint{1.960746in}{5.180465in}}%
\pgfpathlineto{\pgfqpoint{1.961819in}{5.115810in}}%
\pgfpathlineto{\pgfqpoint{1.962356in}{5.211946in}}%
\pgfpathlineto{\pgfqpoint{1.962892in}{5.160143in}}%
\pgfpathlineto{\pgfqpoint{1.963428in}{5.101053in}}%
\pgfpathlineto{\pgfqpoint{1.963965in}{5.149891in}}%
\pgfpathlineto{\pgfqpoint{1.964501in}{5.155399in}}%
\pgfpathlineto{\pgfqpoint{1.965038in}{5.145288in}}%
\pgfpathlineto{\pgfqpoint{1.965574in}{5.075522in}}%
\pgfpathlineto{\pgfqpoint{1.966111in}{5.108620in}}%
\pgfpathlineto{\pgfqpoint{1.966647in}{5.148595in}}%
\pgfpathlineto{\pgfqpoint{1.967184in}{5.148149in}}%
\pgfpathlineto{\pgfqpoint{1.967720in}{5.093280in}}%
\pgfpathlineto{\pgfqpoint{1.968256in}{5.105509in}}%
\pgfpathlineto{\pgfqpoint{1.968793in}{5.173826in}}%
\pgfpathlineto{\pgfqpoint{1.969329in}{5.122781in}}%
\pgfpathlineto{\pgfqpoint{1.969866in}{5.095744in}}%
\pgfpathlineto{\pgfqpoint{1.970939in}{5.170924in}}%
\pgfpathlineto{\pgfqpoint{1.972011in}{5.087777in}}%
\pgfpathlineto{\pgfqpoint{1.973084in}{5.170703in}}%
\pgfpathlineto{\pgfqpoint{1.974157in}{5.064690in}}%
\pgfpathlineto{\pgfqpoint{1.975230in}{5.170091in}}%
\pgfpathlineto{\pgfqpoint{1.976303in}{5.063183in}}%
\pgfpathlineto{\pgfqpoint{1.976839in}{5.165147in}}%
\pgfpathlineto{\pgfqpoint{1.977376in}{5.147983in}}%
\pgfpathlineto{\pgfqpoint{1.977912in}{5.058133in}}%
\pgfpathlineto{\pgfqpoint{1.978449in}{5.071298in}}%
\pgfpathlineto{\pgfqpoint{1.978985in}{5.129708in}}%
\pgfpathlineto{\pgfqpoint{1.979522in}{5.127203in}}%
\pgfpathlineto{\pgfqpoint{1.980594in}{5.052432in}}%
\pgfpathlineto{\pgfqpoint{1.981667in}{5.131292in}}%
\pgfpathlineto{\pgfqpoint{1.982204in}{5.053111in}}%
\pgfpathlineto{\pgfqpoint{1.982740in}{5.057106in}}%
\pgfpathlineto{\pgfqpoint{1.983277in}{5.170736in}}%
\pgfpathlineto{\pgfqpoint{1.983813in}{5.107285in}}%
\pgfpathlineto{\pgfqpoint{1.984349in}{5.041198in}}%
\pgfpathlineto{\pgfqpoint{1.984886in}{5.093832in}}%
\pgfpathlineto{\pgfqpoint{1.985959in}{5.164993in}}%
\pgfpathlineto{\pgfqpoint{1.986495in}{5.003305in}}%
\pgfpathlineto{\pgfqpoint{1.987032in}{5.060408in}}%
\pgfpathlineto{\pgfqpoint{1.987568in}{5.175987in}}%
\pgfpathlineto{\pgfqpoint{1.988105in}{5.113587in}}%
\pgfpathlineto{\pgfqpoint{1.988641in}{5.003230in}}%
\pgfpathlineto{\pgfqpoint{1.989177in}{5.056474in}}%
\pgfpathlineto{\pgfqpoint{1.989714in}{5.163310in}}%
\pgfpathlineto{\pgfqpoint{1.990250in}{5.067698in}}%
\pgfpathlineto{\pgfqpoint{1.990787in}{5.016088in}}%
\pgfpathlineto{\pgfqpoint{1.991323in}{5.053387in}}%
\pgfpathlineto{\pgfqpoint{1.991860in}{5.107728in}}%
\pgfpathlineto{\pgfqpoint{1.992396in}{5.087723in}}%
\pgfpathlineto{\pgfqpoint{1.992932in}{4.943911in}}%
\pgfpathlineto{\pgfqpoint{1.993469in}{5.076705in}}%
\pgfpathlineto{\pgfqpoint{1.994005in}{5.116263in}}%
\pgfpathlineto{\pgfqpoint{1.994542in}{5.077218in}}%
\pgfpathlineto{\pgfqpoint{1.995078in}{4.990372in}}%
\pgfpathlineto{\pgfqpoint{1.996151in}{5.141582in}}%
\pgfpathlineto{\pgfqpoint{1.997224in}{4.999714in}}%
\pgfpathlineto{\pgfqpoint{1.998297in}{5.169818in}}%
\pgfpathlineto{\pgfqpoint{1.999370in}{4.971611in}}%
\pgfpathlineto{\pgfqpoint{2.000443in}{5.159842in}}%
\pgfpathlineto{\pgfqpoint{2.001515in}{4.909167in}}%
\pgfpathlineto{\pgfqpoint{2.002588in}{5.108946in}}%
\pgfpathlineto{\pgfqpoint{2.003661in}{4.918228in}}%
\pgfpathlineto{\pgfqpoint{2.004198in}{5.084091in}}%
\pgfpathlineto{\pgfqpoint{2.004734in}{5.064034in}}%
\pgfpathlineto{\pgfqpoint{2.005807in}{4.906509in}}%
\pgfpathlineto{\pgfqpoint{2.006880in}{5.121370in}}%
\pgfpathlineto{\pgfqpoint{2.007416in}{4.936552in}}%
\pgfpathlineto{\pgfqpoint{2.007953in}{4.987685in}}%
\pgfpathlineto{\pgfqpoint{2.009026in}{5.121899in}}%
\pgfpathlineto{\pgfqpoint{2.010098in}{4.943924in}}%
\pgfpathlineto{\pgfqpoint{2.011171in}{5.131818in}}%
\pgfpathlineto{\pgfqpoint{2.012244in}{4.837040in}}%
\pgfpathlineto{\pgfqpoint{2.012781in}{5.105025in}}%
\pgfpathlineto{\pgfqpoint{2.013317in}{5.088409in}}%
\pgfpathlineto{\pgfqpoint{2.014390in}{4.844301in}}%
\pgfpathlineto{\pgfqpoint{2.015463in}{5.057295in}}%
\pgfpathlineto{\pgfqpoint{2.016536in}{4.880386in}}%
\pgfpathlineto{\pgfqpoint{2.017609in}{5.097410in}}%
\pgfpathlineto{\pgfqpoint{2.018145in}{4.880479in}}%
\pgfpathlineto{\pgfqpoint{2.018681in}{4.938491in}}%
\pgfpathlineto{\pgfqpoint{2.019218in}{5.088792in}}%
\pgfpathlineto{\pgfqpoint{2.019754in}{5.080589in}}%
\pgfpathlineto{\pgfqpoint{2.020291in}{4.892726in}}%
\pgfpathlineto{\pgfqpoint{2.020827in}{4.918438in}}%
\pgfpathlineto{\pgfqpoint{2.021900in}{5.092330in}}%
\pgfpathlineto{\pgfqpoint{2.022973in}{4.792918in}}%
\pgfpathlineto{\pgfqpoint{2.023509in}{5.103697in}}%
\pgfpathlineto{\pgfqpoint{2.024046in}{5.047166in}}%
\pgfpathlineto{\pgfqpoint{2.025119in}{4.796228in}}%
\pgfpathlineto{\pgfqpoint{2.026191in}{5.067030in}}%
\pgfpathlineto{\pgfqpoint{2.027264in}{4.814572in}}%
\pgfpathlineto{\pgfqpoint{2.028337in}{5.058977in}}%
\pgfpathlineto{\pgfqpoint{2.029410in}{4.813704in}}%
\pgfpathlineto{\pgfqpoint{2.029947in}{5.060453in}}%
\pgfpathlineto{\pgfqpoint{2.030483in}{5.029685in}}%
\pgfpathlineto{\pgfqpoint{2.031556in}{4.858038in}}%
\pgfpathlineto{\pgfqpoint{2.032092in}{5.074842in}}%
\pgfpathlineto{\pgfqpoint{2.032629in}{5.026328in}}%
\pgfpathlineto{\pgfqpoint{2.033702in}{4.810294in}}%
\pgfpathlineto{\pgfqpoint{2.034238in}{5.087788in}}%
\pgfpathlineto{\pgfqpoint{2.034774in}{5.039200in}}%
\pgfpathlineto{\pgfqpoint{2.035847in}{4.660758in}}%
\pgfpathlineto{\pgfqpoint{2.036920in}{5.052611in}}%
\pgfpathlineto{\pgfqpoint{2.037993in}{4.691972in}}%
\pgfpathlineto{\pgfqpoint{2.039066in}{5.049869in}}%
\pgfpathlineto{\pgfqpoint{2.040139in}{4.676722in}}%
\pgfpathlineto{\pgfqpoint{2.041212in}{5.040236in}}%
\pgfpathlineto{\pgfqpoint{2.042285in}{4.725496in}}%
\pgfpathlineto{\pgfqpoint{2.042821in}{5.061540in}}%
\pgfpathlineto{\pgfqpoint{2.043357in}{5.002304in}}%
\pgfpathlineto{\pgfqpoint{2.044430in}{4.760965in}}%
\pgfpathlineto{\pgfqpoint{2.044967in}{5.042339in}}%
\pgfpathlineto{\pgfqpoint{2.045503in}{5.013607in}}%
\pgfpathlineto{\pgfqpoint{2.046576in}{4.518706in}}%
\pgfpathlineto{\pgfqpoint{2.047649in}{5.017364in}}%
\pgfpathlineto{\pgfqpoint{2.048722in}{4.424452in}}%
\pgfpathlineto{\pgfqpoint{2.049795in}{5.041535in}}%
\pgfpathlineto{\pgfqpoint{2.050868in}{4.374668in}}%
\pgfpathlineto{\pgfqpoint{2.051940in}{5.061495in}}%
\pgfpathlineto{\pgfqpoint{2.052477in}{4.871273in}}%
\pgfpathlineto{\pgfqpoint{2.053013in}{4.594984in}}%
\pgfpathlineto{\pgfqpoint{2.054086in}{4.967767in}}%
\pgfpathlineto{\pgfqpoint{2.054623in}{4.897778in}}%
\pgfpathlineto{\pgfqpoint{2.055159in}{4.631336in}}%
\pgfpathlineto{\pgfqpoint{2.055695in}{4.980547in}}%
\pgfpathlineto{\pgfqpoint{2.056232in}{4.951231in}}%
\pgfpathlineto{\pgfqpoint{2.057305in}{4.458848in}}%
\pgfpathlineto{\pgfqpoint{2.058378in}{4.977266in}}%
\pgfpathlineto{\pgfqpoint{2.059451in}{4.076630in}}%
\pgfpathlineto{\pgfqpoint{2.060523in}{5.029276in}}%
\pgfpathlineto{\pgfqpoint{2.061060in}{4.861539in}}%
\pgfpathlineto{\pgfqpoint{2.061596in}{4.227272in}}%
\pgfpathlineto{\pgfqpoint{2.062133in}{4.677050in}}%
\pgfpathlineto{\pgfqpoint{2.062669in}{5.058371in}}%
\pgfpathlineto{\pgfqpoint{2.063206in}{4.912296in}}%
\pgfpathlineto{\pgfqpoint{2.063742in}{4.329063in}}%
\pgfpathlineto{\pgfqpoint{2.064278in}{4.738438in}}%
\pgfpathlineto{\pgfqpoint{2.065351in}{4.977025in}}%
\pgfpathlineto{\pgfqpoint{2.065888in}{4.560116in}}%
\pgfpathlineto{\pgfqpoint{2.066424in}{4.842677in}}%
\pgfpathlineto{\pgfqpoint{2.066961in}{4.917775in}}%
\pgfpathlineto{\pgfqpoint{2.067497in}{4.849365in}}%
\pgfpathlineto{\pgfqpoint{2.068034in}{4.493291in}}%
\pgfpathlineto{\pgfqpoint{2.068570in}{4.798214in}}%
\pgfpathlineto{\pgfqpoint{2.069106in}{4.924583in}}%
\pgfpathlineto{\pgfqpoint{2.070179in}{4.475812in}}%
\pgfpathlineto{\pgfqpoint{2.070716in}{4.566792in}}%
\pgfpathlineto{\pgfqpoint{2.071252in}{4.985894in}}%
\pgfpathlineto{\pgfqpoint{2.071789in}{4.885461in}}%
\pgfpathlineto{\pgfqpoint{2.072325in}{4.062542in}}%
\pgfpathlineto{\pgfqpoint{2.072861in}{4.445788in}}%
\pgfpathlineto{\pgfqpoint{2.073398in}{5.035089in}}%
\pgfpathlineto{\pgfqpoint{2.073934in}{4.990350in}}%
\pgfpathlineto{\pgfqpoint{2.075007in}{4.248712in}}%
\pgfpathlineto{\pgfqpoint{2.076080in}{5.020969in}}%
\pgfpathlineto{\pgfqpoint{2.076616in}{4.537557in}}%
\pgfpathlineto{\pgfqpoint{2.077153in}{4.550069in}}%
\pgfpathlineto{\pgfqpoint{2.078226in}{4.867414in}}%
\pgfpathlineto{\pgfqpoint{2.079299in}{4.634277in}}%
\pgfpathlineto{\pgfqpoint{2.079835in}{4.852615in}}%
\pgfpathlineto{\pgfqpoint{2.080372in}{4.819475in}}%
\pgfpathlineto{\pgfqpoint{2.081444in}{4.472210in}}%
\pgfpathlineto{\pgfqpoint{2.081981in}{4.862499in}}%
\pgfpathlineto{\pgfqpoint{2.082517in}{4.775797in}}%
\pgfpathlineto{\pgfqpoint{2.083590in}{4.267037in}}%
\pgfpathlineto{\pgfqpoint{2.084663in}{5.089345in}}%
\pgfpathlineto{\pgfqpoint{2.085736in}{3.782447in}}%
\pgfpathlineto{\pgfqpoint{2.086809in}{5.006345in}}%
\pgfpathlineto{\pgfqpoint{2.087345in}{4.850139in}}%
\pgfpathlineto{\pgfqpoint{2.087882in}{4.317533in}}%
\pgfpathlineto{\pgfqpoint{2.088418in}{4.337822in}}%
\pgfpathlineto{\pgfqpoint{2.088955in}{4.888589in}}%
\pgfpathlineto{\pgfqpoint{2.089491in}{4.823232in}}%
\pgfpathlineto{\pgfqpoint{2.090027in}{4.377144in}}%
\pgfpathlineto{\pgfqpoint{2.090564in}{4.756685in}}%
\pgfpathlineto{\pgfqpoint{2.091100in}{4.590159in}}%
\pgfpathlineto{\pgfqpoint{2.091637in}{4.953478in}}%
\pgfpathlineto{\pgfqpoint{2.092173in}{4.603639in}}%
\pgfpathlineto{\pgfqpoint{2.092710in}{4.609948in}}%
\pgfpathlineto{\pgfqpoint{2.093246in}{4.870914in}}%
\pgfpathlineto{\pgfqpoint{2.093782in}{4.653930in}}%
\pgfpathlineto{\pgfqpoint{2.094319in}{4.541119in}}%
\pgfpathlineto{\pgfqpoint{2.095392in}{4.924548in}}%
\pgfpathlineto{\pgfqpoint{2.095928in}{4.775265in}}%
\pgfpathlineto{\pgfqpoint{2.097001in}{4.200717in}}%
\pgfpathlineto{\pgfqpoint{2.097537in}{5.041126in}}%
\pgfpathlineto{\pgfqpoint{2.098074in}{4.987307in}}%
\pgfpathlineto{\pgfqpoint{2.098610in}{4.007573in}}%
\pgfpathlineto{\pgfqpoint{2.099147in}{4.261383in}}%
\pgfpathlineto{\pgfqpoint{2.100220in}{4.970177in}}%
\pgfpathlineto{\pgfqpoint{2.100756in}{4.639778in}}%
\pgfpathlineto{\pgfqpoint{2.101293in}{4.415354in}}%
\pgfpathlineto{\pgfqpoint{2.101829in}{4.423948in}}%
\pgfpathlineto{\pgfqpoint{2.102365in}{4.923184in}}%
\pgfpathlineto{\pgfqpoint{2.102902in}{4.747231in}}%
\pgfpathlineto{\pgfqpoint{2.103438in}{4.782813in}}%
\pgfpathlineto{\pgfqpoint{2.103975in}{4.711021in}}%
\pgfpathlineto{\pgfqpoint{2.104511in}{4.773380in}}%
\pgfpathlineto{\pgfqpoint{2.105048in}{4.859745in}}%
\pgfpathlineto{\pgfqpoint{2.105584in}{4.601928in}}%
\pgfpathlineto{\pgfqpoint{2.106120in}{4.764991in}}%
\pgfpathlineto{\pgfqpoint{2.106657in}{4.899447in}}%
\pgfpathlineto{\pgfqpoint{2.107193in}{4.409312in}}%
\pgfpathlineto{\pgfqpoint{2.107730in}{4.558176in}}%
\pgfpathlineto{\pgfqpoint{2.108803in}{4.916423in}}%
\pgfpathlineto{\pgfqpoint{2.109339in}{4.767675in}}%
\pgfpathlineto{\pgfqpoint{2.109876in}{4.007146in}}%
\pgfpathlineto{\pgfqpoint{2.110412in}{4.648308in}}%
\pgfpathlineto{\pgfqpoint{2.110948in}{4.981338in}}%
\pgfpathlineto{\pgfqpoint{2.111485in}{4.820261in}}%
\pgfpathlineto{\pgfqpoint{2.112558in}{4.366750in}}%
\pgfpathlineto{\pgfqpoint{2.113631in}{5.009310in}}%
\pgfpathlineto{\pgfqpoint{2.114167in}{4.793169in}}%
\pgfpathlineto{\pgfqpoint{2.114703in}{4.457858in}}%
\pgfpathlineto{\pgfqpoint{2.115240in}{4.748285in}}%
\pgfpathlineto{\pgfqpoint{2.116313in}{4.899254in}}%
\pgfpathlineto{\pgfqpoint{2.117922in}{4.728302in}}%
\pgfpathlineto{\pgfqpoint{2.118459in}{4.758121in}}%
\pgfpathlineto{\pgfqpoint{2.118995in}{4.581652in}}%
\pgfpathlineto{\pgfqpoint{2.119531in}{4.932346in}}%
\pgfpathlineto{\pgfqpoint{2.120068in}{4.875470in}}%
\pgfpathlineto{\pgfqpoint{2.120604in}{4.579164in}}%
\pgfpathlineto{\pgfqpoint{2.121141in}{4.662693in}}%
\pgfpathlineto{\pgfqpoint{2.122214in}{4.939671in}}%
\pgfpathlineto{\pgfqpoint{2.123286in}{4.486692in}}%
\pgfpathlineto{\pgfqpoint{2.123823in}{4.618768in}}%
\pgfpathlineto{\pgfqpoint{2.124359in}{4.981547in}}%
\pgfpathlineto{\pgfqpoint{2.124896in}{4.839684in}}%
\pgfpathlineto{\pgfqpoint{2.125432in}{4.778073in}}%
\pgfpathlineto{\pgfqpoint{2.125969in}{4.580217in}}%
\pgfpathlineto{\pgfqpoint{2.127041in}{5.000877in}}%
\pgfpathlineto{\pgfqpoint{2.127578in}{4.834602in}}%
\pgfpathlineto{\pgfqpoint{2.128114in}{4.572875in}}%
\pgfpathlineto{\pgfqpoint{2.128651in}{4.807785in}}%
\pgfpathlineto{\pgfqpoint{2.129187in}{4.770562in}}%
\pgfpathlineto{\pgfqpoint{2.129724in}{4.871110in}}%
\pgfpathlineto{\pgfqpoint{2.130260in}{4.775665in}}%
\pgfpathlineto{\pgfqpoint{2.130797in}{4.713241in}}%
\pgfpathlineto{\pgfqpoint{2.131333in}{4.817210in}}%
\pgfpathlineto{\pgfqpoint{2.131869in}{4.772623in}}%
\pgfpathlineto{\pgfqpoint{2.132406in}{4.804918in}}%
\pgfpathlineto{\pgfqpoint{2.132942in}{4.990856in}}%
\pgfpathlineto{\pgfqpoint{2.133479in}{4.933678in}}%
\pgfpathlineto{\pgfqpoint{2.134015in}{4.742833in}}%
\pgfpathlineto{\pgfqpoint{2.134552in}{4.784485in}}%
\pgfpathlineto{\pgfqpoint{2.135088in}{4.774386in}}%
\pgfpathlineto{\pgfqpoint{2.135624in}{4.955288in}}%
\pgfpathlineto{\pgfqpoint{2.136161in}{4.892053in}}%
\pgfpathlineto{\pgfqpoint{2.136697in}{4.557283in}}%
\pgfpathlineto{\pgfqpoint{2.137234in}{4.668399in}}%
\pgfpathlineto{\pgfqpoint{2.138307in}{5.004194in}}%
\pgfpathlineto{\pgfqpoint{2.139380in}{4.714497in}}%
\pgfpathlineto{\pgfqpoint{2.139916in}{4.855172in}}%
\pgfpathlineto{\pgfqpoint{2.140452in}{4.988216in}}%
\pgfpathlineto{\pgfqpoint{2.140989in}{4.873638in}}%
\pgfpathlineto{\pgfqpoint{2.142062in}{4.794154in}}%
\pgfpathlineto{\pgfqpoint{2.142598in}{4.834957in}}%
\pgfpathlineto{\pgfqpoint{2.144207in}{4.887489in}}%
\pgfpathlineto{\pgfqpoint{2.144744in}{4.906018in}}%
\pgfpathlineto{\pgfqpoint{2.145280in}{4.859301in}}%
\pgfpathlineto{\pgfqpoint{2.145817in}{4.888532in}}%
\pgfpathlineto{\pgfqpoint{2.146353in}{4.889178in}}%
\pgfpathlineto{\pgfqpoint{2.146890in}{4.943996in}}%
\pgfpathlineto{\pgfqpoint{2.147962in}{4.800937in}}%
\pgfpathlineto{\pgfqpoint{2.148499in}{4.819120in}}%
\pgfpathlineto{\pgfqpoint{2.149572in}{4.981434in}}%
\pgfpathlineto{\pgfqpoint{2.150108in}{4.820700in}}%
\pgfpathlineto{\pgfqpoint{2.150645in}{4.835469in}}%
\pgfpathlineto{\pgfqpoint{2.151718in}{4.966180in}}%
\pgfpathlineto{\pgfqpoint{2.153327in}{4.848836in}}%
\pgfpathlineto{\pgfqpoint{2.153863in}{4.941153in}}%
\pgfpathlineto{\pgfqpoint{2.154936in}{4.939545in}}%
\pgfpathlineto{\pgfqpoint{2.156009in}{4.903535in}}%
\pgfpathlineto{\pgfqpoint{2.156545in}{4.951791in}}%
\pgfpathlineto{\pgfqpoint{2.157082in}{4.936679in}}%
\pgfpathlineto{\pgfqpoint{2.157618in}{4.907759in}}%
\pgfpathlineto{\pgfqpoint{2.158155in}{4.970835in}}%
\pgfpathlineto{\pgfqpoint{2.158691in}{4.873175in}}%
\pgfpathlineto{\pgfqpoint{2.159228in}{4.875316in}}%
\pgfpathlineto{\pgfqpoint{2.160837in}{4.985775in}}%
\pgfpathlineto{\pgfqpoint{2.161373in}{4.889322in}}%
\pgfpathlineto{\pgfqpoint{2.161910in}{4.903558in}}%
\pgfpathlineto{\pgfqpoint{2.162983in}{5.021784in}}%
\pgfpathlineto{\pgfqpoint{2.164056in}{4.885601in}}%
\pgfpathlineto{\pgfqpoint{2.164592in}{4.890972in}}%
\pgfpathlineto{\pgfqpoint{2.165665in}{5.006559in}}%
\pgfpathlineto{\pgfqpoint{2.166738in}{4.894805in}}%
\pgfpathlineto{\pgfqpoint{2.167811in}{5.021562in}}%
\pgfpathlineto{\pgfqpoint{2.168347in}{5.010658in}}%
\pgfpathlineto{\pgfqpoint{2.169956in}{4.944343in}}%
\pgfpathlineto{\pgfqpoint{2.172102in}{5.000851in}}%
\pgfpathlineto{\pgfqpoint{2.173175in}{4.939548in}}%
\pgfpathlineto{\pgfqpoint{2.174784in}{5.017458in}}%
\pgfpathlineto{\pgfqpoint{2.175321in}{4.945912in}}%
\pgfpathlineto{\pgfqpoint{2.175857in}{4.952352in}}%
\pgfpathlineto{\pgfqpoint{2.176394in}{5.032518in}}%
\pgfpathlineto{\pgfqpoint{2.176930in}{5.019167in}}%
\pgfpathlineto{\pgfqpoint{2.177466in}{5.009076in}}%
\pgfpathlineto{\pgfqpoint{2.178003in}{4.968796in}}%
\pgfpathlineto{\pgfqpoint{2.178003in}{4.968796in}}%
\pgfpathlineto{\pgfqpoint{2.178003in}{4.968796in}}%
\pgfpathlineto{\pgfqpoint{2.179076in}{5.041373in}}%
\pgfpathlineto{\pgfqpoint{2.179612in}{5.014771in}}%
\pgfpathlineto{\pgfqpoint{2.180149in}{5.004515in}}%
\pgfpathlineto{\pgfqpoint{2.180685in}{5.025225in}}%
\pgfpathlineto{\pgfqpoint{2.181222in}{4.996051in}}%
\pgfpathlineto{\pgfqpoint{2.181758in}{5.030190in}}%
\pgfpathlineto{\pgfqpoint{2.182294in}{5.008414in}}%
\pgfpathlineto{\pgfqpoint{2.183367in}{4.994674in}}%
\pgfpathlineto{\pgfqpoint{2.184977in}{5.044030in}}%
\pgfpathlineto{\pgfqpoint{2.186049in}{5.032108in}}%
\pgfpathlineto{\pgfqpoint{2.186586in}{4.999449in}}%
\pgfpathlineto{\pgfqpoint{2.187122in}{5.016458in}}%
\pgfpathlineto{\pgfqpoint{2.187659in}{5.070370in}}%
\pgfpathlineto{\pgfqpoint{2.188195in}{5.033959in}}%
\pgfpathlineto{\pgfqpoint{2.189268in}{5.016420in}}%
\pgfpathlineto{\pgfqpoint{2.190341in}{5.082614in}}%
\pgfpathlineto{\pgfqpoint{2.190877in}{5.047310in}}%
\pgfpathlineto{\pgfqpoint{2.191414in}{5.018375in}}%
\pgfpathlineto{\pgfqpoint{2.191950in}{5.030111in}}%
\pgfpathlineto{\pgfqpoint{2.192487in}{5.027048in}}%
\pgfpathlineto{\pgfqpoint{2.193023in}{5.079203in}}%
\pgfpathlineto{\pgfqpoint{2.193560in}{5.050479in}}%
\pgfpathlineto{\pgfqpoint{2.195705in}{5.078895in}}%
\pgfpathlineto{\pgfqpoint{2.196242in}{5.057528in}}%
\pgfpathlineto{\pgfqpoint{2.196778in}{5.060833in}}%
\pgfpathlineto{\pgfqpoint{2.197315in}{5.080699in}}%
\pgfpathlineto{\pgfqpoint{2.197851in}{5.036355in}}%
\pgfpathlineto{\pgfqpoint{2.198387in}{5.064250in}}%
\pgfpathlineto{\pgfqpoint{2.198924in}{5.067916in}}%
\pgfpathlineto{\pgfqpoint{2.199460in}{5.081228in}}%
\pgfpathlineto{\pgfqpoint{2.201070in}{5.053647in}}%
\pgfpathlineto{\pgfqpoint{2.201606in}{5.098439in}}%
\pgfpathlineto{\pgfqpoint{2.202143in}{5.095333in}}%
\pgfpathlineto{\pgfqpoint{2.203215in}{5.041814in}}%
\pgfpathlineto{\pgfqpoint{2.204288in}{5.094366in}}%
\pgfpathlineto{\pgfqpoint{2.204825in}{5.082143in}}%
\pgfpathlineto{\pgfqpoint{2.205361in}{5.066582in}}%
\pgfpathlineto{\pgfqpoint{2.205898in}{5.075244in}}%
\pgfpathlineto{\pgfqpoint{2.206434in}{5.081394in}}%
\pgfpathlineto{\pgfqpoint{2.206970in}{5.123875in}}%
\pgfpathlineto{\pgfqpoint{2.207507in}{5.086016in}}%
\pgfpathlineto{\pgfqpoint{2.208580in}{5.087900in}}%
\pgfpathlineto{\pgfqpoint{2.210189in}{5.107859in}}%
\pgfpathlineto{\pgfqpoint{2.211798in}{5.096114in}}%
\pgfpathlineto{\pgfqpoint{2.213408in}{5.115300in}}%
\pgfpathlineto{\pgfqpoint{2.213944in}{5.110821in}}%
\pgfpathlineto{\pgfqpoint{2.214481in}{5.086947in}}%
\pgfpathlineto{\pgfqpoint{2.215017in}{5.105929in}}%
\pgfpathlineto{\pgfqpoint{2.215553in}{5.120730in}}%
\pgfpathlineto{\pgfqpoint{2.217163in}{5.095141in}}%
\pgfpathlineto{\pgfqpoint{2.217699in}{5.097266in}}%
\pgfpathlineto{\pgfqpoint{2.218236in}{5.134655in}}%
\pgfpathlineto{\pgfqpoint{2.218772in}{5.133732in}}%
\pgfpathlineto{\pgfqpoint{2.219845in}{5.090463in}}%
\pgfpathlineto{\pgfqpoint{2.220918in}{5.125461in}}%
\pgfpathlineto{\pgfqpoint{2.221454in}{5.124757in}}%
\pgfpathlineto{\pgfqpoint{2.221991in}{5.119934in}}%
\pgfpathlineto{\pgfqpoint{2.222527in}{5.127259in}}%
\pgfpathlineto{\pgfqpoint{2.223064in}{5.117600in}}%
\pgfpathlineto{\pgfqpoint{2.223600in}{5.146552in}}%
\pgfpathlineto{\pgfqpoint{2.224136in}{5.129605in}}%
\pgfpathlineto{\pgfqpoint{2.224673in}{5.136267in}}%
\pgfpathlineto{\pgfqpoint{2.226282in}{5.120478in}}%
\pgfpathlineto{\pgfqpoint{2.226819in}{5.152717in}}%
\pgfpathlineto{\pgfqpoint{2.227355in}{5.142677in}}%
\pgfpathlineto{\pgfqpoint{2.227891in}{5.149675in}}%
\pgfpathlineto{\pgfqpoint{2.228428in}{5.134235in}}%
\pgfpathlineto{\pgfqpoint{2.228428in}{5.134235in}}%
\pgfpathlineto{\pgfqpoint{2.228428in}{5.134235in}}%
\pgfpathlineto{\pgfqpoint{2.230037in}{5.156069in}}%
\pgfpathlineto{\pgfqpoint{2.231110in}{5.126911in}}%
\pgfpathlineto{\pgfqpoint{2.232183in}{5.158255in}}%
\pgfpathlineto{\pgfqpoint{2.233792in}{5.142165in}}%
\pgfpathlineto{\pgfqpoint{2.234329in}{5.138106in}}%
\pgfpathlineto{\pgfqpoint{2.235402in}{5.162474in}}%
\pgfpathlineto{\pgfqpoint{2.235938in}{5.144290in}}%
\pgfpathlineto{\pgfqpoint{2.236474in}{5.149454in}}%
\pgfpathlineto{\pgfqpoint{2.237011in}{5.153253in}}%
\pgfpathlineto{\pgfqpoint{2.237547in}{5.167886in}}%
\pgfpathlineto{\pgfqpoint{2.238084in}{5.145535in}}%
\pgfpathlineto{\pgfqpoint{2.238620in}{5.162271in}}%
\pgfpathlineto{\pgfqpoint{2.239157in}{5.157651in}}%
\pgfpathlineto{\pgfqpoint{2.239693in}{5.161430in}}%
\pgfpathlineto{\pgfqpoint{2.240766in}{5.175701in}}%
\pgfpathlineto{\pgfqpoint{2.241302in}{5.169058in}}%
\pgfpathlineto{\pgfqpoint{2.241839in}{5.169732in}}%
\pgfpathlineto{\pgfqpoint{2.242375in}{5.158869in}}%
\pgfpathlineto{\pgfqpoint{2.242912in}{5.161259in}}%
\pgfpathlineto{\pgfqpoint{2.244521in}{5.180657in}}%
\pgfpathlineto{\pgfqpoint{2.246130in}{5.168294in}}%
\pgfpathlineto{\pgfqpoint{2.246667in}{5.174542in}}%
\pgfpathlineto{\pgfqpoint{2.247203in}{5.173526in}}%
\pgfpathlineto{\pgfqpoint{2.247740in}{5.161132in}}%
\pgfpathlineto{\pgfqpoint{2.248276in}{5.176716in}}%
\pgfpathlineto{\pgfqpoint{2.248812in}{5.169001in}}%
\pgfpathlineto{\pgfqpoint{2.249349in}{5.171263in}}%
\pgfpathlineto{\pgfqpoint{2.250422in}{5.186813in}}%
\pgfpathlineto{\pgfqpoint{2.250958in}{5.165670in}}%
\pgfpathlineto{\pgfqpoint{2.251495in}{5.169756in}}%
\pgfpathlineto{\pgfqpoint{2.253640in}{5.194019in}}%
\pgfpathlineto{\pgfqpoint{2.254177in}{5.186147in}}%
\pgfpathlineto{\pgfqpoint{2.254713in}{5.175371in}}%
\pgfpathlineto{\pgfqpoint{2.255250in}{5.184920in}}%
\pgfpathlineto{\pgfqpoint{2.255786in}{5.184182in}}%
\pgfpathlineto{\pgfqpoint{2.256323in}{5.195176in}}%
\pgfpathlineto{\pgfqpoint{2.256323in}{5.195176in}}%
\pgfpathlineto{\pgfqpoint{2.256323in}{5.195176in}}%
\pgfpathlineto{\pgfqpoint{2.256859in}{5.183695in}}%
\pgfpathlineto{\pgfqpoint{2.257395in}{5.188034in}}%
\pgfpathlineto{\pgfqpoint{2.259005in}{5.199012in}}%
\pgfpathlineto{\pgfqpoint{2.261687in}{5.188120in}}%
\pgfpathlineto{\pgfqpoint{2.263296in}{5.203347in}}%
\pgfpathlineto{\pgfqpoint{2.263833in}{5.189625in}}%
\pgfpathlineto{\pgfqpoint{2.264369in}{5.192251in}}%
\pgfpathlineto{\pgfqpoint{2.264906in}{5.191234in}}%
\pgfpathlineto{\pgfqpoint{2.265442in}{5.193105in}}%
\pgfpathlineto{\pgfqpoint{2.265978in}{5.187084in}}%
\pgfpathlineto{\pgfqpoint{2.266515in}{5.207010in}}%
\pgfpathlineto{\pgfqpoint{2.267051in}{5.197748in}}%
\pgfpathlineto{\pgfqpoint{2.268124in}{5.189107in}}%
\pgfpathlineto{\pgfqpoint{2.269197in}{5.203790in}}%
\pgfpathlineto{\pgfqpoint{2.269733in}{5.198712in}}%
\pgfpathlineto{\pgfqpoint{2.270270in}{5.211273in}}%
\pgfpathlineto{\pgfqpoint{2.271343in}{5.197329in}}%
\pgfpathlineto{\pgfqpoint{2.272952in}{5.210873in}}%
\pgfpathlineto{\pgfqpoint{2.273489in}{5.199165in}}%
\pgfpathlineto{\pgfqpoint{2.274025in}{5.202159in}}%
\pgfpathlineto{\pgfqpoint{2.274561in}{5.203930in}}%
\pgfpathlineto{\pgfqpoint{2.275098in}{5.220840in}}%
\pgfpathlineto{\pgfqpoint{2.275634in}{5.217636in}}%
\pgfpathlineto{\pgfqpoint{2.277780in}{5.204189in}}%
\pgfpathlineto{\pgfqpoint{2.279389in}{5.221160in}}%
\pgfpathlineto{\pgfqpoint{2.280462in}{5.208562in}}%
\pgfpathlineto{\pgfqpoint{2.282072in}{5.219893in}}%
\pgfpathlineto{\pgfqpoint{2.282608in}{5.209739in}}%
\pgfpathlineto{\pgfqpoint{2.283144in}{5.219403in}}%
\pgfpathlineto{\pgfqpoint{2.284217in}{5.202639in}}%
\pgfpathlineto{\pgfqpoint{2.284754in}{5.210528in}}%
\pgfpathlineto{\pgfqpoint{2.285827in}{5.221331in}}%
\pgfpathlineto{\pgfqpoint{2.286363in}{5.209999in}}%
\pgfpathlineto{\pgfqpoint{2.286899in}{5.215710in}}%
\pgfpathlineto{\pgfqpoint{2.287436in}{5.225421in}}%
\pgfpathlineto{\pgfqpoint{2.287972in}{5.214707in}}%
\pgfpathlineto{\pgfqpoint{2.288509in}{5.219822in}}%
\pgfpathlineto{\pgfqpoint{2.289045in}{5.229501in}}%
\pgfpathlineto{\pgfqpoint{2.290118in}{5.214042in}}%
\pgfpathlineto{\pgfqpoint{2.290655in}{5.214374in}}%
\pgfpathlineto{\pgfqpoint{2.292264in}{5.231448in}}%
\pgfpathlineto{\pgfqpoint{2.292800in}{5.223536in}}%
\pgfpathlineto{\pgfqpoint{2.293337in}{5.226064in}}%
\pgfpathlineto{\pgfqpoint{2.293873in}{5.228580in}}%
\pgfpathlineto{\pgfqpoint{2.294946in}{5.220652in}}%
\pgfpathlineto{\pgfqpoint{2.296019in}{5.230123in}}%
\pgfpathlineto{\pgfqpoint{2.297092in}{5.217898in}}%
\pgfpathlineto{\pgfqpoint{2.298165in}{5.235142in}}%
\pgfpathlineto{\pgfqpoint{2.298701in}{5.224908in}}%
\pgfpathlineto{\pgfqpoint{2.299237in}{5.225860in}}%
\pgfpathlineto{\pgfqpoint{2.299774in}{5.229526in}}%
\pgfpathlineto{\pgfqpoint{2.300310in}{5.228676in}}%
\pgfpathlineto{\pgfqpoint{2.300847in}{5.214273in}}%
\pgfpathlineto{\pgfqpoint{2.300847in}{5.214273in}}%
\pgfpathlineto{\pgfqpoint{2.300847in}{5.214273in}}%
\pgfpathlineto{\pgfqpoint{2.301920in}{5.233975in}}%
\pgfpathlineto{\pgfqpoint{2.302993in}{5.219077in}}%
\pgfpathlineto{\pgfqpoint{2.305138in}{5.235692in}}%
\pgfpathlineto{\pgfqpoint{2.306748in}{5.223786in}}%
\pgfpathlineto{\pgfqpoint{2.308893in}{5.235890in}}%
\pgfpathlineto{\pgfqpoint{2.309430in}{5.224129in}}%
\pgfpathlineto{\pgfqpoint{2.309966in}{5.230857in}}%
\pgfpathlineto{\pgfqpoint{2.310503in}{5.236704in}}%
\pgfpathlineto{\pgfqpoint{2.311039in}{5.235800in}}%
\pgfpathlineto{\pgfqpoint{2.311576in}{5.225897in}}%
\pgfpathlineto{\pgfqpoint{2.312112in}{5.236801in}}%
\pgfpathlineto{\pgfqpoint{2.312648in}{5.230290in}}%
\pgfpathlineto{\pgfqpoint{2.313185in}{5.230720in}}%
\pgfpathlineto{\pgfqpoint{2.313721in}{5.221402in}}%
\pgfpathlineto{\pgfqpoint{2.314258in}{5.239537in}}%
\pgfpathlineto{\pgfqpoint{2.314794in}{5.237691in}}%
\pgfpathlineto{\pgfqpoint{2.316403in}{5.229017in}}%
\pgfpathlineto{\pgfqpoint{2.316940in}{5.237623in}}%
\pgfpathlineto{\pgfqpoint{2.317476in}{5.227851in}}%
\pgfpathlineto{\pgfqpoint{2.318013in}{5.237042in}}%
\pgfpathlineto{\pgfqpoint{2.318549in}{5.228818in}}%
\pgfpathlineto{\pgfqpoint{2.319086in}{5.239621in}}%
\pgfpathlineto{\pgfqpoint{2.319622in}{5.224771in}}%
\pgfpathlineto{\pgfqpoint{2.320158in}{5.232145in}}%
\pgfpathlineto{\pgfqpoint{2.321231in}{5.228834in}}%
\pgfpathlineto{\pgfqpoint{2.321768in}{5.234631in}}%
\pgfpathlineto{\pgfqpoint{2.322304in}{5.225343in}}%
\pgfpathlineto{\pgfqpoint{2.323914in}{5.235952in}}%
\pgfpathlineto{\pgfqpoint{2.325523in}{5.229342in}}%
\pgfpathlineto{\pgfqpoint{2.326059in}{5.236871in}}%
\pgfpathlineto{\pgfqpoint{2.326596in}{5.225990in}}%
\pgfpathlineto{\pgfqpoint{2.327132in}{5.230854in}}%
\pgfpathlineto{\pgfqpoint{2.328741in}{5.239139in}}%
\pgfpathlineto{\pgfqpoint{2.330351in}{5.222880in}}%
\pgfpathlineto{\pgfqpoint{2.330887in}{5.239906in}}%
\pgfpathlineto{\pgfqpoint{2.331424in}{5.227285in}}%
\pgfpathlineto{\pgfqpoint{2.332497in}{5.230500in}}%
\pgfpathlineto{\pgfqpoint{2.333033in}{5.230204in}}%
\pgfpathlineto{\pgfqpoint{2.334106in}{5.223353in}}%
\pgfpathlineto{\pgfqpoint{2.334642in}{5.237793in}}%
\pgfpathlineto{\pgfqpoint{2.335179in}{5.223908in}}%
\pgfpathlineto{\pgfqpoint{2.335715in}{5.227239in}}%
\pgfpathlineto{\pgfqpoint{2.336252in}{5.221178in}}%
\pgfpathlineto{\pgfqpoint{2.336788in}{5.235659in}}%
\pgfpathlineto{\pgfqpoint{2.337324in}{5.225731in}}%
\pgfpathlineto{\pgfqpoint{2.337861in}{5.218326in}}%
\pgfpathlineto{\pgfqpoint{2.339470in}{5.232411in}}%
\pgfpathlineto{\pgfqpoint{2.340007in}{5.221281in}}%
\pgfpathlineto{\pgfqpoint{2.340543in}{5.231348in}}%
\pgfpathlineto{\pgfqpoint{2.341080in}{5.231058in}}%
\pgfpathlineto{\pgfqpoint{2.342152in}{5.215901in}}%
\pgfpathlineto{\pgfqpoint{2.342689in}{5.231343in}}%
\pgfpathlineto{\pgfqpoint{2.343225in}{5.222952in}}%
\pgfpathlineto{\pgfqpoint{2.343762in}{5.225970in}}%
\pgfpathlineto{\pgfqpoint{2.344298in}{5.219824in}}%
\pgfpathlineto{\pgfqpoint{2.344835in}{5.222976in}}%
\pgfpathlineto{\pgfqpoint{2.345371in}{5.231492in}}%
\pgfpathlineto{\pgfqpoint{2.346980in}{5.214229in}}%
\pgfpathlineto{\pgfqpoint{2.347517in}{5.228340in}}%
\pgfpathlineto{\pgfqpoint{2.348053in}{5.220024in}}%
\pgfpathlineto{\pgfqpoint{2.349662in}{5.227884in}}%
\pgfpathlineto{\pgfqpoint{2.350735in}{5.214388in}}%
\pgfpathlineto{\pgfqpoint{2.351272in}{5.221815in}}%
\pgfpathlineto{\pgfqpoint{2.352345in}{5.228779in}}%
\pgfpathlineto{\pgfqpoint{2.352881in}{5.222779in}}%
\pgfpathlineto{\pgfqpoint{2.353418in}{5.225585in}}%
\pgfpathlineto{\pgfqpoint{2.353954in}{5.226251in}}%
\pgfpathlineto{\pgfqpoint{2.354490in}{5.215256in}}%
\pgfpathlineto{\pgfqpoint{2.355027in}{5.220029in}}%
\pgfpathlineto{\pgfqpoint{2.355563in}{5.225524in}}%
\pgfpathlineto{\pgfqpoint{2.356100in}{5.222398in}}%
\pgfpathlineto{\pgfqpoint{2.357173in}{5.212589in}}%
\pgfpathlineto{\pgfqpoint{2.357709in}{5.223017in}}%
\pgfpathlineto{\pgfqpoint{2.358245in}{5.221978in}}%
\pgfpathlineto{\pgfqpoint{2.359318in}{5.209939in}}%
\pgfpathlineto{\pgfqpoint{2.359855in}{5.212124in}}%
\pgfpathlineto{\pgfqpoint{2.360391in}{5.215245in}}%
\pgfpathlineto{\pgfqpoint{2.360928in}{5.211424in}}%
\pgfpathlineto{\pgfqpoint{2.361464in}{5.213493in}}%
\pgfpathlineto{\pgfqpoint{2.362001in}{5.225514in}}%
\pgfpathlineto{\pgfqpoint{2.362537in}{5.214772in}}%
\pgfpathlineto{\pgfqpoint{2.363610in}{5.211484in}}%
\pgfpathlineto{\pgfqpoint{2.364146in}{5.211792in}}%
\pgfpathlineto{\pgfqpoint{2.365219in}{5.223557in}}%
\pgfpathlineto{\pgfqpoint{2.365756in}{5.218048in}}%
\pgfpathlineto{\pgfqpoint{2.366828in}{5.217904in}}%
\pgfpathlineto{\pgfqpoint{2.367365in}{5.201813in}}%
\pgfpathlineto{\pgfqpoint{2.367901in}{5.214645in}}%
\pgfpathlineto{\pgfqpoint{2.369511in}{5.211076in}}%
\pgfpathlineto{\pgfqpoint{2.370583in}{5.203126in}}%
\pgfpathlineto{\pgfqpoint{2.372193in}{5.197370in}}%
\pgfpathlineto{\pgfqpoint{2.372729in}{5.196723in}}%
\pgfpathlineto{\pgfqpoint{2.373266in}{5.215685in}}%
\pgfpathlineto{\pgfqpoint{2.373802in}{5.192231in}}%
\pgfpathlineto{\pgfqpoint{2.374339in}{5.195820in}}%
\pgfpathlineto{\pgfqpoint{2.374875in}{5.198912in}}%
\pgfpathlineto{\pgfqpoint{2.375411in}{5.211924in}}%
\pgfpathlineto{\pgfqpoint{2.375948in}{5.196040in}}%
\pgfpathlineto{\pgfqpoint{2.376484in}{5.203320in}}%
\pgfpathlineto{\pgfqpoint{2.377021in}{5.199063in}}%
\pgfpathlineto{\pgfqpoint{2.377557in}{5.199383in}}%
\pgfpathlineto{\pgfqpoint{2.379166in}{5.206422in}}%
\pgfpathlineto{\pgfqpoint{2.380239in}{5.188293in}}%
\pgfpathlineto{\pgfqpoint{2.380776in}{5.191821in}}%
\pgfpathlineto{\pgfqpoint{2.382385in}{5.204145in}}%
\pgfpathlineto{\pgfqpoint{2.383994in}{5.166101in}}%
\pgfpathlineto{\pgfqpoint{2.385604in}{5.189000in}}%
\pgfpathlineto{\pgfqpoint{2.386677in}{5.179101in}}%
\pgfpathlineto{\pgfqpoint{2.387213in}{5.171544in}}%
\pgfpathlineto{\pgfqpoint{2.387749in}{5.177754in}}%
\pgfpathlineto{\pgfqpoint{2.388286in}{5.188825in}}%
\pgfpathlineto{\pgfqpoint{2.388822in}{5.182031in}}%
\pgfpathlineto{\pgfqpoint{2.390968in}{5.191572in}}%
\pgfpathlineto{\pgfqpoint{2.391505in}{5.174691in}}%
\pgfpathlineto{\pgfqpoint{2.392041in}{5.186244in}}%
\pgfpathlineto{\pgfqpoint{2.392577in}{5.184604in}}%
\pgfpathlineto{\pgfqpoint{2.393114in}{5.168829in}}%
\pgfpathlineto{\pgfqpoint{2.393650in}{5.181809in}}%
\pgfpathlineto{\pgfqpoint{2.394187in}{5.184123in}}%
\pgfpathlineto{\pgfqpoint{2.394723in}{5.182349in}}%
\pgfpathlineto{\pgfqpoint{2.395260in}{5.168388in}}%
\pgfpathlineto{\pgfqpoint{2.395796in}{5.183906in}}%
\pgfpathlineto{\pgfqpoint{2.395796in}{5.183906in}}%
\pgfpathlineto{\pgfqpoint{2.395796in}{5.183906in}}%
\pgfpathlineto{\pgfqpoint{2.397405in}{5.162783in}}%
\pgfpathlineto{\pgfqpoint{2.398478in}{5.172195in}}%
\pgfpathlineto{\pgfqpoint{2.400087in}{5.158735in}}%
\pgfpathlineto{\pgfqpoint{2.400624in}{5.164247in}}%
\pgfpathlineto{\pgfqpoint{2.401160in}{5.150174in}}%
\pgfpathlineto{\pgfqpoint{2.401697in}{5.152737in}}%
\pgfpathlineto{\pgfqpoint{2.402770in}{5.178878in}}%
\pgfpathlineto{\pgfqpoint{2.403843in}{5.173088in}}%
\pgfpathlineto{\pgfqpoint{2.404379in}{5.136042in}}%
\pgfpathlineto{\pgfqpoint{2.404915in}{5.155337in}}%
\pgfpathlineto{\pgfqpoint{2.405452in}{5.164983in}}%
\pgfpathlineto{\pgfqpoint{2.405988in}{5.163052in}}%
\pgfpathlineto{\pgfqpoint{2.407061in}{5.160101in}}%
\pgfpathlineto{\pgfqpoint{2.408134in}{5.145634in}}%
\pgfpathlineto{\pgfqpoint{2.408670in}{5.155431in}}%
\pgfpathlineto{\pgfqpoint{2.409207in}{5.138030in}}%
\pgfpathlineto{\pgfqpoint{2.409743in}{5.152042in}}%
\pgfpathlineto{\pgfqpoint{2.410280in}{5.142935in}}%
\pgfpathlineto{\pgfqpoint{2.410816in}{5.158960in}}%
\pgfpathlineto{\pgfqpoint{2.412426in}{5.135765in}}%
\pgfpathlineto{\pgfqpoint{2.412962in}{5.138122in}}%
\pgfpathlineto{\pgfqpoint{2.414035in}{5.116968in}}%
\pgfpathlineto{\pgfqpoint{2.414571in}{5.130224in}}%
\pgfpathlineto{\pgfqpoint{2.415644in}{5.156392in}}%
\pgfpathlineto{\pgfqpoint{2.417253in}{5.109317in}}%
\pgfpathlineto{\pgfqpoint{2.418326in}{5.145443in}}%
\pgfpathlineto{\pgfqpoint{2.419399in}{5.127153in}}%
\pgfpathlineto{\pgfqpoint{2.419936in}{5.138514in}}%
\pgfpathlineto{\pgfqpoint{2.420472in}{5.120099in}}%
\pgfpathlineto{\pgfqpoint{2.421008in}{5.127916in}}%
\pgfpathlineto{\pgfqpoint{2.422081in}{5.117793in}}%
\pgfpathlineto{\pgfqpoint{2.423691in}{5.136807in}}%
\pgfpathlineto{\pgfqpoint{2.425300in}{5.099156in}}%
\pgfpathlineto{\pgfqpoint{2.425836in}{5.109500in}}%
\pgfpathlineto{\pgfqpoint{2.426373in}{5.094364in}}%
\pgfpathlineto{\pgfqpoint{2.427982in}{5.135718in}}%
\pgfpathlineto{\pgfqpoint{2.429055in}{5.085516in}}%
\pgfpathlineto{\pgfqpoint{2.429591in}{5.108772in}}%
\pgfpathlineto{\pgfqpoint{2.429591in}{5.108772in}}%
\pgfpathlineto{\pgfqpoint{2.429591in}{5.108772in}}%
\pgfpathlineto{\pgfqpoint{2.430128in}{5.084401in}}%
\pgfpathlineto{\pgfqpoint{2.430664in}{5.117242in}}%
\pgfpathlineto{\pgfqpoint{2.431201in}{5.108806in}}%
\pgfpathlineto{\pgfqpoint{2.432274in}{5.088809in}}%
\pgfpathlineto{\pgfqpoint{2.432810in}{5.108279in}}%
\pgfpathlineto{\pgfqpoint{2.433347in}{5.104826in}}%
\pgfpathlineto{\pgfqpoint{2.434419in}{5.090385in}}%
\pgfpathlineto{\pgfqpoint{2.434956in}{5.100125in}}%
\pgfpathlineto{\pgfqpoint{2.435492in}{5.087608in}}%
\pgfpathlineto{\pgfqpoint{2.436029in}{5.113960in}}%
\pgfpathlineto{\pgfqpoint{2.436565in}{5.109545in}}%
\pgfpathlineto{\pgfqpoint{2.437638in}{5.048012in}}%
\pgfpathlineto{\pgfqpoint{2.438174in}{5.072053in}}%
\pgfpathlineto{\pgfqpoint{2.439784in}{5.098655in}}%
\pgfpathlineto{\pgfqpoint{2.440320in}{5.055899in}}%
\pgfpathlineto{\pgfqpoint{2.440857in}{5.094696in}}%
\pgfpathlineto{\pgfqpoint{2.441930in}{5.064654in}}%
\pgfpathlineto{\pgfqpoint{2.442466in}{5.068578in}}%
\pgfpathlineto{\pgfqpoint{2.443002in}{5.052559in}}%
\pgfpathlineto{\pgfqpoint{2.443539in}{5.078035in}}%
\pgfpathlineto{\pgfqpoint{2.444075in}{5.074440in}}%
\pgfpathlineto{\pgfqpoint{2.445148in}{5.050139in}}%
\pgfpathlineto{\pgfqpoint{2.446221in}{5.088198in}}%
\pgfpathlineto{\pgfqpoint{2.447294in}{5.048284in}}%
\pgfpathlineto{\pgfqpoint{2.448903in}{5.081581in}}%
\pgfpathlineto{\pgfqpoint{2.450512in}{5.001556in}}%
\pgfpathlineto{\pgfqpoint{2.451049in}{5.077360in}}%
\pgfpathlineto{\pgfqpoint{2.451585in}{5.058170in}}%
\pgfpathlineto{\pgfqpoint{2.452122in}{5.069627in}}%
\pgfpathlineto{\pgfqpoint{2.453195in}{5.009562in}}%
\pgfpathlineto{\pgfqpoint{2.454804in}{5.076638in}}%
\pgfpathlineto{\pgfqpoint{2.456413in}{5.012851in}}%
\pgfpathlineto{\pgfqpoint{2.457486in}{5.021161in}}%
\pgfpathlineto{\pgfqpoint{2.458023in}{5.007323in}}%
\pgfpathlineto{\pgfqpoint{2.459095in}{5.050730in}}%
\pgfpathlineto{\pgfqpoint{2.460168in}{4.998776in}}%
\pgfpathlineto{\pgfqpoint{2.461241in}{5.039244in}}%
\pgfpathlineto{\pgfqpoint{2.461778in}{5.035587in}}%
\pgfpathlineto{\pgfqpoint{2.462314in}{5.020951in}}%
\pgfpathlineto{\pgfqpoint{2.462851in}{4.941273in}}%
\pgfpathlineto{\pgfqpoint{2.463387in}{4.961635in}}%
\pgfpathlineto{\pgfqpoint{2.464460in}{5.058258in}}%
\pgfpathlineto{\pgfqpoint{2.464996in}{5.036792in}}%
\pgfpathlineto{\pgfqpoint{2.465533in}{4.957695in}}%
\pgfpathlineto{\pgfqpoint{2.466069in}{4.974155in}}%
\pgfpathlineto{\pgfqpoint{2.467678in}{5.035334in}}%
\pgfpathlineto{\pgfqpoint{2.468751in}{4.944481in}}%
\pgfpathlineto{\pgfqpoint{2.469288in}{4.964397in}}%
\pgfpathlineto{\pgfqpoint{2.470361in}{4.996496in}}%
\pgfpathlineto{\pgfqpoint{2.470897in}{4.982696in}}%
\pgfpathlineto{\pgfqpoint{2.471433in}{4.972988in}}%
\pgfpathlineto{\pgfqpoint{2.471970in}{5.039926in}}%
\pgfpathlineto{\pgfqpoint{2.472506in}{4.977227in}}%
\pgfpathlineto{\pgfqpoint{2.473579in}{4.965570in}}%
\pgfpathlineto{\pgfqpoint{2.474116in}{4.996679in}}%
\pgfpathlineto{\pgfqpoint{2.474652in}{4.990970in}}%
\pgfpathlineto{\pgfqpoint{2.475189in}{4.981292in}}%
\pgfpathlineto{\pgfqpoint{2.476261in}{4.888342in}}%
\pgfpathlineto{\pgfqpoint{2.477871in}{5.022347in}}%
\pgfpathlineto{\pgfqpoint{2.478407in}{4.896817in}}%
\pgfpathlineto{\pgfqpoint{2.478944in}{4.966133in}}%
\pgfpathlineto{\pgfqpoint{2.479480in}{4.954781in}}%
\pgfpathlineto{\pgfqpoint{2.480016in}{4.981380in}}%
\pgfpathlineto{\pgfqpoint{2.480553in}{4.975888in}}%
\pgfpathlineto{\pgfqpoint{2.481089in}{4.860467in}}%
\pgfpathlineto{\pgfqpoint{2.481626in}{4.873983in}}%
\pgfpathlineto{\pgfqpoint{2.482699in}{5.028687in}}%
\pgfpathlineto{\pgfqpoint{2.483235in}{4.976310in}}%
\pgfpathlineto{\pgfqpoint{2.484308in}{4.858882in}}%
\pgfpathlineto{\pgfqpoint{2.484844in}{4.977033in}}%
\pgfpathlineto{\pgfqpoint{2.485381in}{4.954703in}}%
\pgfpathlineto{\pgfqpoint{2.485917in}{4.958394in}}%
\pgfpathlineto{\pgfqpoint{2.486990in}{4.881518in}}%
\pgfpathlineto{\pgfqpoint{2.488599in}{4.964318in}}%
\pgfpathlineto{\pgfqpoint{2.489136in}{4.865724in}}%
\pgfpathlineto{\pgfqpoint{2.489672in}{4.930251in}}%
\pgfpathlineto{\pgfqpoint{2.490745in}{4.977795in}}%
\pgfpathlineto{\pgfqpoint{2.491282in}{4.842150in}}%
\pgfpathlineto{\pgfqpoint{2.491818in}{4.901153in}}%
\pgfpathlineto{\pgfqpoint{2.492891in}{4.959453in}}%
\pgfpathlineto{\pgfqpoint{2.493427in}{4.952989in}}%
\pgfpathlineto{\pgfqpoint{2.493964in}{4.727522in}}%
\pgfpathlineto{\pgfqpoint{2.494500in}{4.809793in}}%
\pgfpathlineto{\pgfqpoint{2.495037in}{4.802046in}}%
\pgfpathlineto{\pgfqpoint{2.495573in}{5.053991in}}%
\pgfpathlineto{\pgfqpoint{2.496110in}{4.981717in}}%
\pgfpathlineto{\pgfqpoint{2.497182in}{4.752978in}}%
\pgfpathlineto{\pgfqpoint{2.498255in}{4.966571in}}%
\pgfpathlineto{\pgfqpoint{2.498792in}{4.928939in}}%
\pgfpathlineto{\pgfqpoint{2.499328in}{4.899190in}}%
\pgfpathlineto{\pgfqpoint{2.499865in}{4.665490in}}%
\pgfpathlineto{\pgfqpoint{2.500401in}{4.785333in}}%
\pgfpathlineto{\pgfqpoint{2.500937in}{4.823774in}}%
\pgfpathlineto{\pgfqpoint{2.501474in}{5.021800in}}%
\pgfpathlineto{\pgfqpoint{2.502010in}{4.884235in}}%
\pgfpathlineto{\pgfqpoint{2.503620in}{4.850263in}}%
\pgfpathlineto{\pgfqpoint{2.504156in}{4.890869in}}%
\pgfpathlineto{\pgfqpoint{2.504693in}{4.744081in}}%
\pgfpathlineto{\pgfqpoint{2.505229in}{4.859344in}}%
\pgfpathlineto{\pgfqpoint{2.505765in}{4.845411in}}%
\pgfpathlineto{\pgfqpoint{2.506302in}{4.968928in}}%
\pgfpathlineto{\pgfqpoint{2.507911in}{4.744934in}}%
\pgfpathlineto{\pgfqpoint{2.508984in}{5.006108in}}%
\pgfpathlineto{\pgfqpoint{2.510593in}{4.519541in}}%
\pgfpathlineto{\pgfqpoint{2.511130in}{4.934481in}}%
\pgfpathlineto{\pgfqpoint{2.511666in}{4.879187in}}%
\pgfpathlineto{\pgfqpoint{2.512203in}{4.875778in}}%
\pgfpathlineto{\pgfqpoint{2.513276in}{4.576124in}}%
\pgfpathlineto{\pgfqpoint{2.514348in}{4.994738in}}%
\pgfpathlineto{\pgfqpoint{2.514885in}{4.990409in}}%
\pgfpathlineto{\pgfqpoint{2.515421in}{4.508610in}}%
\pgfpathlineto{\pgfqpoint{2.515958in}{4.761789in}}%
\pgfpathlineto{\pgfqpoint{2.516494in}{4.700812in}}%
\pgfpathlineto{\pgfqpoint{2.517031in}{4.910206in}}%
\pgfpathlineto{\pgfqpoint{2.517567in}{4.834583in}}%
\pgfpathlineto{\pgfqpoint{2.518640in}{4.681658in}}%
\pgfpathlineto{\pgfqpoint{2.519176in}{4.716279in}}%
\pgfpathlineto{\pgfqpoint{2.519713in}{4.978113in}}%
\pgfpathlineto{\pgfqpoint{2.520249in}{4.822612in}}%
\pgfpathlineto{\pgfqpoint{2.520786in}{4.827539in}}%
\pgfpathlineto{\pgfqpoint{2.521322in}{4.736101in}}%
\pgfpathlineto{\pgfqpoint{2.521858in}{4.954717in}}%
\pgfpathlineto{\pgfqpoint{2.522395in}{4.668245in}}%
\pgfpathlineto{\pgfqpoint{2.522931in}{4.684533in}}%
\pgfpathlineto{\pgfqpoint{2.523468in}{4.690610in}}%
\pgfpathlineto{\pgfqpoint{2.524004in}{4.736465in}}%
\pgfpathlineto{\pgfqpoint{2.524541in}{4.965789in}}%
\pgfpathlineto{\pgfqpoint{2.525077in}{4.875001in}}%
\pgfpathlineto{\pgfqpoint{2.526150in}{4.255063in}}%
\pgfpathlineto{\pgfqpoint{2.527759in}{5.001597in}}%
\pgfpathlineto{\pgfqpoint{2.528832in}{4.202319in}}%
\pgfpathlineto{\pgfqpoint{2.529369in}{4.400364in}}%
\pgfpathlineto{\pgfqpoint{2.530441in}{5.011085in}}%
\pgfpathlineto{\pgfqpoint{2.530978in}{4.725474in}}%
\pgfpathlineto{\pgfqpoint{2.531514in}{4.667348in}}%
\pgfpathlineto{\pgfqpoint{2.532051in}{4.393341in}}%
\pgfpathlineto{\pgfqpoint{2.533124in}{4.997150in}}%
\pgfpathlineto{\pgfqpoint{2.533660in}{4.842596in}}%
\pgfpathlineto{\pgfqpoint{2.534733in}{4.577054in}}%
\pgfpathlineto{\pgfqpoint{2.536342in}{4.850497in}}%
\pgfpathlineto{\pgfqpoint{2.536879in}{4.785395in}}%
\pgfpathlineto{\pgfqpoint{2.537415in}{4.893266in}}%
\pgfpathlineto{\pgfqpoint{2.537952in}{4.835789in}}%
\pgfpathlineto{\pgfqpoint{2.538488in}{4.783913in}}%
\pgfpathlineto{\pgfqpoint{2.539024in}{4.806107in}}%
\pgfpathlineto{\pgfqpoint{2.539561in}{4.218788in}}%
\pgfpathlineto{\pgfqpoint{2.539561in}{4.218788in}}%
\pgfpathlineto{\pgfqpoint{2.539561in}{4.218788in}}%
\pgfpathlineto{\pgfqpoint{2.540634in}{4.928504in}}%
\pgfpathlineto{\pgfqpoint{2.541170in}{4.854951in}}%
\pgfpathlineto{\pgfqpoint{2.542243in}{4.472635in}}%
\pgfpathlineto{\pgfqpoint{2.543852in}{5.070820in}}%
\pgfpathlineto{\pgfqpoint{2.544925in}{4.046497in}}%
\pgfpathlineto{\pgfqpoint{2.545462in}{4.386380in}}%
\pgfpathlineto{\pgfqpoint{2.546535in}{4.929954in}}%
\pgfpathlineto{\pgfqpoint{2.548144in}{4.390152in}}%
\pgfpathlineto{\pgfqpoint{2.549217in}{4.894802in}}%
\pgfpathlineto{\pgfqpoint{2.549753in}{4.881011in}}%
\pgfpathlineto{\pgfqpoint{2.550290in}{4.573275in}}%
\pgfpathlineto{\pgfqpoint{2.550826in}{4.769496in}}%
\pgfpathlineto{\pgfqpoint{2.551362in}{4.699808in}}%
\pgfpathlineto{\pgfqpoint{2.551899in}{4.710100in}}%
\pgfpathlineto{\pgfqpoint{2.552435in}{4.901224in}}%
\pgfpathlineto{\pgfqpoint{2.552972in}{4.572856in}}%
\pgfpathlineto{\pgfqpoint{2.553508in}{4.706142in}}%
\pgfpathlineto{\pgfqpoint{2.554045in}{4.792008in}}%
\pgfpathlineto{\pgfqpoint{2.554581in}{4.772930in}}%
\pgfpathlineto{\pgfqpoint{2.555118in}{4.691763in}}%
\pgfpathlineto{\pgfqpoint{2.555118in}{4.691763in}}%
\pgfpathlineto{\pgfqpoint{2.555118in}{4.691763in}}%
\pgfpathlineto{\pgfqpoint{2.557263in}{4.888524in}}%
\pgfpathlineto{\pgfqpoint{2.558336in}{4.383043in}}%
\pgfpathlineto{\pgfqpoint{2.559409in}{5.001609in}}%
\pgfpathlineto{\pgfqpoint{2.559945in}{4.970053in}}%
\pgfpathlineto{\pgfqpoint{2.561555in}{4.438904in}}%
\pgfpathlineto{\pgfqpoint{2.562628in}{4.961151in}}%
\pgfpathlineto{\pgfqpoint{2.563164in}{4.904643in}}%
\pgfpathlineto{\pgfqpoint{2.563701in}{4.521441in}}%
\pgfpathlineto{\pgfqpoint{2.564237in}{4.536370in}}%
\pgfpathlineto{\pgfqpoint{2.565846in}{4.957050in}}%
\pgfpathlineto{\pgfqpoint{2.567456in}{4.616150in}}%
\pgfpathlineto{\pgfqpoint{2.568528in}{4.929871in}}%
\pgfpathlineto{\pgfqpoint{2.569065in}{4.727449in}}%
\pgfpathlineto{\pgfqpoint{2.569601in}{4.823898in}}%
\pgfpathlineto{\pgfqpoint{2.570138in}{4.857550in}}%
\pgfpathlineto{\pgfqpoint{2.571211in}{4.670526in}}%
\pgfpathlineto{\pgfqpoint{2.572820in}{4.919870in}}%
\pgfpathlineto{\pgfqpoint{2.573356in}{4.924173in}}%
\pgfpathlineto{\pgfqpoint{2.574966in}{4.667155in}}%
\pgfpathlineto{\pgfqpoint{2.576039in}{4.915888in}}%
\pgfpathlineto{\pgfqpoint{2.577111in}{4.717700in}}%
\pgfpathlineto{\pgfqpoint{2.577648in}{4.596458in}}%
\pgfpathlineto{\pgfqpoint{2.579257in}{4.977709in}}%
\pgfpathlineto{\pgfqpoint{2.580330in}{4.698134in}}%
\pgfpathlineto{\pgfqpoint{2.580866in}{4.738045in}}%
\pgfpathlineto{\pgfqpoint{2.581939in}{5.010330in}}%
\pgfpathlineto{\pgfqpoint{2.582476in}{4.919709in}}%
\pgfpathlineto{\pgfqpoint{2.583549in}{4.776021in}}%
\pgfpathlineto{\pgfqpoint{2.584085in}{4.846431in}}%
\pgfpathlineto{\pgfqpoint{2.585694in}{4.906093in}}%
\pgfpathlineto{\pgfqpoint{2.586231in}{4.899928in}}%
\pgfpathlineto{\pgfqpoint{2.586767in}{4.933892in}}%
\pgfpathlineto{\pgfqpoint{2.587840in}{4.859205in}}%
\pgfpathlineto{\pgfqpoint{2.588377in}{4.870058in}}%
\pgfpathlineto{\pgfqpoint{2.589449in}{4.943885in}}%
\pgfpathlineto{\pgfqpoint{2.591059in}{4.841637in}}%
\pgfpathlineto{\pgfqpoint{2.592668in}{4.966455in}}%
\pgfpathlineto{\pgfqpoint{2.593741in}{4.777125in}}%
\pgfpathlineto{\pgfqpoint{2.595350in}{4.990644in}}%
\pgfpathlineto{\pgfqpoint{2.595887in}{4.948396in}}%
\pgfpathlineto{\pgfqpoint{2.596960in}{4.754214in}}%
\pgfpathlineto{\pgfqpoint{2.597496in}{4.873161in}}%
\pgfpathlineto{\pgfqpoint{2.598032in}{4.960174in}}%
\pgfpathlineto{\pgfqpoint{2.598569in}{4.931883in}}%
\pgfpathlineto{\pgfqpoint{2.599105in}{4.924412in}}%
\pgfpathlineto{\pgfqpoint{2.599642in}{4.868770in}}%
\pgfpathlineto{\pgfqpoint{2.600178in}{4.921753in}}%
\pgfpathlineto{\pgfqpoint{2.600715in}{4.911336in}}%
\pgfpathlineto{\pgfqpoint{2.601251in}{4.973304in}}%
\pgfpathlineto{\pgfqpoint{2.601787in}{4.944665in}}%
\pgfpathlineto{\pgfqpoint{2.602324in}{4.867632in}}%
\pgfpathlineto{\pgfqpoint{2.602860in}{4.910911in}}%
\pgfpathlineto{\pgfqpoint{2.603933in}{4.963428in}}%
\pgfpathlineto{\pgfqpoint{2.604470in}{4.950379in}}%
\pgfpathlineto{\pgfqpoint{2.605006in}{4.956088in}}%
\pgfpathlineto{\pgfqpoint{2.605543in}{4.987513in}}%
\pgfpathlineto{\pgfqpoint{2.607152in}{4.919703in}}%
\pgfpathlineto{\pgfqpoint{2.607688in}{4.933587in}}%
\pgfpathlineto{\pgfqpoint{2.609298in}{5.019785in}}%
\pgfpathlineto{\pgfqpoint{2.610370in}{4.888755in}}%
\pgfpathlineto{\pgfqpoint{2.611980in}{5.035500in}}%
\pgfpathlineto{\pgfqpoint{2.613589in}{4.938894in}}%
\pgfpathlineto{\pgfqpoint{2.614662in}{5.025748in}}%
\pgfpathlineto{\pgfqpoint{2.615198in}{4.990332in}}%
\pgfpathlineto{\pgfqpoint{2.616271in}{4.954011in}}%
\pgfpathlineto{\pgfqpoint{2.617881in}{5.025578in}}%
\pgfpathlineto{\pgfqpoint{2.618953in}{4.964307in}}%
\pgfpathlineto{\pgfqpoint{2.619490in}{4.973196in}}%
\pgfpathlineto{\pgfqpoint{2.621636in}{5.025470in}}%
\pgfpathlineto{\pgfqpoint{2.623245in}{4.998432in}}%
\pgfpathlineto{\pgfqpoint{2.623781in}{5.000842in}}%
\pgfpathlineto{\pgfqpoint{2.624318in}{4.994416in}}%
\pgfpathlineto{\pgfqpoint{2.625391in}{5.070321in}}%
\pgfpathlineto{\pgfqpoint{2.625927in}{5.024911in}}%
\pgfpathlineto{\pgfqpoint{2.627000in}{4.982755in}}%
\pgfpathlineto{\pgfqpoint{2.628073in}{5.048160in}}%
\pgfpathlineto{\pgfqpoint{2.628609in}{5.044841in}}%
\pgfpathlineto{\pgfqpoint{2.629146in}{5.047812in}}%
\pgfpathlineto{\pgfqpoint{2.629682in}{4.992743in}}%
\pgfpathlineto{\pgfqpoint{2.630219in}{5.013765in}}%
\pgfpathlineto{\pgfqpoint{2.631291in}{5.068163in}}%
\pgfpathlineto{\pgfqpoint{2.631828in}{5.060890in}}%
\pgfpathlineto{\pgfqpoint{2.632364in}{5.013343in}}%
\pgfpathlineto{\pgfqpoint{2.632901in}{5.026549in}}%
\pgfpathlineto{\pgfqpoint{2.634510in}{5.066164in}}%
\pgfpathlineto{\pgfqpoint{2.636119in}{5.023817in}}%
\pgfpathlineto{\pgfqpoint{2.637192in}{5.067072in}}%
\pgfpathlineto{\pgfqpoint{2.637729in}{5.061738in}}%
\pgfpathlineto{\pgfqpoint{2.638802in}{5.057233in}}%
\pgfpathlineto{\pgfqpoint{2.639338in}{5.078560in}}%
\pgfpathlineto{\pgfqpoint{2.639338in}{5.078560in}}%
\pgfpathlineto{\pgfqpoint{2.639338in}{5.078560in}}%
\pgfpathlineto{\pgfqpoint{2.640411in}{5.051317in}}%
\pgfpathlineto{\pgfqpoint{2.642020in}{5.082374in}}%
\pgfpathlineto{\pgfqpoint{2.643629in}{5.063404in}}%
\pgfpathlineto{\pgfqpoint{2.644166in}{5.066908in}}%
\pgfpathlineto{\pgfqpoint{2.644702in}{5.106695in}}%
\pgfpathlineto{\pgfqpoint{2.645239in}{5.085091in}}%
\pgfpathlineto{\pgfqpoint{2.645775in}{5.078711in}}%
\pgfpathlineto{\pgfqpoint{2.646312in}{5.054296in}}%
\pgfpathlineto{\pgfqpoint{2.646848in}{5.062172in}}%
\pgfpathlineto{\pgfqpoint{2.648457in}{5.116700in}}%
\pgfpathlineto{\pgfqpoint{2.649530in}{5.073578in}}%
\pgfpathlineto{\pgfqpoint{2.650067in}{5.083241in}}%
\pgfpathlineto{\pgfqpoint{2.650603in}{5.075665in}}%
\pgfpathlineto{\pgfqpoint{2.651140in}{5.114577in}}%
\pgfpathlineto{\pgfqpoint{2.651676in}{5.111583in}}%
\pgfpathlineto{\pgfqpoint{2.652749in}{5.084366in}}%
\pgfpathlineto{\pgfqpoint{2.653285in}{5.096240in}}%
\pgfpathlineto{\pgfqpoint{2.653822in}{5.106295in}}%
\pgfpathlineto{\pgfqpoint{2.654358in}{5.086744in}}%
\pgfpathlineto{\pgfqpoint{2.654358in}{5.086744in}}%
\pgfpathlineto{\pgfqpoint{2.654358in}{5.086744in}}%
\pgfpathlineto{\pgfqpoint{2.655431in}{5.117339in}}%
\pgfpathlineto{\pgfqpoint{2.655968in}{5.112546in}}%
\pgfpathlineto{\pgfqpoint{2.656504in}{5.116336in}}%
\pgfpathlineto{\pgfqpoint{2.658113in}{5.094750in}}%
\pgfpathlineto{\pgfqpoint{2.659186in}{5.133270in}}%
\pgfpathlineto{\pgfqpoint{2.659723in}{5.113588in}}%
\pgfpathlineto{\pgfqpoint{2.660259in}{5.113502in}}%
\pgfpathlineto{\pgfqpoint{2.660795in}{5.110212in}}%
\pgfpathlineto{\pgfqpoint{2.661332in}{5.126295in}}%
\pgfpathlineto{\pgfqpoint{2.661868in}{5.113536in}}%
\pgfpathlineto{\pgfqpoint{2.662941in}{5.117993in}}%
\pgfpathlineto{\pgfqpoint{2.663478in}{5.106058in}}%
\pgfpathlineto{\pgfqpoint{2.665087in}{5.139728in}}%
\pgfpathlineto{\pgfqpoint{2.666160in}{5.111313in}}%
\pgfpathlineto{\pgfqpoint{2.666696in}{5.120786in}}%
\pgfpathlineto{\pgfqpoint{2.667233in}{5.118865in}}%
\pgfpathlineto{\pgfqpoint{2.668842in}{5.141687in}}%
\pgfpathlineto{\pgfqpoint{2.669915in}{5.121983in}}%
\pgfpathlineto{\pgfqpoint{2.671524in}{5.143668in}}%
\pgfpathlineto{\pgfqpoint{2.672597in}{5.141407in}}%
\pgfpathlineto{\pgfqpoint{2.673133in}{5.152722in}}%
\pgfpathlineto{\pgfqpoint{2.674206in}{5.132262in}}%
\pgfpathlineto{\pgfqpoint{2.675816in}{5.153404in}}%
\pgfpathlineto{\pgfqpoint{2.676352in}{5.148700in}}%
\pgfpathlineto{\pgfqpoint{2.676889in}{5.160755in}}%
\pgfpathlineto{\pgfqpoint{2.677425in}{5.149156in}}%
\pgfpathlineto{\pgfqpoint{2.677961in}{5.140070in}}%
\pgfpathlineto{\pgfqpoint{2.679571in}{5.159308in}}%
\pgfpathlineto{\pgfqpoint{2.680107in}{5.158467in}}%
\pgfpathlineto{\pgfqpoint{2.680644in}{5.167672in}}%
\pgfpathlineto{\pgfqpoint{2.681180in}{5.161727in}}%
\pgfpathlineto{\pgfqpoint{2.681716in}{5.162500in}}%
\pgfpathlineto{\pgfqpoint{2.682253in}{5.169303in}}%
\pgfpathlineto{\pgfqpoint{2.682789in}{5.157765in}}%
\pgfpathlineto{\pgfqpoint{2.683326in}{5.162038in}}%
\pgfpathlineto{\pgfqpoint{2.683862in}{5.163448in}}%
\pgfpathlineto{\pgfqpoint{2.684935in}{5.174194in}}%
\pgfpathlineto{\pgfqpoint{2.685472in}{5.174131in}}%
\pgfpathlineto{\pgfqpoint{2.686008in}{5.165418in}}%
\pgfpathlineto{\pgfqpoint{2.686544in}{5.166390in}}%
\pgfpathlineto{\pgfqpoint{2.687617in}{5.180504in}}%
\pgfpathlineto{\pgfqpoint{2.688154in}{5.179242in}}%
\pgfpathlineto{\pgfqpoint{2.688690in}{5.173321in}}%
\pgfpathlineto{\pgfqpoint{2.689227in}{5.183257in}}%
\pgfpathlineto{\pgfqpoint{2.690299in}{5.182886in}}%
\pgfpathlineto{\pgfqpoint{2.690836in}{5.180901in}}%
\pgfpathlineto{\pgfqpoint{2.691372in}{5.184516in}}%
\pgfpathlineto{\pgfqpoint{2.691909in}{5.175772in}}%
\pgfpathlineto{\pgfqpoint{2.692445in}{5.178686in}}%
\pgfpathlineto{\pgfqpoint{2.692982in}{5.195159in}}%
\pgfpathlineto{\pgfqpoint{2.693518in}{5.178678in}}%
\pgfpathlineto{\pgfqpoint{2.694054in}{5.184732in}}%
\pgfpathlineto{\pgfqpoint{2.694591in}{5.184250in}}%
\pgfpathlineto{\pgfqpoint{2.695127in}{5.179340in}}%
\pgfpathlineto{\pgfqpoint{2.695664in}{5.183221in}}%
\pgfpathlineto{\pgfqpoint{2.696737in}{5.202047in}}%
\pgfpathlineto{\pgfqpoint{2.697273in}{5.181649in}}%
\pgfpathlineto{\pgfqpoint{2.697810in}{5.189610in}}%
\pgfpathlineto{\pgfqpoint{2.698346in}{5.191363in}}%
\pgfpathlineto{\pgfqpoint{2.699419in}{5.184260in}}%
\pgfpathlineto{\pgfqpoint{2.700492in}{5.195402in}}%
\pgfpathlineto{\pgfqpoint{2.701028in}{5.184245in}}%
\pgfpathlineto{\pgfqpoint{2.701028in}{5.184245in}}%
\pgfpathlineto{\pgfqpoint{2.701028in}{5.184245in}}%
\pgfpathlineto{\pgfqpoint{2.701565in}{5.195817in}}%
\pgfpathlineto{\pgfqpoint{2.702101in}{5.195650in}}%
\pgfpathlineto{\pgfqpoint{2.703174in}{5.186593in}}%
\pgfpathlineto{\pgfqpoint{2.704783in}{5.197120in}}%
\pgfpathlineto{\pgfqpoint{2.706929in}{5.194101in}}%
\pgfpathlineto{\pgfqpoint{2.707465in}{5.205218in}}%
\pgfpathlineto{\pgfqpoint{2.708002in}{5.199736in}}%
\pgfpathlineto{\pgfqpoint{2.709075in}{5.192811in}}%
\pgfpathlineto{\pgfqpoint{2.711220in}{5.207252in}}%
\pgfpathlineto{\pgfqpoint{2.711757in}{5.195321in}}%
\pgfpathlineto{\pgfqpoint{2.712830in}{5.195786in}}%
\pgfpathlineto{\pgfqpoint{2.713366in}{5.207632in}}%
\pgfpathlineto{\pgfqpoint{2.714439in}{5.206914in}}%
\pgfpathlineto{\pgfqpoint{2.714976in}{5.206590in}}%
\pgfpathlineto{\pgfqpoint{2.715512in}{5.203615in}}%
\pgfpathlineto{\pgfqpoint{2.717121in}{5.211915in}}%
\pgfpathlineto{\pgfqpoint{2.717658in}{5.202659in}}%
\pgfpathlineto{\pgfqpoint{2.718194in}{5.217448in}}%
\pgfpathlineto{\pgfqpoint{2.718731in}{5.207029in}}%
\pgfpathlineto{\pgfqpoint{2.719267in}{5.205479in}}%
\pgfpathlineto{\pgfqpoint{2.720876in}{5.216246in}}%
\pgfpathlineto{\pgfqpoint{2.721413in}{5.216756in}}%
\pgfpathlineto{\pgfqpoint{2.721949in}{5.223629in}}%
\pgfpathlineto{\pgfqpoint{2.723022in}{5.211443in}}%
\pgfpathlineto{\pgfqpoint{2.724631in}{5.222295in}}%
\pgfpathlineto{\pgfqpoint{2.725168in}{5.218557in}}%
\pgfpathlineto{\pgfqpoint{2.725704in}{5.227944in}}%
\pgfpathlineto{\pgfqpoint{2.726241in}{5.216796in}}%
\pgfpathlineto{\pgfqpoint{2.726777in}{5.218600in}}%
\pgfpathlineto{\pgfqpoint{2.727850in}{5.228728in}}%
\pgfpathlineto{\pgfqpoint{2.728923in}{5.220416in}}%
\pgfpathlineto{\pgfqpoint{2.730532in}{5.228710in}}%
\pgfpathlineto{\pgfqpoint{2.731069in}{5.222690in}}%
\pgfpathlineto{\pgfqpoint{2.731605in}{5.235483in}}%
\pgfpathlineto{\pgfqpoint{2.732141in}{5.223043in}}%
\pgfpathlineto{\pgfqpoint{2.732678in}{5.218382in}}%
\pgfpathlineto{\pgfqpoint{2.733751in}{5.231270in}}%
\pgfpathlineto{\pgfqpoint{2.734287in}{5.230230in}}%
\pgfpathlineto{\pgfqpoint{2.734824in}{5.224254in}}%
\pgfpathlineto{\pgfqpoint{2.735360in}{5.238128in}}%
\pgfpathlineto{\pgfqpoint{2.735897in}{5.227229in}}%
\pgfpathlineto{\pgfqpoint{2.736969in}{5.223928in}}%
\pgfpathlineto{\pgfqpoint{2.737506in}{5.232908in}}%
\pgfpathlineto{\pgfqpoint{2.738042in}{5.231652in}}%
\pgfpathlineto{\pgfqpoint{2.738579in}{5.221870in}}%
\pgfpathlineto{\pgfqpoint{2.739115in}{5.240163in}}%
\pgfpathlineto{\pgfqpoint{2.739652in}{5.228846in}}%
\pgfpathlineto{\pgfqpoint{2.740724in}{5.223737in}}%
\pgfpathlineto{\pgfqpoint{2.741261in}{5.238000in}}%
\pgfpathlineto{\pgfqpoint{2.741797in}{5.233944in}}%
\pgfpathlineto{\pgfqpoint{2.742334in}{5.228116in}}%
\pgfpathlineto{\pgfqpoint{2.742870in}{5.235091in}}%
\pgfpathlineto{\pgfqpoint{2.743407in}{5.233068in}}%
\pgfpathlineto{\pgfqpoint{2.743943in}{5.231633in}}%
\pgfpathlineto{\pgfqpoint{2.744479in}{5.223842in}}%
\pgfpathlineto{\pgfqpoint{2.745016in}{5.241755in}}%
\pgfpathlineto{\pgfqpoint{2.745552in}{5.235824in}}%
\pgfpathlineto{\pgfqpoint{2.746625in}{5.228721in}}%
\pgfpathlineto{\pgfqpoint{2.747162in}{5.231653in}}%
\pgfpathlineto{\pgfqpoint{2.747698in}{5.229731in}}%
\pgfpathlineto{\pgfqpoint{2.748235in}{5.230408in}}%
\pgfpathlineto{\pgfqpoint{2.748771in}{5.239727in}}%
\pgfpathlineto{\pgfqpoint{2.749307in}{5.234122in}}%
\pgfpathlineto{\pgfqpoint{2.749844in}{5.232612in}}%
\pgfpathlineto{\pgfqpoint{2.750380in}{5.227694in}}%
\pgfpathlineto{\pgfqpoint{2.751453in}{5.238631in}}%
\pgfpathlineto{\pgfqpoint{2.751990in}{5.232962in}}%
\pgfpathlineto{\pgfqpoint{2.751990in}{5.232962in}}%
\pgfpathlineto{\pgfqpoint{2.751990in}{5.232962in}}%
\pgfpathlineto{\pgfqpoint{2.752526in}{5.238837in}}%
\pgfpathlineto{\pgfqpoint{2.753062in}{5.230321in}}%
\pgfpathlineto{\pgfqpoint{2.753599in}{5.234702in}}%
\pgfpathlineto{\pgfqpoint{2.754135in}{5.231386in}}%
\pgfpathlineto{\pgfqpoint{2.754672in}{5.234124in}}%
\pgfpathlineto{\pgfqpoint{2.755208in}{5.237910in}}%
\pgfpathlineto{\pgfqpoint{2.755745in}{5.236863in}}%
\pgfpathlineto{\pgfqpoint{2.756818in}{5.229362in}}%
\pgfpathlineto{\pgfqpoint{2.758427in}{5.237692in}}%
\pgfpathlineto{\pgfqpoint{2.758963in}{5.238752in}}%
\pgfpathlineto{\pgfqpoint{2.759500in}{5.237838in}}%
\pgfpathlineto{\pgfqpoint{2.760573in}{5.231229in}}%
\pgfpathlineto{\pgfqpoint{2.761109in}{5.232259in}}%
\pgfpathlineto{\pgfqpoint{2.762718in}{5.238264in}}%
\pgfpathlineto{\pgfqpoint{2.763255in}{5.238627in}}%
\pgfpathlineto{\pgfqpoint{2.764328in}{5.229976in}}%
\pgfpathlineto{\pgfqpoint{2.764864in}{5.230510in}}%
\pgfpathlineto{\pgfqpoint{2.766473in}{5.239394in}}%
\pgfpathlineto{\pgfqpoint{2.768083in}{5.231112in}}%
\pgfpathlineto{\pgfqpoint{2.768619in}{5.231006in}}%
\pgfpathlineto{\pgfqpoint{2.769156in}{5.239701in}}%
\pgfpathlineto{\pgfqpoint{2.769692in}{5.234435in}}%
\pgfpathlineto{\pgfqpoint{2.770228in}{5.235726in}}%
\pgfpathlineto{\pgfqpoint{2.770765in}{5.234991in}}%
\pgfpathlineto{\pgfqpoint{2.771301in}{5.232177in}}%
\pgfpathlineto{\pgfqpoint{2.771838in}{5.233324in}}%
\pgfpathlineto{\pgfqpoint{2.772911in}{5.239431in}}%
\pgfpathlineto{\pgfqpoint{2.774520in}{5.230470in}}%
\pgfpathlineto{\pgfqpoint{2.776129in}{5.240581in}}%
\pgfpathlineto{\pgfqpoint{2.776666in}{5.240356in}}%
\pgfpathlineto{\pgfqpoint{2.777202in}{5.235621in}}%
\pgfpathlineto{\pgfqpoint{2.777739in}{5.241626in}}%
\pgfpathlineto{\pgfqpoint{2.778811in}{5.228672in}}%
\pgfpathlineto{\pgfqpoint{2.779884in}{5.244232in}}%
\pgfpathlineto{\pgfqpoint{2.780421in}{5.235888in}}%
\pgfpathlineto{\pgfqpoint{2.780957in}{5.237367in}}%
\pgfpathlineto{\pgfqpoint{2.782030in}{5.236240in}}%
\pgfpathlineto{\pgfqpoint{2.782566in}{5.228461in}}%
\pgfpathlineto{\pgfqpoint{2.783103in}{5.235480in}}%
\pgfpathlineto{\pgfqpoint{2.783639in}{5.243885in}}%
\pgfpathlineto{\pgfqpoint{2.784176in}{5.239936in}}%
\pgfpathlineto{\pgfqpoint{2.785249in}{5.232804in}}%
\pgfpathlineto{\pgfqpoint{2.785785in}{5.237114in}}%
\pgfpathlineto{\pgfqpoint{2.786322in}{5.223931in}}%
\pgfpathlineto{\pgfqpoint{2.786858in}{5.233453in}}%
\pgfpathlineto{\pgfqpoint{2.787394in}{5.240071in}}%
\pgfpathlineto{\pgfqpoint{2.787931in}{5.235972in}}%
\pgfpathlineto{\pgfqpoint{2.789004in}{5.223697in}}%
\pgfpathlineto{\pgfqpoint{2.789540in}{5.236952in}}%
\pgfpathlineto{\pgfqpoint{2.790077in}{5.225168in}}%
\pgfpathlineto{\pgfqpoint{2.791686in}{5.235047in}}%
\pgfpathlineto{\pgfqpoint{2.792759in}{5.217255in}}%
\pgfpathlineto{\pgfqpoint{2.793295in}{5.234205in}}%
\pgfpathlineto{\pgfqpoint{2.793832in}{5.224471in}}%
\pgfpathlineto{\pgfqpoint{2.794368in}{5.228889in}}%
\pgfpathlineto{\pgfqpoint{2.794904in}{5.226715in}}%
\pgfpathlineto{\pgfqpoint{2.795441in}{5.226747in}}%
\pgfpathlineto{\pgfqpoint{2.795977in}{5.225324in}}%
\pgfpathlineto{\pgfqpoint{2.796514in}{5.214873in}}%
\pgfpathlineto{\pgfqpoint{2.797050in}{5.233883in}}%
\pgfpathlineto{\pgfqpoint{2.797587in}{5.225606in}}%
\pgfpathlineto{\pgfqpoint{2.798123in}{5.229206in}}%
\pgfpathlineto{\pgfqpoint{2.798660in}{5.221079in}}%
\pgfpathlineto{\pgfqpoint{2.799196in}{5.226837in}}%
\pgfpathlineto{\pgfqpoint{2.800269in}{5.215943in}}%
\pgfpathlineto{\pgfqpoint{2.801878in}{5.230610in}}%
\pgfpathlineto{\pgfqpoint{2.802415in}{5.215134in}}%
\pgfpathlineto{\pgfqpoint{2.802951in}{5.220117in}}%
\pgfpathlineto{\pgfqpoint{2.804560in}{5.227301in}}%
\pgfpathlineto{\pgfqpoint{2.805097in}{5.222633in}}%
\pgfpathlineto{\pgfqpoint{2.805633in}{5.233910in}}%
\pgfpathlineto{\pgfqpoint{2.806170in}{5.216702in}}%
\pgfpathlineto{\pgfqpoint{2.806706in}{5.220332in}}%
\pgfpathlineto{\pgfqpoint{2.807243in}{5.223436in}}%
\pgfpathlineto{\pgfqpoint{2.807779in}{5.220192in}}%
\pgfpathlineto{\pgfqpoint{2.808315in}{5.219957in}}%
\pgfpathlineto{\pgfqpoint{2.808852in}{5.221270in}}%
\pgfpathlineto{\pgfqpoint{2.809388in}{5.228476in}}%
\pgfpathlineto{\pgfqpoint{2.809925in}{5.207361in}}%
\pgfpathlineto{\pgfqpoint{2.810461in}{5.212356in}}%
\pgfpathlineto{\pgfqpoint{2.811534in}{5.218754in}}%
\pgfpathlineto{\pgfqpoint{2.812070in}{5.217338in}}%
\pgfpathlineto{\pgfqpoint{2.813143in}{5.222279in}}%
\pgfpathlineto{\pgfqpoint{2.814753in}{5.202263in}}%
\pgfpathlineto{\pgfqpoint{2.816362in}{5.219724in}}%
\pgfpathlineto{\pgfqpoint{2.818508in}{5.194504in}}%
\pgfpathlineto{\pgfqpoint{2.820117in}{5.213454in}}%
\pgfpathlineto{\pgfqpoint{2.821726in}{5.198794in}}%
\pgfpathlineto{\pgfqpoint{2.822263in}{5.186623in}}%
\pgfpathlineto{\pgfqpoint{2.823336in}{5.208846in}}%
\pgfpathlineto{\pgfqpoint{2.824408in}{5.194038in}}%
\pgfpathlineto{\pgfqpoint{2.824945in}{5.194377in}}%
\pgfpathlineto{\pgfqpoint{2.825481in}{5.189979in}}%
\pgfpathlineto{\pgfqpoint{2.827091in}{5.195439in}}%
\pgfpathlineto{\pgfqpoint{2.827627in}{5.198348in}}%
\pgfpathlineto{\pgfqpoint{2.829236in}{5.185493in}}%
\pgfpathlineto{\pgfqpoint{2.830309in}{5.188224in}}%
\pgfpathlineto{\pgfqpoint{2.830846in}{5.201271in}}%
\pgfpathlineto{\pgfqpoint{2.831382in}{5.198389in}}%
\pgfpathlineto{\pgfqpoint{2.832991in}{5.180328in}}%
\pgfpathlineto{\pgfqpoint{2.833528in}{5.195311in}}%
\pgfpathlineto{\pgfqpoint{2.834064in}{5.193281in}}%
\pgfpathlineto{\pgfqpoint{2.834601in}{5.188262in}}%
\pgfpathlineto{\pgfqpoint{2.835137in}{5.207712in}}%
\pgfpathlineto{\pgfqpoint{2.836210in}{5.185793in}}%
\pgfpathlineto{\pgfqpoint{2.836747in}{5.186421in}}%
\pgfpathlineto{\pgfqpoint{2.837283in}{5.188753in}}%
\pgfpathlineto{\pgfqpoint{2.837819in}{5.180625in}}%
\pgfpathlineto{\pgfqpoint{2.838356in}{5.186269in}}%
\pgfpathlineto{\pgfqpoint{2.838892in}{5.188663in}}%
\pgfpathlineto{\pgfqpoint{2.839429in}{5.173573in}}%
\pgfpathlineto{\pgfqpoint{2.839965in}{5.182324in}}%
\pgfpathlineto{\pgfqpoint{2.840502in}{5.180515in}}%
\pgfpathlineto{\pgfqpoint{2.841038in}{5.189739in}}%
\pgfpathlineto{\pgfqpoint{2.842111in}{5.168764in}}%
\pgfpathlineto{\pgfqpoint{2.842647in}{5.173583in}}%
\pgfpathlineto{\pgfqpoint{2.843184in}{5.174796in}}%
\pgfpathlineto{\pgfqpoint{2.843720in}{5.181714in}}%
\pgfpathlineto{\pgfqpoint{2.844257in}{5.176477in}}%
\pgfpathlineto{\pgfqpoint{2.845329in}{5.166692in}}%
\pgfpathlineto{\pgfqpoint{2.845866in}{5.176650in}}%
\pgfpathlineto{\pgfqpoint{2.846402in}{5.160777in}}%
\pgfpathlineto{\pgfqpoint{2.846939in}{5.166474in}}%
\pgfpathlineto{\pgfqpoint{2.847475in}{5.179368in}}%
\pgfpathlineto{\pgfqpoint{2.848012in}{5.158596in}}%
\pgfpathlineto{\pgfqpoint{2.848548in}{5.159762in}}%
\pgfpathlineto{\pgfqpoint{2.849621in}{5.165295in}}%
\pgfpathlineto{\pgfqpoint{2.850157in}{5.147543in}}%
\pgfpathlineto{\pgfqpoint{2.850694in}{5.157038in}}%
\pgfpathlineto{\pgfqpoint{2.851230in}{5.161407in}}%
\pgfpathlineto{\pgfqpoint{2.851767in}{5.150589in}}%
\pgfpathlineto{\pgfqpoint{2.852303in}{5.154575in}}%
\pgfpathlineto{\pgfqpoint{2.852840in}{5.157909in}}%
\pgfpathlineto{\pgfqpoint{2.853912in}{5.142495in}}%
\pgfpathlineto{\pgfqpoint{2.854449in}{5.144016in}}%
\pgfpathlineto{\pgfqpoint{2.856058in}{5.166438in}}%
\pgfpathlineto{\pgfqpoint{2.857131in}{5.129835in}}%
\pgfpathlineto{\pgfqpoint{2.857668in}{5.137729in}}%
\pgfpathlineto{\pgfqpoint{2.858740in}{5.159958in}}%
\pgfpathlineto{\pgfqpoint{2.859277in}{5.155280in}}%
\pgfpathlineto{\pgfqpoint{2.860886in}{5.117609in}}%
\pgfpathlineto{\pgfqpoint{2.862495in}{5.150931in}}%
\pgfpathlineto{\pgfqpoint{2.863032in}{5.150334in}}%
\pgfpathlineto{\pgfqpoint{2.864105in}{5.129557in}}%
\pgfpathlineto{\pgfqpoint{2.864641in}{5.130161in}}%
\pgfpathlineto{\pgfqpoint{2.865178in}{5.133253in}}%
\pgfpathlineto{\pgfqpoint{2.865714in}{5.129392in}}%
\pgfpathlineto{\pgfqpoint{2.866787in}{5.147184in}}%
\pgfpathlineto{\pgfqpoint{2.867323in}{5.121773in}}%
\pgfpathlineto{\pgfqpoint{2.867860in}{5.133354in}}%
\pgfpathlineto{\pgfqpoint{2.868396in}{5.138303in}}%
\pgfpathlineto{\pgfqpoint{2.869469in}{5.112605in}}%
\pgfpathlineto{\pgfqpoint{2.871078in}{5.138566in}}%
\pgfpathlineto{\pgfqpoint{2.872688in}{5.104613in}}%
\pgfpathlineto{\pgfqpoint{2.873224in}{5.105939in}}%
\pgfpathlineto{\pgfqpoint{2.873761in}{5.134528in}}%
\pgfpathlineto{\pgfqpoint{2.874297in}{5.123704in}}%
\pgfpathlineto{\pgfqpoint{2.875906in}{5.082382in}}%
\pgfpathlineto{\pgfqpoint{2.876979in}{5.111590in}}%
\pgfpathlineto{\pgfqpoint{2.877516in}{5.135210in}}%
\pgfpathlineto{\pgfqpoint{2.878589in}{5.095492in}}%
\pgfpathlineto{\pgfqpoint{2.879125in}{5.100836in}}%
\pgfpathlineto{\pgfqpoint{2.879661in}{5.075680in}}%
\pgfpathlineto{\pgfqpoint{2.880198in}{5.091607in}}%
\pgfpathlineto{\pgfqpoint{2.881271in}{5.115883in}}%
\pgfpathlineto{\pgfqpoint{2.882344in}{5.073235in}}%
\pgfpathlineto{\pgfqpoint{2.882880in}{5.102538in}}%
\pgfpathlineto{\pgfqpoint{2.882880in}{5.102538in}}%
\pgfpathlineto{\pgfqpoint{2.882880in}{5.102538in}}%
\pgfpathlineto{\pgfqpoint{2.883416in}{5.071692in}}%
\pgfpathlineto{\pgfqpoint{2.883953in}{5.097084in}}%
\pgfpathlineto{\pgfqpoint{2.885562in}{5.071105in}}%
\pgfpathlineto{\pgfqpoint{2.886099in}{5.073656in}}%
\pgfpathlineto{\pgfqpoint{2.886635in}{5.100063in}}%
\pgfpathlineto{\pgfqpoint{2.887172in}{5.074443in}}%
\pgfpathlineto{\pgfqpoint{2.888781in}{5.062802in}}%
\pgfpathlineto{\pgfqpoint{2.889317in}{5.078385in}}%
\pgfpathlineto{\pgfqpoint{2.890927in}{5.048722in}}%
\pgfpathlineto{\pgfqpoint{2.891463in}{5.054348in}}%
\pgfpathlineto{\pgfqpoint{2.891999in}{5.082544in}}%
\pgfpathlineto{\pgfqpoint{2.892536in}{5.065408in}}%
\pgfpathlineto{\pgfqpoint{2.893072in}{5.071887in}}%
\pgfpathlineto{\pgfqpoint{2.893609in}{5.032835in}}%
\pgfpathlineto{\pgfqpoint{2.894145in}{5.034200in}}%
\pgfpathlineto{\pgfqpoint{2.895754in}{5.078727in}}%
\pgfpathlineto{\pgfqpoint{2.896291in}{5.086338in}}%
\pgfpathlineto{\pgfqpoint{2.897364in}{5.023267in}}%
\pgfpathlineto{\pgfqpoint{2.897900in}{5.033194in}}%
\pgfpathlineto{\pgfqpoint{2.898973in}{5.066159in}}%
\pgfpathlineto{\pgfqpoint{2.899510in}{5.049979in}}%
\pgfpathlineto{\pgfqpoint{2.900046in}{5.060404in}}%
\pgfpathlineto{\pgfqpoint{2.901119in}{4.991264in}}%
\pgfpathlineto{\pgfqpoint{2.902192in}{5.058676in}}%
\pgfpathlineto{\pgfqpoint{2.903801in}{5.012607in}}%
\pgfpathlineto{\pgfqpoint{2.904874in}{5.024795in}}%
\pgfpathlineto{\pgfqpoint{2.905410in}{5.068410in}}%
\pgfpathlineto{\pgfqpoint{2.905947in}{5.028801in}}%
\pgfpathlineto{\pgfqpoint{2.907020in}{4.982777in}}%
\pgfpathlineto{\pgfqpoint{2.907556in}{4.997705in}}%
\pgfpathlineto{\pgfqpoint{2.909165in}{5.050625in}}%
\pgfpathlineto{\pgfqpoint{2.909702in}{5.004506in}}%
\pgfpathlineto{\pgfqpoint{2.910238in}{5.022351in}}%
\pgfpathlineto{\pgfqpoint{2.910775in}{5.026676in}}%
\pgfpathlineto{\pgfqpoint{2.912384in}{4.957864in}}%
\pgfpathlineto{\pgfqpoint{2.912920in}{5.012331in}}%
\pgfpathlineto{\pgfqpoint{2.913993in}{5.009670in}}%
\pgfpathlineto{\pgfqpoint{2.914530in}{5.040774in}}%
\pgfpathlineto{\pgfqpoint{2.916139in}{4.943930in}}%
\pgfpathlineto{\pgfqpoint{2.917748in}{5.030160in}}%
\pgfpathlineto{\pgfqpoint{2.918285in}{5.011466in}}%
\pgfpathlineto{\pgfqpoint{2.919358in}{4.925605in}}%
\pgfpathlineto{\pgfqpoint{2.920967in}{5.013713in}}%
\pgfpathlineto{\pgfqpoint{2.921503in}{4.970776in}}%
\pgfpathlineto{\pgfqpoint{2.922576in}{4.917202in}}%
\pgfpathlineto{\pgfqpoint{2.924186in}{5.006188in}}%
\pgfpathlineto{\pgfqpoint{2.924722in}{4.995233in}}%
\pgfpathlineto{\pgfqpoint{2.925795in}{4.883733in}}%
\pgfpathlineto{\pgfqpoint{2.927941in}{5.023541in}}%
\pgfpathlineto{\pgfqpoint{2.929550in}{4.894082in}}%
\pgfpathlineto{\pgfqpoint{2.930623in}{4.950499in}}%
\pgfpathlineto{\pgfqpoint{2.931159in}{4.934098in}}%
\pgfpathlineto{\pgfqpoint{2.931696in}{4.962814in}}%
\pgfpathlineto{\pgfqpoint{2.932232in}{4.917012in}}%
\pgfpathlineto{\pgfqpoint{2.932769in}{4.959497in}}%
\pgfpathlineto{\pgfqpoint{2.933305in}{4.959933in}}%
\pgfpathlineto{\pgfqpoint{2.934914in}{4.855680in}}%
\pgfpathlineto{\pgfqpoint{2.935987in}{4.991428in}}%
\pgfpathlineto{\pgfqpoint{2.936524in}{4.958585in}}%
\pgfpathlineto{\pgfqpoint{2.938133in}{4.798769in}}%
\pgfpathlineto{\pgfqpoint{2.938669in}{4.845263in}}%
\pgfpathlineto{\pgfqpoint{2.939206in}{4.961913in}}%
\pgfpathlineto{\pgfqpoint{2.939742in}{4.957169in}}%
\pgfpathlineto{\pgfqpoint{2.941352in}{4.789408in}}%
\pgfpathlineto{\pgfqpoint{2.941888in}{4.887069in}}%
\pgfpathlineto{\pgfqpoint{2.942961in}{5.013300in}}%
\pgfpathlineto{\pgfqpoint{2.943497in}{4.917286in}}%
\pgfpathlineto{\pgfqpoint{2.945107in}{4.708104in}}%
\pgfpathlineto{\pgfqpoint{2.946179in}{5.049894in}}%
\pgfpathlineto{\pgfqpoint{2.948325in}{4.706143in}}%
\pgfpathlineto{\pgfqpoint{2.949935in}{4.993744in}}%
\pgfpathlineto{\pgfqpoint{2.951007in}{4.737505in}}%
\pgfpathlineto{\pgfqpoint{2.952080in}{4.765319in}}%
\pgfpathlineto{\pgfqpoint{2.952617in}{4.765693in}}%
\pgfpathlineto{\pgfqpoint{2.953153in}{4.914642in}}%
\pgfpathlineto{\pgfqpoint{2.953690in}{4.816138in}}%
\pgfpathlineto{\pgfqpoint{2.954762in}{4.898003in}}%
\pgfpathlineto{\pgfqpoint{2.955299in}{4.808291in}}%
\pgfpathlineto{\pgfqpoint{2.955835in}{4.923318in}}%
\pgfpathlineto{\pgfqpoint{2.956908in}{4.675928in}}%
\pgfpathlineto{\pgfqpoint{2.958518in}{4.919076in}}%
\pgfpathlineto{\pgfqpoint{2.959054in}{4.887995in}}%
\pgfpathlineto{\pgfqpoint{2.960127in}{4.620570in}}%
\pgfpathlineto{\pgfqpoint{2.960663in}{4.639404in}}%
\pgfpathlineto{\pgfqpoint{2.961736in}{5.027295in}}%
\pgfpathlineto{\pgfqpoint{2.963882in}{4.457594in}}%
\pgfpathlineto{\pgfqpoint{2.964955in}{4.980074in}}%
\pgfpathlineto{\pgfqpoint{2.965491in}{4.923928in}}%
\pgfpathlineto{\pgfqpoint{2.966028in}{4.757149in}}%
\pgfpathlineto{\pgfqpoint{2.966564in}{4.214773in}}%
\pgfpathlineto{\pgfqpoint{2.967100in}{4.478798in}}%
\pgfpathlineto{\pgfqpoint{2.968710in}{5.018607in}}%
\pgfpathlineto{\pgfqpoint{2.969246in}{4.921452in}}%
\pgfpathlineto{\pgfqpoint{2.970319in}{4.454995in}}%
\pgfpathlineto{\pgfqpoint{2.971928in}{4.902413in}}%
\pgfpathlineto{\pgfqpoint{2.972465in}{4.877186in}}%
\pgfpathlineto{\pgfqpoint{2.974074in}{4.525646in}}%
\pgfpathlineto{\pgfqpoint{2.975147in}{4.903779in}}%
\pgfpathlineto{\pgfqpoint{2.975683in}{4.682113in}}%
\pgfpathlineto{\pgfqpoint{2.976220in}{4.802805in}}%
\pgfpathlineto{\pgfqpoint{2.977293in}{4.879305in}}%
\pgfpathlineto{\pgfqpoint{2.978902in}{4.557438in}}%
\pgfpathlineto{\pgfqpoint{2.981048in}{4.957273in}}%
\pgfpathlineto{\pgfqpoint{2.982657in}{4.423491in}}%
\pgfpathlineto{\pgfqpoint{2.983730in}{4.975034in}}%
\pgfpathlineto{\pgfqpoint{2.984266in}{4.832986in}}%
\pgfpathlineto{\pgfqpoint{2.984803in}{4.889194in}}%
\pgfpathlineto{\pgfqpoint{2.985876in}{4.219615in}}%
\pgfpathlineto{\pgfqpoint{2.987485in}{4.965087in}}%
\pgfpathlineto{\pgfqpoint{2.988022in}{4.940265in}}%
\pgfpathlineto{\pgfqpoint{2.989631in}{4.150825in}}%
\pgfpathlineto{\pgfqpoint{2.990704in}{5.047759in}}%
\pgfpathlineto{\pgfqpoint{2.991240in}{4.950854in}}%
\pgfpathlineto{\pgfqpoint{2.992849in}{4.370499in}}%
\pgfpathlineto{\pgfqpoint{2.994459in}{4.989018in}}%
\pgfpathlineto{\pgfqpoint{2.994995in}{4.840138in}}%
\pgfpathlineto{\pgfqpoint{2.996068in}{4.587919in}}%
\pgfpathlineto{\pgfqpoint{2.996604in}{4.665885in}}%
\pgfpathlineto{\pgfqpoint{2.998214in}{4.921406in}}%
\pgfpathlineto{\pgfqpoint{2.998750in}{4.908547in}}%
\pgfpathlineto{\pgfqpoint{3.000360in}{4.663719in}}%
\pgfpathlineto{\pgfqpoint{3.001432in}{4.755924in}}%
\pgfpathlineto{\pgfqpoint{3.001969in}{4.540006in}}%
\pgfpathlineto{\pgfqpoint{3.003042in}{4.960697in}}%
\pgfpathlineto{\pgfqpoint{3.005187in}{4.531855in}}%
\pgfpathlineto{\pgfqpoint{3.006797in}{5.036520in}}%
\pgfpathlineto{\pgfqpoint{3.007333in}{4.824001in}}%
\pgfpathlineto{\pgfqpoint{3.008406in}{4.576927in}}%
\pgfpathlineto{\pgfqpoint{3.008943in}{4.617955in}}%
\pgfpathlineto{\pgfqpoint{3.010015in}{4.985767in}}%
\pgfpathlineto{\pgfqpoint{3.010552in}{4.964089in}}%
\pgfpathlineto{\pgfqpoint{3.011625in}{4.433903in}}%
\pgfpathlineto{\pgfqpoint{3.012161in}{4.640797in}}%
\pgfpathlineto{\pgfqpoint{3.013770in}{4.944593in}}%
\pgfpathlineto{\pgfqpoint{3.014307in}{4.893102in}}%
\pgfpathlineto{\pgfqpoint{3.015380in}{4.367075in}}%
\pgfpathlineto{\pgfqpoint{3.016989in}{4.865933in}}%
\pgfpathlineto{\pgfqpoint{3.017525in}{4.923921in}}%
\pgfpathlineto{\pgfqpoint{3.019135in}{4.720985in}}%
\pgfpathlineto{\pgfqpoint{3.019671in}{4.688366in}}%
\pgfpathlineto{\pgfqpoint{3.020208in}{4.921534in}}%
\pgfpathlineto{\pgfqpoint{3.020744in}{4.858324in}}%
\pgfpathlineto{\pgfqpoint{3.021281in}{4.818473in}}%
\pgfpathlineto{\pgfqpoint{3.021817in}{4.854459in}}%
\pgfpathlineto{\pgfqpoint{3.023426in}{4.775445in}}%
\pgfpathlineto{\pgfqpoint{3.023963in}{4.847661in}}%
\pgfpathlineto{\pgfqpoint{3.023963in}{4.847661in}}%
\pgfpathlineto{\pgfqpoint{3.023963in}{4.847661in}}%
\pgfpathlineto{\pgfqpoint{3.024499in}{4.774612in}}%
\pgfpathlineto{\pgfqpoint{3.026108in}{4.942173in}}%
\pgfpathlineto{\pgfqpoint{3.027718in}{4.728245in}}%
\pgfpathlineto{\pgfqpoint{3.029864in}{4.965860in}}%
\pgfpathlineto{\pgfqpoint{3.031473in}{4.702491in}}%
\pgfpathlineto{\pgfqpoint{3.032546in}{5.018237in}}%
\pgfpathlineto{\pgfqpoint{3.033082in}{4.944647in}}%
\pgfpathlineto{\pgfqpoint{3.033619in}{4.936208in}}%
\pgfpathlineto{\pgfqpoint{3.034691in}{4.742476in}}%
\pgfpathlineto{\pgfqpoint{3.035228in}{4.807121in}}%
\pgfpathlineto{\pgfqpoint{3.036301in}{5.031321in}}%
\pgfpathlineto{\pgfqpoint{3.036837in}{4.974749in}}%
\pgfpathlineto{\pgfqpoint{3.038447in}{4.818735in}}%
\pgfpathlineto{\pgfqpoint{3.040056in}{5.017926in}}%
\pgfpathlineto{\pgfqpoint{3.040592in}{4.960048in}}%
\pgfpathlineto{\pgfqpoint{3.042202in}{4.876097in}}%
\pgfpathlineto{\pgfqpoint{3.043811in}{4.980221in}}%
\pgfpathlineto{\pgfqpoint{3.045420in}{4.925986in}}%
\pgfpathlineto{\pgfqpoint{3.045957in}{4.930250in}}%
\pgfpathlineto{\pgfqpoint{3.046493in}{4.952419in}}%
\pgfpathlineto{\pgfqpoint{3.047029in}{4.905774in}}%
\pgfpathlineto{\pgfqpoint{3.048639in}{4.977316in}}%
\pgfpathlineto{\pgfqpoint{3.049712in}{4.929096in}}%
\pgfpathlineto{\pgfqpoint{3.050248in}{4.956769in}}%
\pgfpathlineto{\pgfqpoint{3.050785in}{4.893876in}}%
\pgfpathlineto{\pgfqpoint{3.052394in}{5.006540in}}%
\pgfpathlineto{\pgfqpoint{3.053467in}{4.911052in}}%
\pgfpathlineto{\pgfqpoint{3.054003in}{4.928129in}}%
\pgfpathlineto{\pgfqpoint{3.054540in}{4.914641in}}%
\pgfpathlineto{\pgfqpoint{3.055612in}{5.030692in}}%
\pgfpathlineto{\pgfqpoint{3.056149in}{5.022658in}}%
\pgfpathlineto{\pgfqpoint{3.057222in}{4.909416in}}%
\pgfpathlineto{\pgfqpoint{3.057758in}{4.914944in}}%
\pgfpathlineto{\pgfqpoint{3.058295in}{4.941416in}}%
\pgfpathlineto{\pgfqpoint{3.059368in}{5.039256in}}%
\pgfpathlineto{\pgfqpoint{3.060977in}{4.921232in}}%
\pgfpathlineto{\pgfqpoint{3.061513in}{4.942485in}}%
\pgfpathlineto{\pgfqpoint{3.062586in}{5.041048in}}%
\pgfpathlineto{\pgfqpoint{3.063659in}{5.013611in}}%
\pgfpathlineto{\pgfqpoint{3.064732in}{4.946129in}}%
\pgfpathlineto{\pgfqpoint{3.066341in}{5.058083in}}%
\pgfpathlineto{\pgfqpoint{3.066878in}{5.028693in}}%
\pgfpathlineto{\pgfqpoint{3.068487in}{4.984360in}}%
\pgfpathlineto{\pgfqpoint{3.070096in}{5.056346in}}%
\pgfpathlineto{\pgfqpoint{3.070633in}{5.026798in}}%
\pgfpathlineto{\pgfqpoint{3.071169in}{5.035076in}}%
\pgfpathlineto{\pgfqpoint{3.071706in}{5.033813in}}%
\pgfpathlineto{\pgfqpoint{3.073315in}{4.999227in}}%
\pgfpathlineto{\pgfqpoint{3.074924in}{5.072832in}}%
\pgfpathlineto{\pgfqpoint{3.077070in}{5.009299in}}%
\pgfpathlineto{\pgfqpoint{3.078679in}{5.091280in}}%
\pgfpathlineto{\pgfqpoint{3.080825in}{5.016892in}}%
\pgfpathlineto{\pgfqpoint{3.082434in}{5.101910in}}%
\pgfpathlineto{\pgfqpoint{3.083507in}{5.026400in}}%
\pgfpathlineto{\pgfqpoint{3.084580in}{5.036189in}}%
\pgfpathlineto{\pgfqpoint{3.086189in}{5.107373in}}%
\pgfpathlineto{\pgfqpoint{3.087262in}{5.042581in}}%
\pgfpathlineto{\pgfqpoint{3.087799in}{5.061587in}}%
\pgfpathlineto{\pgfqpoint{3.088335in}{5.052186in}}%
\pgfpathlineto{\pgfqpoint{3.089408in}{5.102958in}}%
\pgfpathlineto{\pgfqpoint{3.089944in}{5.102913in}}%
\pgfpathlineto{\pgfqpoint{3.091017in}{5.061502in}}%
\pgfpathlineto{\pgfqpoint{3.091554in}{5.078171in}}%
\pgfpathlineto{\pgfqpoint{3.092090in}{5.061135in}}%
\pgfpathlineto{\pgfqpoint{3.093699in}{5.107859in}}%
\pgfpathlineto{\pgfqpoint{3.094772in}{5.075965in}}%
\pgfpathlineto{\pgfqpoint{3.095309in}{5.087166in}}%
\pgfpathlineto{\pgfqpoint{3.095845in}{5.073955in}}%
\pgfpathlineto{\pgfqpoint{3.097454in}{5.111349in}}%
\pgfpathlineto{\pgfqpoint{3.098527in}{5.093704in}}%
\pgfpathlineto{\pgfqpoint{3.099064in}{5.104896in}}%
\pgfpathlineto{\pgfqpoint{3.099600in}{5.086037in}}%
\pgfpathlineto{\pgfqpoint{3.101210in}{5.123407in}}%
\pgfpathlineto{\pgfqpoint{3.101746in}{5.107346in}}%
\pgfpathlineto{\pgfqpoint{3.102282in}{5.109914in}}%
\pgfpathlineto{\pgfqpoint{3.102819in}{5.118969in}}%
\pgfpathlineto{\pgfqpoint{3.103355in}{5.100116in}}%
\pgfpathlineto{\pgfqpoint{3.103892in}{5.108820in}}%
\pgfpathlineto{\pgfqpoint{3.104428in}{5.113930in}}%
\pgfpathlineto{\pgfqpoint{3.104965in}{5.140943in}}%
\pgfpathlineto{\pgfqpoint{3.105501in}{5.125453in}}%
\pgfpathlineto{\pgfqpoint{3.107110in}{5.106122in}}%
\pgfpathlineto{\pgfqpoint{3.108720in}{5.154198in}}%
\pgfpathlineto{\pgfqpoint{3.110329in}{5.124328in}}%
\pgfpathlineto{\pgfqpoint{3.110865in}{5.113982in}}%
\pgfpathlineto{\pgfqpoint{3.112475in}{5.155571in}}%
\pgfpathlineto{\pgfqpoint{3.113011in}{5.139130in}}%
\pgfpathlineto{\pgfqpoint{3.114620in}{5.120244in}}%
\pgfpathlineto{\pgfqpoint{3.116230in}{5.159018in}}%
\pgfpathlineto{\pgfqpoint{3.118375in}{5.128765in}}%
\pgfpathlineto{\pgfqpoint{3.119985in}{5.160738in}}%
\pgfpathlineto{\pgfqpoint{3.121058in}{5.142380in}}%
\pgfpathlineto{\pgfqpoint{3.121594in}{5.159361in}}%
\pgfpathlineto{\pgfqpoint{3.122131in}{5.142858in}}%
\pgfpathlineto{\pgfqpoint{3.123740in}{5.167060in}}%
\pgfpathlineto{\pgfqpoint{3.124813in}{5.152366in}}%
\pgfpathlineto{\pgfqpoint{3.125349in}{5.165361in}}%
\pgfpathlineto{\pgfqpoint{3.125886in}{5.153647in}}%
\pgfpathlineto{\pgfqpoint{3.127495in}{5.172632in}}%
\pgfpathlineto{\pgfqpoint{3.128568in}{5.162377in}}%
\pgfpathlineto{\pgfqpoint{3.129104in}{5.175973in}}%
\pgfpathlineto{\pgfqpoint{3.129641in}{5.167290in}}%
\pgfpathlineto{\pgfqpoint{3.130177in}{5.163337in}}%
\pgfpathlineto{\pgfqpoint{3.131250in}{5.174972in}}%
\pgfpathlineto{\pgfqpoint{3.131786in}{5.174036in}}%
\pgfpathlineto{\pgfqpoint{3.132323in}{5.170096in}}%
\pgfpathlineto{\pgfqpoint{3.132859in}{5.183170in}}%
\pgfpathlineto{\pgfqpoint{3.133396in}{5.173149in}}%
\pgfpathlineto{\pgfqpoint{3.134469in}{5.167969in}}%
\pgfpathlineto{\pgfqpoint{3.136078in}{5.179067in}}%
\pgfpathlineto{\pgfqpoint{3.136614in}{5.183983in}}%
\pgfpathlineto{\pgfqpoint{3.137151in}{5.179642in}}%
\pgfpathlineto{\pgfqpoint{3.138224in}{5.169261in}}%
\pgfpathlineto{\pgfqpoint{3.139296in}{5.184388in}}%
\pgfpathlineto{\pgfqpoint{3.139833in}{5.183535in}}%
\pgfpathlineto{\pgfqpoint{3.140369in}{5.189776in}}%
\pgfpathlineto{\pgfqpoint{3.140906in}{5.185758in}}%
\pgfpathlineto{\pgfqpoint{3.141442in}{5.186452in}}%
\pgfpathlineto{\pgfqpoint{3.141979in}{5.179929in}}%
\pgfpathlineto{\pgfqpoint{3.143052in}{5.187728in}}%
\pgfpathlineto{\pgfqpoint{3.143588in}{5.186456in}}%
\pgfpathlineto{\pgfqpoint{3.144124in}{5.197789in}}%
\pgfpathlineto{\pgfqpoint{3.144661in}{5.188742in}}%
\pgfpathlineto{\pgfqpoint{3.145197in}{5.191211in}}%
\pgfpathlineto{\pgfqpoint{3.145734in}{5.187999in}}%
\pgfpathlineto{\pgfqpoint{3.146270in}{5.197865in}}%
\pgfpathlineto{\pgfqpoint{3.146807in}{5.191328in}}%
\pgfpathlineto{\pgfqpoint{3.147343in}{5.189131in}}%
\pgfpathlineto{\pgfqpoint{3.147879in}{5.203717in}}%
\pgfpathlineto{\pgfqpoint{3.148416in}{5.194287in}}%
\pgfpathlineto{\pgfqpoint{3.150562in}{5.200864in}}%
\pgfpathlineto{\pgfqpoint{3.151098in}{5.193096in}}%
\pgfpathlineto{\pgfqpoint{3.151635in}{5.206313in}}%
\pgfpathlineto{\pgfqpoint{3.152171in}{5.196976in}}%
\pgfpathlineto{\pgfqpoint{3.153780in}{5.204692in}}%
\pgfpathlineto{\pgfqpoint{3.154317in}{5.204138in}}%
\pgfpathlineto{\pgfqpoint{3.154853in}{5.199834in}}%
\pgfpathlineto{\pgfqpoint{3.156462in}{5.211992in}}%
\pgfpathlineto{\pgfqpoint{3.156999in}{5.205643in}}%
\pgfpathlineto{\pgfqpoint{3.157535in}{5.207421in}}%
\pgfpathlineto{\pgfqpoint{3.158072in}{5.207845in}}%
\pgfpathlineto{\pgfqpoint{3.158608in}{5.204163in}}%
\pgfpathlineto{\pgfqpoint{3.160218in}{5.211517in}}%
\pgfpathlineto{\pgfqpoint{3.160754in}{5.207889in}}%
\pgfpathlineto{\pgfqpoint{3.161290in}{5.209538in}}%
\pgfpathlineto{\pgfqpoint{3.161827in}{5.211977in}}%
\pgfpathlineto{\pgfqpoint{3.162363in}{5.205815in}}%
\pgfpathlineto{\pgfqpoint{3.162900in}{5.217487in}}%
\pgfpathlineto{\pgfqpoint{3.163436in}{5.210493in}}%
\pgfpathlineto{\pgfqpoint{3.163973in}{5.215939in}}%
\pgfpathlineto{\pgfqpoint{3.164509in}{5.212469in}}%
\pgfpathlineto{\pgfqpoint{3.165045in}{5.214328in}}%
\pgfpathlineto{\pgfqpoint{3.165582in}{5.213429in}}%
\pgfpathlineto{\pgfqpoint{3.166118in}{5.211565in}}%
\pgfpathlineto{\pgfqpoint{3.166655in}{5.224006in}}%
\pgfpathlineto{\pgfqpoint{3.167191in}{5.215745in}}%
\pgfpathlineto{\pgfqpoint{3.167728in}{5.220325in}}%
\pgfpathlineto{\pgfqpoint{3.168264in}{5.217883in}}%
\pgfpathlineto{\pgfqpoint{3.168800in}{5.215456in}}%
\pgfpathlineto{\pgfqpoint{3.169337in}{5.217212in}}%
\pgfpathlineto{\pgfqpoint{3.169873in}{5.219271in}}%
\pgfpathlineto{\pgfqpoint{3.170410in}{5.228430in}}%
\pgfpathlineto{\pgfqpoint{3.170410in}{5.228430in}}%
\pgfpathlineto{\pgfqpoint{3.170410in}{5.228430in}}%
\pgfpathlineto{\pgfqpoint{3.170946in}{5.218831in}}%
\pgfpathlineto{\pgfqpoint{3.171483in}{5.225802in}}%
\pgfpathlineto{\pgfqpoint{3.173092in}{5.217336in}}%
\pgfpathlineto{\pgfqpoint{3.174165in}{5.227864in}}%
\pgfpathlineto{\pgfqpoint{3.174701in}{5.221725in}}%
\pgfpathlineto{\pgfqpoint{3.175238in}{5.230080in}}%
\pgfpathlineto{\pgfqpoint{3.175774in}{5.224262in}}%
\pgfpathlineto{\pgfqpoint{3.176847in}{5.219753in}}%
\pgfpathlineto{\pgfqpoint{3.177920in}{5.233225in}}%
\pgfpathlineto{\pgfqpoint{3.178456in}{5.221539in}}%
\pgfpathlineto{\pgfqpoint{3.178993in}{5.232513in}}%
\pgfpathlineto{\pgfqpoint{3.179529in}{5.223321in}}%
\pgfpathlineto{\pgfqpoint{3.180066in}{5.225322in}}%
\pgfpathlineto{\pgfqpoint{3.181139in}{5.230391in}}%
\pgfpathlineto{\pgfqpoint{3.181675in}{5.235717in}}%
\pgfpathlineto{\pgfqpoint{3.182211in}{5.223984in}}%
\pgfpathlineto{\pgfqpoint{3.182211in}{5.223984in}}%
\pgfpathlineto{\pgfqpoint{3.182211in}{5.223984in}}%
\pgfpathlineto{\pgfqpoint{3.182748in}{5.236062in}}%
\pgfpathlineto{\pgfqpoint{3.183284in}{5.230588in}}%
\pgfpathlineto{\pgfqpoint{3.183821in}{5.230367in}}%
\pgfpathlineto{\pgfqpoint{3.184894in}{5.232268in}}%
\pgfpathlineto{\pgfqpoint{3.185430in}{5.238026in}}%
\pgfpathlineto{\pgfqpoint{3.185966in}{5.227843in}}%
\pgfpathlineto{\pgfqpoint{3.186503in}{5.234742in}}%
\pgfpathlineto{\pgfqpoint{3.188112in}{5.231496in}}%
\pgfpathlineto{\pgfqpoint{3.188649in}{5.236957in}}%
\pgfpathlineto{\pgfqpoint{3.189185in}{5.236773in}}%
\pgfpathlineto{\pgfqpoint{3.189721in}{5.226708in}}%
\pgfpathlineto{\pgfqpoint{3.190258in}{5.232230in}}%
\pgfpathlineto{\pgfqpoint{3.190794in}{5.236864in}}%
\pgfpathlineto{\pgfqpoint{3.191331in}{5.234806in}}%
\pgfpathlineto{\pgfqpoint{3.191867in}{5.232392in}}%
\pgfpathlineto{\pgfqpoint{3.192940in}{5.238313in}}%
\pgfpathlineto{\pgfqpoint{3.193477in}{5.229146in}}%
\pgfpathlineto{\pgfqpoint{3.194013in}{5.235639in}}%
\pgfpathlineto{\pgfqpoint{3.194549in}{5.237328in}}%
\pgfpathlineto{\pgfqpoint{3.195622in}{5.231969in}}%
\pgfpathlineto{\pgfqpoint{3.196695in}{5.241637in}}%
\pgfpathlineto{\pgfqpoint{3.197232in}{5.230866in}}%
\pgfpathlineto{\pgfqpoint{3.197768in}{5.240733in}}%
\pgfpathlineto{\pgfqpoint{3.198304in}{5.241211in}}%
\pgfpathlineto{\pgfqpoint{3.199377in}{5.236478in}}%
\pgfpathlineto{\pgfqpoint{3.200450in}{5.242680in}}%
\pgfpathlineto{\pgfqpoint{3.200987in}{5.231091in}}%
\pgfpathlineto{\pgfqpoint{3.201523in}{5.244550in}}%
\pgfpathlineto{\pgfqpoint{3.202060in}{5.242783in}}%
\pgfpathlineto{\pgfqpoint{3.203132in}{5.237078in}}%
\pgfpathlineto{\pgfqpoint{3.204205in}{5.246199in}}%
\pgfpathlineto{\pgfqpoint{3.204742in}{5.232987in}}%
\pgfpathlineto{\pgfqpoint{3.205278in}{5.240567in}}%
\pgfpathlineto{\pgfqpoint{3.205815in}{5.241323in}}%
\pgfpathlineto{\pgfqpoint{3.206351in}{5.236009in}}%
\pgfpathlineto{\pgfqpoint{3.206887in}{5.239536in}}%
\pgfpathlineto{\pgfqpoint{3.207960in}{5.246140in}}%
\pgfpathlineto{\pgfqpoint{3.208497in}{5.235757in}}%
\pgfpathlineto{\pgfqpoint{3.209033in}{5.242267in}}%
\pgfpathlineto{\pgfqpoint{3.209570in}{5.242609in}}%
\pgfpathlineto{\pgfqpoint{3.210106in}{5.237112in}}%
\pgfpathlineto{\pgfqpoint{3.210643in}{5.238690in}}%
\pgfpathlineto{\pgfqpoint{3.211715in}{5.246807in}}%
\pgfpathlineto{\pgfqpoint{3.212252in}{5.238010in}}%
\pgfpathlineto{\pgfqpoint{3.212788in}{5.240527in}}%
\pgfpathlineto{\pgfqpoint{3.213325in}{5.241722in}}%
\pgfpathlineto{\pgfqpoint{3.213861in}{5.241239in}}%
\pgfpathlineto{\pgfqpoint{3.214398in}{5.236672in}}%
\pgfpathlineto{\pgfqpoint{3.214934in}{5.247215in}}%
\pgfpathlineto{\pgfqpoint{3.215470in}{5.246754in}}%
\pgfpathlineto{\pgfqpoint{3.216007in}{5.238806in}}%
\pgfpathlineto{\pgfqpoint{3.216543in}{5.239930in}}%
\pgfpathlineto{\pgfqpoint{3.217080in}{5.243406in}}%
\pgfpathlineto{\pgfqpoint{3.217616in}{5.241182in}}%
\pgfpathlineto{\pgfqpoint{3.218153in}{5.235490in}}%
\pgfpathlineto{\pgfqpoint{3.218689in}{5.245112in}}%
\pgfpathlineto{\pgfqpoint{3.219225in}{5.242068in}}%
\pgfpathlineto{\pgfqpoint{3.219762in}{5.234876in}}%
\pgfpathlineto{\pgfqpoint{3.220298in}{5.239682in}}%
\pgfpathlineto{\pgfqpoint{3.220835in}{5.243671in}}%
\pgfpathlineto{\pgfqpoint{3.221371in}{5.242345in}}%
\pgfpathlineto{\pgfqpoint{3.221908in}{5.239057in}}%
\pgfpathlineto{\pgfqpoint{3.222444in}{5.245949in}}%
\pgfpathlineto{\pgfqpoint{3.222981in}{5.241795in}}%
\pgfpathlineto{\pgfqpoint{3.223517in}{5.233623in}}%
\pgfpathlineto{\pgfqpoint{3.224053in}{5.234624in}}%
\pgfpathlineto{\pgfqpoint{3.225663in}{5.240975in}}%
\pgfpathlineto{\pgfqpoint{3.226199in}{5.243584in}}%
\pgfpathlineto{\pgfqpoint{3.226736in}{5.241639in}}%
\pgfpathlineto{\pgfqpoint{3.227808in}{5.232852in}}%
\pgfpathlineto{\pgfqpoint{3.228345in}{5.242548in}}%
\pgfpathlineto{\pgfqpoint{3.228881in}{5.238300in}}%
\pgfpathlineto{\pgfqpoint{3.229954in}{5.241149in}}%
\pgfpathlineto{\pgfqpoint{3.231027in}{5.239832in}}%
\pgfpathlineto{\pgfqpoint{3.231564in}{5.231048in}}%
\pgfpathlineto{\pgfqpoint{3.231564in}{5.231048in}}%
\pgfpathlineto{\pgfqpoint{3.231564in}{5.231048in}}%
\pgfpathlineto{\pgfqpoint{3.232100in}{5.240445in}}%
\pgfpathlineto{\pgfqpoint{3.232636in}{5.236078in}}%
\pgfpathlineto{\pgfqpoint{3.233709in}{5.239032in}}%
\pgfpathlineto{\pgfqpoint{3.235319in}{5.233101in}}%
\pgfpathlineto{\pgfqpoint{3.235855in}{5.238681in}}%
\pgfpathlineto{\pgfqpoint{3.236391in}{5.234377in}}%
\pgfpathlineto{\pgfqpoint{3.236928in}{5.237909in}}%
\pgfpathlineto{\pgfqpoint{3.237464in}{5.234533in}}%
\pgfpathlineto{\pgfqpoint{3.239074in}{5.229096in}}%
\pgfpathlineto{\pgfqpoint{3.240683in}{5.239798in}}%
\pgfpathlineto{\pgfqpoint{3.241756in}{5.228002in}}%
\pgfpathlineto{\pgfqpoint{3.242292in}{5.232268in}}%
\pgfpathlineto{\pgfqpoint{3.242829in}{5.229136in}}%
\pgfpathlineto{\pgfqpoint{3.243365in}{5.231062in}}%
\pgfpathlineto{\pgfqpoint{3.244438in}{5.234021in}}%
\pgfpathlineto{\pgfqpoint{3.246584in}{5.226640in}}%
\pgfpathlineto{\pgfqpoint{3.247657in}{5.230317in}}%
\pgfpathlineto{\pgfqpoint{3.248193in}{5.232463in}}%
\pgfpathlineto{\pgfqpoint{3.250339in}{5.222220in}}%
\pgfpathlineto{\pgfqpoint{3.252485in}{5.228949in}}%
\pgfpathlineto{\pgfqpoint{3.254094in}{5.220239in}}%
\pgfpathlineto{\pgfqpoint{3.254630in}{5.219020in}}%
\pgfpathlineto{\pgfqpoint{3.255703in}{5.224609in}}%
\pgfpathlineto{\pgfqpoint{3.256240in}{5.222982in}}%
\pgfpathlineto{\pgfqpoint{3.257849in}{5.212724in}}%
\pgfpathlineto{\pgfqpoint{3.259458in}{5.222346in}}%
\pgfpathlineto{\pgfqpoint{3.261604in}{5.206462in}}%
\pgfpathlineto{\pgfqpoint{3.262677in}{5.218529in}}%
\pgfpathlineto{\pgfqpoint{3.263213in}{5.217236in}}%
\pgfpathlineto{\pgfqpoint{3.264286in}{5.206123in}}%
\pgfpathlineto{\pgfqpoint{3.264823in}{5.212846in}}%
\pgfpathlineto{\pgfqpoint{3.265359in}{5.205466in}}%
\pgfpathlineto{\pgfqpoint{3.265895in}{5.205972in}}%
\pgfpathlineto{\pgfqpoint{3.266968in}{5.215389in}}%
\pgfpathlineto{\pgfqpoint{3.268041in}{5.202587in}}%
\pgfpathlineto{\pgfqpoint{3.268578in}{5.210321in}}%
\pgfpathlineto{\pgfqpoint{3.269650in}{5.195953in}}%
\pgfpathlineto{\pgfqpoint{3.270723in}{5.214587in}}%
\pgfpathlineto{\pgfqpoint{3.273406in}{5.184207in}}%
\pgfpathlineto{\pgfqpoint{3.274478in}{5.209776in}}%
\pgfpathlineto{\pgfqpoint{3.275015in}{5.204439in}}%
\pgfpathlineto{\pgfqpoint{3.276624in}{5.182768in}}%
\pgfpathlineto{\pgfqpoint{3.277161in}{5.183243in}}%
\pgfpathlineto{\pgfqpoint{3.278233in}{5.205595in}}%
\pgfpathlineto{\pgfqpoint{3.278770in}{5.198459in}}%
\pgfpathlineto{\pgfqpoint{3.280379in}{5.177939in}}%
\pgfpathlineto{\pgfqpoint{3.280916in}{5.178400in}}%
\pgfpathlineto{\pgfqpoint{3.281989in}{5.192944in}}%
\pgfpathlineto{\pgfqpoint{3.282525in}{5.188013in}}%
\pgfpathlineto{\pgfqpoint{3.284134in}{5.171936in}}%
\pgfpathlineto{\pgfqpoint{3.284671in}{5.171662in}}%
\pgfpathlineto{\pgfqpoint{3.285744in}{5.191280in}}%
\pgfpathlineto{\pgfqpoint{3.286280in}{5.180358in}}%
\pgfpathlineto{\pgfqpoint{3.288426in}{5.160653in}}%
\pgfpathlineto{\pgfqpoint{3.289499in}{5.180107in}}%
\pgfpathlineto{\pgfqpoint{3.290035in}{5.176067in}}%
\pgfpathlineto{\pgfqpoint{3.291644in}{5.156858in}}%
\pgfpathlineto{\pgfqpoint{3.292181in}{5.158574in}}%
\pgfpathlineto{\pgfqpoint{3.293254in}{5.173592in}}%
\pgfpathlineto{\pgfqpoint{3.293790in}{5.172619in}}%
\pgfpathlineto{\pgfqpoint{3.295399in}{5.150486in}}%
\pgfpathlineto{\pgfqpoint{3.296472in}{5.164911in}}%
\pgfpathlineto{\pgfqpoint{3.297545in}{5.163348in}}%
\pgfpathlineto{\pgfqpoint{3.298082in}{5.149554in}}%
\pgfpathlineto{\pgfqpoint{3.298618in}{5.167758in}}%
\pgfpathlineto{\pgfqpoint{3.299154in}{5.140996in}}%
\pgfpathlineto{\pgfqpoint{3.299691in}{5.150518in}}%
\pgfpathlineto{\pgfqpoint{3.300227in}{5.162377in}}%
\pgfpathlineto{\pgfqpoint{3.300764in}{5.160515in}}%
\pgfpathlineto{\pgfqpoint{3.301837in}{5.140305in}}%
\pgfpathlineto{\pgfqpoint{3.302910in}{5.142015in}}%
\pgfpathlineto{\pgfqpoint{3.303982in}{5.157730in}}%
\pgfpathlineto{\pgfqpoint{3.305592in}{5.127355in}}%
\pgfpathlineto{\pgfqpoint{3.306128in}{5.143558in}}%
\pgfpathlineto{\pgfqpoint{3.306665in}{5.141717in}}%
\pgfpathlineto{\pgfqpoint{3.307201in}{5.135176in}}%
\pgfpathlineto{\pgfqpoint{3.307737in}{5.153838in}}%
\pgfpathlineto{\pgfqpoint{3.308274in}{5.140757in}}%
\pgfpathlineto{\pgfqpoint{3.309347in}{5.118217in}}%
\pgfpathlineto{\pgfqpoint{3.309883in}{5.135225in}}%
\pgfpathlineto{\pgfqpoint{3.310420in}{5.133845in}}%
\pgfpathlineto{\pgfqpoint{3.310956in}{5.133785in}}%
\pgfpathlineto{\pgfqpoint{3.311493in}{5.142671in}}%
\pgfpathlineto{\pgfqpoint{3.313102in}{5.102174in}}%
\pgfpathlineto{\pgfqpoint{3.314711in}{5.130084in}}%
\pgfpathlineto{\pgfqpoint{3.315248in}{5.139451in}}%
\pgfpathlineto{\pgfqpoint{3.316857in}{5.091436in}}%
\pgfpathlineto{\pgfqpoint{3.318466in}{5.123041in}}%
\pgfpathlineto{\pgfqpoint{3.319003in}{5.131019in}}%
\pgfpathlineto{\pgfqpoint{3.320612in}{5.081155in}}%
\pgfpathlineto{\pgfqpoint{3.322221in}{5.106720in}}%
\pgfpathlineto{\pgfqpoint{3.322758in}{5.119710in}}%
\pgfpathlineto{\pgfqpoint{3.323294in}{5.110823in}}%
\pgfpathlineto{\pgfqpoint{3.324367in}{5.077567in}}%
\pgfpathlineto{\pgfqpoint{3.325976in}{5.097267in}}%
\pgfpathlineto{\pgfqpoint{3.326513in}{5.106968in}}%
\pgfpathlineto{\pgfqpoint{3.328122in}{5.073277in}}%
\pgfpathlineto{\pgfqpoint{3.329731in}{5.092878in}}%
\pgfpathlineto{\pgfqpoint{3.330268in}{5.097241in}}%
\pgfpathlineto{\pgfqpoint{3.331877in}{5.065983in}}%
\pgfpathlineto{\pgfqpoint{3.332414in}{5.081802in}}%
\pgfpathlineto{\pgfqpoint{3.332950in}{5.066077in}}%
\pgfpathlineto{\pgfqpoint{3.334023in}{5.088858in}}%
\pgfpathlineto{\pgfqpoint{3.334559in}{5.047877in}}%
\pgfpathlineto{\pgfqpoint{3.335096in}{5.076070in}}%
\pgfpathlineto{\pgfqpoint{3.335632in}{5.059792in}}%
\pgfpathlineto{\pgfqpoint{3.336169in}{5.092891in}}%
\pgfpathlineto{\pgfqpoint{3.336705in}{5.063199in}}%
\pgfpathlineto{\pgfqpoint{3.337778in}{5.069985in}}%
\pgfpathlineto{\pgfqpoint{3.338314in}{5.037972in}}%
\pgfpathlineto{\pgfqpoint{3.338851in}{5.061034in}}%
\pgfpathlineto{\pgfqpoint{3.339387in}{5.048876in}}%
\pgfpathlineto{\pgfqpoint{3.339924in}{5.090066in}}%
\pgfpathlineto{\pgfqpoint{3.340460in}{5.059953in}}%
\pgfpathlineto{\pgfqpoint{3.342069in}{5.016661in}}%
\pgfpathlineto{\pgfqpoint{3.343679in}{5.079266in}}%
\pgfpathlineto{\pgfqpoint{3.345824in}{4.996059in}}%
\pgfpathlineto{\pgfqpoint{3.347434in}{5.057830in}}%
\pgfpathlineto{\pgfqpoint{3.347970in}{5.053481in}}%
\pgfpathlineto{\pgfqpoint{3.349579in}{4.978029in}}%
\pgfpathlineto{\pgfqpoint{3.350116in}{5.014031in}}%
\pgfpathlineto{\pgfqpoint{3.351189in}{5.050195in}}%
\pgfpathlineto{\pgfqpoint{3.351725in}{5.042701in}}%
\pgfpathlineto{\pgfqpoint{3.353335in}{4.970875in}}%
\pgfpathlineto{\pgfqpoint{3.353871in}{4.999205in}}%
\pgfpathlineto{\pgfqpoint{3.354407in}{5.005007in}}%
\pgfpathlineto{\pgfqpoint{3.354944in}{5.047721in}}%
\pgfpathlineto{\pgfqpoint{3.355480in}{5.036495in}}%
\pgfpathlineto{\pgfqpoint{3.357090in}{4.964081in}}%
\pgfpathlineto{\pgfqpoint{3.358699in}{5.019576in}}%
\pgfpathlineto{\pgfqpoint{3.360845in}{4.960485in}}%
\pgfpathlineto{\pgfqpoint{3.361381in}{4.977413in}}%
\pgfpathlineto{\pgfqpoint{3.361918in}{4.966639in}}%
\pgfpathlineto{\pgfqpoint{3.362454in}{4.999348in}}%
\pgfpathlineto{\pgfqpoint{3.362990in}{4.976555in}}%
\pgfpathlineto{\pgfqpoint{3.363527in}{4.969220in}}%
\pgfpathlineto{\pgfqpoint{3.365136in}{4.997497in}}%
\pgfpathlineto{\pgfqpoint{3.366745in}{4.940434in}}%
\pgfpathlineto{\pgfqpoint{3.368891in}{4.996708in}}%
\pgfpathlineto{\pgfqpoint{3.369428in}{4.978152in}}%
\pgfpathlineto{\pgfqpoint{3.370500in}{4.888878in}}%
\pgfpathlineto{\pgfqpoint{3.371037in}{4.939363in}}%
\pgfpathlineto{\pgfqpoint{3.372646in}{5.023908in}}%
\pgfpathlineto{\pgfqpoint{3.373183in}{4.996534in}}%
\pgfpathlineto{\pgfqpoint{3.374792in}{4.887522in}}%
\pgfpathlineto{\pgfqpoint{3.376401in}{5.045760in}}%
\pgfpathlineto{\pgfqpoint{3.376938in}{4.991088in}}%
\pgfpathlineto{\pgfqpoint{3.378547in}{4.821378in}}%
\pgfpathlineto{\pgfqpoint{3.380156in}{5.026166in}}%
\pgfpathlineto{\pgfqpoint{3.380693in}{4.978780in}}%
\pgfpathlineto{\pgfqpoint{3.382302in}{4.763852in}}%
\pgfpathlineto{\pgfqpoint{3.382839in}{4.834859in}}%
\pgfpathlineto{\pgfqpoint{3.384448in}{4.987237in}}%
\pgfpathlineto{\pgfqpoint{3.386057in}{4.728782in}}%
\pgfpathlineto{\pgfqpoint{3.388203in}{4.998275in}}%
\pgfpathlineto{\pgfqpoint{3.389812in}{4.740239in}}%
\pgfpathlineto{\pgfqpoint{3.390349in}{4.830321in}}%
\pgfpathlineto{\pgfqpoint{3.390885in}{4.849351in}}%
\pgfpathlineto{\pgfqpoint{3.391421in}{4.968257in}}%
\pgfpathlineto{\pgfqpoint{3.391958in}{4.950717in}}%
\pgfpathlineto{\pgfqpoint{3.394104in}{4.757408in}}%
\pgfpathlineto{\pgfqpoint{3.395177in}{4.968062in}}%
\pgfpathlineto{\pgfqpoint{3.395713in}{4.928476in}}%
\pgfpathlineto{\pgfqpoint{3.397322in}{4.813334in}}%
\pgfpathlineto{\pgfqpoint{3.398395in}{4.843943in}}%
\pgfpathlineto{\pgfqpoint{3.398932in}{4.925402in}}%
\pgfpathlineto{\pgfqpoint{3.399468in}{4.819197in}}%
\pgfpathlineto{\pgfqpoint{3.400541in}{4.824179in}}%
\pgfpathlineto{\pgfqpoint{3.401077in}{4.789512in}}%
\pgfpathlineto{\pgfqpoint{3.402150in}{4.890850in}}%
\pgfpathlineto{\pgfqpoint{3.403760in}{4.769090in}}%
\pgfpathlineto{\pgfqpoint{3.404832in}{4.712367in}}%
\pgfpathlineto{\pgfqpoint{3.405369in}{4.921691in}}%
\pgfpathlineto{\pgfqpoint{3.405905in}{4.852971in}}%
\pgfpathlineto{\pgfqpoint{3.408051in}{4.523798in}}%
\pgfpathlineto{\pgfqpoint{3.410197in}{4.961242in}}%
\pgfpathlineto{\pgfqpoint{3.411806in}{4.468756in}}%
\pgfpathlineto{\pgfqpoint{3.413952in}{5.024359in}}%
\pgfpathlineto{\pgfqpoint{3.415025in}{4.353794in}}%
\pgfpathlineto{\pgfqpoint{3.416098in}{4.530919in}}%
\pgfpathlineto{\pgfqpoint{3.417707in}{5.094379in}}%
\pgfpathlineto{\pgfqpoint{3.419316in}{4.172217in}}%
\pgfpathlineto{\pgfqpoint{3.419853in}{4.505305in}}%
\pgfpathlineto{\pgfqpoint{3.421462in}{4.990486in}}%
\pgfpathlineto{\pgfqpoint{3.421998in}{4.965712in}}%
\pgfpathlineto{\pgfqpoint{3.423071in}{4.072736in}}%
\pgfpathlineto{\pgfqpoint{3.423608in}{3.280213in}}%
\pgfpathlineto{\pgfqpoint{3.425217in}{4.959058in}}%
\pgfpathlineto{\pgfqpoint{3.425753in}{4.936336in}}%
\pgfpathlineto{\pgfqpoint{3.426290in}{4.776326in}}%
\pgfpathlineto{\pgfqpoint{3.426826in}{4.243016in}}%
\pgfpathlineto{\pgfqpoint{3.427363in}{4.349721in}}%
\pgfpathlineto{\pgfqpoint{3.429508in}{4.954773in}}%
\pgfpathlineto{\pgfqpoint{3.430581in}{4.415633in}}%
\pgfpathlineto{\pgfqpoint{3.431118in}{4.730183in}}%
\pgfpathlineto{\pgfqpoint{3.431654in}{4.466118in}}%
\pgfpathlineto{\pgfqpoint{3.432727in}{4.974646in}}%
\pgfpathlineto{\pgfqpoint{3.433264in}{4.883371in}}%
\pgfpathlineto{\pgfqpoint{3.433800in}{4.545940in}}%
\pgfpathlineto{\pgfqpoint{3.434336in}{4.700560in}}%
\pgfpathlineto{\pgfqpoint{3.435409in}{4.814314in}}%
\pgfpathlineto{\pgfqpoint{3.435946in}{4.615630in}}%
\pgfpathlineto{\pgfqpoint{3.436482in}{4.922208in}}%
\pgfpathlineto{\pgfqpoint{3.437019in}{4.903027in}}%
\pgfpathlineto{\pgfqpoint{3.437555in}{4.647254in}}%
\pgfpathlineto{\pgfqpoint{3.438628in}{4.664642in}}%
\pgfpathlineto{\pgfqpoint{3.439164in}{4.879274in}}%
\pgfpathlineto{\pgfqpoint{3.439701in}{4.879056in}}%
\pgfpathlineto{\pgfqpoint{3.440237in}{4.678436in}}%
\pgfpathlineto{\pgfqpoint{3.440774in}{4.777233in}}%
\pgfpathlineto{\pgfqpoint{3.442383in}{4.638582in}}%
\pgfpathlineto{\pgfqpoint{3.442919in}{4.937507in}}%
\pgfpathlineto{\pgfqpoint{3.443456in}{4.925153in}}%
\pgfpathlineto{\pgfqpoint{3.444529in}{4.472275in}}%
\pgfpathlineto{\pgfqpoint{3.445065in}{4.759521in}}%
\pgfpathlineto{\pgfqpoint{3.445602in}{4.704949in}}%
\pgfpathlineto{\pgfqpoint{3.446138in}{4.668893in}}%
\pgfpathlineto{\pgfqpoint{3.446674in}{4.945450in}}%
\pgfpathlineto{\pgfqpoint{3.447211in}{4.924887in}}%
\pgfpathlineto{\pgfqpoint{3.448820in}{4.462912in}}%
\pgfpathlineto{\pgfqpoint{3.450429in}{4.947055in}}%
\pgfpathlineto{\pgfqpoint{3.452575in}{4.446982in}}%
\pgfpathlineto{\pgfqpoint{3.453112in}{4.462118in}}%
\pgfpathlineto{\pgfqpoint{3.454721in}{4.929207in}}%
\pgfpathlineto{\pgfqpoint{3.455257in}{4.942017in}}%
\pgfpathlineto{\pgfqpoint{3.456330in}{4.633759in}}%
\pgfpathlineto{\pgfqpoint{3.456867in}{4.340167in}}%
\pgfpathlineto{\pgfqpoint{3.459012in}{5.026624in}}%
\pgfpathlineto{\pgfqpoint{3.461158in}{4.396427in}}%
\pgfpathlineto{\pgfqpoint{3.462767in}{5.022343in}}%
\pgfpathlineto{\pgfqpoint{3.464913in}{4.601580in}}%
\pgfpathlineto{\pgfqpoint{3.465450in}{4.643115in}}%
\pgfpathlineto{\pgfqpoint{3.466523in}{5.010392in}}%
\pgfpathlineto{\pgfqpoint{3.467059in}{4.934309in}}%
\pgfpathlineto{\pgfqpoint{3.468132in}{4.710763in}}%
\pgfpathlineto{\pgfqpoint{3.468668in}{4.747669in}}%
\pgfpathlineto{\pgfqpoint{3.469741in}{4.881682in}}%
\pgfpathlineto{\pgfqpoint{3.470814in}{4.984802in}}%
\pgfpathlineto{\pgfqpoint{3.472423in}{4.804680in}}%
\pgfpathlineto{\pgfqpoint{3.474569in}{4.953604in}}%
\pgfpathlineto{\pgfqpoint{3.475642in}{4.855380in}}%
\pgfpathlineto{\pgfqpoint{3.476178in}{4.889812in}}%
\pgfpathlineto{\pgfqpoint{3.476715in}{4.886705in}}%
\pgfpathlineto{\pgfqpoint{3.477251in}{4.900886in}}%
\pgfpathlineto{\pgfqpoint{3.478324in}{4.862932in}}%
\pgfpathlineto{\pgfqpoint{3.478861in}{4.876917in}}%
\pgfpathlineto{\pgfqpoint{3.479397in}{4.873547in}}%
\pgfpathlineto{\pgfqpoint{3.479933in}{4.926740in}}%
\pgfpathlineto{\pgfqpoint{3.480470in}{4.924338in}}%
\pgfpathlineto{\pgfqpoint{3.481543in}{4.907465in}}%
\pgfpathlineto{\pgfqpoint{3.482616in}{4.820121in}}%
\pgfpathlineto{\pgfqpoint{3.484225in}{4.966266in}}%
\pgfpathlineto{\pgfqpoint{3.486371in}{4.814928in}}%
\pgfpathlineto{\pgfqpoint{3.487980in}{4.983616in}}%
\pgfpathlineto{\pgfqpoint{3.490126in}{4.876660in}}%
\pgfpathlineto{\pgfqpoint{3.492271in}{5.001187in}}%
\pgfpathlineto{\pgfqpoint{3.492808in}{4.996016in}}%
\pgfpathlineto{\pgfqpoint{3.493881in}{4.919409in}}%
\pgfpathlineto{\pgfqpoint{3.494417in}{4.925384in}}%
\pgfpathlineto{\pgfqpoint{3.494954in}{4.894289in}}%
\pgfpathlineto{\pgfqpoint{3.496563in}{5.024183in}}%
\pgfpathlineto{\pgfqpoint{3.498709in}{4.887148in}}%
\pgfpathlineto{\pgfqpoint{3.500318in}{5.035265in}}%
\pgfpathlineto{\pgfqpoint{3.500854in}{5.011806in}}%
\pgfpathlineto{\pgfqpoint{3.501391in}{4.998075in}}%
\pgfpathlineto{\pgfqpoint{3.502464in}{4.918515in}}%
\pgfpathlineto{\pgfqpoint{3.504073in}{5.047687in}}%
\pgfpathlineto{\pgfqpoint{3.505146in}{5.033515in}}%
\pgfpathlineto{\pgfqpoint{3.506219in}{4.948973in}}%
\pgfpathlineto{\pgfqpoint{3.506755in}{4.964555in}}%
\pgfpathlineto{\pgfqpoint{3.507292in}{4.986518in}}%
\pgfpathlineto{\pgfqpoint{3.507828in}{5.057361in}}%
\pgfpathlineto{\pgfqpoint{3.508365in}{5.022889in}}%
\pgfpathlineto{\pgfqpoint{3.508901in}{5.043595in}}%
\pgfpathlineto{\pgfqpoint{3.510510in}{4.965808in}}%
\pgfpathlineto{\pgfqpoint{3.512656in}{5.058006in}}%
\pgfpathlineto{\pgfqpoint{3.514265in}{4.998392in}}%
\pgfpathlineto{\pgfqpoint{3.515338in}{5.033037in}}%
\pgfpathlineto{\pgfqpoint{3.515875in}{5.010190in}}%
\pgfpathlineto{\pgfqpoint{3.516411in}{5.052507in}}%
\pgfpathlineto{\pgfqpoint{3.516948in}{5.031322in}}%
\pgfpathlineto{\pgfqpoint{3.517484in}{5.044988in}}%
\pgfpathlineto{\pgfqpoint{3.518020in}{5.032034in}}%
\pgfpathlineto{\pgfqpoint{3.519093in}{5.048630in}}%
\pgfpathlineto{\pgfqpoint{3.519630in}{5.021982in}}%
\pgfpathlineto{\pgfqpoint{3.520166in}{5.043578in}}%
\pgfpathlineto{\pgfqpoint{3.520703in}{5.045990in}}%
\pgfpathlineto{\pgfqpoint{3.521239in}{5.066749in}}%
\pgfpathlineto{\pgfqpoint{3.521775in}{5.046430in}}%
\pgfpathlineto{\pgfqpoint{3.522848in}{5.060708in}}%
\pgfpathlineto{\pgfqpoint{3.523385in}{5.044174in}}%
\pgfpathlineto{\pgfqpoint{3.523921in}{5.046127in}}%
\pgfpathlineto{\pgfqpoint{3.524994in}{5.078059in}}%
\pgfpathlineto{\pgfqpoint{3.525531in}{5.050555in}}%
\pgfpathlineto{\pgfqpoint{3.526067in}{5.070566in}}%
\pgfpathlineto{\pgfqpoint{3.526603in}{5.074706in}}%
\pgfpathlineto{\pgfqpoint{3.526603in}{5.074706in}}%
\pgfpathlineto{\pgfqpoint{3.526603in}{5.074706in}}%
\pgfpathlineto{\pgfqpoint{3.527140in}{5.070335in}}%
\pgfpathlineto{\pgfqpoint{3.527676in}{5.056166in}}%
\pgfpathlineto{\pgfqpoint{3.528213in}{5.067712in}}%
\pgfpathlineto{\pgfqpoint{3.528749in}{5.070744in}}%
\pgfpathlineto{\pgfqpoint{3.529286in}{5.062203in}}%
\pgfpathlineto{\pgfqpoint{3.530895in}{5.091019in}}%
\pgfpathlineto{\pgfqpoint{3.532504in}{5.072116in}}%
\pgfpathlineto{\pgfqpoint{3.533041in}{5.071544in}}%
\pgfpathlineto{\pgfqpoint{3.533577in}{5.073504in}}%
\pgfpathlineto{\pgfqpoint{3.534650in}{5.107257in}}%
\pgfpathlineto{\pgfqpoint{3.536796in}{5.075184in}}%
\pgfpathlineto{\pgfqpoint{3.537332in}{5.068137in}}%
\pgfpathlineto{\pgfqpoint{3.538405in}{5.124397in}}%
\pgfpathlineto{\pgfqpoint{3.538941in}{5.107751in}}%
\pgfpathlineto{\pgfqpoint{3.541087in}{5.073915in}}%
\pgfpathlineto{\pgfqpoint{3.542160in}{5.133330in}}%
\pgfpathlineto{\pgfqpoint{3.542696in}{5.119388in}}%
\pgfpathlineto{\pgfqpoint{3.543233in}{5.118371in}}%
\pgfpathlineto{\pgfqpoint{3.544842in}{5.082718in}}%
\pgfpathlineto{\pgfqpoint{3.546452in}{5.134321in}}%
\pgfpathlineto{\pgfqpoint{3.548597in}{5.099941in}}%
\pgfpathlineto{\pgfqpoint{3.550207in}{5.146216in}}%
\pgfpathlineto{\pgfqpoint{3.550743in}{5.120460in}}%
\pgfpathlineto{\pgfqpoint{3.551816in}{5.120907in}}%
\pgfpathlineto{\pgfqpoint{3.552352in}{5.118784in}}%
\pgfpathlineto{\pgfqpoint{3.553962in}{5.143189in}}%
\pgfpathlineto{\pgfqpoint{3.554498in}{5.127332in}}%
\pgfpathlineto{\pgfqpoint{3.555035in}{5.129836in}}%
\pgfpathlineto{\pgfqpoint{3.556644in}{5.138683in}}%
\pgfpathlineto{\pgfqpoint{3.557180in}{5.127369in}}%
\pgfpathlineto{\pgfqpoint{3.557717in}{5.142032in}}%
\pgfpathlineto{\pgfqpoint{3.558253in}{5.139589in}}%
\pgfpathlineto{\pgfqpoint{3.558790in}{5.136854in}}%
\pgfpathlineto{\pgfqpoint{3.559862in}{5.143928in}}%
\pgfpathlineto{\pgfqpoint{3.560399in}{5.142335in}}%
\pgfpathlineto{\pgfqpoint{3.560935in}{5.137782in}}%
\pgfpathlineto{\pgfqpoint{3.562008in}{5.149415in}}%
\pgfpathlineto{\pgfqpoint{3.562545in}{5.149087in}}%
\pgfpathlineto{\pgfqpoint{3.563081in}{5.141136in}}%
\pgfpathlineto{\pgfqpoint{3.563617in}{5.156645in}}%
\pgfpathlineto{\pgfqpoint{3.564154in}{5.150392in}}%
\pgfpathlineto{\pgfqpoint{3.564690in}{5.149447in}}%
\pgfpathlineto{\pgfqpoint{3.565227in}{5.161293in}}%
\pgfpathlineto{\pgfqpoint{3.565763in}{5.156245in}}%
\pgfpathlineto{\pgfqpoint{3.566300in}{5.158348in}}%
\pgfpathlineto{\pgfqpoint{3.566836in}{5.145609in}}%
\pgfpathlineto{\pgfqpoint{3.567373in}{5.165727in}}%
\pgfpathlineto{\pgfqpoint{3.567909in}{5.163474in}}%
\pgfpathlineto{\pgfqpoint{3.568445in}{5.158583in}}%
\pgfpathlineto{\pgfqpoint{3.569518in}{5.170626in}}%
\pgfpathlineto{\pgfqpoint{3.570591in}{5.152482in}}%
\pgfpathlineto{\pgfqpoint{3.571664in}{5.175264in}}%
\pgfpathlineto{\pgfqpoint{3.572200in}{5.169523in}}%
\pgfpathlineto{\pgfqpoint{3.572737in}{5.171140in}}%
\pgfpathlineto{\pgfqpoint{3.573273in}{5.178511in}}%
\pgfpathlineto{\pgfqpoint{3.573810in}{5.172893in}}%
\pgfpathlineto{\pgfqpoint{3.574346in}{5.162510in}}%
\pgfpathlineto{\pgfqpoint{3.574883in}{5.171542in}}%
\pgfpathlineto{\pgfqpoint{3.575419in}{5.181815in}}%
\pgfpathlineto{\pgfqpoint{3.575956in}{5.177587in}}%
\pgfpathlineto{\pgfqpoint{3.576492in}{5.178181in}}%
\pgfpathlineto{\pgfqpoint{3.577028in}{5.184803in}}%
\pgfpathlineto{\pgfqpoint{3.577565in}{5.179096in}}%
\pgfpathlineto{\pgfqpoint{3.578101in}{5.172518in}}%
\pgfpathlineto{\pgfqpoint{3.579711in}{5.187570in}}%
\pgfpathlineto{\pgfqpoint{3.580247in}{5.183087in}}%
\pgfpathlineto{\pgfqpoint{3.580783in}{5.194197in}}%
\pgfpathlineto{\pgfqpoint{3.581320in}{5.186119in}}%
\pgfpathlineto{\pgfqpoint{3.581856in}{5.184331in}}%
\pgfpathlineto{\pgfqpoint{3.582929in}{5.188753in}}%
\pgfpathlineto{\pgfqpoint{3.583466in}{5.188481in}}%
\pgfpathlineto{\pgfqpoint{3.584002in}{5.190268in}}%
\pgfpathlineto{\pgfqpoint{3.584539in}{5.200490in}}%
\pgfpathlineto{\pgfqpoint{3.585075in}{5.198354in}}%
\pgfpathlineto{\pgfqpoint{3.585611in}{5.191943in}}%
\pgfpathlineto{\pgfqpoint{3.586148in}{5.192725in}}%
\pgfpathlineto{\pgfqpoint{3.586684in}{5.193365in}}%
\pgfpathlineto{\pgfqpoint{3.587221in}{5.189768in}}%
\pgfpathlineto{\pgfqpoint{3.589366in}{5.205714in}}%
\pgfpathlineto{\pgfqpoint{3.590976in}{5.194310in}}%
\pgfpathlineto{\pgfqpoint{3.593121in}{5.212535in}}%
\pgfpathlineto{\pgfqpoint{3.593658in}{5.208738in}}%
\pgfpathlineto{\pgfqpoint{3.594731in}{5.200111in}}%
\pgfpathlineto{\pgfqpoint{3.595267in}{5.200874in}}%
\pgfpathlineto{\pgfqpoint{3.595804in}{5.199852in}}%
\pgfpathlineto{\pgfqpoint{3.597413in}{5.215289in}}%
\pgfpathlineto{\pgfqpoint{3.599022in}{5.204330in}}%
\pgfpathlineto{\pgfqpoint{3.599559in}{5.202696in}}%
\pgfpathlineto{\pgfqpoint{3.601168in}{5.215027in}}%
\pgfpathlineto{\pgfqpoint{3.601704in}{5.204691in}}%
\pgfpathlineto{\pgfqpoint{3.602241in}{5.206992in}}%
\pgfpathlineto{\pgfqpoint{3.602777in}{5.210741in}}%
\pgfpathlineto{\pgfqpoint{3.603314in}{5.206529in}}%
\pgfpathlineto{\pgfqpoint{3.603850in}{5.217619in}}%
\pgfpathlineto{\pgfqpoint{3.604387in}{5.214203in}}%
\pgfpathlineto{\pgfqpoint{3.604923in}{5.215875in}}%
\pgfpathlineto{\pgfqpoint{3.605460in}{5.204484in}}%
\pgfpathlineto{\pgfqpoint{3.605996in}{5.211207in}}%
\pgfpathlineto{\pgfqpoint{3.606532in}{5.212568in}}%
\pgfpathlineto{\pgfqpoint{3.607069in}{5.212289in}}%
\pgfpathlineto{\pgfqpoint{3.607605in}{5.222793in}}%
\pgfpathlineto{\pgfqpoint{3.608142in}{5.217527in}}%
\pgfpathlineto{\pgfqpoint{3.608678in}{5.220978in}}%
\pgfpathlineto{\pgfqpoint{3.609215in}{5.208671in}}%
\pgfpathlineto{\pgfqpoint{3.609751in}{5.216355in}}%
\pgfpathlineto{\pgfqpoint{3.610824in}{5.213090in}}%
\pgfpathlineto{\pgfqpoint{3.611360in}{5.226075in}}%
\pgfpathlineto{\pgfqpoint{3.612433in}{5.225364in}}%
\pgfpathlineto{\pgfqpoint{3.614042in}{5.213863in}}%
\pgfpathlineto{\pgfqpoint{3.614579in}{5.214126in}}%
\pgfpathlineto{\pgfqpoint{3.616188in}{5.230064in}}%
\pgfpathlineto{\pgfqpoint{3.617798in}{5.219196in}}%
\pgfpathlineto{\pgfqpoint{3.618334in}{5.220762in}}%
\pgfpathlineto{\pgfqpoint{3.618870in}{5.235464in}}%
\pgfpathlineto{\pgfqpoint{3.619407in}{5.228869in}}%
\pgfpathlineto{\pgfqpoint{3.619943in}{5.233897in}}%
\pgfpathlineto{\pgfqpoint{3.621553in}{5.222068in}}%
\pgfpathlineto{\pgfqpoint{3.622089in}{5.226657in}}%
\pgfpathlineto{\pgfqpoint{3.622625in}{5.242189in}}%
\pgfpathlineto{\pgfqpoint{3.623162in}{5.230566in}}%
\pgfpathlineto{\pgfqpoint{3.623698in}{5.234258in}}%
\pgfpathlineto{\pgfqpoint{3.624235in}{5.231664in}}%
\pgfpathlineto{\pgfqpoint{3.625308in}{5.226680in}}%
\pgfpathlineto{\pgfqpoint{3.625844in}{5.230054in}}%
\pgfpathlineto{\pgfqpoint{3.626381in}{5.246298in}}%
\pgfpathlineto{\pgfqpoint{3.626917in}{5.232473in}}%
\pgfpathlineto{\pgfqpoint{3.627453in}{5.237523in}}%
\pgfpathlineto{\pgfqpoint{3.627990in}{5.236889in}}%
\pgfpathlineto{\pgfqpoint{3.629599in}{5.230511in}}%
\pgfpathlineto{\pgfqpoint{3.630136in}{5.246239in}}%
\pgfpathlineto{\pgfqpoint{3.630672in}{5.235590in}}%
\pgfpathlineto{\pgfqpoint{3.631208in}{5.242877in}}%
\pgfpathlineto{\pgfqpoint{3.631745in}{5.240475in}}%
\pgfpathlineto{\pgfqpoint{3.633354in}{5.231730in}}%
\pgfpathlineto{\pgfqpoint{3.634964in}{5.247357in}}%
\pgfpathlineto{\pgfqpoint{3.637109in}{5.233479in}}%
\pgfpathlineto{\pgfqpoint{3.638719in}{5.246163in}}%
\pgfpathlineto{\pgfqpoint{3.639791in}{5.235486in}}%
\pgfpathlineto{\pgfqpoint{3.640328in}{5.241041in}}%
\pgfpathlineto{\pgfqpoint{3.640864in}{5.237749in}}%
\pgfpathlineto{\pgfqpoint{3.641937in}{5.246707in}}%
\pgfpathlineto{\pgfqpoint{3.642474in}{5.245883in}}%
\pgfpathlineto{\pgfqpoint{3.643546in}{5.235149in}}%
\pgfpathlineto{\pgfqpoint{3.644083in}{5.246677in}}%
\pgfpathlineto{\pgfqpoint{3.644619in}{5.240986in}}%
\pgfpathlineto{\pgfqpoint{3.645692in}{5.248157in}}%
\pgfpathlineto{\pgfqpoint{3.646229in}{5.244966in}}%
\pgfpathlineto{\pgfqpoint{3.646765in}{5.243776in}}%
\pgfpathlineto{\pgfqpoint{3.647302in}{5.238195in}}%
\pgfpathlineto{\pgfqpoint{3.647838in}{5.248710in}}%
\pgfpathlineto{\pgfqpoint{3.648374in}{5.241177in}}%
\pgfpathlineto{\pgfqpoint{3.650520in}{5.246228in}}%
\pgfpathlineto{\pgfqpoint{3.651057in}{5.240816in}}%
\pgfpathlineto{\pgfqpoint{3.651593in}{5.247232in}}%
\pgfpathlineto{\pgfqpoint{3.652129in}{5.240937in}}%
\pgfpathlineto{\pgfqpoint{3.652666in}{5.239977in}}%
\pgfpathlineto{\pgfqpoint{3.654275in}{5.245800in}}%
\pgfpathlineto{\pgfqpoint{3.656421in}{5.238084in}}%
\pgfpathlineto{\pgfqpoint{3.656957in}{5.247829in}}%
\pgfpathlineto{\pgfqpoint{3.657494in}{5.240594in}}%
\pgfpathlineto{\pgfqpoint{3.658567in}{5.245360in}}%
\pgfpathlineto{\pgfqpoint{3.660176in}{5.238136in}}%
\pgfpathlineto{\pgfqpoint{3.660712in}{5.249062in}}%
\pgfpathlineto{\pgfqpoint{3.661249in}{5.234453in}}%
\pgfpathlineto{\pgfqpoint{3.661785in}{5.244936in}}%
\pgfpathlineto{\pgfqpoint{3.662322in}{5.245881in}}%
\pgfpathlineto{\pgfqpoint{3.663931in}{5.236956in}}%
\pgfpathlineto{\pgfqpoint{3.664467in}{5.245060in}}%
\pgfpathlineto{\pgfqpoint{3.665004in}{5.228726in}}%
\pgfpathlineto{\pgfqpoint{3.665004in}{5.228726in}}%
\pgfpathlineto{\pgfqpoint{3.665004in}{5.228726in}}%
\pgfpathlineto{\pgfqpoint{3.665540in}{5.245930in}}%
\pgfpathlineto{\pgfqpoint{3.666077in}{5.244804in}}%
\pgfpathlineto{\pgfqpoint{3.667686in}{5.235041in}}%
\pgfpathlineto{\pgfqpoint{3.668223in}{5.240655in}}%
\pgfpathlineto{\pgfqpoint{3.668759in}{5.227949in}}%
\pgfpathlineto{\pgfqpoint{3.669832in}{5.245104in}}%
\pgfpathlineto{\pgfqpoint{3.671441in}{5.235825in}}%
\pgfpathlineto{\pgfqpoint{3.671978in}{5.236343in}}%
\pgfpathlineto{\pgfqpoint{3.672514in}{5.228859in}}%
\pgfpathlineto{\pgfqpoint{3.673587in}{5.247224in}}%
\pgfpathlineto{\pgfqpoint{3.675196in}{5.234071in}}%
\pgfpathlineto{\pgfqpoint{3.675733in}{5.233062in}}%
\pgfpathlineto{\pgfqpoint{3.676269in}{5.228922in}}%
\pgfpathlineto{\pgfqpoint{3.677342in}{5.246214in}}%
\pgfpathlineto{\pgfqpoint{3.678951in}{5.230740in}}%
\pgfpathlineto{\pgfqpoint{3.680024in}{5.229369in}}%
\pgfpathlineto{\pgfqpoint{3.681097in}{5.242825in}}%
\pgfpathlineto{\pgfqpoint{3.681633in}{5.230218in}}%
\pgfpathlineto{\pgfqpoint{3.682170in}{5.244619in}}%
\pgfpathlineto{\pgfqpoint{3.682706in}{5.225171in}}%
\pgfpathlineto{\pgfqpoint{3.683243in}{5.229783in}}%
\pgfpathlineto{\pgfqpoint{3.684316in}{5.230569in}}%
\pgfpathlineto{\pgfqpoint{3.685925in}{5.241486in}}%
\pgfpathlineto{\pgfqpoint{3.686461in}{5.225426in}}%
\pgfpathlineto{\pgfqpoint{3.686998in}{5.225916in}}%
\pgfpathlineto{\pgfqpoint{3.688607in}{5.234057in}}%
\pgfpathlineto{\pgfqpoint{3.689144in}{5.229926in}}%
\pgfpathlineto{\pgfqpoint{3.689680in}{5.235809in}}%
\pgfpathlineto{\pgfqpoint{3.690753in}{5.219709in}}%
\pgfpathlineto{\pgfqpoint{3.692362in}{5.228854in}}%
\pgfpathlineto{\pgfqpoint{3.692899in}{5.226640in}}%
\pgfpathlineto{\pgfqpoint{3.693435in}{5.231882in}}%
\pgfpathlineto{\pgfqpoint{3.694508in}{5.216594in}}%
\pgfpathlineto{\pgfqpoint{3.695044in}{5.227440in}}%
\pgfpathlineto{\pgfqpoint{3.695581in}{5.218330in}}%
\pgfpathlineto{\pgfqpoint{3.697190in}{5.227169in}}%
\pgfpathlineto{\pgfqpoint{3.698263in}{5.213072in}}%
\pgfpathlineto{\pgfqpoint{3.698799in}{5.224837in}}%
\pgfpathlineto{\pgfqpoint{3.698799in}{5.224837in}}%
\pgfpathlineto{\pgfqpoint{3.698799in}{5.224837in}}%
\pgfpathlineto{\pgfqpoint{3.699336in}{5.212499in}}%
\pgfpathlineto{\pgfqpoint{3.699872in}{5.221507in}}%
\pgfpathlineto{\pgfqpoint{3.700945in}{5.222160in}}%
\pgfpathlineto{\pgfqpoint{3.702018in}{5.212783in}}%
\pgfpathlineto{\pgfqpoint{3.702554in}{5.221023in}}%
\pgfpathlineto{\pgfqpoint{3.703091in}{5.208172in}}%
\pgfpathlineto{\pgfqpoint{3.703627in}{5.216702in}}%
\pgfpathlineto{\pgfqpoint{3.704164in}{5.222836in}}%
\pgfpathlineto{\pgfqpoint{3.704700in}{5.219606in}}%
\pgfpathlineto{\pgfqpoint{3.705773in}{5.210999in}}%
\pgfpathlineto{\pgfqpoint{3.706310in}{5.218414in}}%
\pgfpathlineto{\pgfqpoint{3.706846in}{5.206654in}}%
\pgfpathlineto{\pgfqpoint{3.707382in}{5.209195in}}%
\pgfpathlineto{\pgfqpoint{3.707919in}{5.220898in}}%
\pgfpathlineto{\pgfqpoint{3.708455in}{5.220056in}}%
\pgfpathlineto{\pgfqpoint{3.709528in}{5.209006in}}%
\pgfpathlineto{\pgfqpoint{3.710065in}{5.217222in}}%
\pgfpathlineto{\pgfqpoint{3.710601in}{5.202410in}}%
\pgfpathlineto{\pgfqpoint{3.711137in}{5.204017in}}%
\pgfpathlineto{\pgfqpoint{3.712210in}{5.215894in}}%
\pgfpathlineto{\pgfqpoint{3.713283in}{5.207218in}}%
\pgfpathlineto{\pgfqpoint{3.713820in}{5.212486in}}%
\pgfpathlineto{\pgfqpoint{3.714356in}{5.199885in}}%
\pgfpathlineto{\pgfqpoint{3.714892in}{5.201695in}}%
\pgfpathlineto{\pgfqpoint{3.715965in}{5.208125in}}%
\pgfpathlineto{\pgfqpoint{3.716502in}{5.207870in}}%
\pgfpathlineto{\pgfqpoint{3.718111in}{5.199931in}}%
\pgfpathlineto{\pgfqpoint{3.718648in}{5.200142in}}%
\pgfpathlineto{\pgfqpoint{3.719184in}{5.200990in}}%
\pgfpathlineto{\pgfqpoint{3.719720in}{5.200182in}}%
\pgfpathlineto{\pgfqpoint{3.720257in}{5.204942in}}%
\pgfpathlineto{\pgfqpoint{3.720793in}{5.203665in}}%
\pgfpathlineto{\pgfqpoint{3.722939in}{5.192204in}}%
\pgfpathlineto{\pgfqpoint{3.723475in}{5.193635in}}%
\pgfpathlineto{\pgfqpoint{3.724012in}{5.194352in}}%
\pgfpathlineto{\pgfqpoint{3.724548in}{5.201903in}}%
\pgfpathlineto{\pgfqpoint{3.725085in}{5.196229in}}%
\pgfpathlineto{\pgfqpoint{3.725621in}{5.194188in}}%
\pgfpathlineto{\pgfqpoint{3.726158in}{5.194976in}}%
\pgfpathlineto{\pgfqpoint{3.726694in}{5.181974in}}%
\pgfpathlineto{\pgfqpoint{3.727231in}{5.187258in}}%
\pgfpathlineto{\pgfqpoint{3.727767in}{5.187220in}}%
\pgfpathlineto{\pgfqpoint{3.729376in}{5.196059in}}%
\pgfpathlineto{\pgfqpoint{3.730449in}{5.173475in}}%
\pgfpathlineto{\pgfqpoint{3.731522in}{5.181686in}}%
\pgfpathlineto{\pgfqpoint{3.733131in}{5.194023in}}%
\pgfpathlineto{\pgfqpoint{3.734204in}{5.169663in}}%
\pgfpathlineto{\pgfqpoint{3.735277in}{5.173872in}}%
\pgfpathlineto{\pgfqpoint{3.736886in}{5.186070in}}%
\pgfpathlineto{\pgfqpoint{3.737423in}{5.180936in}}%
\pgfpathlineto{\pgfqpoint{3.737959in}{5.158705in}}%
\pgfpathlineto{\pgfqpoint{3.738496in}{5.162281in}}%
\pgfpathlineto{\pgfqpoint{3.739569in}{5.175146in}}%
\pgfpathlineto{\pgfqpoint{3.740105in}{5.162854in}}%
\pgfpathlineto{\pgfqpoint{3.740105in}{5.162854in}}%
\pgfpathlineto{\pgfqpoint{3.740105in}{5.162854in}}%
\pgfpathlineto{\pgfqpoint{3.740641in}{5.175606in}}%
\pgfpathlineto{\pgfqpoint{3.741178in}{5.172083in}}%
\pgfpathlineto{\pgfqpoint{3.742251in}{5.150593in}}%
\pgfpathlineto{\pgfqpoint{3.743324in}{5.170062in}}%
\pgfpathlineto{\pgfqpoint{3.743860in}{5.153835in}}%
\pgfpathlineto{\pgfqpoint{3.744933in}{5.154785in}}%
\pgfpathlineto{\pgfqpoint{3.746006in}{5.138003in}}%
\pgfpathlineto{\pgfqpoint{3.747079in}{5.163072in}}%
\pgfpathlineto{\pgfqpoint{3.748688in}{5.145281in}}%
\pgfpathlineto{\pgfqpoint{3.749761in}{5.128308in}}%
\pgfpathlineto{\pgfqpoint{3.750297in}{5.139251in}}%
\pgfpathlineto{\pgfqpoint{3.750834in}{5.154127in}}%
\pgfpathlineto{\pgfqpoint{3.751370in}{5.144359in}}%
\pgfpathlineto{\pgfqpoint{3.751907in}{5.148122in}}%
\pgfpathlineto{\pgfqpoint{3.753516in}{5.124016in}}%
\pgfpathlineto{\pgfqpoint{3.754052in}{5.129433in}}%
\pgfpathlineto{\pgfqpoint{3.754589in}{5.147963in}}%
\pgfpathlineto{\pgfqpoint{3.755125in}{5.133669in}}%
\pgfpathlineto{\pgfqpoint{3.755662in}{5.139592in}}%
\pgfpathlineto{\pgfqpoint{3.757807in}{5.119180in}}%
\pgfpathlineto{\pgfqpoint{3.758344in}{5.135978in}}%
\pgfpathlineto{\pgfqpoint{3.758880in}{5.124988in}}%
\pgfpathlineto{\pgfqpoint{3.759417in}{5.130694in}}%
\pgfpathlineto{\pgfqpoint{3.759953in}{5.116467in}}%
\pgfpathlineto{\pgfqpoint{3.759953in}{5.116467in}}%
\pgfpathlineto{\pgfqpoint{3.759953in}{5.116467in}}%
\pgfpathlineto{\pgfqpoint{3.760490in}{5.131420in}}%
\pgfpathlineto{\pgfqpoint{3.760490in}{5.131420in}}%
\pgfpathlineto{\pgfqpoint{3.760490in}{5.131420in}}%
\pgfpathlineto{\pgfqpoint{3.761026in}{5.116323in}}%
\pgfpathlineto{\pgfqpoint{3.762099in}{5.116566in}}%
\pgfpathlineto{\pgfqpoint{3.763708in}{5.109703in}}%
\pgfpathlineto{\pgfqpoint{3.764245in}{5.126559in}}%
\pgfpathlineto{\pgfqpoint{3.764781in}{5.114739in}}%
\pgfpathlineto{\pgfqpoint{3.766927in}{5.095776in}}%
\pgfpathlineto{\pgfqpoint{3.767463in}{5.094712in}}%
\pgfpathlineto{\pgfqpoint{3.768000in}{5.115886in}}%
\pgfpathlineto{\pgfqpoint{3.768536in}{5.114620in}}%
\pgfpathlineto{\pgfqpoint{3.769073in}{5.109515in}}%
\pgfpathlineto{\pgfqpoint{3.770682in}{5.078778in}}%
\pgfpathlineto{\pgfqpoint{3.771218in}{5.079456in}}%
\pgfpathlineto{\pgfqpoint{3.772291in}{5.114070in}}%
\pgfpathlineto{\pgfqpoint{3.772828in}{5.111490in}}%
\pgfpathlineto{\pgfqpoint{3.774437in}{5.071424in}}%
\pgfpathlineto{\pgfqpoint{3.774973in}{5.058629in}}%
\pgfpathlineto{\pgfqpoint{3.776046in}{5.108360in}}%
\pgfpathlineto{\pgfqpoint{3.776583in}{5.101202in}}%
\pgfpathlineto{\pgfqpoint{3.778728in}{5.048922in}}%
\pgfpathlineto{\pgfqpoint{3.779801in}{5.093902in}}%
\pgfpathlineto{\pgfqpoint{3.781411in}{5.082147in}}%
\pgfpathlineto{\pgfqpoint{3.781947in}{5.076755in}}%
\pgfpathlineto{\pgfqpoint{3.783020in}{5.038955in}}%
\pgfpathlineto{\pgfqpoint{3.784629in}{5.076397in}}%
\pgfpathlineto{\pgfqpoint{3.785166in}{5.083810in}}%
\pgfpathlineto{\pgfqpoint{3.785702in}{5.080348in}}%
\pgfpathlineto{\pgfqpoint{3.786775in}{5.033660in}}%
\pgfpathlineto{\pgfqpoint{3.787311in}{5.047498in}}%
\pgfpathlineto{\pgfqpoint{3.787848in}{5.032038in}}%
\pgfpathlineto{\pgfqpoint{3.789457in}{5.080416in}}%
\pgfpathlineto{\pgfqpoint{3.791603in}{5.005731in}}%
\pgfpathlineto{\pgfqpoint{3.793212in}{5.074034in}}%
\pgfpathlineto{\pgfqpoint{3.794285in}{5.037810in}}%
\pgfpathlineto{\pgfqpoint{3.795358in}{4.999455in}}%
\pgfpathlineto{\pgfqpoint{3.795894in}{5.012454in}}%
\pgfpathlineto{\pgfqpoint{3.796431in}{5.016233in}}%
\pgfpathlineto{\pgfqpoint{3.796967in}{5.058663in}}%
\pgfpathlineto{\pgfqpoint{3.797504in}{5.054404in}}%
\pgfpathlineto{\pgfqpoint{3.798040in}{5.048997in}}%
\pgfpathlineto{\pgfqpoint{3.799649in}{4.998793in}}%
\pgfpathlineto{\pgfqpoint{3.800186in}{4.996367in}}%
\pgfpathlineto{\pgfqpoint{3.801795in}{5.046517in}}%
\pgfpathlineto{\pgfqpoint{3.803941in}{4.970660in}}%
\pgfpathlineto{\pgfqpoint{3.805550in}{5.036343in}}%
\pgfpathlineto{\pgfqpoint{3.806087in}{4.998250in}}%
\pgfpathlineto{\pgfqpoint{3.806623in}{5.030518in}}%
\pgfpathlineto{\pgfqpoint{3.807696in}{4.947716in}}%
\pgfpathlineto{\pgfqpoint{3.808769in}{4.960947in}}%
\pgfpathlineto{\pgfqpoint{3.810378in}{5.028881in}}%
\pgfpathlineto{\pgfqpoint{3.811987in}{4.922692in}}%
\pgfpathlineto{\pgfqpoint{3.812524in}{4.932799in}}%
\pgfpathlineto{\pgfqpoint{3.814133in}{5.017098in}}%
\pgfpathlineto{\pgfqpoint{3.815742in}{4.910493in}}%
\pgfpathlineto{\pgfqpoint{3.816279in}{4.921903in}}%
\pgfpathlineto{\pgfqpoint{3.817888in}{4.991365in}}%
\pgfpathlineto{\pgfqpoint{3.819498in}{4.913556in}}%
\pgfpathlineto{\pgfqpoint{3.820034in}{4.920599in}}%
\pgfpathlineto{\pgfqpoint{3.821643in}{4.954457in}}%
\pgfpathlineto{\pgfqpoint{3.822180in}{4.861125in}}%
\pgfpathlineto{\pgfqpoint{3.822716in}{4.916050in}}%
\pgfpathlineto{\pgfqpoint{3.823253in}{4.912426in}}%
\pgfpathlineto{\pgfqpoint{3.824325in}{4.937161in}}%
\pgfpathlineto{\pgfqpoint{3.825398in}{4.917982in}}%
\pgfpathlineto{\pgfqpoint{3.825935in}{4.822714in}}%
\pgfpathlineto{\pgfqpoint{3.826471in}{4.909518in}}%
\pgfpathlineto{\pgfqpoint{3.828081in}{4.919112in}}%
\pgfpathlineto{\pgfqpoint{3.829153in}{4.890209in}}%
\pgfpathlineto{\pgfqpoint{3.829690in}{4.821598in}}%
\pgfpathlineto{\pgfqpoint{3.830763in}{4.917956in}}%
\pgfpathlineto{\pgfqpoint{3.831836in}{4.852807in}}%
\pgfpathlineto{\pgfqpoint{3.832372in}{4.895371in}}%
\pgfpathlineto{\pgfqpoint{3.832908in}{4.866780in}}%
\pgfpathlineto{\pgfqpoint{3.833445in}{4.846990in}}%
\pgfpathlineto{\pgfqpoint{3.833981in}{4.942530in}}%
\pgfpathlineto{\pgfqpoint{3.834518in}{4.940715in}}%
\pgfpathlineto{\pgfqpoint{3.835591in}{4.775213in}}%
\pgfpathlineto{\pgfqpoint{3.836127in}{4.853165in}}%
\pgfpathlineto{\pgfqpoint{3.836663in}{4.855634in}}%
\pgfpathlineto{\pgfqpoint{3.838273in}{4.963110in}}%
\pgfpathlineto{\pgfqpoint{3.839346in}{4.697769in}}%
\pgfpathlineto{\pgfqpoint{3.839882in}{4.811996in}}%
\pgfpathlineto{\pgfqpoint{3.840419in}{4.818271in}}%
\pgfpathlineto{\pgfqpoint{3.842028in}{4.960961in}}%
\pgfpathlineto{\pgfqpoint{3.842564in}{4.930551in}}%
\pgfpathlineto{\pgfqpoint{3.843101in}{4.711718in}}%
\pgfpathlineto{\pgfqpoint{3.843637in}{4.747953in}}%
\pgfpathlineto{\pgfqpoint{3.845783in}{4.971579in}}%
\pgfpathlineto{\pgfqpoint{3.846319in}{4.939786in}}%
\pgfpathlineto{\pgfqpoint{3.847392in}{4.750690in}}%
\pgfpathlineto{\pgfqpoint{3.847929in}{4.755601in}}%
\pgfpathlineto{\pgfqpoint{3.850074in}{4.937685in}}%
\pgfpathlineto{\pgfqpoint{3.852757in}{4.572085in}}%
\pgfpathlineto{\pgfqpoint{3.853829in}{4.963766in}}%
\pgfpathlineto{\pgfqpoint{3.854366in}{4.933361in}}%
\pgfpathlineto{\pgfqpoint{3.855975in}{4.698281in}}%
\pgfpathlineto{\pgfqpoint{3.856512in}{4.404286in}}%
\pgfpathlineto{\pgfqpoint{3.857048in}{4.535888in}}%
\pgfpathlineto{\pgfqpoint{3.858121in}{4.980978in}}%
\pgfpathlineto{\pgfqpoint{3.860267in}{4.453521in}}%
\pgfpathlineto{\pgfqpoint{3.860803in}{4.244952in}}%
\pgfpathlineto{\pgfqpoint{3.861876in}{4.947980in}}%
\pgfpathlineto{\pgfqpoint{3.862412in}{4.913691in}}%
\pgfpathlineto{\pgfqpoint{3.862949in}{4.943236in}}%
\pgfpathlineto{\pgfqpoint{3.864022in}{4.705479in}}%
\pgfpathlineto{\pgfqpoint{3.864558in}{4.054919in}}%
\pgfpathlineto{\pgfqpoint{3.865095in}{4.182115in}}%
\pgfpathlineto{\pgfqpoint{3.866704in}{4.930222in}}%
\pgfpathlineto{\pgfqpoint{3.867777in}{4.895844in}}%
\pgfpathlineto{\pgfqpoint{3.868850in}{3.719220in}}%
\pgfpathlineto{\pgfqpoint{3.869386in}{4.297505in}}%
\pgfpathlineto{\pgfqpoint{3.870995in}{4.949134in}}%
\pgfpathlineto{\pgfqpoint{3.871532in}{4.999835in}}%
\pgfpathlineto{\pgfqpoint{3.873141in}{4.001873in}}%
\pgfpathlineto{\pgfqpoint{3.873678in}{4.358365in}}%
\pgfpathlineto{\pgfqpoint{3.875287in}{5.039632in}}%
\pgfpathlineto{\pgfqpoint{3.877433in}{4.261796in}}%
\pgfpathlineto{\pgfqpoint{3.877969in}{4.383306in}}%
\pgfpathlineto{\pgfqpoint{3.879042in}{4.999714in}}%
\pgfpathlineto{\pgfqpoint{3.880115in}{4.927852in}}%
\pgfpathlineto{\pgfqpoint{3.882261in}{4.180375in}}%
\pgfpathlineto{\pgfqpoint{3.883333in}{4.991312in}}%
\pgfpathlineto{\pgfqpoint{3.883870in}{4.938021in}}%
\pgfpathlineto{\pgfqpoint{3.884406in}{4.819808in}}%
\pgfpathlineto{\pgfqpoint{3.884943in}{4.825844in}}%
\pgfpathlineto{\pgfqpoint{3.886016in}{3.878049in}}%
\pgfpathlineto{\pgfqpoint{3.886552in}{4.224262in}}%
\pgfpathlineto{\pgfqpoint{3.887625in}{5.002011in}}%
\pgfpathlineto{\pgfqpoint{3.890307in}{4.188230in}}%
\pgfpathlineto{\pgfqpoint{3.891380in}{4.978821in}}%
\pgfpathlineto{\pgfqpoint{3.891916in}{4.895291in}}%
\pgfpathlineto{\pgfqpoint{3.892453in}{4.970427in}}%
\pgfpathlineto{\pgfqpoint{3.894599in}{4.482901in}}%
\pgfpathlineto{\pgfqpoint{3.896208in}{4.944685in}}%
\pgfpathlineto{\pgfqpoint{3.896744in}{4.935079in}}%
\pgfpathlineto{\pgfqpoint{3.897281in}{4.938529in}}%
\pgfpathlineto{\pgfqpoint{3.898890in}{4.382042in}}%
\pgfpathlineto{\pgfqpoint{3.900499in}{4.896411in}}%
\pgfpathlineto{\pgfqpoint{3.901036in}{4.981057in}}%
\pgfpathlineto{\pgfqpoint{3.902645in}{4.432712in}}%
\pgfpathlineto{\pgfqpoint{3.903182in}{4.596053in}}%
\pgfpathlineto{\pgfqpoint{3.904791in}{4.974397in}}%
\pgfpathlineto{\pgfqpoint{3.906937in}{4.679937in}}%
\pgfpathlineto{\pgfqpoint{3.907473in}{4.681440in}}%
\pgfpathlineto{\pgfqpoint{3.908546in}{4.939732in}}%
\pgfpathlineto{\pgfqpoint{3.909619in}{4.939178in}}%
\pgfpathlineto{\pgfqpoint{3.911228in}{4.773041in}}%
\pgfpathlineto{\pgfqpoint{3.911765in}{4.781813in}}%
\pgfpathlineto{\pgfqpoint{3.913374in}{4.928197in}}%
\pgfpathlineto{\pgfqpoint{3.913910in}{4.904620in}}%
\pgfpathlineto{\pgfqpoint{3.914447in}{4.942963in}}%
\pgfpathlineto{\pgfqpoint{3.916056in}{4.727093in}}%
\pgfpathlineto{\pgfqpoint{3.917129in}{4.912957in}}%
\pgfpathlineto{\pgfqpoint{3.917665in}{4.895522in}}%
\pgfpathlineto{\pgfqpoint{3.918202in}{4.954571in}}%
\pgfpathlineto{\pgfqpoint{3.918202in}{4.954571in}}%
\pgfpathlineto{\pgfqpoint{3.918202in}{4.954571in}}%
\pgfpathlineto{\pgfqpoint{3.919811in}{4.735285in}}%
\pgfpathlineto{\pgfqpoint{3.920348in}{4.784395in}}%
\pgfpathlineto{\pgfqpoint{3.921957in}{4.928830in}}%
\pgfpathlineto{\pgfqpoint{3.922493in}{4.917823in}}%
\pgfpathlineto{\pgfqpoint{3.924103in}{4.809445in}}%
\pgfpathlineto{\pgfqpoint{3.925175in}{4.963450in}}%
\pgfpathlineto{\pgfqpoint{3.925712in}{4.900266in}}%
\pgfpathlineto{\pgfqpoint{3.926248in}{4.932570in}}%
\pgfpathlineto{\pgfqpoint{3.926785in}{4.921231in}}%
\pgfpathlineto{\pgfqpoint{3.927321in}{4.917677in}}%
\pgfpathlineto{\pgfqpoint{3.927858in}{4.882291in}}%
\pgfpathlineto{\pgfqpoint{3.928931in}{4.950921in}}%
\pgfpathlineto{\pgfqpoint{3.929467in}{4.900981in}}%
\pgfpathlineto{\pgfqpoint{3.930003in}{4.942106in}}%
\pgfpathlineto{\pgfqpoint{3.931076in}{4.958734in}}%
\pgfpathlineto{\pgfqpoint{3.933222in}{4.890142in}}%
\pgfpathlineto{\pgfqpoint{3.934831in}{4.968173in}}%
\pgfpathlineto{\pgfqpoint{3.935368in}{4.966471in}}%
\pgfpathlineto{\pgfqpoint{3.935904in}{4.958791in}}%
\pgfpathlineto{\pgfqpoint{3.936441in}{4.899977in}}%
\pgfpathlineto{\pgfqpoint{3.936977in}{4.904226in}}%
\pgfpathlineto{\pgfqpoint{3.937513in}{4.916167in}}%
\pgfpathlineto{\pgfqpoint{3.938050in}{4.968939in}}%
\pgfpathlineto{\pgfqpoint{3.938586in}{4.963906in}}%
\pgfpathlineto{\pgfqpoint{3.939123in}{4.963202in}}%
\pgfpathlineto{\pgfqpoint{3.939659in}{4.978997in}}%
\pgfpathlineto{\pgfqpoint{3.941269in}{4.922590in}}%
\pgfpathlineto{\pgfqpoint{3.941805in}{5.003647in}}%
\pgfpathlineto{\pgfqpoint{3.942341in}{4.973018in}}%
\pgfpathlineto{\pgfqpoint{3.942878in}{4.957908in}}%
\pgfpathlineto{\pgfqpoint{3.944487in}{4.994137in}}%
\pgfpathlineto{\pgfqpoint{3.945024in}{4.966095in}}%
\pgfpathlineto{\pgfqpoint{3.945560in}{5.031082in}}%
\pgfpathlineto{\pgfqpoint{3.946096in}{4.976855in}}%
\pgfpathlineto{\pgfqpoint{3.946633in}{4.967798in}}%
\pgfpathlineto{\pgfqpoint{3.948242in}{5.016177in}}%
\pgfpathlineto{\pgfqpoint{3.948779in}{4.997162in}}%
\pgfpathlineto{\pgfqpoint{3.949315in}{5.034374in}}%
\pgfpathlineto{\pgfqpoint{3.949852in}{4.973550in}}%
\pgfpathlineto{\pgfqpoint{3.950388in}{5.000445in}}%
\pgfpathlineto{\pgfqpoint{3.950924in}{4.985532in}}%
\pgfpathlineto{\pgfqpoint{3.951461in}{4.996470in}}%
\pgfpathlineto{\pgfqpoint{3.951997in}{5.025106in}}%
\pgfpathlineto{\pgfqpoint{3.952534in}{5.006762in}}%
\pgfpathlineto{\pgfqpoint{3.953070in}{5.020543in}}%
\pgfpathlineto{\pgfqpoint{3.953607in}{4.979957in}}%
\pgfpathlineto{\pgfqpoint{3.954143in}{5.030912in}}%
\pgfpathlineto{\pgfqpoint{3.954679in}{4.990144in}}%
\pgfpathlineto{\pgfqpoint{3.955752in}{5.019617in}}%
\pgfpathlineto{\pgfqpoint{3.956289in}{5.000468in}}%
\pgfpathlineto{\pgfqpoint{3.956825in}{5.002639in}}%
\pgfpathlineto{\pgfqpoint{3.957362in}{5.002523in}}%
\pgfpathlineto{\pgfqpoint{3.957898in}{5.055804in}}%
\pgfpathlineto{\pgfqpoint{3.958434in}{5.013804in}}%
\pgfpathlineto{\pgfqpoint{3.958971in}{5.035429in}}%
\pgfpathlineto{\pgfqpoint{3.958971in}{5.035429in}}%
\pgfpathlineto{\pgfqpoint{3.958971in}{5.035429in}}%
\pgfpathlineto{\pgfqpoint{3.960044in}{5.000258in}}%
\pgfpathlineto{\pgfqpoint{3.960580in}{5.011138in}}%
\pgfpathlineto{\pgfqpoint{3.961653in}{5.060298in}}%
\pgfpathlineto{\pgfqpoint{3.962726in}{5.054603in}}%
\pgfpathlineto{\pgfqpoint{3.963262in}{5.015280in}}%
\pgfpathlineto{\pgfqpoint{3.963799in}{5.022551in}}%
\pgfpathlineto{\pgfqpoint{3.964335in}{5.019421in}}%
\pgfpathlineto{\pgfqpoint{3.966481in}{5.062773in}}%
\pgfpathlineto{\pgfqpoint{3.967017in}{5.027287in}}%
\pgfpathlineto{\pgfqpoint{3.967554in}{5.047867in}}%
\pgfpathlineto{\pgfqpoint{3.968090in}{5.040215in}}%
\pgfpathlineto{\pgfqpoint{3.968627in}{5.043857in}}%
\pgfpathlineto{\pgfqpoint{3.970236in}{5.065031in}}%
\pgfpathlineto{\pgfqpoint{3.970773in}{5.046906in}}%
\pgfpathlineto{\pgfqpoint{3.971309in}{5.063920in}}%
\pgfpathlineto{\pgfqpoint{3.971845in}{5.063628in}}%
\pgfpathlineto{\pgfqpoint{3.972918in}{5.053071in}}%
\pgfpathlineto{\pgfqpoint{3.975064in}{5.079489in}}%
\pgfpathlineto{\pgfqpoint{3.975600in}{5.090483in}}%
\pgfpathlineto{\pgfqpoint{3.977210in}{5.060425in}}%
\pgfpathlineto{\pgfqpoint{3.977746in}{5.065795in}}%
\pgfpathlineto{\pgfqpoint{3.979356in}{5.113487in}}%
\pgfpathlineto{\pgfqpoint{3.980428in}{5.069102in}}%
\pgfpathlineto{\pgfqpoint{3.981501in}{5.079432in}}%
\pgfpathlineto{\pgfqpoint{3.982574in}{5.095002in}}%
\pgfpathlineto{\pgfqpoint{3.983111in}{5.120540in}}%
\pgfpathlineto{\pgfqpoint{3.983647in}{5.109853in}}%
\pgfpathlineto{\pgfqpoint{3.984720in}{5.087431in}}%
\pgfpathlineto{\pgfqpoint{3.986866in}{5.118235in}}%
\pgfpathlineto{\pgfqpoint{3.988475in}{5.098504in}}%
\pgfpathlineto{\pgfqpoint{3.989011in}{5.119412in}}%
\pgfpathlineto{\pgfqpoint{3.989011in}{5.119412in}}%
\pgfpathlineto{\pgfqpoint{3.989011in}{5.119412in}}%
\pgfpathlineto{\pgfqpoint{3.989548in}{5.097706in}}%
\pgfpathlineto{\pgfqpoint{3.990084in}{5.106648in}}%
\pgfpathlineto{\pgfqpoint{3.991157in}{5.114724in}}%
\pgfpathlineto{\pgfqpoint{3.991694in}{5.138121in}}%
\pgfpathlineto{\pgfqpoint{3.992230in}{5.106115in}}%
\pgfpathlineto{\pgfqpoint{3.992766in}{5.137804in}}%
\pgfpathlineto{\pgfqpoint{3.993839in}{5.113474in}}%
\pgfpathlineto{\pgfqpoint{3.994376in}{5.113642in}}%
\pgfpathlineto{\pgfqpoint{3.994912in}{5.120134in}}%
\pgfpathlineto{\pgfqpoint{3.996521in}{5.144802in}}%
\pgfpathlineto{\pgfqpoint{3.998131in}{5.121426in}}%
\pgfpathlineto{\pgfqpoint{3.998667in}{5.123467in}}%
\pgfpathlineto{\pgfqpoint{4.000277in}{5.142286in}}%
\pgfpathlineto{\pgfqpoint{4.000813in}{5.143377in}}%
\pgfpathlineto{\pgfqpoint{4.001349in}{5.149912in}}%
\pgfpathlineto{\pgfqpoint{4.001886in}{5.131853in}}%
\pgfpathlineto{\pgfqpoint{4.002422in}{5.136830in}}%
\pgfpathlineto{\pgfqpoint{4.002959in}{5.136014in}}%
\pgfpathlineto{\pgfqpoint{4.005104in}{5.165226in}}%
\pgfpathlineto{\pgfqpoint{4.006714in}{5.141634in}}%
\pgfpathlineto{\pgfqpoint{4.008859in}{5.172079in}}%
\pgfpathlineto{\pgfqpoint{4.009932in}{5.149595in}}%
\pgfpathlineto{\pgfqpoint{4.010469in}{5.151035in}}%
\pgfpathlineto{\pgfqpoint{4.011005in}{5.150801in}}%
\pgfpathlineto{\pgfqpoint{4.011542in}{5.151817in}}%
\pgfpathlineto{\pgfqpoint{4.012615in}{5.169792in}}%
\pgfpathlineto{\pgfqpoint{4.013151in}{5.165696in}}%
\pgfpathlineto{\pgfqpoint{4.014224in}{5.156690in}}%
\pgfpathlineto{\pgfqpoint{4.014760in}{5.158320in}}%
\pgfpathlineto{\pgfqpoint{4.015297in}{5.159092in}}%
\pgfpathlineto{\pgfqpoint{4.016906in}{5.169534in}}%
\pgfpathlineto{\pgfqpoint{4.017979in}{5.160405in}}%
\pgfpathlineto{\pgfqpoint{4.019588in}{5.172080in}}%
\pgfpathlineto{\pgfqpoint{4.020125in}{5.166744in}}%
\pgfpathlineto{\pgfqpoint{4.020661in}{5.170717in}}%
\pgfpathlineto{\pgfqpoint{4.021198in}{5.177451in}}%
\pgfpathlineto{\pgfqpoint{4.021734in}{5.165406in}}%
\pgfpathlineto{\pgfqpoint{4.021734in}{5.165406in}}%
\pgfpathlineto{\pgfqpoint{4.021734in}{5.165406in}}%
\pgfpathlineto{\pgfqpoint{4.022807in}{5.184184in}}%
\pgfpathlineto{\pgfqpoint{4.023343in}{5.178414in}}%
\pgfpathlineto{\pgfqpoint{4.023880in}{5.171080in}}%
\pgfpathlineto{\pgfqpoint{4.024416in}{5.174803in}}%
\pgfpathlineto{\pgfqpoint{4.026562in}{5.199494in}}%
\pgfpathlineto{\pgfqpoint{4.028171in}{5.177843in}}%
\pgfpathlineto{\pgfqpoint{4.030317in}{5.209924in}}%
\pgfpathlineto{\pgfqpoint{4.031926in}{5.185805in}}%
\pgfpathlineto{\pgfqpoint{4.034072in}{5.210975in}}%
\pgfpathlineto{\pgfqpoint{4.035681in}{5.196238in}}%
\pgfpathlineto{\pgfqpoint{4.037827in}{5.207568in}}%
\pgfpathlineto{\pgfqpoint{4.038363in}{5.209921in}}%
\pgfpathlineto{\pgfqpoint{4.038900in}{5.218065in}}%
\pgfpathlineto{\pgfqpoint{4.040509in}{5.203579in}}%
\pgfpathlineto{\pgfqpoint{4.042655in}{5.221794in}}%
\pgfpathlineto{\pgfqpoint{4.044264in}{5.209294in}}%
\pgfpathlineto{\pgfqpoint{4.044801in}{5.214609in}}%
\pgfpathlineto{\pgfqpoint{4.045337in}{5.207047in}}%
\pgfpathlineto{\pgfqpoint{4.045874in}{5.208982in}}%
\pgfpathlineto{\pgfqpoint{4.046410in}{5.219548in}}%
\pgfpathlineto{\pgfqpoint{4.046946in}{5.216637in}}%
\pgfpathlineto{\pgfqpoint{4.048019in}{5.214581in}}%
\pgfpathlineto{\pgfqpoint{4.048556in}{5.222453in}}%
\pgfpathlineto{\pgfqpoint{4.049629in}{5.209793in}}%
\pgfpathlineto{\pgfqpoint{4.050165in}{5.218227in}}%
\pgfpathlineto{\pgfqpoint{4.051238in}{5.217922in}}%
\pgfpathlineto{\pgfqpoint{4.051774in}{5.219243in}}%
\pgfpathlineto{\pgfqpoint{4.052311in}{5.224455in}}%
\pgfpathlineto{\pgfqpoint{4.053384in}{5.211921in}}%
\pgfpathlineto{\pgfqpoint{4.053920in}{5.216723in}}%
\pgfpathlineto{\pgfqpoint{4.054993in}{5.219570in}}%
\pgfpathlineto{\pgfqpoint{4.056066in}{5.223655in}}%
\pgfpathlineto{\pgfqpoint{4.057139in}{5.218046in}}%
\pgfpathlineto{\pgfqpoint{4.059284in}{5.227843in}}%
\pgfpathlineto{\pgfqpoint{4.059821in}{5.224232in}}%
\pgfpathlineto{\pgfqpoint{4.060357in}{5.217261in}}%
\pgfpathlineto{\pgfqpoint{4.060894in}{5.225159in}}%
\pgfpathlineto{\pgfqpoint{4.061430in}{5.223938in}}%
\pgfpathlineto{\pgfqpoint{4.061967in}{5.224051in}}%
\pgfpathlineto{\pgfqpoint{4.063040in}{5.231514in}}%
\pgfpathlineto{\pgfqpoint{4.064112in}{5.217824in}}%
\pgfpathlineto{\pgfqpoint{4.064649in}{5.233343in}}%
\pgfpathlineto{\pgfqpoint{4.065185in}{5.225671in}}%
\pgfpathlineto{\pgfqpoint{4.066795in}{5.235609in}}%
\pgfpathlineto{\pgfqpoint{4.067867in}{5.224463in}}%
\pgfpathlineto{\pgfqpoint{4.068404in}{5.239155in}}%
\pgfpathlineto{\pgfqpoint{4.068940in}{5.226158in}}%
\pgfpathlineto{\pgfqpoint{4.070013in}{5.237780in}}%
\pgfpathlineto{\pgfqpoint{4.070550in}{5.237672in}}%
\pgfpathlineto{\pgfqpoint{4.071623in}{5.232125in}}%
\pgfpathlineto{\pgfqpoint{4.072159in}{5.239368in}}%
\pgfpathlineto{\pgfqpoint{4.072695in}{5.228096in}}%
\pgfpathlineto{\pgfqpoint{4.073232in}{5.236262in}}%
\pgfpathlineto{\pgfqpoint{4.073768in}{5.243169in}}%
\pgfpathlineto{\pgfqpoint{4.074305in}{5.236304in}}%
\pgfpathlineto{\pgfqpoint{4.075378in}{5.236528in}}%
\pgfpathlineto{\pgfqpoint{4.075914in}{5.239825in}}%
\pgfpathlineto{\pgfqpoint{4.076450in}{5.231292in}}%
\pgfpathlineto{\pgfqpoint{4.076987in}{5.239516in}}%
\pgfpathlineto{\pgfqpoint{4.077523in}{5.248506in}}%
\pgfpathlineto{\pgfqpoint{4.078060in}{5.235883in}}%
\pgfpathlineto{\pgfqpoint{4.078596in}{5.239737in}}%
\pgfpathlineto{\pgfqpoint{4.080206in}{5.237088in}}%
\pgfpathlineto{\pgfqpoint{4.081278in}{5.249671in}}%
\pgfpathlineto{\pgfqpoint{4.081815in}{5.234244in}}%
\pgfpathlineto{\pgfqpoint{4.082351in}{5.242733in}}%
\pgfpathlineto{\pgfqpoint{4.082888in}{5.238722in}}%
\pgfpathlineto{\pgfqpoint{4.083424in}{5.242153in}}%
\pgfpathlineto{\pgfqpoint{4.083961in}{5.241112in}}%
\pgfpathlineto{\pgfqpoint{4.085033in}{5.248107in}}%
\pgfpathlineto{\pgfqpoint{4.085570in}{5.235630in}}%
\pgfpathlineto{\pgfqpoint{4.086106in}{5.247300in}}%
\pgfpathlineto{\pgfqpoint{4.086643in}{5.237988in}}%
\pgfpathlineto{\pgfqpoint{4.087179in}{5.244141in}}%
\pgfpathlineto{\pgfqpoint{4.087716in}{5.243436in}}%
\pgfpathlineto{\pgfqpoint{4.088252in}{5.246449in}}%
\pgfpathlineto{\pgfqpoint{4.088788in}{5.244038in}}%
\pgfpathlineto{\pgfqpoint{4.089325in}{5.237369in}}%
\pgfpathlineto{\pgfqpoint{4.089861in}{5.248371in}}%
\pgfpathlineto{\pgfqpoint{4.090398in}{5.239351in}}%
\pgfpathlineto{\pgfqpoint{4.092007in}{5.247859in}}%
\pgfpathlineto{\pgfqpoint{4.093080in}{5.238933in}}%
\pgfpathlineto{\pgfqpoint{4.093616in}{5.246743in}}%
\pgfpathlineto{\pgfqpoint{4.094153in}{5.240776in}}%
\pgfpathlineto{\pgfqpoint{4.094689in}{5.240684in}}%
\pgfpathlineto{\pgfqpoint{4.095762in}{5.247880in}}%
\pgfpathlineto{\pgfqpoint{4.096299in}{5.238871in}}%
\pgfpathlineto{\pgfqpoint{4.096835in}{5.238975in}}%
\pgfpathlineto{\pgfqpoint{4.097371in}{5.243888in}}%
\pgfpathlineto{\pgfqpoint{4.097908in}{5.241888in}}%
\pgfpathlineto{\pgfqpoint{4.098444in}{5.238679in}}%
\pgfpathlineto{\pgfqpoint{4.098981in}{5.247226in}}%
\pgfpathlineto{\pgfqpoint{4.099517in}{5.245984in}}%
\pgfpathlineto{\pgfqpoint{4.100054in}{5.238506in}}%
\pgfpathlineto{\pgfqpoint{4.100590in}{5.239219in}}%
\pgfpathlineto{\pgfqpoint{4.101663in}{5.241829in}}%
\pgfpathlineto{\pgfqpoint{4.102199in}{5.238444in}}%
\pgfpathlineto{\pgfqpoint{4.102736in}{5.249298in}}%
\pgfpathlineto{\pgfqpoint{4.103272in}{5.241008in}}%
\pgfpathlineto{\pgfqpoint{4.103809in}{5.239243in}}%
\pgfpathlineto{\pgfqpoint{4.104345in}{5.240041in}}%
\pgfpathlineto{\pgfqpoint{4.105418in}{5.239727in}}%
\pgfpathlineto{\pgfqpoint{4.106491in}{5.248810in}}%
\pgfpathlineto{\pgfqpoint{4.107027in}{5.236271in}}%
\pgfpathlineto{\pgfqpoint{4.107564in}{5.239868in}}%
\pgfpathlineto{\pgfqpoint{4.108100in}{5.241405in}}%
\pgfpathlineto{\pgfqpoint{4.109173in}{5.237940in}}%
\pgfpathlineto{\pgfqpoint{4.110246in}{5.245400in}}%
\pgfpathlineto{\pgfqpoint{4.110782in}{5.234730in}}%
\pgfpathlineto{\pgfqpoint{4.111319in}{5.241545in}}%
\pgfpathlineto{\pgfqpoint{4.111855in}{5.244857in}}%
\pgfpathlineto{\pgfqpoint{4.112392in}{5.237150in}}%
\pgfpathlineto{\pgfqpoint{4.112928in}{5.237306in}}%
\pgfpathlineto{\pgfqpoint{4.113465in}{5.246180in}}%
\pgfpathlineto{\pgfqpoint{4.114001in}{5.241750in}}%
\pgfpathlineto{\pgfqpoint{4.114537in}{5.237566in}}%
\pgfpathlineto{\pgfqpoint{4.115074in}{5.241477in}}%
\pgfpathlineto{\pgfqpoint{4.115610in}{5.246375in}}%
\pgfpathlineto{\pgfqpoint{4.116147in}{5.234843in}}%
\pgfpathlineto{\pgfqpoint{4.116683in}{5.239592in}}%
\pgfpathlineto{\pgfqpoint{4.117220in}{5.245613in}}%
\pgfpathlineto{\pgfqpoint{4.117756in}{5.239047in}}%
\pgfpathlineto{\pgfqpoint{4.118829in}{5.239379in}}%
\pgfpathlineto{\pgfqpoint{4.119365in}{5.246945in}}%
\pgfpathlineto{\pgfqpoint{4.119902in}{5.234275in}}%
\pgfpathlineto{\pgfqpoint{4.120438in}{5.244385in}}%
\pgfpathlineto{\pgfqpoint{4.120975in}{5.243212in}}%
\pgfpathlineto{\pgfqpoint{4.122584in}{5.235063in}}%
\pgfpathlineto{\pgfqpoint{4.123120in}{5.243606in}}%
\pgfpathlineto{\pgfqpoint{4.123120in}{5.243606in}}%
\pgfpathlineto{\pgfqpoint{4.123120in}{5.243606in}}%
\pgfpathlineto{\pgfqpoint{4.123657in}{5.234572in}}%
\pgfpathlineto{\pgfqpoint{4.124193in}{5.248077in}}%
\pgfpathlineto{\pgfqpoint{4.124730in}{5.240528in}}%
\pgfpathlineto{\pgfqpoint{4.126339in}{5.232841in}}%
\pgfpathlineto{\pgfqpoint{4.127948in}{5.247131in}}%
\pgfpathlineto{\pgfqpoint{4.129021in}{5.232317in}}%
\pgfpathlineto{\pgfqpoint{4.129558in}{5.237592in}}%
\pgfpathlineto{\pgfqpoint{4.129558in}{5.237592in}}%
\pgfpathlineto{\pgfqpoint{4.129558in}{5.237592in}}%
\pgfpathlineto{\pgfqpoint{4.130094in}{5.232007in}}%
\pgfpathlineto{\pgfqpoint{4.130631in}{5.233463in}}%
\pgfpathlineto{\pgfqpoint{4.131167in}{5.234160in}}%
\pgfpathlineto{\pgfqpoint{4.131703in}{5.242116in}}%
\pgfpathlineto{\pgfqpoint{4.132776in}{5.225963in}}%
\pgfpathlineto{\pgfqpoint{4.133313in}{5.233367in}}%
\pgfpathlineto{\pgfqpoint{4.133849in}{5.229666in}}%
\pgfpathlineto{\pgfqpoint{4.134386in}{5.229339in}}%
\pgfpathlineto{\pgfqpoint{4.135458in}{5.235947in}}%
\pgfpathlineto{\pgfqpoint{4.136531in}{5.221034in}}%
\pgfpathlineto{\pgfqpoint{4.137068in}{5.229541in}}%
\pgfpathlineto{\pgfqpoint{4.137604in}{5.225417in}}%
\pgfpathlineto{\pgfqpoint{4.139213in}{5.229761in}}%
\pgfpathlineto{\pgfqpoint{4.140286in}{5.217758in}}%
\pgfpathlineto{\pgfqpoint{4.140823in}{5.226972in}}%
\pgfpathlineto{\pgfqpoint{4.140823in}{5.226972in}}%
\pgfpathlineto{\pgfqpoint{4.140823in}{5.226972in}}%
\pgfpathlineto{\pgfqpoint{4.141359in}{5.217236in}}%
\pgfpathlineto{\pgfqpoint{4.141896in}{5.224404in}}%
\pgfpathlineto{\pgfqpoint{4.142432in}{5.225337in}}%
\pgfpathlineto{\pgfqpoint{4.145114in}{5.212239in}}%
\pgfpathlineto{\pgfqpoint{4.146187in}{5.222611in}}%
\pgfpathlineto{\pgfqpoint{4.147796in}{5.215814in}}%
\pgfpathlineto{\pgfqpoint{4.148333in}{5.216189in}}%
\pgfpathlineto{\pgfqpoint{4.148869in}{5.208078in}}%
\pgfpathlineto{\pgfqpoint{4.149942in}{5.220083in}}%
\pgfpathlineto{\pgfqpoint{4.151015in}{5.211722in}}%
\pgfpathlineto{\pgfqpoint{4.151552in}{5.212100in}}%
\pgfpathlineto{\pgfqpoint{4.152624in}{5.208659in}}%
\pgfpathlineto{\pgfqpoint{4.153697in}{5.216132in}}%
\pgfpathlineto{\pgfqpoint{4.155843in}{5.201917in}}%
\pgfpathlineto{\pgfqpoint{4.156916in}{5.213220in}}%
\pgfpathlineto{\pgfqpoint{4.157452in}{5.209541in}}%
\pgfpathlineto{\pgfqpoint{4.159062in}{5.205994in}}%
\pgfpathlineto{\pgfqpoint{4.159598in}{5.198146in}}%
\pgfpathlineto{\pgfqpoint{4.160134in}{5.204112in}}%
\pgfpathlineto{\pgfqpoint{4.160671in}{5.208561in}}%
\pgfpathlineto{\pgfqpoint{4.161207in}{5.201523in}}%
\pgfpathlineto{\pgfqpoint{4.161744in}{5.208108in}}%
\pgfpathlineto{\pgfqpoint{4.162280in}{5.205440in}}%
\pgfpathlineto{\pgfqpoint{4.162817in}{5.211778in}}%
\pgfpathlineto{\pgfqpoint{4.163890in}{5.197081in}}%
\pgfpathlineto{\pgfqpoint{4.164426in}{5.204405in}}%
\pgfpathlineto{\pgfqpoint{4.164962in}{5.194095in}}%
\pgfpathlineto{\pgfqpoint{4.165499in}{5.202283in}}%
\pgfpathlineto{\pgfqpoint{4.166035in}{5.200460in}}%
\pgfpathlineto{\pgfqpoint{4.166572in}{5.212701in}}%
\pgfpathlineto{\pgfqpoint{4.167645in}{5.192709in}}%
\pgfpathlineto{\pgfqpoint{4.168181in}{5.197381in}}%
\pgfpathlineto{\pgfqpoint{4.168717in}{5.187011in}}%
\pgfpathlineto{\pgfqpoint{4.169254in}{5.195894in}}%
\pgfpathlineto{\pgfqpoint{4.169790in}{5.188647in}}%
\pgfpathlineto{\pgfqpoint{4.170327in}{5.203563in}}%
\pgfpathlineto{\pgfqpoint{4.170863in}{5.202754in}}%
\pgfpathlineto{\pgfqpoint{4.172473in}{5.185775in}}%
\pgfpathlineto{\pgfqpoint{4.173009in}{5.188120in}}%
\pgfpathlineto{\pgfqpoint{4.173545in}{5.178134in}}%
\pgfpathlineto{\pgfqpoint{4.175155in}{5.199005in}}%
\pgfpathlineto{\pgfqpoint{4.177300in}{5.167810in}}%
\pgfpathlineto{\pgfqpoint{4.178373in}{5.183993in}}%
\pgfpathlineto{\pgfqpoint{4.178910in}{5.197225in}}%
\pgfpathlineto{\pgfqpoint{4.180519in}{5.173350in}}%
\pgfpathlineto{\pgfqpoint{4.181592in}{5.158674in}}%
\pgfpathlineto{\pgfqpoint{4.182665in}{5.182615in}}%
\pgfpathlineto{\pgfqpoint{4.183201in}{5.171532in}}%
\pgfpathlineto{\pgfqpoint{4.183738in}{5.160612in}}%
\pgfpathlineto{\pgfqpoint{4.184274in}{5.166196in}}%
\pgfpathlineto{\pgfqpoint{4.185347in}{5.153738in}}%
\pgfpathlineto{\pgfqpoint{4.185883in}{5.160999in}}%
\pgfpathlineto{\pgfqpoint{4.186420in}{5.165253in}}%
\pgfpathlineto{\pgfqpoint{4.186420in}{5.165253in}}%
\pgfpathlineto{\pgfqpoint{4.186420in}{5.165253in}}%
\pgfpathlineto{\pgfqpoint{4.187493in}{5.148674in}}%
\pgfpathlineto{\pgfqpoint{4.188029in}{5.157541in}}%
\pgfpathlineto{\pgfqpoint{4.188566in}{5.158001in}}%
\pgfpathlineto{\pgfqpoint{4.190175in}{5.149134in}}%
\pgfpathlineto{\pgfqpoint{4.190711in}{5.149299in}}%
\pgfpathlineto{\pgfqpoint{4.191248in}{5.140951in}}%
\pgfpathlineto{\pgfqpoint{4.191784in}{5.147751in}}%
\pgfpathlineto{\pgfqpoint{4.192857in}{5.156775in}}%
\pgfpathlineto{\pgfqpoint{4.195003in}{5.134355in}}%
\pgfpathlineto{\pgfqpoint{4.196076in}{5.146466in}}%
\pgfpathlineto{\pgfqpoint{4.196612in}{5.158785in}}%
\pgfpathlineto{\pgfqpoint{4.198221in}{5.128673in}}%
\pgfpathlineto{\pgfqpoint{4.198758in}{5.131281in}}%
\pgfpathlineto{\pgfqpoint{4.199294in}{5.127910in}}%
\pgfpathlineto{\pgfqpoint{4.199831in}{5.133302in}}%
\pgfpathlineto{\pgfqpoint{4.200367in}{5.160225in}}%
\pgfpathlineto{\pgfqpoint{4.200904in}{5.135455in}}%
\pgfpathlineto{\pgfqpoint{4.202513in}{5.125336in}}%
\pgfpathlineto{\pgfqpoint{4.203049in}{5.123195in}}%
\pgfpathlineto{\pgfqpoint{4.203586in}{5.124724in}}%
\pgfpathlineto{\pgfqpoint{4.204122in}{5.147562in}}%
\pgfpathlineto{\pgfqpoint{4.204659in}{5.128557in}}%
\pgfpathlineto{\pgfqpoint{4.205195in}{5.130110in}}%
\pgfpathlineto{\pgfqpoint{4.207341in}{5.111187in}}%
\pgfpathlineto{\pgfqpoint{4.207877in}{5.127072in}}%
\pgfpathlineto{\pgfqpoint{4.208950in}{5.126376in}}%
\pgfpathlineto{\pgfqpoint{4.210559in}{5.107517in}}%
\pgfpathlineto{\pgfqpoint{4.211096in}{5.097704in}}%
\pgfpathlineto{\pgfqpoint{4.211632in}{5.106268in}}%
\pgfpathlineto{\pgfqpoint{4.212169in}{5.102129in}}%
\pgfpathlineto{\pgfqpoint{4.212705in}{5.123778in}}%
\pgfpathlineto{\pgfqpoint{4.213242in}{5.097259in}}%
\pgfpathlineto{\pgfqpoint{4.213778in}{5.121692in}}%
\pgfpathlineto{\pgfqpoint{4.215387in}{5.086578in}}%
\pgfpathlineto{\pgfqpoint{4.215924in}{5.087761in}}%
\pgfpathlineto{\pgfqpoint{4.216460in}{5.115359in}}%
\pgfpathlineto{\pgfqpoint{4.216997in}{5.083363in}}%
\pgfpathlineto{\pgfqpoint{4.217533in}{5.112515in}}%
\pgfpathlineto{\pgfqpoint{4.219679in}{5.072193in}}%
\pgfpathlineto{\pgfqpoint{4.220215in}{5.094358in}}%
\pgfpathlineto{\pgfqpoint{4.220215in}{5.094358in}}%
\pgfpathlineto{\pgfqpoint{4.220215in}{5.094358in}}%
\pgfpathlineto{\pgfqpoint{4.220752in}{5.072153in}}%
\pgfpathlineto{\pgfqpoint{4.221288in}{5.088139in}}%
\pgfpathlineto{\pgfqpoint{4.221825in}{5.085415in}}%
\pgfpathlineto{\pgfqpoint{4.222361in}{5.094526in}}%
\pgfpathlineto{\pgfqpoint{4.224507in}{5.061766in}}%
\pgfpathlineto{\pgfqpoint{4.225043in}{5.058675in}}%
\pgfpathlineto{\pgfqpoint{4.226116in}{5.101032in}}%
\pgfpathlineto{\pgfqpoint{4.227725in}{5.052130in}}%
\pgfpathlineto{\pgfqpoint{4.228798in}{5.032908in}}%
\pgfpathlineto{\pgfqpoint{4.229871in}{5.083323in}}%
\pgfpathlineto{\pgfqpoint{4.231481in}{5.034700in}}%
\pgfpathlineto{\pgfqpoint{4.232017in}{5.034730in}}%
\pgfpathlineto{\pgfqpoint{4.232553in}{5.032409in}}%
\pgfpathlineto{\pgfqpoint{4.233626in}{5.054999in}}%
\pgfpathlineto{\pgfqpoint{4.235236in}{5.030015in}}%
\pgfpathlineto{\pgfqpoint{4.235772in}{5.039915in}}%
\pgfpathlineto{\pgfqpoint{4.236308in}{5.037928in}}%
\pgfpathlineto{\pgfqpoint{4.236845in}{5.029664in}}%
\pgfpathlineto{\pgfqpoint{4.237381in}{5.029744in}}%
\pgfpathlineto{\pgfqpoint{4.237918in}{5.032533in}}%
\pgfpathlineto{\pgfqpoint{4.238454in}{5.010088in}}%
\pgfpathlineto{\pgfqpoint{4.238454in}{5.010088in}}%
\pgfpathlineto{\pgfqpoint{4.238454in}{5.010088in}}%
\pgfpathlineto{\pgfqpoint{4.239527in}{5.045513in}}%
\pgfpathlineto{\pgfqpoint{4.240063in}{5.036245in}}%
\pgfpathlineto{\pgfqpoint{4.242209in}{4.999247in}}%
\pgfpathlineto{\pgfqpoint{4.242746in}{5.038634in}}%
\pgfpathlineto{\pgfqpoint{4.243282in}{5.025183in}}%
\pgfpathlineto{\pgfqpoint{4.243819in}{5.031380in}}%
\pgfpathlineto{\pgfqpoint{4.245964in}{4.990791in}}%
\pgfpathlineto{\pgfqpoint{4.246501in}{5.028345in}}%
\pgfpathlineto{\pgfqpoint{4.247037in}{5.000530in}}%
\pgfpathlineto{\pgfqpoint{4.247574in}{5.022731in}}%
\pgfpathlineto{\pgfqpoint{4.248646in}{4.978545in}}%
\pgfpathlineto{\pgfqpoint{4.249183in}{4.981787in}}%
\pgfpathlineto{\pgfqpoint{4.250256in}{5.008410in}}%
\pgfpathlineto{\pgfqpoint{4.250792in}{4.975770in}}%
\pgfpathlineto{\pgfqpoint{4.251329in}{5.011668in}}%
\pgfpathlineto{\pgfqpoint{4.251865in}{4.965572in}}%
\pgfpathlineto{\pgfqpoint{4.252402in}{4.976259in}}%
\pgfpathlineto{\pgfqpoint{4.252938in}{4.977907in}}%
\pgfpathlineto{\pgfqpoint{4.253474in}{4.992884in}}%
\pgfpathlineto{\pgfqpoint{4.254011in}{4.988943in}}%
\pgfpathlineto{\pgfqpoint{4.255620in}{4.943040in}}%
\pgfpathlineto{\pgfqpoint{4.257229in}{5.002913in}}%
\pgfpathlineto{\pgfqpoint{4.259375in}{4.952493in}}%
\pgfpathlineto{\pgfqpoint{4.259912in}{5.001689in}}%
\pgfpathlineto{\pgfqpoint{4.260448in}{4.959260in}}%
\pgfpathlineto{\pgfqpoint{4.260984in}{4.991611in}}%
\pgfpathlineto{\pgfqpoint{4.261521in}{4.964916in}}%
\pgfpathlineto{\pgfqpoint{4.262057in}{4.966034in}}%
\pgfpathlineto{\pgfqpoint{4.262594in}{4.937344in}}%
\pgfpathlineto{\pgfqpoint{4.263130in}{4.949206in}}%
\pgfpathlineto{\pgfqpoint{4.263667in}{5.006165in}}%
\pgfpathlineto{\pgfqpoint{4.264203in}{4.925039in}}%
\pgfpathlineto{\pgfqpoint{4.265276in}{4.929919in}}%
\pgfpathlineto{\pgfqpoint{4.265812in}{4.959503in}}%
\pgfpathlineto{\pgfqpoint{4.266349in}{4.936955in}}%
\pgfpathlineto{\pgfqpoint{4.266885in}{4.936148in}}%
\pgfpathlineto{\pgfqpoint{4.267422in}{4.969573in}}%
\pgfpathlineto{\pgfqpoint{4.268495in}{4.874698in}}%
\pgfpathlineto{\pgfqpoint{4.269031in}{4.878243in}}%
\pgfpathlineto{\pgfqpoint{4.269567in}{4.956827in}}%
\pgfpathlineto{\pgfqpoint{4.270104in}{4.951917in}}%
\pgfpathlineto{\pgfqpoint{4.271713in}{4.880850in}}%
\pgfpathlineto{\pgfqpoint{4.272250in}{4.822752in}}%
\pgfpathlineto{\pgfqpoint{4.272786in}{4.827156in}}%
\pgfpathlineto{\pgfqpoint{4.273859in}{4.951898in}}%
\pgfpathlineto{\pgfqpoint{4.274395in}{4.945032in}}%
\pgfpathlineto{\pgfqpoint{4.276541in}{4.827247in}}%
\pgfpathlineto{\pgfqpoint{4.278150in}{4.949770in}}%
\pgfpathlineto{\pgfqpoint{4.281369in}{4.802493in}}%
\pgfpathlineto{\pgfqpoint{4.282978in}{4.931739in}}%
\pgfpathlineto{\pgfqpoint{4.283515in}{4.921751in}}%
\pgfpathlineto{\pgfqpoint{4.284051in}{4.942721in}}%
\pgfpathlineto{\pgfqpoint{4.285124in}{4.696319in}}%
\pgfpathlineto{\pgfqpoint{4.285661in}{4.733848in}}%
\pgfpathlineto{\pgfqpoint{4.287806in}{4.946153in}}%
\pgfpathlineto{\pgfqpoint{4.289416in}{4.619753in}}%
\pgfpathlineto{\pgfqpoint{4.289952in}{4.688230in}}%
\pgfpathlineto{\pgfqpoint{4.291561in}{4.928546in}}%
\pgfpathlineto{\pgfqpoint{4.292098in}{4.934649in}}%
\pgfpathlineto{\pgfqpoint{4.293707in}{4.643809in}}%
\pgfpathlineto{\pgfqpoint{4.295853in}{4.952736in}}%
\pgfpathlineto{\pgfqpoint{4.297462in}{4.777285in}}%
\pgfpathlineto{\pgfqpoint{4.297999in}{4.593266in}}%
\pgfpathlineto{\pgfqpoint{4.298535in}{4.742843in}}%
\pgfpathlineto{\pgfqpoint{4.299608in}{4.941400in}}%
\pgfpathlineto{\pgfqpoint{4.300144in}{4.920549in}}%
\pgfpathlineto{\pgfqpoint{4.301217in}{4.881492in}}%
\pgfpathlineto{\pgfqpoint{4.302290in}{4.520730in}}%
\pgfpathlineto{\pgfqpoint{4.302827in}{4.551677in}}%
\pgfpathlineto{\pgfqpoint{4.303899in}{4.918849in}}%
\pgfpathlineto{\pgfqpoint{4.304436in}{4.915533in}}%
\pgfpathlineto{\pgfqpoint{4.305509in}{4.806376in}}%
\pgfpathlineto{\pgfqpoint{4.306582in}{4.083162in}}%
\pgfpathlineto{\pgfqpoint{4.308191in}{4.934711in}}%
\pgfpathlineto{\pgfqpoint{4.308727in}{4.864168in}}%
\pgfpathlineto{\pgfqpoint{4.309264in}{4.944906in}}%
\pgfpathlineto{\pgfqpoint{4.309800in}{4.742236in}}%
\pgfpathlineto{\pgfqpoint{4.310337in}{3.724496in}}%
\pgfpathlineto{\pgfqpoint{4.310873in}{4.425991in}}%
\pgfpathlineto{\pgfqpoint{4.313019in}{4.989622in}}%
\pgfpathlineto{\pgfqpoint{4.313555in}{4.964781in}}%
\pgfpathlineto{\pgfqpoint{4.315165in}{4.218489in}}%
\pgfpathlineto{\pgfqpoint{4.316774in}{4.937152in}}%
\pgfpathlineto{\pgfqpoint{4.317310in}{5.002751in}}%
\pgfpathlineto{\pgfqpoint{4.317847in}{4.916775in}}%
\pgfpathlineto{\pgfqpoint{4.318920in}{3.824786in}}%
\pgfpathlineto{\pgfqpoint{4.319456in}{4.284735in}}%
\pgfpathlineto{\pgfqpoint{4.321602in}{5.030801in}}%
\pgfpathlineto{\pgfqpoint{4.323748in}{3.910240in}}%
\pgfpathlineto{\pgfqpoint{4.325357in}{5.003144in}}%
\pgfpathlineto{\pgfqpoint{4.325893in}{4.950921in}}%
\pgfpathlineto{\pgfqpoint{4.326966in}{4.640761in}}%
\pgfpathlineto{\pgfqpoint{4.327503in}{3.959160in}}%
\pgfpathlineto{\pgfqpoint{4.328039in}{4.335159in}}%
\pgfpathlineto{\pgfqpoint{4.329112in}{4.976547in}}%
\pgfpathlineto{\pgfqpoint{4.329648in}{4.974228in}}%
\pgfpathlineto{\pgfqpoint{4.330721in}{4.866635in}}%
\pgfpathlineto{\pgfqpoint{4.331794in}{3.988452in}}%
\pgfpathlineto{\pgfqpoint{4.332330in}{4.247082in}}%
\pgfpathlineto{\pgfqpoint{4.333940in}{4.974424in}}%
\pgfpathlineto{\pgfqpoint{4.335013in}{4.831585in}}%
\pgfpathlineto{\pgfqpoint{4.336086in}{3.774929in}}%
\pgfpathlineto{\pgfqpoint{4.337695in}{4.940264in}}%
\pgfpathlineto{\pgfqpoint{4.338231in}{4.910003in}}%
\pgfpathlineto{\pgfqpoint{4.338768in}{4.910541in}}%
\pgfpathlineto{\pgfqpoint{4.339841in}{4.416933in}}%
\pgfpathlineto{\pgfqpoint{4.340377in}{4.468694in}}%
\pgfpathlineto{\pgfqpoint{4.342523in}{4.945517in}}%
\pgfpathlineto{\pgfqpoint{4.343059in}{4.948578in}}%
\pgfpathlineto{\pgfqpoint{4.344132in}{4.565064in}}%
\pgfpathlineto{\pgfqpoint{4.344669in}{4.614545in}}%
\pgfpathlineto{\pgfqpoint{4.346814in}{4.967965in}}%
\pgfpathlineto{\pgfqpoint{4.347351in}{4.918672in}}%
\pgfpathlineto{\pgfqpoint{4.348424in}{4.613061in}}%
\pgfpathlineto{\pgfqpoint{4.348960in}{4.672423in}}%
\pgfpathlineto{\pgfqpoint{4.349496in}{4.688819in}}%
\pgfpathlineto{\pgfqpoint{4.351106in}{4.940988in}}%
\pgfpathlineto{\pgfqpoint{4.352715in}{4.668650in}}%
\pgfpathlineto{\pgfqpoint{4.353252in}{4.710004in}}%
\pgfpathlineto{\pgfqpoint{4.354861in}{4.936726in}}%
\pgfpathlineto{\pgfqpoint{4.357007in}{4.752605in}}%
\pgfpathlineto{\pgfqpoint{4.358616in}{4.887557in}}%
\pgfpathlineto{\pgfqpoint{4.359152in}{4.883994in}}%
\pgfpathlineto{\pgfqpoint{4.359689in}{4.877436in}}%
\pgfpathlineto{\pgfqpoint{4.360225in}{4.848618in}}%
\pgfpathlineto{\pgfqpoint{4.360762in}{4.884661in}}%
\pgfpathlineto{\pgfqpoint{4.361298in}{4.797507in}}%
\pgfpathlineto{\pgfqpoint{4.361834in}{4.824631in}}%
\pgfpathlineto{\pgfqpoint{4.363980in}{4.916863in}}%
\pgfpathlineto{\pgfqpoint{4.364517in}{4.941582in}}%
\pgfpathlineto{\pgfqpoint{4.366126in}{4.814867in}}%
\pgfpathlineto{\pgfqpoint{4.367199in}{4.939133in}}%
\pgfpathlineto{\pgfqpoint{4.367735in}{4.936331in}}%
\pgfpathlineto{\pgfqpoint{4.368272in}{4.966031in}}%
\pgfpathlineto{\pgfqpoint{4.369345in}{4.849618in}}%
\pgfpathlineto{\pgfqpoint{4.369881in}{4.850093in}}%
\pgfpathlineto{\pgfqpoint{4.371490in}{4.954660in}}%
\pgfpathlineto{\pgfqpoint{4.372027in}{4.947249in}}%
\pgfpathlineto{\pgfqpoint{4.372563in}{4.887317in}}%
\pgfpathlineto{\pgfqpoint{4.373100in}{4.894534in}}%
\pgfpathlineto{\pgfqpoint{4.374709in}{4.966716in}}%
\pgfpathlineto{\pgfqpoint{4.375245in}{4.950308in}}%
\pgfpathlineto{\pgfqpoint{4.376855in}{4.906274in}}%
\pgfpathlineto{\pgfqpoint{4.377928in}{4.981847in}}%
\pgfpathlineto{\pgfqpoint{4.378464in}{4.972961in}}%
\pgfpathlineto{\pgfqpoint{4.379000in}{4.968077in}}%
\pgfpathlineto{\pgfqpoint{4.379537in}{4.904203in}}%
\pgfpathlineto{\pgfqpoint{4.380073in}{4.924762in}}%
\pgfpathlineto{\pgfqpoint{4.380610in}{4.926581in}}%
\pgfpathlineto{\pgfqpoint{4.381683in}{4.995762in}}%
\pgfpathlineto{\pgfqpoint{4.382219in}{4.982816in}}%
\pgfpathlineto{\pgfqpoint{4.382755in}{4.978207in}}%
\pgfpathlineto{\pgfqpoint{4.383292in}{4.899506in}}%
\pgfpathlineto{\pgfqpoint{4.383828in}{4.958931in}}%
\pgfpathlineto{\pgfqpoint{4.384365in}{4.938597in}}%
\pgfpathlineto{\pgfqpoint{4.384901in}{4.955121in}}%
\pgfpathlineto{\pgfqpoint{4.385438in}{4.985023in}}%
\pgfpathlineto{\pgfqpoint{4.386511in}{4.983768in}}%
\pgfpathlineto{\pgfqpoint{4.387047in}{4.937684in}}%
\pgfpathlineto{\pgfqpoint{4.387583in}{4.991532in}}%
\pgfpathlineto{\pgfqpoint{4.388120in}{4.970092in}}%
\pgfpathlineto{\pgfqpoint{4.389193in}{4.956892in}}%
\pgfpathlineto{\pgfqpoint{4.389729in}{4.963748in}}%
\pgfpathlineto{\pgfqpoint{4.391338in}{5.027187in}}%
\pgfpathlineto{\pgfqpoint{4.392948in}{4.948367in}}%
\pgfpathlineto{\pgfqpoint{4.394557in}{4.995939in}}%
\pgfpathlineto{\pgfqpoint{4.395094in}{5.044006in}}%
\pgfpathlineto{\pgfqpoint{4.395630in}{5.010007in}}%
\pgfpathlineto{\pgfqpoint{4.396166in}{5.000424in}}%
\pgfpathlineto{\pgfqpoint{4.396703in}{4.953019in}}%
\pgfpathlineto{\pgfqpoint{4.397239in}{4.962097in}}%
\pgfpathlineto{\pgfqpoint{4.398312in}{5.007391in}}%
\pgfpathlineto{\pgfqpoint{4.398849in}{5.063644in}}%
\pgfpathlineto{\pgfqpoint{4.399385in}{5.025098in}}%
\pgfpathlineto{\pgfqpoint{4.399921in}{5.031535in}}%
\pgfpathlineto{\pgfqpoint{4.400458in}{4.971897in}}%
\pgfpathlineto{\pgfqpoint{4.401531in}{4.973095in}}%
\pgfpathlineto{\pgfqpoint{4.402604in}{5.055715in}}%
\pgfpathlineto{\pgfqpoint{4.403677in}{5.053812in}}%
\pgfpathlineto{\pgfqpoint{4.405286in}{4.984283in}}%
\pgfpathlineto{\pgfqpoint{4.407432in}{5.065838in}}%
\pgfpathlineto{\pgfqpoint{4.409041in}{5.002651in}}%
\pgfpathlineto{\pgfqpoint{4.411187in}{5.065111in}}%
\pgfpathlineto{\pgfqpoint{4.412259in}{5.080591in}}%
\pgfpathlineto{\pgfqpoint{4.412796in}{5.027901in}}%
\pgfpathlineto{\pgfqpoint{4.413332in}{5.047202in}}%
\pgfpathlineto{\pgfqpoint{4.414405in}{5.038209in}}%
\pgfpathlineto{\pgfqpoint{4.416015in}{5.091719in}}%
\pgfpathlineto{\pgfqpoint{4.417624in}{5.047481in}}%
\pgfpathlineto{\pgfqpoint{4.418160in}{5.045243in}}%
\pgfpathlineto{\pgfqpoint{4.419770in}{5.099761in}}%
\pgfpathlineto{\pgfqpoint{4.420842in}{5.094946in}}%
\pgfpathlineto{\pgfqpoint{4.422452in}{5.065165in}}%
\pgfpathlineto{\pgfqpoint{4.424598in}{5.105406in}}%
\pgfpathlineto{\pgfqpoint{4.426207in}{5.072012in}}%
\pgfpathlineto{\pgfqpoint{4.426743in}{5.077930in}}%
\pgfpathlineto{\pgfqpoint{4.427816in}{5.111370in}}%
\pgfpathlineto{\pgfqpoint{4.428353in}{5.105390in}}%
\pgfpathlineto{\pgfqpoint{4.429425in}{5.111433in}}%
\pgfpathlineto{\pgfqpoint{4.430498in}{5.083968in}}%
\pgfpathlineto{\pgfqpoint{4.433180in}{5.128753in}}%
\pgfpathlineto{\pgfqpoint{4.434253in}{5.093444in}}%
\pgfpathlineto{\pgfqpoint{4.434790in}{5.094113in}}%
\pgfpathlineto{\pgfqpoint{4.436399in}{5.121548in}}%
\pgfpathlineto{\pgfqpoint{4.436936in}{5.133846in}}%
\pgfpathlineto{\pgfqpoint{4.438008in}{5.106501in}}%
\pgfpathlineto{\pgfqpoint{4.438545in}{5.108660in}}%
\pgfpathlineto{\pgfqpoint{4.440691in}{5.140210in}}%
\pgfpathlineto{\pgfqpoint{4.442300in}{5.120551in}}%
\pgfpathlineto{\pgfqpoint{4.442836in}{5.131735in}}%
\pgfpathlineto{\pgfqpoint{4.443373in}{5.123575in}}%
\pgfpathlineto{\pgfqpoint{4.443909in}{5.123997in}}%
\pgfpathlineto{\pgfqpoint{4.444446in}{5.142233in}}%
\pgfpathlineto{\pgfqpoint{4.444982in}{5.138713in}}%
\pgfpathlineto{\pgfqpoint{4.445519in}{5.132018in}}%
\pgfpathlineto{\pgfqpoint{4.446055in}{5.136234in}}%
\pgfpathlineto{\pgfqpoint{4.446591in}{5.141673in}}%
\pgfpathlineto{\pgfqpoint{4.447128in}{5.137973in}}%
\pgfpathlineto{\pgfqpoint{4.447664in}{5.132858in}}%
\pgfpathlineto{\pgfqpoint{4.448737in}{5.145814in}}%
\pgfpathlineto{\pgfqpoint{4.449274in}{5.132594in}}%
\pgfpathlineto{\pgfqpoint{4.450883in}{5.153737in}}%
\pgfpathlineto{\pgfqpoint{4.451419in}{5.136260in}}%
\pgfpathlineto{\pgfqpoint{4.451956in}{5.146645in}}%
\pgfpathlineto{\pgfqpoint{4.452492in}{5.149027in}}%
\pgfpathlineto{\pgfqpoint{4.453029in}{5.140927in}}%
\pgfpathlineto{\pgfqpoint{4.454638in}{5.165201in}}%
\pgfpathlineto{\pgfqpoint{4.455174in}{5.146398in}}%
\pgfpathlineto{\pgfqpoint{4.455711in}{5.150206in}}%
\pgfpathlineto{\pgfqpoint{4.456247in}{5.150641in}}%
\pgfpathlineto{\pgfqpoint{4.456784in}{5.148812in}}%
\pgfpathlineto{\pgfqpoint{4.458393in}{5.172062in}}%
\pgfpathlineto{\pgfqpoint{4.458929in}{5.153218in}}%
\pgfpathlineto{\pgfqpoint{4.459466in}{5.154365in}}%
\pgfpathlineto{\pgfqpoint{4.460539in}{5.157502in}}%
\pgfpathlineto{\pgfqpoint{4.462148in}{5.178288in}}%
\pgfpathlineto{\pgfqpoint{4.463221in}{5.158974in}}%
\pgfpathlineto{\pgfqpoint{4.463757in}{5.163603in}}%
\pgfpathlineto{\pgfqpoint{4.464294in}{5.168531in}}%
\pgfpathlineto{\pgfqpoint{4.465903in}{5.186863in}}%
\pgfpathlineto{\pgfqpoint{4.466976in}{5.168981in}}%
\pgfpathlineto{\pgfqpoint{4.467512in}{5.171335in}}%
\pgfpathlineto{\pgfqpoint{4.468585in}{5.193681in}}%
\pgfpathlineto{\pgfqpoint{4.470195in}{5.179560in}}%
\pgfpathlineto{\pgfqpoint{4.470731in}{5.176558in}}%
\pgfpathlineto{\pgfqpoint{4.471267in}{5.177971in}}%
\pgfpathlineto{\pgfqpoint{4.472340in}{5.203463in}}%
\pgfpathlineto{\pgfqpoint{4.472877in}{5.189581in}}%
\pgfpathlineto{\pgfqpoint{4.473413in}{5.196011in}}%
\pgfpathlineto{\pgfqpoint{4.473950in}{5.190813in}}%
\pgfpathlineto{\pgfqpoint{4.474486in}{5.183188in}}%
\pgfpathlineto{\pgfqpoint{4.475023in}{5.187219in}}%
\pgfpathlineto{\pgfqpoint{4.476095in}{5.207270in}}%
\pgfpathlineto{\pgfqpoint{4.477168in}{5.200575in}}%
\pgfpathlineto{\pgfqpoint{4.478241in}{5.188574in}}%
\pgfpathlineto{\pgfqpoint{4.478778in}{5.191555in}}%
\pgfpathlineto{\pgfqpoint{4.480923in}{5.210768in}}%
\pgfpathlineto{\pgfqpoint{4.481996in}{5.198618in}}%
\pgfpathlineto{\pgfqpoint{4.482533in}{5.199086in}}%
\pgfpathlineto{\pgfqpoint{4.483605in}{5.207665in}}%
\pgfpathlineto{\pgfqpoint{4.484142in}{5.205534in}}%
\pgfpathlineto{\pgfqpoint{4.484678in}{5.212797in}}%
\pgfpathlineto{\pgfqpoint{4.485215in}{5.210069in}}%
\pgfpathlineto{\pgfqpoint{4.485751in}{5.202938in}}%
\pgfpathlineto{\pgfqpoint{4.486288in}{5.208473in}}%
\pgfpathlineto{\pgfqpoint{4.486824in}{5.210584in}}%
\pgfpathlineto{\pgfqpoint{4.487361in}{5.208874in}}%
\pgfpathlineto{\pgfqpoint{4.487897in}{5.209371in}}%
\pgfpathlineto{\pgfqpoint{4.488970in}{5.219105in}}%
\pgfpathlineto{\pgfqpoint{4.489506in}{5.211748in}}%
\pgfpathlineto{\pgfqpoint{4.490043in}{5.217638in}}%
\pgfpathlineto{\pgfqpoint{4.491116in}{5.210576in}}%
\pgfpathlineto{\pgfqpoint{4.492725in}{5.221805in}}%
\pgfpathlineto{\pgfqpoint{4.493261in}{5.220586in}}%
\pgfpathlineto{\pgfqpoint{4.493798in}{5.224061in}}%
\pgfpathlineto{\pgfqpoint{4.494871in}{5.211226in}}%
\pgfpathlineto{\pgfqpoint{4.495407in}{5.221984in}}%
\pgfpathlineto{\pgfqpoint{4.495944in}{5.216796in}}%
\pgfpathlineto{\pgfqpoint{4.497553in}{5.231056in}}%
\pgfpathlineto{\pgfqpoint{4.498626in}{5.211612in}}%
\pgfpathlineto{\pgfqpoint{4.499162in}{5.225080in}}%
\pgfpathlineto{\pgfqpoint{4.499699in}{5.217647in}}%
\pgfpathlineto{\pgfqpoint{4.500235in}{5.220070in}}%
\pgfpathlineto{\pgfqpoint{4.501308in}{5.233621in}}%
\pgfpathlineto{\pgfqpoint{4.502381in}{5.213992in}}%
\pgfpathlineto{\pgfqpoint{4.502917in}{5.226579in}}%
\pgfpathlineto{\pgfqpoint{4.503454in}{5.219015in}}%
\pgfpathlineto{\pgfqpoint{4.505063in}{5.233895in}}%
\pgfpathlineto{\pgfqpoint{4.505599in}{5.229089in}}%
\pgfpathlineto{\pgfqpoint{4.506136in}{5.214880in}}%
\pgfpathlineto{\pgfqpoint{4.506672in}{5.231194in}}%
\pgfpathlineto{\pgfqpoint{4.507209in}{5.219451in}}%
\pgfpathlineto{\pgfqpoint{4.508282in}{5.233535in}}%
\pgfpathlineto{\pgfqpoint{4.508818in}{5.232092in}}%
\pgfpathlineto{\pgfqpoint{4.509354in}{5.231773in}}%
\pgfpathlineto{\pgfqpoint{4.509891in}{5.217664in}}%
\pgfpathlineto{\pgfqpoint{4.510427in}{5.233091in}}%
\pgfpathlineto{\pgfqpoint{4.510964in}{5.220734in}}%
\pgfpathlineto{\pgfqpoint{4.512037in}{5.238035in}}%
\pgfpathlineto{\pgfqpoint{4.512573in}{5.232474in}}%
\pgfpathlineto{\pgfqpoint{4.513109in}{5.233053in}}%
\pgfpathlineto{\pgfqpoint{4.513646in}{5.222618in}}%
\pgfpathlineto{\pgfqpoint{4.514182in}{5.236173in}}%
\pgfpathlineto{\pgfqpoint{4.514182in}{5.236173in}}%
\pgfpathlineto{\pgfqpoint{4.514182in}{5.236173in}}%
\pgfpathlineto{\pgfqpoint{4.514719in}{5.222219in}}%
\pgfpathlineto{\pgfqpoint{4.515255in}{5.232837in}}%
\pgfpathlineto{\pgfqpoint{4.515792in}{5.240258in}}%
\pgfpathlineto{\pgfqpoint{4.516328in}{5.234165in}}%
\pgfpathlineto{\pgfqpoint{4.516865in}{5.236182in}}%
\pgfpathlineto{\pgfqpoint{4.518474in}{5.227818in}}%
\pgfpathlineto{\pgfqpoint{4.519010in}{5.231500in}}%
\pgfpathlineto{\pgfqpoint{4.519547in}{5.243627in}}%
\pgfpathlineto{\pgfqpoint{4.520083in}{5.236137in}}%
\pgfpathlineto{\pgfqpoint{4.520620in}{5.239852in}}%
\pgfpathlineto{\pgfqpoint{4.521156in}{5.231002in}}%
\pgfpathlineto{\pgfqpoint{4.521692in}{5.237166in}}%
\pgfpathlineto{\pgfqpoint{4.522765in}{5.231824in}}%
\pgfpathlineto{\pgfqpoint{4.523302in}{5.248157in}}%
\pgfpathlineto{\pgfqpoint{4.523838in}{5.234366in}}%
\pgfpathlineto{\pgfqpoint{4.524375in}{5.242752in}}%
\pgfpathlineto{\pgfqpoint{4.524375in}{5.242752in}}%
\pgfpathlineto{\pgfqpoint{4.524375in}{5.242752in}}%
\pgfpathlineto{\pgfqpoint{4.524911in}{5.233989in}}%
\pgfpathlineto{\pgfqpoint{4.525448in}{5.237878in}}%
\pgfpathlineto{\pgfqpoint{4.526520in}{5.233311in}}%
\pgfpathlineto{\pgfqpoint{4.527057in}{5.249004in}}%
\pgfpathlineto{\pgfqpoint{4.527057in}{5.249004in}}%
\pgfpathlineto{\pgfqpoint{4.527057in}{5.249004in}}%
\pgfpathlineto{\pgfqpoint{4.527593in}{5.232698in}}%
\pgfpathlineto{\pgfqpoint{4.528130in}{5.246330in}}%
\pgfpathlineto{\pgfqpoint{4.529739in}{5.236104in}}%
\pgfpathlineto{\pgfqpoint{4.530275in}{5.237493in}}%
\pgfpathlineto{\pgfqpoint{4.530812in}{5.245632in}}%
\pgfpathlineto{\pgfqpoint{4.531348in}{5.232627in}}%
\pgfpathlineto{\pgfqpoint{4.531885in}{5.249146in}}%
\pgfpathlineto{\pgfqpoint{4.532421in}{5.242104in}}%
\pgfpathlineto{\pgfqpoint{4.532958in}{5.236271in}}%
\pgfpathlineto{\pgfqpoint{4.533494in}{5.237004in}}%
\pgfpathlineto{\pgfqpoint{4.534567in}{5.243720in}}%
\pgfpathlineto{\pgfqpoint{4.535103in}{5.236507in}}%
\pgfpathlineto{\pgfqpoint{4.535640in}{5.247509in}}%
\pgfpathlineto{\pgfqpoint{4.536176in}{5.244070in}}%
\pgfpathlineto{\pgfqpoint{4.536713in}{5.234072in}}%
\pgfpathlineto{\pgfqpoint{4.537249in}{5.240349in}}%
\pgfpathlineto{\pgfqpoint{4.537786in}{5.237871in}}%
\pgfpathlineto{\pgfqpoint{4.539395in}{5.244577in}}%
\pgfpathlineto{\pgfqpoint{4.539931in}{5.245757in}}%
\pgfpathlineto{\pgfqpoint{4.540468in}{5.232082in}}%
\pgfpathlineto{\pgfqpoint{4.541004in}{5.242309in}}%
\pgfpathlineto{\pgfqpoint{4.541541in}{5.237393in}}%
\pgfpathlineto{\pgfqpoint{4.543150in}{5.244400in}}%
\pgfpathlineto{\pgfqpoint{4.543686in}{5.244730in}}%
\pgfpathlineto{\pgfqpoint{4.544223in}{5.233787in}}%
\pgfpathlineto{\pgfqpoint{4.544759in}{5.246287in}}%
\pgfpathlineto{\pgfqpoint{4.545296in}{5.237341in}}%
\pgfpathlineto{\pgfqpoint{4.546905in}{5.244655in}}%
\pgfpathlineto{\pgfqpoint{4.547441in}{5.246048in}}%
\pgfpathlineto{\pgfqpoint{4.547978in}{5.238134in}}%
\pgfpathlineto{\pgfqpoint{4.548514in}{5.247327in}}%
\pgfpathlineto{\pgfqpoint{4.549051in}{5.236823in}}%
\pgfpathlineto{\pgfqpoint{4.549587in}{5.240556in}}%
\pgfpathlineto{\pgfqpoint{4.550660in}{5.241637in}}%
\pgfpathlineto{\pgfqpoint{4.551196in}{5.246201in}}%
\pgfpathlineto{\pgfqpoint{4.552806in}{5.235155in}}%
\pgfpathlineto{\pgfqpoint{4.554951in}{5.248187in}}%
\pgfpathlineto{\pgfqpoint{4.556561in}{5.236417in}}%
\pgfpathlineto{\pgfqpoint{4.557634in}{5.245206in}}%
\pgfpathlineto{\pgfqpoint{4.558170in}{5.237125in}}%
\pgfpathlineto{\pgfqpoint{4.558707in}{5.247031in}}%
\pgfpathlineto{\pgfqpoint{4.559243in}{5.241163in}}%
\pgfpathlineto{\pgfqpoint{4.559779in}{5.243930in}}%
\pgfpathlineto{\pgfqpoint{4.560316in}{5.238274in}}%
\pgfpathlineto{\pgfqpoint{4.560852in}{5.240918in}}%
\pgfpathlineto{\pgfqpoint{4.561389in}{5.246988in}}%
\pgfpathlineto{\pgfqpoint{4.561925in}{5.234071in}}%
\pgfpathlineto{\pgfqpoint{4.562462in}{5.242888in}}%
\pgfpathlineto{\pgfqpoint{4.562998in}{5.242171in}}%
\pgfpathlineto{\pgfqpoint{4.563534in}{5.246670in}}%
\pgfpathlineto{\pgfqpoint{4.564071in}{5.237298in}}%
\pgfpathlineto{\pgfqpoint{4.564607in}{5.238204in}}%
\pgfpathlineto{\pgfqpoint{4.565144in}{5.243214in}}%
\pgfpathlineto{\pgfqpoint{4.565680in}{5.230970in}}%
\pgfpathlineto{\pgfqpoint{4.566217in}{5.241389in}}%
\pgfpathlineto{\pgfqpoint{4.566753in}{5.241694in}}%
\pgfpathlineto{\pgfqpoint{4.567290in}{5.244462in}}%
\pgfpathlineto{\pgfqpoint{4.568362in}{5.234738in}}%
\pgfpathlineto{\pgfqpoint{4.568899in}{5.237749in}}%
\pgfpathlineto{\pgfqpoint{4.569435in}{5.232416in}}%
\pgfpathlineto{\pgfqpoint{4.569972in}{5.241336in}}%
\pgfpathlineto{\pgfqpoint{4.571045in}{5.241257in}}%
\pgfpathlineto{\pgfqpoint{4.572117in}{5.230598in}}%
\pgfpathlineto{\pgfqpoint{4.572654in}{5.234777in}}%
\pgfpathlineto{\pgfqpoint{4.573190in}{5.233665in}}%
\pgfpathlineto{\pgfqpoint{4.574263in}{5.241294in}}%
\pgfpathlineto{\pgfqpoint{4.575873in}{5.228202in}}%
\pgfpathlineto{\pgfqpoint{4.576945in}{5.230254in}}%
\pgfpathlineto{\pgfqpoint{4.577482in}{5.236449in}}%
\pgfpathlineto{\pgfqpoint{4.578018in}{5.235127in}}%
\pgfpathlineto{\pgfqpoint{4.578555in}{5.235006in}}%
\pgfpathlineto{\pgfqpoint{4.579091in}{5.238354in}}%
\pgfpathlineto{\pgfqpoint{4.580700in}{5.222823in}}%
\pgfpathlineto{\pgfqpoint{4.581237in}{5.230535in}}%
\pgfpathlineto{\pgfqpoint{4.581773in}{5.228917in}}%
\pgfpathlineto{\pgfqpoint{4.582310in}{5.229918in}}%
\pgfpathlineto{\pgfqpoint{4.582846in}{5.237073in}}%
\pgfpathlineto{\pgfqpoint{4.584455in}{5.220244in}}%
\pgfpathlineto{\pgfqpoint{4.586601in}{5.231279in}}%
\pgfpathlineto{\pgfqpoint{4.587674in}{5.215806in}}%
\pgfpathlineto{\pgfqpoint{4.588211in}{5.219114in}}%
\pgfpathlineto{\pgfqpoint{4.588747in}{5.219810in}}%
\pgfpathlineto{\pgfqpoint{4.589283in}{5.225005in}}%
\pgfpathlineto{\pgfqpoint{4.589820in}{5.222507in}}%
\pgfpathlineto{\pgfqpoint{4.590356in}{5.221339in}}%
\pgfpathlineto{\pgfqpoint{4.591429in}{5.211276in}}%
\pgfpathlineto{\pgfqpoint{4.593038in}{5.219012in}}%
\pgfpathlineto{\pgfqpoint{4.594111in}{5.217952in}}%
\pgfpathlineto{\pgfqpoint{4.594648in}{5.205595in}}%
\pgfpathlineto{\pgfqpoint{4.595184in}{5.206876in}}%
\pgfpathlineto{\pgfqpoint{4.595721in}{5.211499in}}%
\pgfpathlineto{\pgfqpoint{4.596257in}{5.206077in}}%
\pgfpathlineto{\pgfqpoint{4.597866in}{5.215253in}}%
\pgfpathlineto{\pgfqpoint{4.598403in}{5.202520in}}%
\pgfpathlineto{\pgfqpoint{4.598939in}{5.205002in}}%
\pgfpathlineto{\pgfqpoint{4.599476in}{5.207092in}}%
\pgfpathlineto{\pgfqpoint{4.600012in}{5.199253in}}%
\pgfpathlineto{\pgfqpoint{4.601085in}{5.210818in}}%
\pgfpathlineto{\pgfqpoint{4.601621in}{5.210123in}}%
\pgfpathlineto{\pgfqpoint{4.602158in}{5.200425in}}%
\pgfpathlineto{\pgfqpoint{4.602694in}{5.203145in}}%
\pgfpathlineto{\pgfqpoint{4.603231in}{5.204898in}}%
\pgfpathlineto{\pgfqpoint{4.603767in}{5.193485in}}%
\pgfpathlineto{\pgfqpoint{4.603767in}{5.193485in}}%
\pgfpathlineto{\pgfqpoint{4.603767in}{5.193485in}}%
\pgfpathlineto{\pgfqpoint{4.604304in}{5.205420in}}%
\pgfpathlineto{\pgfqpoint{4.605376in}{5.205036in}}%
\pgfpathlineto{\pgfqpoint{4.605913in}{5.192395in}}%
\pgfpathlineto{\pgfqpoint{4.606449in}{5.198196in}}%
\pgfpathlineto{\pgfqpoint{4.606986in}{5.199982in}}%
\pgfpathlineto{\pgfqpoint{4.607522in}{5.193547in}}%
\pgfpathlineto{\pgfqpoint{4.608059in}{5.204897in}}%
\pgfpathlineto{\pgfqpoint{4.608595in}{5.204870in}}%
\pgfpathlineto{\pgfqpoint{4.609668in}{5.184475in}}%
\pgfpathlineto{\pgfqpoint{4.610204in}{5.192479in}}%
\pgfpathlineto{\pgfqpoint{4.611814in}{5.201050in}}%
\pgfpathlineto{\pgfqpoint{4.612350in}{5.205958in}}%
\pgfpathlineto{\pgfqpoint{4.613423in}{5.180302in}}%
\pgfpathlineto{\pgfqpoint{4.613959in}{5.186472in}}%
\pgfpathlineto{\pgfqpoint{4.615569in}{5.192978in}}%
\pgfpathlineto{\pgfqpoint{4.616105in}{5.203580in}}%
\pgfpathlineto{\pgfqpoint{4.616105in}{5.203580in}}%
\pgfpathlineto{\pgfqpoint{4.616105in}{5.203580in}}%
\pgfpathlineto{\pgfqpoint{4.617178in}{5.177859in}}%
\pgfpathlineto{\pgfqpoint{4.618251in}{5.181615in}}%
\pgfpathlineto{\pgfqpoint{4.618787in}{5.185016in}}%
\pgfpathlineto{\pgfqpoint{4.619324in}{5.179931in}}%
\pgfpathlineto{\pgfqpoint{4.620397in}{5.191330in}}%
\pgfpathlineto{\pgfqpoint{4.622006in}{5.175522in}}%
\pgfpathlineto{\pgfqpoint{4.622542in}{5.180347in}}%
\pgfpathlineto{\pgfqpoint{4.623615in}{5.180085in}}%
\pgfpathlineto{\pgfqpoint{4.624152in}{5.181064in}}%
\pgfpathlineto{\pgfqpoint{4.625225in}{5.172471in}}%
\pgfpathlineto{\pgfqpoint{4.625761in}{5.174751in}}%
\pgfpathlineto{\pgfqpoint{4.626298in}{5.174806in}}%
\pgfpathlineto{\pgfqpoint{4.626834in}{5.173638in}}%
\pgfpathlineto{\pgfqpoint{4.627370in}{5.168593in}}%
\pgfpathlineto{\pgfqpoint{4.627907in}{5.170418in}}%
\pgfpathlineto{\pgfqpoint{4.628443in}{5.172478in}}%
\pgfpathlineto{\pgfqpoint{4.628980in}{5.162073in}}%
\pgfpathlineto{\pgfqpoint{4.629516in}{5.172373in}}%
\pgfpathlineto{\pgfqpoint{4.630053in}{5.171731in}}%
\pgfpathlineto{\pgfqpoint{4.631125in}{5.156899in}}%
\pgfpathlineto{\pgfqpoint{4.631662in}{5.163001in}}%
\pgfpathlineto{\pgfqpoint{4.632198in}{5.162427in}}%
\pgfpathlineto{\pgfqpoint{4.632735in}{5.155127in}}%
\pgfpathlineto{\pgfqpoint{4.633271in}{5.161075in}}%
\pgfpathlineto{\pgfqpoint{4.633808in}{5.168654in}}%
\pgfpathlineto{\pgfqpoint{4.634880in}{5.143851in}}%
\pgfpathlineto{\pgfqpoint{4.635417in}{5.157955in}}%
\pgfpathlineto{\pgfqpoint{4.635953in}{5.151364in}}%
\pgfpathlineto{\pgfqpoint{4.637026in}{5.145855in}}%
\pgfpathlineto{\pgfqpoint{4.637563in}{5.157811in}}%
\pgfpathlineto{\pgfqpoint{4.638099in}{5.156338in}}%
\pgfpathlineto{\pgfqpoint{4.638636in}{5.133214in}}%
\pgfpathlineto{\pgfqpoint{4.639172in}{5.146306in}}%
\pgfpathlineto{\pgfqpoint{4.640781in}{5.136378in}}%
\pgfpathlineto{\pgfqpoint{4.641854in}{5.152639in}}%
\pgfpathlineto{\pgfqpoint{4.642391in}{5.127932in}}%
\pgfpathlineto{\pgfqpoint{4.643463in}{5.129702in}}%
\pgfpathlineto{\pgfqpoint{4.644536in}{5.124708in}}%
\pgfpathlineto{\pgfqpoint{4.645609in}{5.144373in}}%
\pgfpathlineto{\pgfqpoint{4.647219in}{5.120674in}}%
\pgfpathlineto{\pgfqpoint{4.648291in}{5.110469in}}%
\pgfpathlineto{\pgfqpoint{4.649901in}{5.129281in}}%
\pgfpathlineto{\pgfqpoint{4.652046in}{5.096834in}}%
\pgfpathlineto{\pgfqpoint{4.653656in}{5.120519in}}%
\pgfpathlineto{\pgfqpoint{4.654192in}{5.119856in}}%
\pgfpathlineto{\pgfqpoint{4.655801in}{5.093018in}}%
\pgfpathlineto{\pgfqpoint{4.657411in}{5.104002in}}%
\pgfpathlineto{\pgfqpoint{4.657947in}{5.109793in}}%
\pgfpathlineto{\pgfqpoint{4.658484in}{5.091926in}}%
\pgfpathlineto{\pgfqpoint{4.659020in}{5.121240in}}%
\pgfpathlineto{\pgfqpoint{4.659557in}{5.092388in}}%
\pgfpathlineto{\pgfqpoint{4.660629in}{5.073737in}}%
\pgfpathlineto{\pgfqpoint{4.662775in}{5.107420in}}%
\pgfpathlineto{\pgfqpoint{4.664384in}{5.056291in}}%
\pgfpathlineto{\pgfqpoint{4.665457in}{5.073214in}}%
\pgfpathlineto{\pgfqpoint{4.667067in}{5.101406in}}%
\pgfpathlineto{\pgfqpoint{4.668140in}{5.045025in}}%
\pgfpathlineto{\pgfqpoint{4.668676in}{5.062114in}}%
\pgfpathlineto{\pgfqpoint{4.669749in}{5.077407in}}%
\pgfpathlineto{\pgfqpoint{4.670822in}{5.093878in}}%
\pgfpathlineto{\pgfqpoint{4.671895in}{5.040004in}}%
\pgfpathlineto{\pgfqpoint{4.672431in}{5.055123in}}%
\pgfpathlineto{\pgfqpoint{4.672967in}{5.054794in}}%
\pgfpathlineto{\pgfqpoint{4.674577in}{5.071017in}}%
\pgfpathlineto{\pgfqpoint{4.676186in}{5.042278in}}%
\pgfpathlineto{\pgfqpoint{4.677259in}{5.055467in}}%
\pgfpathlineto{\pgfqpoint{4.678868in}{5.037286in}}%
\pgfpathlineto{\pgfqpoint{4.679405in}{5.054224in}}%
\pgfpathlineto{\pgfqpoint{4.679941in}{5.030781in}}%
\pgfpathlineto{\pgfqpoint{4.680478in}{5.057035in}}%
\pgfpathlineto{\pgfqpoint{4.681014in}{5.053103in}}%
\pgfpathlineto{\pgfqpoint{4.682623in}{5.025942in}}%
\pgfpathlineto{\pgfqpoint{4.684233in}{5.070154in}}%
\pgfpathlineto{\pgfqpoint{4.684769in}{5.051842in}}%
\pgfpathlineto{\pgfqpoint{4.686378in}{5.004324in}}%
\pgfpathlineto{\pgfqpoint{4.687451in}{5.028906in}}%
\pgfpathlineto{\pgfqpoint{4.687988in}{5.057778in}}%
\pgfpathlineto{\pgfqpoint{4.687988in}{5.057778in}}%
\pgfpathlineto{\pgfqpoint{4.687988in}{5.057778in}}%
\pgfpathlineto{\pgfqpoint{4.689597in}{4.981785in}}%
\pgfpathlineto{\pgfqpoint{4.690133in}{4.983925in}}%
\pgfpathlineto{\pgfqpoint{4.691206in}{4.999917in}}%
\pgfpathlineto{\pgfqpoint{4.691743in}{5.036789in}}%
\pgfpathlineto{\pgfqpoint{4.692279in}{5.031419in}}%
\pgfpathlineto{\pgfqpoint{4.692816in}{5.029704in}}%
\pgfpathlineto{\pgfqpoint{4.693352in}{4.961577in}}%
\pgfpathlineto{\pgfqpoint{4.694425in}{4.964615in}}%
\pgfpathlineto{\pgfqpoint{4.694961in}{4.953691in}}%
\pgfpathlineto{\pgfqpoint{4.696571in}{5.047816in}}%
\pgfpathlineto{\pgfqpoint{4.698180in}{4.936889in}}%
\pgfpathlineto{\pgfqpoint{4.698716in}{4.923895in}}%
\pgfpathlineto{\pgfqpoint{4.700326in}{5.040102in}}%
\pgfpathlineto{\pgfqpoint{4.700862in}{5.002027in}}%
\pgfpathlineto{\pgfqpoint{4.701399in}{4.996309in}}%
\pgfpathlineto{\pgfqpoint{4.702471in}{4.901944in}}%
\pgfpathlineto{\pgfqpoint{4.704081in}{5.025207in}}%
\pgfpathlineto{\pgfqpoint{4.705154in}{4.997937in}}%
\pgfpathlineto{\pgfqpoint{4.705690in}{4.870517in}}%
\pgfpathlineto{\pgfqpoint{4.706226in}{4.882430in}}%
\pgfpathlineto{\pgfqpoint{4.706763in}{4.895896in}}%
\pgfpathlineto{\pgfqpoint{4.707836in}{5.001753in}}%
\pgfpathlineto{\pgfqpoint{4.708372in}{4.991123in}}%
\pgfpathlineto{\pgfqpoint{4.708909in}{4.971038in}}%
\pgfpathlineto{\pgfqpoint{4.710518in}{4.852979in}}%
\pgfpathlineto{\pgfqpoint{4.711591in}{4.966070in}}%
\pgfpathlineto{\pgfqpoint{4.712127in}{4.948900in}}%
\pgfpathlineto{\pgfqpoint{4.712664in}{4.966021in}}%
\pgfpathlineto{\pgfqpoint{4.714273in}{4.865266in}}%
\pgfpathlineto{\pgfqpoint{4.714809in}{4.940278in}}%
\pgfpathlineto{\pgfqpoint{4.715346in}{4.933679in}}%
\pgfpathlineto{\pgfqpoint{4.715882in}{4.903358in}}%
\pgfpathlineto{\pgfqpoint{4.716419in}{4.931946in}}%
\pgfpathlineto{\pgfqpoint{4.716955in}{4.926597in}}%
\pgfpathlineto{\pgfqpoint{4.718028in}{4.944287in}}%
\pgfpathlineto{\pgfqpoint{4.718565in}{4.941980in}}%
\pgfpathlineto{\pgfqpoint{4.720174in}{4.876949in}}%
\pgfpathlineto{\pgfqpoint{4.720710in}{4.871535in}}%
\pgfpathlineto{\pgfqpoint{4.721783in}{4.953153in}}%
\pgfpathlineto{\pgfqpoint{4.722320in}{4.945869in}}%
\pgfpathlineto{\pgfqpoint{4.722856in}{4.807093in}}%
\pgfpathlineto{\pgfqpoint{4.723392in}{4.855427in}}%
\pgfpathlineto{\pgfqpoint{4.723929in}{4.856962in}}%
\pgfpathlineto{\pgfqpoint{4.724465in}{4.811803in}}%
\pgfpathlineto{\pgfqpoint{4.726075in}{4.942050in}}%
\pgfpathlineto{\pgfqpoint{4.727148in}{4.751157in}}%
\pgfpathlineto{\pgfqpoint{4.729293in}{4.913875in}}%
\pgfpathlineto{\pgfqpoint{4.729830in}{4.946557in}}%
\pgfpathlineto{\pgfqpoint{4.729830in}{4.946557in}}%
\pgfpathlineto{\pgfqpoint{4.729830in}{4.946557in}}%
\pgfpathlineto{\pgfqpoint{4.730903in}{4.840524in}}%
\pgfpathlineto{\pgfqpoint{4.731439in}{4.723223in}}%
\pgfpathlineto{\pgfqpoint{4.731975in}{4.764319in}}%
\pgfpathlineto{\pgfqpoint{4.732512in}{4.773161in}}%
\pgfpathlineto{\pgfqpoint{4.734121in}{4.963205in}}%
\pgfpathlineto{\pgfqpoint{4.734658in}{4.968946in}}%
\pgfpathlineto{\pgfqpoint{4.735730in}{4.673155in}}%
\pgfpathlineto{\pgfqpoint{4.736267in}{4.761497in}}%
\pgfpathlineto{\pgfqpoint{4.736803in}{4.768211in}}%
\pgfpathlineto{\pgfqpoint{4.738413in}{5.010639in}}%
\pgfpathlineto{\pgfqpoint{4.740022in}{4.617335in}}%
\pgfpathlineto{\pgfqpoint{4.740558in}{4.691106in}}%
\pgfpathlineto{\pgfqpoint{4.742168in}{4.956529in}}%
\pgfpathlineto{\pgfqpoint{4.743241in}{4.758634in}}%
\pgfpathlineto{\pgfqpoint{4.743777in}{4.328824in}}%
\pgfpathlineto{\pgfqpoint{4.744313in}{4.428871in}}%
\pgfpathlineto{\pgfqpoint{4.745923in}{4.934091in}}%
\pgfpathlineto{\pgfqpoint{4.746459in}{4.933525in}}%
\pgfpathlineto{\pgfqpoint{4.746996in}{4.858465in}}%
\pgfpathlineto{\pgfqpoint{4.748069in}{4.410679in}}%
\pgfpathlineto{\pgfqpoint{4.750214in}{4.997219in}}%
\pgfpathlineto{\pgfqpoint{4.752360in}{4.555920in}}%
\pgfpathlineto{\pgfqpoint{4.753969in}{4.936842in}}%
\pgfpathlineto{\pgfqpoint{4.755042in}{4.635219in}}%
\pgfpathlineto{\pgfqpoint{4.755579in}{4.902669in}}%
\pgfpathlineto{\pgfqpoint{4.756115in}{4.681497in}}%
\pgfpathlineto{\pgfqpoint{4.756651in}{4.570337in}}%
\pgfpathlineto{\pgfqpoint{4.757188in}{4.905733in}}%
\pgfpathlineto{\pgfqpoint{4.757724in}{4.904985in}}%
\pgfpathlineto{\pgfqpoint{4.758261in}{4.676508in}}%
\pgfpathlineto{\pgfqpoint{4.758797in}{4.725520in}}%
\pgfpathlineto{\pgfqpoint{4.759870in}{4.853137in}}%
\pgfpathlineto{\pgfqpoint{4.760407in}{4.695593in}}%
\pgfpathlineto{\pgfqpoint{4.760943in}{4.743171in}}%
\pgfpathlineto{\pgfqpoint{4.761479in}{4.828090in}}%
\pgfpathlineto{\pgfqpoint{4.763089in}{4.634003in}}%
\pgfpathlineto{\pgfqpoint{4.764162in}{4.917573in}}%
\pgfpathlineto{\pgfqpoint{4.764698in}{4.547166in}}%
\pgfpathlineto{\pgfqpoint{4.765234in}{4.620100in}}%
\pgfpathlineto{\pgfqpoint{4.766307in}{4.806373in}}%
\pgfpathlineto{\pgfqpoint{4.766844in}{4.731781in}}%
\pgfpathlineto{\pgfqpoint{4.767917in}{4.948464in}}%
\pgfpathlineto{\pgfqpoint{4.768990in}{4.357341in}}%
\pgfpathlineto{\pgfqpoint{4.769526in}{4.775226in}}%
\pgfpathlineto{\pgfqpoint{4.770599in}{4.746769in}}%
\pgfpathlineto{\pgfqpoint{4.771135in}{4.992159in}}%
\pgfpathlineto{\pgfqpoint{4.771672in}{4.855231in}}%
\pgfpathlineto{\pgfqpoint{4.772208in}{4.869227in}}%
\pgfpathlineto{\pgfqpoint{4.773281in}{4.526627in}}%
\pgfpathlineto{\pgfqpoint{4.773817in}{4.535926in}}%
\pgfpathlineto{\pgfqpoint{4.774890in}{4.910227in}}%
\pgfpathlineto{\pgfqpoint{4.775427in}{4.853568in}}%
\pgfpathlineto{\pgfqpoint{4.775963in}{4.845545in}}%
\pgfpathlineto{\pgfqpoint{4.776500in}{4.863210in}}%
\pgfpathlineto{\pgfqpoint{4.777573in}{4.170583in}}%
\pgfpathlineto{\pgfqpoint{4.779182in}{4.857378in}}%
\pgfpathlineto{\pgfqpoint{4.779718in}{4.985799in}}%
\pgfpathlineto{\pgfqpoint{4.781328in}{4.417280in}}%
\pgfpathlineto{\pgfqpoint{4.781864in}{4.411242in}}%
\pgfpathlineto{\pgfqpoint{4.783473in}{4.982989in}}%
\pgfpathlineto{\pgfqpoint{4.784010in}{4.797625in}}%
\pgfpathlineto{\pgfqpoint{4.785619in}{4.475445in}}%
\pgfpathlineto{\pgfqpoint{4.787228in}{4.998494in}}%
\pgfpathlineto{\pgfqpoint{4.787765in}{4.838060in}}%
\pgfpathlineto{\pgfqpoint{4.788838in}{4.596965in}}%
\pgfpathlineto{\pgfqpoint{4.789374in}{4.728947in}}%
\pgfpathlineto{\pgfqpoint{4.790447in}{4.840525in}}%
\pgfpathlineto{\pgfqpoint{4.791520in}{4.959316in}}%
\pgfpathlineto{\pgfqpoint{4.792056in}{4.876741in}}%
\pgfpathlineto{\pgfqpoint{4.792593in}{4.617362in}}%
\pgfpathlineto{\pgfqpoint{4.793129in}{4.850669in}}%
\pgfpathlineto{\pgfqpoint{4.794202in}{4.809308in}}%
\pgfpathlineto{\pgfqpoint{4.795275in}{4.962265in}}%
\pgfpathlineto{\pgfqpoint{4.796348in}{4.784102in}}%
\pgfpathlineto{\pgfqpoint{4.796884in}{4.850108in}}%
\pgfpathlineto{\pgfqpoint{4.797421in}{4.845709in}}%
\pgfpathlineto{\pgfqpoint{4.797957in}{4.765953in}}%
\pgfpathlineto{\pgfqpoint{4.797957in}{4.765953in}}%
\pgfpathlineto{\pgfqpoint{4.797957in}{4.765953in}}%
\pgfpathlineto{\pgfqpoint{4.799030in}{4.916152in}}%
\pgfpathlineto{\pgfqpoint{4.799566in}{4.795515in}}%
\pgfpathlineto{\pgfqpoint{4.800103in}{4.811742in}}%
\pgfpathlineto{\pgfqpoint{4.800639in}{4.916820in}}%
\pgfpathlineto{\pgfqpoint{4.801176in}{4.840050in}}%
\pgfpathlineto{\pgfqpoint{4.801712in}{4.879640in}}%
\pgfpathlineto{\pgfqpoint{4.803321in}{4.769622in}}%
\pgfpathlineto{\pgfqpoint{4.803858in}{4.905223in}}%
\pgfpathlineto{\pgfqpoint{4.804931in}{4.897990in}}%
\pgfpathlineto{\pgfqpoint{4.805467in}{4.909881in}}%
\pgfpathlineto{\pgfqpoint{4.806540in}{4.703596in}}%
\pgfpathlineto{\pgfqpoint{4.807076in}{4.778779in}}%
\pgfpathlineto{\pgfqpoint{4.808686in}{4.975965in}}%
\pgfpathlineto{\pgfqpoint{4.809222in}{4.930453in}}%
\pgfpathlineto{\pgfqpoint{4.810295in}{4.786779in}}%
\pgfpathlineto{\pgfqpoint{4.810832in}{4.791467in}}%
\pgfpathlineto{\pgfqpoint{4.811368in}{4.814042in}}%
\pgfpathlineto{\pgfqpoint{4.812441in}{5.007112in}}%
\pgfpathlineto{\pgfqpoint{4.812977in}{4.980973in}}%
\pgfpathlineto{\pgfqpoint{4.814587in}{4.768732in}}%
\pgfpathlineto{\pgfqpoint{4.816196in}{5.007632in}}%
\pgfpathlineto{\pgfqpoint{4.816732in}{4.975471in}}%
\pgfpathlineto{\pgfqpoint{4.818342in}{4.799930in}}%
\pgfpathlineto{\pgfqpoint{4.818878in}{4.867338in}}%
\pgfpathlineto{\pgfqpoint{4.820487in}{5.004864in}}%
\pgfpathlineto{\pgfqpoint{4.822097in}{4.831511in}}%
\pgfpathlineto{\pgfqpoint{4.824242in}{4.995420in}}%
\pgfpathlineto{\pgfqpoint{4.825852in}{4.882944in}}%
\pgfpathlineto{\pgfqpoint{4.827461in}{4.966020in}}%
\pgfpathlineto{\pgfqpoint{4.827997in}{4.999031in}}%
\pgfpathlineto{\pgfqpoint{4.829607in}{4.928423in}}%
\pgfpathlineto{\pgfqpoint{4.831753in}{4.996005in}}%
\pgfpathlineto{\pgfqpoint{4.833362in}{4.942027in}}%
\pgfpathlineto{\pgfqpoint{4.834435in}{5.008389in}}%
\pgfpathlineto{\pgfqpoint{4.836044in}{4.969326in}}%
\pgfpathlineto{\pgfqpoint{4.838190in}{5.028746in}}%
\pgfpathlineto{\pgfqpoint{4.839799in}{4.961397in}}%
\pgfpathlineto{\pgfqpoint{4.841408in}{5.017722in}}%
\pgfpathlineto{\pgfqpoint{4.841945in}{5.043841in}}%
\pgfpathlineto{\pgfqpoint{4.843554in}{4.946583in}}%
\pgfpathlineto{\pgfqpoint{4.845700in}{5.061654in}}%
\pgfpathlineto{\pgfqpoint{4.847309in}{4.956408in}}%
\pgfpathlineto{\pgfqpoint{4.847846in}{4.984582in}}%
\pgfpathlineto{\pgfqpoint{4.849455in}{5.063087in}}%
\pgfpathlineto{\pgfqpoint{4.851064in}{4.973676in}}%
\pgfpathlineto{\pgfqpoint{4.851601in}{4.988452in}}%
\pgfpathlineto{\pgfqpoint{4.853210in}{5.079439in}}%
\pgfpathlineto{\pgfqpoint{4.854283in}{5.070080in}}%
\pgfpathlineto{\pgfqpoint{4.854819in}{5.003712in}}%
\pgfpathlineto{\pgfqpoint{4.855356in}{5.007947in}}%
\pgfpathlineto{\pgfqpoint{4.856965in}{5.083003in}}%
\pgfpathlineto{\pgfqpoint{4.858038in}{5.061278in}}%
\pgfpathlineto{\pgfqpoint{4.858574in}{5.014366in}}%
\pgfpathlineto{\pgfqpoint{4.859111in}{5.024202in}}%
\pgfpathlineto{\pgfqpoint{4.860184in}{5.069389in}}%
\pgfpathlineto{\pgfqpoint{4.860720in}{5.084111in}}%
\pgfpathlineto{\pgfqpoint{4.861257in}{5.071925in}}%
\pgfpathlineto{\pgfqpoint{4.861793in}{5.075964in}}%
\pgfpathlineto{\pgfqpoint{4.862329in}{5.039376in}}%
\pgfpathlineto{\pgfqpoint{4.862866in}{5.051083in}}%
\pgfpathlineto{\pgfqpoint{4.863402in}{5.052152in}}%
\pgfpathlineto{\pgfqpoint{4.864475in}{5.087983in}}%
\pgfpathlineto{\pgfqpoint{4.865548in}{5.084777in}}%
\pgfpathlineto{\pgfqpoint{4.866084in}{5.050202in}}%
\pgfpathlineto{\pgfqpoint{4.866621in}{5.068829in}}%
\pgfpathlineto{\pgfqpoint{4.868230in}{5.084783in}}%
\pgfpathlineto{\pgfqpoint{4.868767in}{5.079899in}}%
\pgfpathlineto{\pgfqpoint{4.869303in}{5.100651in}}%
\pgfpathlineto{\pgfqpoint{4.869840in}{5.071767in}}%
\pgfpathlineto{\pgfqpoint{4.870376in}{5.085402in}}%
\pgfpathlineto{\pgfqpoint{4.871985in}{5.094533in}}%
\pgfpathlineto{\pgfqpoint{4.872522in}{5.083505in}}%
\pgfpathlineto{\pgfqpoint{4.874131in}{5.108098in}}%
\pgfpathlineto{\pgfqpoint{4.875740in}{5.099095in}}%
\pgfpathlineto{\pgfqpoint{4.876277in}{5.087306in}}%
\pgfpathlineto{\pgfqpoint{4.876813in}{5.111692in}}%
\pgfpathlineto{\pgfqpoint{4.877350in}{5.096675in}}%
\pgfpathlineto{\pgfqpoint{4.878422in}{5.114366in}}%
\pgfpathlineto{\pgfqpoint{4.878959in}{5.111992in}}%
\pgfpathlineto{\pgfqpoint{4.880032in}{5.100893in}}%
\pgfpathlineto{\pgfqpoint{4.882178in}{5.132376in}}%
\pgfpathlineto{\pgfqpoint{4.883250in}{5.105754in}}%
\pgfpathlineto{\pgfqpoint{4.883787in}{5.106583in}}%
\pgfpathlineto{\pgfqpoint{4.885933in}{5.149379in}}%
\pgfpathlineto{\pgfqpoint{4.887005in}{5.115991in}}%
\pgfpathlineto{\pgfqpoint{4.887542in}{5.116833in}}%
\pgfpathlineto{\pgfqpoint{4.889151in}{5.137530in}}%
\pgfpathlineto{\pgfqpoint{4.889688in}{5.151970in}}%
\pgfpathlineto{\pgfqpoint{4.891297in}{5.121732in}}%
\pgfpathlineto{\pgfqpoint{4.893443in}{5.156755in}}%
\pgfpathlineto{\pgfqpoint{4.894516in}{5.135373in}}%
\pgfpathlineto{\pgfqpoint{4.895052in}{5.135530in}}%
\pgfpathlineto{\pgfqpoint{4.897198in}{5.161360in}}%
\pgfpathlineto{\pgfqpoint{4.898271in}{5.140272in}}%
\pgfpathlineto{\pgfqpoint{4.900416in}{5.162169in}}%
\pgfpathlineto{\pgfqpoint{4.900953in}{5.165476in}}%
\pgfpathlineto{\pgfqpoint{4.902026in}{5.150207in}}%
\pgfpathlineto{\pgfqpoint{4.902562in}{5.165492in}}%
\pgfpathlineto{\pgfqpoint{4.903099in}{5.160189in}}%
\pgfpathlineto{\pgfqpoint{4.904171in}{5.165352in}}%
\pgfpathlineto{\pgfqpoint{4.904708in}{5.172381in}}%
\pgfpathlineto{\pgfqpoint{4.905244in}{5.166304in}}%
\pgfpathlineto{\pgfqpoint{4.905781in}{5.160238in}}%
\pgfpathlineto{\pgfqpoint{4.907390in}{5.172568in}}%
\pgfpathlineto{\pgfqpoint{4.907926in}{5.171947in}}%
\pgfpathlineto{\pgfqpoint{4.908463in}{5.174407in}}%
\pgfpathlineto{\pgfqpoint{4.908999in}{5.173905in}}%
\pgfpathlineto{\pgfqpoint{4.909536in}{5.169754in}}%
\pgfpathlineto{\pgfqpoint{4.911682in}{5.183987in}}%
\pgfpathlineto{\pgfqpoint{4.912754in}{5.182322in}}%
\pgfpathlineto{\pgfqpoint{4.913291in}{5.174783in}}%
\pgfpathlineto{\pgfqpoint{4.913827in}{5.178326in}}%
\pgfpathlineto{\pgfqpoint{4.915437in}{5.195539in}}%
\pgfpathlineto{\pgfqpoint{4.917046in}{5.180019in}}%
\pgfpathlineto{\pgfqpoint{4.919192in}{5.198476in}}%
\pgfpathlineto{\pgfqpoint{4.920801in}{5.183942in}}%
\pgfpathlineto{\pgfqpoint{4.922947in}{5.204679in}}%
\pgfpathlineto{\pgfqpoint{4.923483in}{5.197358in}}%
\pgfpathlineto{\pgfqpoint{4.924020in}{5.199026in}}%
\pgfpathlineto{\pgfqpoint{4.924556in}{5.194140in}}%
\pgfpathlineto{\pgfqpoint{4.925092in}{5.198574in}}%
\pgfpathlineto{\pgfqpoint{4.926165in}{5.197614in}}%
\pgfpathlineto{\pgfqpoint{4.926702in}{5.205770in}}%
\pgfpathlineto{\pgfqpoint{4.927775in}{5.205347in}}%
\pgfpathlineto{\pgfqpoint{4.929920in}{5.197803in}}%
\pgfpathlineto{\pgfqpoint{4.930457in}{5.213102in}}%
\pgfpathlineto{\pgfqpoint{4.930993in}{5.207021in}}%
\pgfpathlineto{\pgfqpoint{4.931530in}{5.211937in}}%
\pgfpathlineto{\pgfqpoint{4.932603in}{5.204540in}}%
\pgfpathlineto{\pgfqpoint{4.933139in}{5.205892in}}%
\pgfpathlineto{\pgfqpoint{4.934748in}{5.218859in}}%
\pgfpathlineto{\pgfqpoint{4.935285in}{5.216778in}}%
\pgfpathlineto{\pgfqpoint{4.935821in}{5.207537in}}%
\pgfpathlineto{\pgfqpoint{4.936894in}{5.208190in}}%
\pgfpathlineto{\pgfqpoint{4.939040in}{5.223473in}}%
\pgfpathlineto{\pgfqpoint{4.940113in}{5.211606in}}%
\pgfpathlineto{\pgfqpoint{4.941722in}{5.226974in}}%
\pgfpathlineto{\pgfqpoint{4.942795in}{5.226014in}}%
\pgfpathlineto{\pgfqpoint{4.943868in}{5.215017in}}%
\pgfpathlineto{\pgfqpoint{4.944404in}{5.219383in}}%
\pgfpathlineto{\pgfqpoint{4.944941in}{5.220942in}}%
\pgfpathlineto{\pgfqpoint{4.946550in}{5.234249in}}%
\pgfpathlineto{\pgfqpoint{4.947623in}{5.218646in}}%
\pgfpathlineto{\pgfqpoint{4.948696in}{5.220694in}}%
\pgfpathlineto{\pgfqpoint{4.950305in}{5.236950in}}%
\pgfpathlineto{\pgfqpoint{4.951378in}{5.220650in}}%
\pgfpathlineto{\pgfqpoint{4.951914in}{5.223629in}}%
\pgfpathlineto{\pgfqpoint{4.952451in}{5.225709in}}%
\pgfpathlineto{\pgfqpoint{4.954060in}{5.237013in}}%
\pgfpathlineto{\pgfqpoint{4.955133in}{5.226344in}}%
\pgfpathlineto{\pgfqpoint{4.956206in}{5.229212in}}%
\pgfpathlineto{\pgfqpoint{4.956742in}{5.240608in}}%
\pgfpathlineto{\pgfqpoint{4.957279in}{5.227011in}}%
\pgfpathlineto{\pgfqpoint{4.957815in}{5.235153in}}%
\pgfpathlineto{\pgfqpoint{4.958888in}{5.228305in}}%
\pgfpathlineto{\pgfqpoint{4.959424in}{5.231632in}}%
\pgfpathlineto{\pgfqpoint{4.959961in}{5.234195in}}%
\pgfpathlineto{\pgfqpoint{4.960497in}{5.242370in}}%
\pgfpathlineto{\pgfqpoint{4.961034in}{5.231710in}}%
\pgfpathlineto{\pgfqpoint{4.961570in}{5.235630in}}%
\pgfpathlineto{\pgfqpoint{4.962107in}{5.235849in}}%
\pgfpathlineto{\pgfqpoint{4.962643in}{5.229457in}}%
\pgfpathlineto{\pgfqpoint{4.963179in}{5.232384in}}%
\pgfpathlineto{\pgfqpoint{4.964252in}{5.240612in}}%
\pgfpathlineto{\pgfqpoint{4.964789in}{5.233392in}}%
\pgfpathlineto{\pgfqpoint{4.965325in}{5.237863in}}%
\pgfpathlineto{\pgfqpoint{4.966398in}{5.232338in}}%
\pgfpathlineto{\pgfqpoint{4.966934in}{5.233647in}}%
\pgfpathlineto{\pgfqpoint{4.968007in}{5.242752in}}%
\pgfpathlineto{\pgfqpoint{4.968544in}{5.238693in}}%
\pgfpathlineto{\pgfqpoint{4.969080in}{5.241395in}}%
\pgfpathlineto{\pgfqpoint{4.970690in}{5.232395in}}%
\pgfpathlineto{\pgfqpoint{4.973372in}{5.243551in}}%
\pgfpathlineto{\pgfqpoint{4.974445in}{5.237688in}}%
\pgfpathlineto{\pgfqpoint{4.974981in}{5.238506in}}%
\pgfpathlineto{\pgfqpoint{4.975517in}{5.243369in}}%
\pgfpathlineto{\pgfqpoint{4.976054in}{5.236858in}}%
\pgfpathlineto{\pgfqpoint{4.976590in}{5.239015in}}%
\pgfpathlineto{\pgfqpoint{4.977127in}{5.244240in}}%
\pgfpathlineto{\pgfqpoint{4.977663in}{5.236452in}}%
\pgfpathlineto{\pgfqpoint{4.978200in}{5.239236in}}%
\pgfpathlineto{\pgfqpoint{4.978736in}{5.240312in}}%
\pgfpathlineto{\pgfqpoint{4.979272in}{5.244708in}}%
\pgfpathlineto{\pgfqpoint{4.979809in}{5.240577in}}%
\pgfpathlineto{\pgfqpoint{4.980345in}{5.239514in}}%
\pgfpathlineto{\pgfqpoint{4.980882in}{5.244488in}}%
\pgfpathlineto{\pgfqpoint{4.981418in}{5.238723in}}%
\pgfpathlineto{\pgfqpoint{4.981955in}{5.240900in}}%
\pgfpathlineto{\pgfqpoint{4.982491in}{5.240034in}}%
\pgfpathlineto{\pgfqpoint{4.983028in}{5.244631in}}%
\pgfpathlineto{\pgfqpoint{4.983564in}{5.242424in}}%
\pgfpathlineto{\pgfqpoint{4.984100in}{5.238311in}}%
\pgfpathlineto{\pgfqpoint{4.984637in}{5.243348in}}%
\pgfpathlineto{\pgfqpoint{4.985173in}{5.238605in}}%
\pgfpathlineto{\pgfqpoint{4.985710in}{5.234377in}}%
\pgfpathlineto{\pgfqpoint{4.986783in}{5.247003in}}%
\pgfpathlineto{\pgfqpoint{4.987319in}{5.244227in}}%
\pgfpathlineto{\pgfqpoint{4.987855in}{5.239612in}}%
\pgfpathlineto{\pgfqpoint{4.988392in}{5.242848in}}%
\pgfpathlineto{\pgfqpoint{4.988928in}{5.243036in}}%
\pgfpathlineto{\pgfqpoint{4.989465in}{5.231527in}}%
\pgfpathlineto{\pgfqpoint{4.990001in}{5.239904in}}%
\pgfpathlineto{\pgfqpoint{4.990538in}{5.246790in}}%
\pgfpathlineto{\pgfqpoint{4.991074in}{5.240718in}}%
\pgfpathlineto{\pgfqpoint{4.991611in}{5.240653in}}%
\pgfpathlineto{\pgfqpoint{4.992683in}{5.244640in}}%
\pgfpathlineto{\pgfqpoint{4.993220in}{5.233667in}}%
\pgfpathlineto{\pgfqpoint{4.993756in}{5.242199in}}%
\pgfpathlineto{\pgfqpoint{4.994293in}{5.249268in}}%
\pgfpathlineto{\pgfqpoint{4.995366in}{5.239095in}}%
\pgfpathlineto{\pgfqpoint{4.995902in}{5.240225in}}%
\pgfpathlineto{\pgfqpoint{4.996438in}{5.245404in}}%
\pgfpathlineto{\pgfqpoint{4.996975in}{5.232665in}}%
\pgfpathlineto{\pgfqpoint{4.997511in}{5.242540in}}%
\pgfpathlineto{\pgfqpoint{4.998048in}{5.244520in}}%
\pgfpathlineto{\pgfqpoint{4.999121in}{5.235743in}}%
\pgfpathlineto{\pgfqpoint{5.000194in}{5.243703in}}%
\pgfpathlineto{\pgfqpoint{5.000730in}{5.232235in}}%
\pgfpathlineto{\pgfqpoint{5.001266in}{5.240565in}}%
\pgfpathlineto{\pgfqpoint{5.001803in}{5.241414in}}%
\pgfpathlineto{\pgfqpoint{5.002339in}{5.239026in}}%
\pgfpathlineto{\pgfqpoint{5.002876in}{5.230290in}}%
\pgfpathlineto{\pgfqpoint{5.003412in}{5.238473in}}%
\pgfpathlineto{\pgfqpoint{5.003949in}{5.241315in}}%
\pgfpathlineto{\pgfqpoint{5.004485in}{5.230169in}}%
\pgfpathlineto{\pgfqpoint{5.005021in}{5.239883in}}%
\pgfpathlineto{\pgfqpoint{5.005558in}{5.238918in}}%
\pgfpathlineto{\pgfqpoint{5.006094in}{5.239725in}}%
\pgfpathlineto{\pgfqpoint{5.006631in}{5.228353in}}%
\pgfpathlineto{\pgfqpoint{5.007167in}{5.238931in}}%
\pgfpathlineto{\pgfqpoint{5.007704in}{5.245802in}}%
\pgfpathlineto{\pgfqpoint{5.008240in}{5.227756in}}%
\pgfpathlineto{\pgfqpoint{5.008776in}{5.237170in}}%
\pgfpathlineto{\pgfqpoint{5.009313in}{5.236456in}}%
\pgfpathlineto{\pgfqpoint{5.009849in}{5.239718in}}%
\pgfpathlineto{\pgfqpoint{5.010386in}{5.231954in}}%
\pgfpathlineto{\pgfqpoint{5.010922in}{5.239347in}}%
\pgfpathlineto{\pgfqpoint{5.011459in}{5.244288in}}%
\pgfpathlineto{\pgfqpoint{5.011995in}{5.226103in}}%
\pgfpathlineto{\pgfqpoint{5.012532in}{5.235468in}}%
\pgfpathlineto{\pgfqpoint{5.013068in}{5.235308in}}%
\pgfpathlineto{\pgfqpoint{5.013604in}{5.236968in}}%
\pgfpathlineto{\pgfqpoint{5.014141in}{5.226720in}}%
\pgfpathlineto{\pgfqpoint{5.014677in}{5.235715in}}%
\pgfpathlineto{\pgfqpoint{5.015214in}{5.241339in}}%
\pgfpathlineto{\pgfqpoint{5.015750in}{5.222303in}}%
\pgfpathlineto{\pgfqpoint{5.016287in}{5.230289in}}%
\pgfpathlineto{\pgfqpoint{5.016823in}{5.234539in}}%
\pgfpathlineto{\pgfqpoint{5.017359in}{5.233016in}}%
\pgfpathlineto{\pgfqpoint{5.017896in}{5.227072in}}%
\pgfpathlineto{\pgfqpoint{5.018432in}{5.234614in}}%
\pgfpathlineto{\pgfqpoint{5.018969in}{5.233942in}}%
\pgfpathlineto{\pgfqpoint{5.019505in}{5.218963in}}%
\pgfpathlineto{\pgfqpoint{5.020042in}{5.225214in}}%
\pgfpathlineto{\pgfqpoint{5.021115in}{5.232248in}}%
\pgfpathlineto{\pgfqpoint{5.021651in}{5.224175in}}%
\pgfpathlineto{\pgfqpoint{5.022187in}{5.231483in}}%
\pgfpathlineto{\pgfqpoint{5.022724in}{5.231634in}}%
\pgfpathlineto{\pgfqpoint{5.023260in}{5.219851in}}%
\pgfpathlineto{\pgfqpoint{5.023797in}{5.225525in}}%
\pgfpathlineto{\pgfqpoint{5.024333in}{5.224021in}}%
\pgfpathlineto{\pgfqpoint{5.024870in}{5.231720in}}%
\pgfpathlineto{\pgfqpoint{5.025406in}{5.221626in}}%
\pgfpathlineto{\pgfqpoint{5.025942in}{5.229255in}}%
\pgfpathlineto{\pgfqpoint{5.026479in}{5.230057in}}%
\pgfpathlineto{\pgfqpoint{5.027015in}{5.217164in}}%
\pgfpathlineto{\pgfqpoint{5.027552in}{5.221009in}}%
\pgfpathlineto{\pgfqpoint{5.028088in}{5.219364in}}%
\pgfpathlineto{\pgfqpoint{5.028625in}{5.227228in}}%
\pgfpathlineto{\pgfqpoint{5.029161in}{5.220294in}}%
\pgfpathlineto{\pgfqpoint{5.030234in}{5.226180in}}%
\pgfpathlineto{\pgfqpoint{5.030770in}{5.214443in}}%
\pgfpathlineto{\pgfqpoint{5.031843in}{5.214498in}}%
\pgfpathlineto{\pgfqpoint{5.032380in}{5.221836in}}%
\pgfpathlineto{\pgfqpoint{5.032916in}{5.214856in}}%
\pgfpathlineto{\pgfqpoint{5.033989in}{5.224938in}}%
\pgfpathlineto{\pgfqpoint{5.034525in}{5.211097in}}%
\pgfpathlineto{\pgfqpoint{5.035062in}{5.211988in}}%
\pgfpathlineto{\pgfqpoint{5.036135in}{5.215843in}}%
\pgfpathlineto{\pgfqpoint{5.036671in}{5.208141in}}%
\pgfpathlineto{\pgfqpoint{5.037744in}{5.221875in}}%
\pgfpathlineto{\pgfqpoint{5.038280in}{5.204849in}}%
\pgfpathlineto{\pgfqpoint{5.038817in}{5.211645in}}%
\pgfpathlineto{\pgfqpoint{5.039353in}{5.210523in}}%
\pgfpathlineto{\pgfqpoint{5.039890in}{5.215531in}}%
\pgfpathlineto{\pgfqpoint{5.040426in}{5.204955in}}%
\pgfpathlineto{\pgfqpoint{5.040963in}{5.214703in}}%
\pgfpathlineto{\pgfqpoint{5.041499in}{5.213340in}}%
\pgfpathlineto{\pgfqpoint{5.042036in}{5.204164in}}%
\pgfpathlineto{\pgfqpoint{5.042572in}{5.206583in}}%
\pgfpathlineto{\pgfqpoint{5.043108in}{5.206213in}}%
\pgfpathlineto{\pgfqpoint{5.043645in}{5.211769in}}%
\pgfpathlineto{\pgfqpoint{5.044181in}{5.204456in}}%
\pgfpathlineto{\pgfqpoint{5.044718in}{5.210635in}}%
\pgfpathlineto{\pgfqpoint{5.045791in}{5.197945in}}%
\pgfpathlineto{\pgfqpoint{5.046863in}{5.198626in}}%
\pgfpathlineto{\pgfqpoint{5.047400in}{5.207719in}}%
\pgfpathlineto{\pgfqpoint{5.047936in}{5.201375in}}%
\pgfpathlineto{\pgfqpoint{5.048473in}{5.201757in}}%
\pgfpathlineto{\pgfqpoint{5.050619in}{5.192499in}}%
\pgfpathlineto{\pgfqpoint{5.051155in}{5.205467in}}%
\pgfpathlineto{\pgfqpoint{5.051691in}{5.195566in}}%
\pgfpathlineto{\pgfqpoint{5.052228in}{5.196746in}}%
\pgfpathlineto{\pgfqpoint{5.053837in}{5.183073in}}%
\pgfpathlineto{\pgfqpoint{5.054374in}{5.187835in}}%
\pgfpathlineto{\pgfqpoint{5.054910in}{5.199275in}}%
\pgfpathlineto{\pgfqpoint{5.055446in}{5.189657in}}%
\pgfpathlineto{\pgfqpoint{5.055983in}{5.192548in}}%
\pgfpathlineto{\pgfqpoint{5.056519in}{5.191996in}}%
\pgfpathlineto{\pgfqpoint{5.058129in}{5.179408in}}%
\pgfpathlineto{\pgfqpoint{5.058665in}{5.191416in}}%
\pgfpathlineto{\pgfqpoint{5.059201in}{5.185799in}}%
\pgfpathlineto{\pgfqpoint{5.059738in}{5.187457in}}%
\pgfpathlineto{\pgfqpoint{5.061347in}{5.180304in}}%
\pgfpathlineto{\pgfqpoint{5.061884in}{5.173674in}}%
\pgfpathlineto{\pgfqpoint{5.063493in}{5.184161in}}%
\pgfpathlineto{\pgfqpoint{5.065639in}{5.170596in}}%
\pgfpathlineto{\pgfqpoint{5.066175in}{5.172713in}}%
\pgfpathlineto{\pgfqpoint{5.066712in}{5.177004in}}%
\pgfpathlineto{\pgfqpoint{5.067248in}{5.175056in}}%
\pgfpathlineto{\pgfqpoint{5.068321in}{5.173193in}}%
\pgfpathlineto{\pgfqpoint{5.069394in}{5.163303in}}%
\pgfpathlineto{\pgfqpoint{5.070467in}{5.171153in}}%
\pgfpathlineto{\pgfqpoint{5.072612in}{5.156635in}}%
\pgfpathlineto{\pgfqpoint{5.074222in}{5.162165in}}%
\pgfpathlineto{\pgfqpoint{5.074758in}{5.153608in}}%
\pgfpathlineto{\pgfqpoint{5.075295in}{5.159080in}}%
\pgfpathlineto{\pgfqpoint{5.076367in}{5.148210in}}%
\pgfpathlineto{\pgfqpoint{5.076904in}{5.155440in}}%
\pgfpathlineto{\pgfqpoint{5.077440in}{5.159138in}}%
\pgfpathlineto{\pgfqpoint{5.077977in}{5.156946in}}%
\pgfpathlineto{\pgfqpoint{5.078513in}{5.142883in}}%
\pgfpathlineto{\pgfqpoint{5.079050in}{5.143270in}}%
\pgfpathlineto{\pgfqpoint{5.080659in}{5.152183in}}%
\pgfpathlineto{\pgfqpoint{5.082268in}{5.140122in}}%
\pgfpathlineto{\pgfqpoint{5.082805in}{5.132399in}}%
\pgfpathlineto{\pgfqpoint{5.084414in}{5.149769in}}%
\pgfpathlineto{\pgfqpoint{5.086560in}{5.126830in}}%
\pgfpathlineto{\pgfqpoint{5.087096in}{5.132926in}}%
\pgfpathlineto{\pgfqpoint{5.087633in}{5.141977in}}%
\pgfpathlineto{\pgfqpoint{5.088169in}{5.138669in}}%
\pgfpathlineto{\pgfqpoint{5.089242in}{5.129905in}}%
\pgfpathlineto{\pgfqpoint{5.089778in}{5.118462in}}%
\pgfpathlineto{\pgfqpoint{5.090315in}{5.120211in}}%
\pgfpathlineto{\pgfqpoint{5.091388in}{5.133813in}}%
\pgfpathlineto{\pgfqpoint{5.091924in}{5.128720in}}%
\pgfpathlineto{\pgfqpoint{5.093533in}{5.108970in}}%
\pgfpathlineto{\pgfqpoint{5.095143in}{5.123318in}}%
\pgfpathlineto{\pgfqpoint{5.095679in}{5.107341in}}%
\pgfpathlineto{\pgfqpoint{5.096216in}{5.107361in}}%
\pgfpathlineto{\pgfqpoint{5.096752in}{5.114746in}}%
\pgfpathlineto{\pgfqpoint{5.097288in}{5.103618in}}%
\pgfpathlineto{\pgfqpoint{5.097825in}{5.111544in}}%
\pgfpathlineto{\pgfqpoint{5.099971in}{5.097587in}}%
\pgfpathlineto{\pgfqpoint{5.100507in}{5.114903in}}%
\pgfpathlineto{\pgfqpoint{5.101043in}{5.093172in}}%
\pgfpathlineto{\pgfqpoint{5.101580in}{5.107179in}}%
\pgfpathlineto{\pgfqpoint{5.103189in}{5.090855in}}%
\pgfpathlineto{\pgfqpoint{5.104262in}{5.114807in}}%
\pgfpathlineto{\pgfqpoint{5.104799in}{5.081517in}}%
\pgfpathlineto{\pgfqpoint{5.105335in}{5.092919in}}%
\pgfpathlineto{\pgfqpoint{5.106408in}{5.077154in}}%
\pgfpathlineto{\pgfqpoint{5.108017in}{5.110103in}}%
\pgfpathlineto{\pgfqpoint{5.108554in}{5.069092in}}%
\pgfpathlineto{\pgfqpoint{5.109626in}{5.071743in}}%
\pgfpathlineto{\pgfqpoint{5.110163in}{5.072210in}}%
\pgfpathlineto{\pgfqpoint{5.111772in}{5.103371in}}%
\pgfpathlineto{\pgfqpoint{5.112309in}{5.050449in}}%
\pgfpathlineto{\pgfqpoint{5.112845in}{5.061205in}}%
\pgfpathlineto{\pgfqpoint{5.113382in}{5.055424in}}%
\pgfpathlineto{\pgfqpoint{5.113382in}{5.055424in}}%
\pgfpathlineto{\pgfqpoint{5.113382in}{5.055424in}}%
\pgfpathlineto{\pgfqpoint{5.113918in}{5.061523in}}%
\pgfpathlineto{\pgfqpoint{5.114454in}{5.088610in}}%
\pgfpathlineto{\pgfqpoint{5.114991in}{5.070299in}}%
\pgfpathlineto{\pgfqpoint{5.115527in}{5.081921in}}%
\pgfpathlineto{\pgfqpoint{5.116064in}{5.031674in}}%
\pgfpathlineto{\pgfqpoint{5.116600in}{5.050518in}}%
\pgfpathlineto{\pgfqpoint{5.117137in}{5.036179in}}%
\pgfpathlineto{\pgfqpoint{5.118209in}{5.083565in}}%
\pgfpathlineto{\pgfqpoint{5.119819in}{5.018156in}}%
\pgfpathlineto{\pgfqpoint{5.121965in}{5.070283in}}%
\pgfpathlineto{\pgfqpoint{5.123574in}{5.019767in}}%
\pgfpathlineto{\pgfqpoint{5.124110in}{5.041800in}}%
\pgfpathlineto{\pgfqpoint{5.124647in}{5.032170in}}%
\pgfpathlineto{\pgfqpoint{5.125720in}{5.041052in}}%
\pgfpathlineto{\pgfqpoint{5.126256in}{5.016852in}}%
\pgfpathlineto{\pgfqpoint{5.126792in}{5.044396in}}%
\pgfpathlineto{\pgfqpoint{5.127329in}{5.029873in}}%
\pgfpathlineto{\pgfqpoint{5.127865in}{5.038085in}}%
\pgfpathlineto{\pgfqpoint{5.129475in}{5.000111in}}%
\pgfpathlineto{\pgfqpoint{5.130011in}{5.001092in}}%
\pgfpathlineto{\pgfqpoint{5.130547in}{5.047102in}}%
\pgfpathlineto{\pgfqpoint{5.131084in}{5.026556in}}%
\pgfpathlineto{\pgfqpoint{5.131620in}{5.024096in}}%
\pgfpathlineto{\pgfqpoint{5.132693in}{4.983376in}}%
\pgfpathlineto{\pgfqpoint{5.133230in}{4.997563in}}%
\pgfpathlineto{\pgfqpoint{5.133766in}{5.002529in}}%
\pgfpathlineto{\pgfqpoint{5.134303in}{5.067364in}}%
\pgfpathlineto{\pgfqpoint{5.134839in}{5.021001in}}%
\pgfpathlineto{\pgfqpoint{5.135375in}{5.007413in}}%
\pgfpathlineto{\pgfqpoint{5.135912in}{4.960953in}}%
\pgfpathlineto{\pgfqpoint{5.136448in}{4.966446in}}%
\pgfpathlineto{\pgfqpoint{5.138058in}{5.051882in}}%
\pgfpathlineto{\pgfqpoint{5.139667in}{4.939936in}}%
\pgfpathlineto{\pgfqpoint{5.141276in}{5.011124in}}%
\pgfpathlineto{\pgfqpoint{5.141813in}{5.027045in}}%
\pgfpathlineto{\pgfqpoint{5.143422in}{4.928836in}}%
\pgfpathlineto{\pgfqpoint{5.145568in}{5.003461in}}%
\pgfpathlineto{\pgfqpoint{5.147177in}{4.929726in}}%
\pgfpathlineto{\pgfqpoint{5.147713in}{4.941660in}}%
\pgfpathlineto{\pgfqpoint{5.148250in}{4.985868in}}%
\pgfpathlineto{\pgfqpoint{5.148786in}{4.980937in}}%
\pgfpathlineto{\pgfqpoint{5.149859in}{4.916787in}}%
\pgfpathlineto{\pgfqpoint{5.150396in}{4.962159in}}%
\pgfpathlineto{\pgfqpoint{5.150932in}{4.922390in}}%
\pgfpathlineto{\pgfqpoint{5.151468in}{4.913482in}}%
\pgfpathlineto{\pgfqpoint{5.152005in}{4.958048in}}%
\pgfpathlineto{\pgfqpoint{5.152541in}{4.937774in}}%
\pgfpathlineto{\pgfqpoint{5.154151in}{4.961412in}}%
\pgfpathlineto{\pgfqpoint{5.155224in}{4.847846in}}%
\pgfpathlineto{\pgfqpoint{5.156296in}{4.893469in}}%
\pgfpathlineto{\pgfqpoint{5.157369in}{4.973456in}}%
\pgfpathlineto{\pgfqpoint{5.157906in}{4.961369in}}%
\pgfpathlineto{\pgfqpoint{5.158979in}{4.797861in}}%
\pgfpathlineto{\pgfqpoint{5.159515in}{4.840929in}}%
\pgfpathlineto{\pgfqpoint{5.160051in}{4.859707in}}%
\pgfpathlineto{\pgfqpoint{5.160588in}{5.001909in}}%
\pgfpathlineto{\pgfqpoint{5.161124in}{4.980989in}}%
\pgfpathlineto{\pgfqpoint{5.162734in}{4.751408in}}%
\pgfpathlineto{\pgfqpoint{5.163270in}{4.836616in}}%
\pgfpathlineto{\pgfqpoint{5.164343in}{4.998971in}}%
\pgfpathlineto{\pgfqpoint{5.164879in}{4.957258in}}%
\pgfpathlineto{\pgfqpoint{5.165952in}{4.739698in}}%
\pgfpathlineto{\pgfqpoint{5.166489in}{4.765934in}}%
\pgfpathlineto{\pgfqpoint{5.168098in}{4.989551in}}%
\pgfpathlineto{\pgfqpoint{5.169707in}{4.776839in}}%
\pgfpathlineto{\pgfqpoint{5.170244in}{4.792783in}}%
\pgfpathlineto{\pgfqpoint{5.171317in}{4.922280in}}%
\pgfpathlineto{\pgfqpoint{5.171853in}{4.894854in}}%
\pgfpathlineto{\pgfqpoint{5.172390in}{4.906810in}}%
\pgfpathlineto{\pgfqpoint{5.173462in}{4.796794in}}%
\pgfpathlineto{\pgfqpoint{5.173999in}{4.849499in}}%
\pgfpathlineto{\pgfqpoint{5.174535in}{4.904072in}}%
\pgfpathlineto{\pgfqpoint{5.175072in}{4.866739in}}%
\pgfpathlineto{\pgfqpoint{5.175608in}{4.839982in}}%
\pgfpathlineto{\pgfqpoint{5.176145in}{4.860832in}}%
\pgfpathlineto{\pgfqpoint{5.176681in}{4.968764in}}%
\pgfpathlineto{\pgfqpoint{5.177754in}{4.814060in}}%
\pgfpathlineto{\pgfqpoint{5.178290in}{4.839902in}}%
\pgfpathlineto{\pgfqpoint{5.179363in}{4.829527in}}%
\pgfpathlineto{\pgfqpoint{5.180436in}{4.982512in}}%
\pgfpathlineto{\pgfqpoint{5.181509in}{4.700731in}}%
\pgfpathlineto{\pgfqpoint{5.182045in}{4.762031in}}%
\pgfpathlineto{\pgfqpoint{5.182582in}{4.771980in}}%
\pgfpathlineto{\pgfqpoint{5.183118in}{4.947635in}}%
\pgfpathlineto{\pgfqpoint{5.184191in}{4.942736in}}%
\pgfpathlineto{\pgfqpoint{5.184728in}{4.792915in}}%
\pgfpathlineto{\pgfqpoint{5.185264in}{4.307723in}}%
\pgfpathlineto{\pgfqpoint{5.185800in}{4.690184in}}%
\pgfpathlineto{\pgfqpoint{5.187410in}{4.948256in}}%
\pgfpathlineto{\pgfqpoint{5.189019in}{4.407541in}}%
\pgfpathlineto{\pgfqpoint{5.189555in}{4.565446in}}%
\pgfpathlineto{\pgfqpoint{5.190628in}{5.000999in}}%
\pgfpathlineto{\pgfqpoint{5.192774in}{4.466045in}}%
\pgfpathlineto{\pgfqpoint{5.193847in}{4.959386in}}%
\pgfpathlineto{\pgfqpoint{5.194383in}{4.900747in}}%
\pgfpathlineto{\pgfqpoint{5.195456in}{4.625308in}}%
\pgfpathlineto{\pgfqpoint{5.195993in}{4.700277in}}%
\pgfpathlineto{\pgfqpoint{5.196529in}{4.709163in}}%
\pgfpathlineto{\pgfqpoint{5.197066in}{4.675454in}}%
\pgfpathlineto{\pgfqpoint{5.198138in}{4.911120in}}%
\pgfpathlineto{\pgfqpoint{5.198675in}{4.664010in}}%
\pgfpathlineto{\pgfqpoint{5.199211in}{4.789115in}}%
\pgfpathlineto{\pgfqpoint{5.199748in}{4.781305in}}%
\pgfpathlineto{\pgfqpoint{5.200284in}{4.639625in}}%
\pgfpathlineto{\pgfqpoint{5.200821in}{4.814290in}}%
\pgfpathlineto{\pgfqpoint{5.201357in}{4.546802in}}%
\pgfpathlineto{\pgfqpoint{5.201893in}{4.703331in}}%
\pgfpathlineto{\pgfqpoint{5.202430in}{4.961688in}}%
\pgfpathlineto{\pgfqpoint{5.202966in}{4.730351in}}%
\pgfpathlineto{\pgfqpoint{5.203503in}{4.802960in}}%
\pgfpathlineto{\pgfqpoint{5.204576in}{4.513921in}}%
\pgfpathlineto{\pgfqpoint{5.205112in}{4.594641in}}%
\pgfpathlineto{\pgfqpoint{5.206185in}{5.055353in}}%
\pgfpathlineto{\pgfqpoint{5.206721in}{4.886965in}}%
\pgfpathlineto{\pgfqpoint{5.208331in}{4.419883in}}%
\pgfpathlineto{\pgfqpoint{5.209940in}{4.994293in}}%
\pgfpathlineto{\pgfqpoint{5.210476in}{4.821534in}}%
\pgfpathlineto{\pgfqpoint{5.211013in}{4.257607in}}%
\pgfpathlineto{\pgfqpoint{5.211549in}{4.360412in}}%
\pgfpathlineto{\pgfqpoint{5.213159in}{4.936848in}}%
\pgfpathlineto{\pgfqpoint{5.214232in}{4.764609in}}%
\pgfpathlineto{\pgfqpoint{5.214768in}{3.936643in}}%
\pgfpathlineto{\pgfqpoint{5.215304in}{4.349935in}}%
\pgfpathlineto{\pgfqpoint{5.216914in}{4.944336in}}%
\pgfpathlineto{\pgfqpoint{5.219059in}{4.392050in}}%
\pgfpathlineto{\pgfqpoint{5.219596in}{4.951086in}}%
\pgfpathlineto{\pgfqpoint{5.220132in}{4.927577in}}%
\pgfpathlineto{\pgfqpoint{5.220669in}{4.561956in}}%
\pgfpathlineto{\pgfqpoint{5.221205in}{4.828411in}}%
\pgfpathlineto{\pgfqpoint{5.222278in}{4.739574in}}%
\pgfpathlineto{\pgfqpoint{5.222815in}{4.786425in}}%
\pgfpathlineto{\pgfqpoint{5.223351in}{4.809315in}}%
\pgfpathlineto{\pgfqpoint{5.223887in}{4.658434in}}%
\pgfpathlineto{\pgfqpoint{5.224424in}{4.674743in}}%
\pgfpathlineto{\pgfqpoint{5.224960in}{4.709174in}}%
\pgfpathlineto{\pgfqpoint{5.225497in}{4.984314in}}%
\pgfpathlineto{\pgfqpoint{5.226033in}{4.812568in}}%
\pgfpathlineto{\pgfqpoint{5.226570in}{4.623034in}}%
\pgfpathlineto{\pgfqpoint{5.227106in}{4.736684in}}%
\pgfpathlineto{\pgfqpoint{5.227642in}{4.674815in}}%
\pgfpathlineto{\pgfqpoint{5.228715in}{4.926460in}}%
\pgfpathlineto{\pgfqpoint{5.230861in}{4.502977in}}%
\pgfpathlineto{\pgfqpoint{5.231934in}{4.913137in}}%
\pgfpathlineto{\pgfqpoint{5.232470in}{4.898435in}}%
\pgfpathlineto{\pgfqpoint{5.234080in}{4.478583in}}%
\pgfpathlineto{\pgfqpoint{5.235689in}{4.979504in}}%
\pgfpathlineto{\pgfqpoint{5.236762in}{4.447733in}}%
\pgfpathlineto{\pgfqpoint{5.237835in}{4.647482in}}%
\pgfpathlineto{\pgfqpoint{5.238908in}{5.003923in}}%
\pgfpathlineto{\pgfqpoint{5.239444in}{4.890902in}}%
\pgfpathlineto{\pgfqpoint{5.240517in}{4.691944in}}%
\pgfpathlineto{\pgfqpoint{5.241053in}{4.713215in}}%
\pgfpathlineto{\pgfqpoint{5.242126in}{4.936228in}}%
\pgfpathlineto{\pgfqpoint{5.243736in}{4.762822in}}%
\pgfpathlineto{\pgfqpoint{5.244272in}{4.787658in}}%
\pgfpathlineto{\pgfqpoint{5.244808in}{4.911034in}}%
\pgfpathlineto{\pgfqpoint{5.245345in}{4.796444in}}%
\pgfpathlineto{\pgfqpoint{5.245881in}{4.824298in}}%
\pgfpathlineto{\pgfqpoint{5.246418in}{4.782910in}}%
\pgfpathlineto{\pgfqpoint{5.246954in}{4.801488in}}%
\pgfpathlineto{\pgfqpoint{5.247491in}{4.967039in}}%
\pgfpathlineto{\pgfqpoint{5.248027in}{4.895753in}}%
\pgfpathlineto{\pgfqpoint{5.249100in}{4.874763in}}%
\pgfpathlineto{\pgfqpoint{5.249636in}{4.746645in}}%
\pgfpathlineto{\pgfqpoint{5.250173in}{4.837385in}}%
\pgfpathlineto{\pgfqpoint{5.250709in}{4.904798in}}%
\pgfpathlineto{\pgfqpoint{5.251246in}{4.877957in}}%
\pgfpathlineto{\pgfqpoint{5.251782in}{4.891273in}}%
\pgfpathlineto{\pgfqpoint{5.252855in}{4.751422in}}%
\pgfpathlineto{\pgfqpoint{5.254464in}{4.992646in}}%
\pgfpathlineto{\pgfqpoint{5.255001in}{4.956068in}}%
\pgfpathlineto{\pgfqpoint{5.255537in}{4.700303in}}%
\pgfpathlineto{\pgfqpoint{5.256074in}{4.779448in}}%
\pgfpathlineto{\pgfqpoint{5.256610in}{4.810059in}}%
\pgfpathlineto{\pgfqpoint{5.257146in}{4.997937in}}%
\pgfpathlineto{\pgfqpoint{5.257683in}{4.993012in}}%
\pgfpathlineto{\pgfqpoint{5.259292in}{4.692318in}}%
\pgfpathlineto{\pgfqpoint{5.259829in}{4.774214in}}%
\pgfpathlineto{\pgfqpoint{5.260901in}{4.987609in}}%
\pgfpathlineto{\pgfqpoint{5.261438in}{4.981328in}}%
\pgfpathlineto{\pgfqpoint{5.263047in}{4.860848in}}%
\pgfpathlineto{\pgfqpoint{5.263584in}{4.886530in}}%
\pgfpathlineto{\pgfqpoint{5.264657in}{4.970374in}}%
\pgfpathlineto{\pgfqpoint{5.265729in}{4.874672in}}%
\pgfpathlineto{\pgfqpoint{5.267339in}{4.956303in}}%
\pgfpathlineto{\pgfqpoint{5.267875in}{4.953337in}}%
\pgfpathlineto{\pgfqpoint{5.269484in}{4.911452in}}%
\pgfpathlineto{\pgfqpoint{5.270021in}{4.980630in}}%
\pgfpathlineto{\pgfqpoint{5.270557in}{4.953706in}}%
\pgfpathlineto{\pgfqpoint{5.271094in}{4.882984in}}%
\pgfpathlineto{\pgfqpoint{5.271630in}{4.924460in}}%
\pgfpathlineto{\pgfqpoint{5.273240in}{4.996626in}}%
\pgfpathlineto{\pgfqpoint{5.274312in}{4.914594in}}%
\pgfpathlineto{\pgfqpoint{5.274849in}{4.927676in}}%
\pgfpathlineto{\pgfqpoint{5.275385in}{4.949000in}}%
\pgfpathlineto{\pgfqpoint{5.275922in}{5.025979in}}%
\pgfpathlineto{\pgfqpoint{5.276458in}{4.998184in}}%
\pgfpathlineto{\pgfqpoint{5.278067in}{4.858499in}}%
\pgfpathlineto{\pgfqpoint{5.279677in}{5.051202in}}%
\pgfpathlineto{\pgfqpoint{5.280213in}{5.033547in}}%
\pgfpathlineto{\pgfqpoint{5.281822in}{4.922472in}}%
\pgfpathlineto{\pgfqpoint{5.283432in}{5.039598in}}%
\pgfpathlineto{\pgfqpoint{5.284505in}{4.942131in}}%
\pgfpathlineto{\pgfqpoint{5.286114in}{5.033276in}}%
\pgfpathlineto{\pgfqpoint{5.287723in}{4.995139in}}%
\pgfpathlineto{\pgfqpoint{5.288260in}{4.993934in}}%
\pgfpathlineto{\pgfqpoint{5.288796in}{5.039490in}}%
\pgfpathlineto{\pgfqpoint{5.289333in}{4.990018in}}%
\pgfpathlineto{\pgfqpoint{5.289869in}{5.005150in}}%
\pgfpathlineto{\pgfqpoint{5.290405in}{5.020019in}}%
\pgfpathlineto{\pgfqpoint{5.290942in}{4.989852in}}%
\pgfpathlineto{\pgfqpoint{5.291478in}{5.011013in}}%
\pgfpathlineto{\pgfqpoint{5.292015in}{5.003343in}}%
\pgfpathlineto{\pgfqpoint{5.292551in}{5.033225in}}%
\pgfpathlineto{\pgfqpoint{5.293088in}{5.019617in}}%
\pgfpathlineto{\pgfqpoint{5.293624in}{5.014192in}}%
\pgfpathlineto{\pgfqpoint{5.294697in}{5.043022in}}%
\pgfpathlineto{\pgfqpoint{5.296843in}{4.992947in}}%
\pgfpathlineto{\pgfqpoint{5.297916in}{5.078940in}}%
\pgfpathlineto{\pgfqpoint{5.298452in}{5.068175in}}%
\pgfpathlineto{\pgfqpoint{5.300061in}{5.013303in}}%
\pgfpathlineto{\pgfqpoint{5.301671in}{5.078608in}}%
\pgfpathlineto{\pgfqpoint{5.302207in}{5.063368in}}%
\pgfpathlineto{\pgfqpoint{5.302743in}{5.021045in}}%
\pgfpathlineto{\pgfqpoint{5.303280in}{5.033466in}}%
\pgfpathlineto{\pgfqpoint{5.304889in}{5.071873in}}%
\pgfpathlineto{\pgfqpoint{5.306499in}{5.043242in}}%
\pgfpathlineto{\pgfqpoint{5.307035in}{5.036484in}}%
\pgfpathlineto{\pgfqpoint{5.307571in}{5.082363in}}%
\pgfpathlineto{\pgfqpoint{5.308644in}{5.080002in}}%
\pgfpathlineto{\pgfqpoint{5.309717in}{5.047216in}}%
\pgfpathlineto{\pgfqpoint{5.310254in}{5.053567in}}%
\pgfpathlineto{\pgfqpoint{5.310790in}{5.053169in}}%
\pgfpathlineto{\pgfqpoint{5.311326in}{5.095304in}}%
\pgfpathlineto{\pgfqpoint{5.311863in}{5.083010in}}%
\pgfpathlineto{\pgfqpoint{5.312399in}{5.063998in}}%
\pgfpathlineto{\pgfqpoint{5.312936in}{5.088474in}}%
\pgfpathlineto{\pgfqpoint{5.313472in}{5.075335in}}%
\pgfpathlineto{\pgfqpoint{5.314009in}{5.086375in}}%
\pgfpathlineto{\pgfqpoint{5.314545in}{5.054141in}}%
\pgfpathlineto{\pgfqpoint{5.316154in}{5.093956in}}%
\pgfpathlineto{\pgfqpoint{5.316691in}{5.123116in}}%
\pgfpathlineto{\pgfqpoint{5.316691in}{5.123116in}}%
\pgfpathlineto{\pgfqpoint{5.316691in}{5.123116in}}%
\pgfpathlineto{\pgfqpoint{5.318300in}{5.067045in}}%
\pgfpathlineto{\pgfqpoint{5.320446in}{5.121523in}}%
\pgfpathlineto{\pgfqpoint{5.321519in}{5.080443in}}%
\pgfpathlineto{\pgfqpoint{5.323665in}{5.125255in}}%
\pgfpathlineto{\pgfqpoint{5.324737in}{5.095562in}}%
\pgfpathlineto{\pgfqpoint{5.325274in}{5.105491in}}%
\pgfpathlineto{\pgfqpoint{5.326883in}{5.132287in}}%
\pgfpathlineto{\pgfqpoint{5.328492in}{5.099714in}}%
\pgfpathlineto{\pgfqpoint{5.329565in}{5.125189in}}%
\pgfpathlineto{\pgfqpoint{5.330102in}{5.124344in}}%
\pgfpathlineto{\pgfqpoint{5.330638in}{5.118880in}}%
\pgfpathlineto{\pgfqpoint{5.331175in}{5.137344in}}%
\pgfpathlineto{\pgfqpoint{5.331711in}{5.119351in}}%
\pgfpathlineto{\pgfqpoint{5.332784in}{5.134821in}}%
\pgfpathlineto{\pgfqpoint{5.333320in}{5.116229in}}%
\pgfpathlineto{\pgfqpoint{5.333857in}{5.124166in}}%
\pgfpathlineto{\pgfqpoint{5.335466in}{5.145000in}}%
\pgfpathlineto{\pgfqpoint{5.337075in}{5.119313in}}%
\pgfpathlineto{\pgfqpoint{5.338685in}{5.141031in}}%
\pgfpathlineto{\pgfqpoint{5.339221in}{5.140771in}}%
\pgfpathlineto{\pgfqpoint{5.340830in}{5.132584in}}%
\pgfpathlineto{\pgfqpoint{5.341367in}{5.152781in}}%
\pgfpathlineto{\pgfqpoint{5.341903in}{5.149515in}}%
\pgfpathlineto{\pgfqpoint{5.342440in}{5.125965in}}%
\pgfpathlineto{\pgfqpoint{5.342976in}{5.152112in}}%
\pgfpathlineto{\pgfqpoint{5.343513in}{5.139941in}}%
\pgfpathlineto{\pgfqpoint{5.344049in}{5.143884in}}%
\pgfpathlineto{\pgfqpoint{5.344586in}{5.159052in}}%
\pgfpathlineto{\pgfqpoint{5.345122in}{5.149284in}}%
\pgfpathlineto{\pgfqpoint{5.346195in}{5.142617in}}%
\pgfpathlineto{\pgfqpoint{5.348341in}{5.163641in}}%
\pgfpathlineto{\pgfqpoint{5.349413in}{5.152430in}}%
\pgfpathlineto{\pgfqpoint{5.349950in}{5.157270in}}%
\pgfpathlineto{\pgfqpoint{5.350486in}{5.166837in}}%
\pgfpathlineto{\pgfqpoint{5.351023in}{5.164579in}}%
\pgfpathlineto{\pgfqpoint{5.351559in}{5.164846in}}%
\pgfpathlineto{\pgfqpoint{5.352096in}{5.171426in}}%
\pgfpathlineto{\pgfqpoint{5.352632in}{5.156411in}}%
\pgfpathlineto{\pgfqpoint{5.353168in}{5.167181in}}%
\pgfpathlineto{\pgfqpoint{5.353705in}{5.171767in}}%
\pgfpathlineto{\pgfqpoint{5.354241in}{5.170643in}}%
\pgfpathlineto{\pgfqpoint{5.354778in}{5.169157in}}%
\pgfpathlineto{\pgfqpoint{5.355314in}{5.182424in}}%
\pgfpathlineto{\pgfqpoint{5.355851in}{5.166252in}}%
\pgfpathlineto{\pgfqpoint{5.356387in}{5.176684in}}%
\pgfpathlineto{\pgfqpoint{5.356924in}{5.181659in}}%
\pgfpathlineto{\pgfqpoint{5.357460in}{5.171921in}}%
\pgfpathlineto{\pgfqpoint{5.357996in}{5.175235in}}%
\pgfpathlineto{\pgfqpoint{5.358533in}{5.175573in}}%
\pgfpathlineto{\pgfqpoint{5.359069in}{5.193247in}}%
\pgfpathlineto{\pgfqpoint{5.359606in}{5.186604in}}%
\pgfpathlineto{\pgfqpoint{5.361215in}{5.174506in}}%
\pgfpathlineto{\pgfqpoint{5.362824in}{5.200955in}}%
\pgfpathlineto{\pgfqpoint{5.364434in}{5.181351in}}%
\pgfpathlineto{\pgfqpoint{5.364970in}{5.169948in}}%
\pgfpathlineto{\pgfqpoint{5.366043in}{5.192923in}}%
\pgfpathlineto{\pgfqpoint{5.366579in}{5.188428in}}%
\pgfpathlineto{\pgfqpoint{5.367116in}{5.191792in}}%
\pgfpathlineto{\pgfqpoint{5.368725in}{5.181477in}}%
\pgfpathlineto{\pgfqpoint{5.369262in}{5.194485in}}%
\pgfpathlineto{\pgfqpoint{5.369798in}{5.187931in}}%
\pgfpathlineto{\pgfqpoint{5.370871in}{5.196233in}}%
\pgfpathlineto{\pgfqpoint{5.371407in}{5.194942in}}%
\pgfpathlineto{\pgfqpoint{5.371944in}{5.188445in}}%
\pgfpathlineto{\pgfqpoint{5.372480in}{5.189025in}}%
\pgfpathlineto{\pgfqpoint{5.374626in}{5.204958in}}%
\pgfpathlineto{\pgfqpoint{5.375699in}{5.181574in}}%
\pgfpathlineto{\pgfqpoint{5.376235in}{5.196397in}}%
\pgfpathlineto{\pgfqpoint{5.378381in}{5.210770in}}%
\pgfpathlineto{\pgfqpoint{5.378917in}{5.207694in}}%
\pgfpathlineto{\pgfqpoint{5.379454in}{5.195575in}}%
\pgfpathlineto{\pgfqpoint{5.379990in}{5.198105in}}%
\pgfpathlineto{\pgfqpoint{5.381063in}{5.218472in}}%
\pgfpathlineto{\pgfqpoint{5.382136in}{5.212239in}}%
\pgfpathlineto{\pgfqpoint{5.383209in}{5.207412in}}%
\pgfpathlineto{\pgfqpoint{5.384818in}{5.221693in}}%
\pgfpathlineto{\pgfqpoint{5.385355in}{5.213017in}}%
\pgfpathlineto{\pgfqpoint{5.385891in}{5.223264in}}%
\pgfpathlineto{\pgfqpoint{5.386428in}{5.214882in}}%
\pgfpathlineto{\pgfqpoint{5.386964in}{5.218358in}}%
\pgfpathlineto{\pgfqpoint{5.388037in}{5.212327in}}%
\pgfpathlineto{\pgfqpoint{5.388573in}{5.228314in}}%
\pgfpathlineto{\pgfqpoint{5.389110in}{5.212685in}}%
\pgfpathlineto{\pgfqpoint{5.389646in}{5.220419in}}%
\pgfpathlineto{\pgfqpoint{5.390183in}{5.216951in}}%
\pgfpathlineto{\pgfqpoint{5.391255in}{5.220114in}}%
\pgfpathlineto{\pgfqpoint{5.391792in}{5.218565in}}%
\pgfpathlineto{\pgfqpoint{5.392328in}{5.226795in}}%
\pgfpathlineto{\pgfqpoint{5.392865in}{5.216652in}}%
\pgfpathlineto{\pgfqpoint{5.393401in}{5.229775in}}%
\pgfpathlineto{\pgfqpoint{5.393938in}{5.214201in}}%
\pgfpathlineto{\pgfqpoint{5.394474in}{5.218445in}}%
\pgfpathlineto{\pgfqpoint{5.395011in}{5.222892in}}%
\pgfpathlineto{\pgfqpoint{5.395547in}{5.215048in}}%
\pgfpathlineto{\pgfqpoint{5.395547in}{5.215048in}}%
\pgfpathlineto{\pgfqpoint{5.395547in}{5.215048in}}%
\pgfpathlineto{\pgfqpoint{5.397156in}{5.226194in}}%
\pgfpathlineto{\pgfqpoint{5.397693in}{5.211573in}}%
\pgfpathlineto{\pgfqpoint{5.398229in}{5.218632in}}%
\pgfpathlineto{\pgfqpoint{5.398766in}{5.228905in}}%
\pgfpathlineto{\pgfqpoint{5.399302in}{5.223510in}}%
\pgfpathlineto{\pgfqpoint{5.400375in}{5.221122in}}%
\pgfpathlineto{\pgfqpoint{5.400911in}{5.237498in}}%
\pgfpathlineto{\pgfqpoint{5.401448in}{5.214084in}}%
\pgfpathlineto{\pgfqpoint{5.401984in}{5.223655in}}%
\pgfpathlineto{\pgfqpoint{5.403057in}{5.228763in}}%
\pgfpathlineto{\pgfqpoint{5.404130in}{5.220724in}}%
\pgfpathlineto{\pgfqpoint{5.404666in}{5.231896in}}%
\pgfpathlineto{\pgfqpoint{5.405203in}{5.219639in}}%
\pgfpathlineto{\pgfqpoint{5.405739in}{5.229686in}}%
\pgfpathlineto{\pgfqpoint{5.406276in}{5.236117in}}%
\pgfpathlineto{\pgfqpoint{5.406812in}{5.235392in}}%
\pgfpathlineto{\pgfqpoint{5.407349in}{5.222215in}}%
\pgfpathlineto{\pgfqpoint{5.407885in}{5.229704in}}%
\pgfpathlineto{\pgfqpoint{5.408421in}{5.234065in}}%
\pgfpathlineto{\pgfqpoint{5.408958in}{5.228685in}}%
\pgfpathlineto{\pgfqpoint{5.409494in}{5.231792in}}%
\pgfpathlineto{\pgfqpoint{5.410567in}{5.240360in}}%
\pgfpathlineto{\pgfqpoint{5.411640in}{5.223591in}}%
\pgfpathlineto{\pgfqpoint{5.412713in}{5.238284in}}%
\pgfpathlineto{\pgfqpoint{5.413249in}{5.234003in}}%
\pgfpathlineto{\pgfqpoint{5.414322in}{5.241338in}}%
\pgfpathlineto{\pgfqpoint{5.415395in}{5.230869in}}%
\pgfpathlineto{\pgfqpoint{5.415932in}{5.232188in}}%
\pgfpathlineto{\pgfqpoint{5.416468in}{5.238679in}}%
\pgfpathlineto{\pgfqpoint{5.417004in}{5.237566in}}%
\pgfpathlineto{\pgfqpoint{5.417541in}{5.235491in}}%
\pgfpathlineto{\pgfqpoint{5.418077in}{5.239315in}}%
\pgfpathlineto{\pgfqpoint{5.419687in}{5.227163in}}%
\pgfpathlineto{\pgfqpoint{5.421296in}{5.237900in}}%
\pgfpathlineto{\pgfqpoint{5.421832in}{5.239756in}}%
\pgfpathlineto{\pgfqpoint{5.423442in}{5.226270in}}%
\pgfpathlineto{\pgfqpoint{5.425051in}{5.239554in}}%
\pgfpathlineto{\pgfqpoint{5.426124in}{5.230290in}}%
\pgfpathlineto{\pgfqpoint{5.428806in}{5.241174in}}%
\pgfpathlineto{\pgfqpoint{5.429879in}{5.230326in}}%
\pgfpathlineto{\pgfqpoint{5.431488in}{5.237231in}}%
\pgfpathlineto{\pgfqpoint{5.432025in}{5.240014in}}%
\pgfpathlineto{\pgfqpoint{5.433634in}{5.232780in}}%
\pgfpathlineto{\pgfqpoint{5.434707in}{5.238318in}}%
\pgfpathlineto{\pgfqpoint{5.435243in}{5.237259in}}%
\pgfpathlineto{\pgfqpoint{5.435780in}{5.236101in}}%
\pgfpathlineto{\pgfqpoint{5.436316in}{5.239128in}}%
\pgfpathlineto{\pgfqpoint{5.436853in}{5.234759in}}%
\pgfpathlineto{\pgfqpoint{5.437389in}{5.236072in}}%
\pgfpathlineto{\pgfqpoint{5.437925in}{5.239987in}}%
\pgfpathlineto{\pgfqpoint{5.438462in}{5.238857in}}%
\pgfpathlineto{\pgfqpoint{5.440071in}{5.233642in}}%
\pgfpathlineto{\pgfqpoint{5.440608in}{5.239035in}}%
\pgfpathlineto{\pgfqpoint{5.441144in}{5.238649in}}%
\pgfpathlineto{\pgfqpoint{5.442753in}{5.235231in}}%
\pgfpathlineto{\pgfqpoint{5.443290in}{5.237447in}}%
\pgfpathlineto{\pgfqpoint{5.443826in}{5.234520in}}%
\pgfpathlineto{\pgfqpoint{5.444363in}{5.237483in}}%
\pgfpathlineto{\pgfqpoint{5.444899in}{5.237841in}}%
\pgfpathlineto{\pgfqpoint{5.445436in}{5.244664in}}%
\pgfpathlineto{\pgfqpoint{5.445972in}{5.232749in}}%
\pgfpathlineto{\pgfqpoint{5.446508in}{5.236012in}}%
\pgfpathlineto{\pgfqpoint{5.447045in}{5.236422in}}%
\pgfpathlineto{\pgfqpoint{5.447581in}{5.231713in}}%
\pgfpathlineto{\pgfqpoint{5.448118in}{5.232691in}}%
\pgfpathlineto{\pgfqpoint{5.449191in}{5.244594in}}%
\pgfpathlineto{\pgfqpoint{5.450800in}{5.230072in}}%
\pgfpathlineto{\pgfqpoint{5.451336in}{5.234210in}}%
\pgfpathlineto{\pgfqpoint{5.451873in}{5.225800in}}%
\pgfpathlineto{\pgfqpoint{5.452946in}{5.248129in}}%
\pgfpathlineto{\pgfqpoint{5.454018in}{5.230557in}}%
\pgfpathlineto{\pgfqpoint{5.454555in}{5.232326in}}%
\pgfpathlineto{\pgfqpoint{5.455091in}{5.235416in}}%
\pgfpathlineto{\pgfqpoint{5.455628in}{5.227910in}}%
\pgfpathlineto{\pgfqpoint{5.456701in}{5.242954in}}%
\pgfpathlineto{\pgfqpoint{5.458310in}{5.227979in}}%
\pgfpathlineto{\pgfqpoint{5.458846in}{5.240140in}}%
\pgfpathlineto{\pgfqpoint{5.459383in}{5.223519in}}%
\pgfpathlineto{\pgfqpoint{5.459919in}{5.229941in}}%
\pgfpathlineto{\pgfqpoint{5.460456in}{5.235250in}}%
\pgfpathlineto{\pgfqpoint{5.460992in}{5.231601in}}%
\pgfpathlineto{\pgfqpoint{5.461529in}{5.228902in}}%
\pgfpathlineto{\pgfqpoint{5.462065in}{5.229938in}}%
\pgfpathlineto{\pgfqpoint{5.462601in}{5.238810in}}%
\pgfpathlineto{\pgfqpoint{5.463138in}{5.227549in}}%
\pgfpathlineto{\pgfqpoint{5.464211in}{5.228049in}}%
\pgfpathlineto{\pgfqpoint{5.464747in}{5.233578in}}%
\pgfpathlineto{\pgfqpoint{5.465820in}{5.223464in}}%
\pgfpathlineto{\pgfqpoint{5.466357in}{5.231206in}}%
\pgfpathlineto{\pgfqpoint{5.466893in}{5.228261in}}%
\pgfpathlineto{\pgfqpoint{5.467966in}{5.223486in}}%
\pgfpathlineto{\pgfqpoint{5.468502in}{5.229058in}}%
\pgfpathlineto{\pgfqpoint{5.469039in}{5.220902in}}%
\pgfpathlineto{\pgfqpoint{5.469575in}{5.221916in}}%
\pgfpathlineto{\pgfqpoint{5.470648in}{5.228052in}}%
\pgfpathlineto{\pgfqpoint{5.471721in}{5.214494in}}%
\pgfpathlineto{\pgfqpoint{5.472257in}{5.228740in}}%
\pgfpathlineto{\pgfqpoint{5.472794in}{5.217745in}}%
\pgfpathlineto{\pgfqpoint{5.473330in}{5.222314in}}%
\pgfpathlineto{\pgfqpoint{5.473867in}{5.215127in}}%
\pgfpathlineto{\pgfqpoint{5.474403in}{5.223716in}}%
\pgfpathlineto{\pgfqpoint{5.475476in}{5.210199in}}%
\pgfpathlineto{\pgfqpoint{5.476012in}{5.222542in}}%
\pgfpathlineto{\pgfqpoint{5.476549in}{5.212784in}}%
\pgfpathlineto{\pgfqpoint{5.477085in}{5.218437in}}%
\pgfpathlineto{\pgfqpoint{5.477622in}{5.209303in}}%
\pgfpathlineto{\pgfqpoint{5.478158in}{5.219329in}}%
\pgfpathlineto{\pgfqpoint{5.478695in}{5.209257in}}%
\pgfpathlineto{\pgfqpoint{5.479231in}{5.206798in}}%
\pgfpathlineto{\pgfqpoint{5.480840in}{5.218211in}}%
\pgfpathlineto{\pgfqpoint{5.481377in}{5.206652in}}%
\pgfpathlineto{\pgfqpoint{5.481913in}{5.215173in}}%
\pgfpathlineto{\pgfqpoint{5.482986in}{5.202860in}}%
\pgfpathlineto{\pgfqpoint{5.483522in}{5.205657in}}%
\pgfpathlineto{\pgfqpoint{5.484059in}{5.206014in}}%
\pgfpathlineto{\pgfqpoint{5.484595in}{5.211999in}}%
\pgfpathlineto{\pgfqpoint{5.485132in}{5.202477in}}%
\pgfpathlineto{\pgfqpoint{5.485668in}{5.210569in}}%
\pgfpathlineto{\pgfqpoint{5.487278in}{5.199887in}}%
\pgfpathlineto{\pgfqpoint{5.488350in}{5.204351in}}%
\pgfpathlineto{\pgfqpoint{5.489423in}{5.200292in}}%
\pgfpathlineto{\pgfqpoint{5.489960in}{5.207598in}}%
\pgfpathlineto{\pgfqpoint{5.490496in}{5.206215in}}%
\pgfpathlineto{\pgfqpoint{5.491033in}{5.196847in}}%
\pgfpathlineto{\pgfqpoint{5.491569in}{5.202804in}}%
\pgfpathlineto{\pgfqpoint{5.493178in}{5.198831in}}%
\pgfpathlineto{\pgfqpoint{5.494251in}{5.205128in}}%
\pgfpathlineto{\pgfqpoint{5.494788in}{5.191930in}}%
\pgfpathlineto{\pgfqpoint{5.495324in}{5.199051in}}%
\pgfpathlineto{\pgfqpoint{5.495861in}{5.201690in}}%
\pgfpathlineto{\pgfqpoint{5.496933in}{5.191147in}}%
\pgfpathlineto{\pgfqpoint{5.497470in}{5.206250in}}%
\pgfpathlineto{\pgfqpoint{5.498006in}{5.198990in}}%
\pgfpathlineto{\pgfqpoint{5.498543in}{5.191222in}}%
\pgfpathlineto{\pgfqpoint{5.499079in}{5.200673in}}%
\pgfpathlineto{\pgfqpoint{5.499616in}{5.196983in}}%
\pgfpathlineto{\pgfqpoint{5.500152in}{5.187221in}}%
\pgfpathlineto{\pgfqpoint{5.500688in}{5.190479in}}%
\pgfpathlineto{\pgfqpoint{5.501225in}{5.202091in}}%
\pgfpathlineto{\pgfqpoint{5.501761in}{5.199809in}}%
\pgfpathlineto{\pgfqpoint{5.502834in}{5.186663in}}%
\pgfpathlineto{\pgfqpoint{5.503371in}{5.187085in}}%
\pgfpathlineto{\pgfqpoint{5.503907in}{5.182294in}}%
\pgfpathlineto{\pgfqpoint{5.504443in}{5.186663in}}%
\pgfpathlineto{\pgfqpoint{5.504980in}{5.194515in}}%
\pgfpathlineto{\pgfqpoint{5.506053in}{5.184337in}}%
\pgfpathlineto{\pgfqpoint{5.506589in}{5.187289in}}%
\pgfpathlineto{\pgfqpoint{5.508199in}{5.179615in}}%
\pgfpathlineto{\pgfqpoint{5.508735in}{5.179341in}}%
\pgfpathlineto{\pgfqpoint{5.509271in}{5.183573in}}%
\pgfpathlineto{\pgfqpoint{5.509808in}{5.182924in}}%
\pgfpathlineto{\pgfqpoint{5.511417in}{5.166984in}}%
\pgfpathlineto{\pgfqpoint{5.511954in}{5.170739in}}%
\pgfpathlineto{\pgfqpoint{5.512490in}{5.170694in}}%
\pgfpathlineto{\pgfqpoint{5.513026in}{5.172273in}}%
\pgfpathlineto{\pgfqpoint{5.513563in}{5.170440in}}%
\pgfpathlineto{\pgfqpoint{5.514636in}{5.159118in}}%
\pgfpathlineto{\pgfqpoint{5.515172in}{5.166898in}}%
\pgfpathlineto{\pgfqpoint{5.515709in}{5.171292in}}%
\pgfpathlineto{\pgfqpoint{5.516245in}{5.154254in}}%
\pgfpathlineto{\pgfqpoint{5.516782in}{5.155425in}}%
\pgfpathlineto{\pgfqpoint{5.518391in}{5.164001in}}%
\pgfpathlineto{\pgfqpoint{5.518927in}{5.164162in}}%
\pgfpathlineto{\pgfqpoint{5.520000in}{5.138096in}}%
\pgfpathlineto{\pgfqpoint{5.520537in}{5.140026in}}%
\pgfpathlineto{\pgfqpoint{5.521073in}{5.158408in}}%
\pgfpathlineto{\pgfqpoint{5.521609in}{5.156633in}}%
\pgfpathlineto{\pgfqpoint{5.522146in}{5.155812in}}%
\pgfpathlineto{\pgfqpoint{5.523219in}{5.137499in}}%
\pgfpathlineto{\pgfqpoint{5.523755in}{5.140631in}}%
\pgfpathlineto{\pgfqpoint{5.524292in}{5.138870in}}%
\pgfpathlineto{\pgfqpoint{5.525901in}{5.147525in}}%
\pgfpathlineto{\pgfqpoint{5.526437in}{5.147342in}}%
\pgfpathlineto{\pgfqpoint{5.528047in}{5.126228in}}%
\pgfpathlineto{\pgfqpoint{5.529656in}{5.147630in}}%
\pgfpathlineto{\pgfqpoint{5.530192in}{5.143040in}}%
\pgfpathlineto{\pgfqpoint{5.530729in}{5.116590in}}%
\pgfpathlineto{\pgfqpoint{5.531265in}{5.127959in}}%
\pgfpathlineto{\pgfqpoint{5.532875in}{5.140421in}}%
\pgfpathlineto{\pgfqpoint{5.533411in}{5.137413in}}%
\pgfpathlineto{\pgfqpoint{5.533947in}{5.117117in}}%
\pgfpathlineto{\pgfqpoint{5.534484in}{5.119457in}}%
\pgfpathlineto{\pgfqpoint{5.535020in}{5.137302in}}%
\pgfpathlineto{\pgfqpoint{5.535557in}{5.133887in}}%
\pgfpathlineto{\pgfqpoint{5.536093in}{5.111393in}}%
\pgfpathlineto{\pgfqpoint{5.537166in}{5.111428in}}%
\pgfpathlineto{\pgfqpoint{5.537703in}{5.112293in}}%
\pgfpathlineto{\pgfqpoint{5.538775in}{5.134197in}}%
\pgfpathlineto{\pgfqpoint{5.539312in}{5.126112in}}%
\pgfpathlineto{\pgfqpoint{5.539848in}{5.093846in}}%
\pgfpathlineto{\pgfqpoint{5.540385in}{5.108925in}}%
\pgfpathlineto{\pgfqpoint{5.540921in}{5.112581in}}%
\pgfpathlineto{\pgfqpoint{5.541458in}{5.109452in}}%
\pgfpathlineto{\pgfqpoint{5.541994in}{5.129336in}}%
\pgfpathlineto{\pgfqpoint{5.542530in}{5.113098in}}%
\pgfpathlineto{\pgfqpoint{5.543067in}{5.116555in}}%
\pgfpathlineto{\pgfqpoint{5.543603in}{5.090915in}}%
\pgfpathlineto{\pgfqpoint{5.544140in}{5.110406in}}%
\pgfpathlineto{\pgfqpoint{5.545213in}{5.086873in}}%
\pgfpathlineto{\pgfqpoint{5.545749in}{5.120118in}}%
\pgfpathlineto{\pgfqpoint{5.546286in}{5.099659in}}%
\pgfpathlineto{\pgfqpoint{5.546822in}{5.107068in}}%
\pgfpathlineto{\pgfqpoint{5.547358in}{5.082303in}}%
\pgfpathlineto{\pgfqpoint{5.548431in}{5.083968in}}%
\pgfpathlineto{\pgfqpoint{5.548968in}{5.088836in}}%
\pgfpathlineto{\pgfqpoint{5.549504in}{5.121571in}}%
\pgfpathlineto{\pgfqpoint{5.550041in}{5.089270in}}%
\pgfpathlineto{\pgfqpoint{5.551113in}{5.069920in}}%
\pgfpathlineto{\pgfqpoint{5.552186in}{5.092915in}}%
\pgfpathlineto{\pgfqpoint{5.553259in}{5.090532in}}%
\pgfpathlineto{\pgfqpoint{5.554332in}{5.043769in}}%
\pgfpathlineto{\pgfqpoint{5.555405in}{5.088268in}}%
\pgfpathlineto{\pgfqpoint{5.555941in}{5.079975in}}%
\pgfpathlineto{\pgfqpoint{5.557014in}{5.054046in}}%
\pgfpathlineto{\pgfqpoint{5.557551in}{5.054405in}}%
\pgfpathlineto{\pgfqpoint{5.558087in}{5.054406in}}%
\pgfpathlineto{\pgfqpoint{5.558624in}{5.081741in}}%
\pgfpathlineto{\pgfqpoint{5.559160in}{5.060207in}}%
\pgfpathlineto{\pgfqpoint{5.560769in}{5.056236in}}%
\pgfpathlineto{\pgfqpoint{5.561306in}{5.060529in}}%
\pgfpathlineto{\pgfqpoint{5.561842in}{5.030889in}}%
\pgfpathlineto{\pgfqpoint{5.562379in}{5.055981in}}%
\pgfpathlineto{\pgfqpoint{5.562915in}{5.053132in}}%
\pgfpathlineto{\pgfqpoint{5.563451in}{5.062212in}}%
\pgfpathlineto{\pgfqpoint{5.564524in}{5.031578in}}%
\pgfpathlineto{\pgfqpoint{5.565061in}{5.045988in}}%
\pgfpathlineto{\pgfqpoint{5.566134in}{5.062660in}}%
\pgfpathlineto{\pgfqpoint{5.567743in}{5.017799in}}%
\pgfpathlineto{\pgfqpoint{5.568279in}{5.019163in}}%
\pgfpathlineto{\pgfqpoint{5.568816in}{5.064911in}}%
\pgfpathlineto{\pgfqpoint{5.569352in}{5.061598in}}%
\pgfpathlineto{\pgfqpoint{5.570962in}{4.993907in}}%
\pgfpathlineto{\pgfqpoint{5.571498in}{5.023473in}}%
\pgfpathlineto{\pgfqpoint{5.572034in}{5.068332in}}%
\pgfpathlineto{\pgfqpoint{5.572571in}{5.060989in}}%
\pgfpathlineto{\pgfqpoint{5.573644in}{4.996020in}}%
\pgfpathlineto{\pgfqpoint{5.574180in}{5.011483in}}%
\pgfpathlineto{\pgfqpoint{5.575789in}{5.039852in}}%
\pgfpathlineto{\pgfqpoint{5.576326in}{5.000910in}}%
\pgfpathlineto{\pgfqpoint{5.576862in}{5.006550in}}%
\pgfpathlineto{\pgfqpoint{5.577399in}{5.027926in}}%
\pgfpathlineto{\pgfqpoint{5.577935in}{5.025903in}}%
\pgfpathlineto{\pgfqpoint{5.579008in}{4.984060in}}%
\pgfpathlineto{\pgfqpoint{5.579545in}{5.030424in}}%
\pgfpathlineto{\pgfqpoint{5.580081in}{5.004252in}}%
\pgfpathlineto{\pgfqpoint{5.580617in}{4.978680in}}%
\pgfpathlineto{\pgfqpoint{5.582227in}{5.014417in}}%
\pgfpathlineto{\pgfqpoint{5.582763in}{5.009914in}}%
\pgfpathlineto{\pgfqpoint{5.583300in}{5.014200in}}%
\pgfpathlineto{\pgfqpoint{5.584372in}{4.913454in}}%
\pgfpathlineto{\pgfqpoint{5.585982in}{5.021113in}}%
\pgfpathlineto{\pgfqpoint{5.587591in}{4.917472in}}%
\pgfpathlineto{\pgfqpoint{5.588128in}{4.932266in}}%
\pgfpathlineto{\pgfqpoint{5.588664in}{5.017804in}}%
\pgfpathlineto{\pgfqpoint{5.589200in}{4.984760in}}%
\pgfpathlineto{\pgfqpoint{5.590810in}{4.926761in}}%
\pgfpathlineto{\pgfqpoint{5.591346in}{4.989625in}}%
\pgfpathlineto{\pgfqpoint{5.591883in}{4.982936in}}%
\pgfpathlineto{\pgfqpoint{5.592419in}{4.965282in}}%
\pgfpathlineto{\pgfqpoint{5.593492in}{4.893808in}}%
\pgfpathlineto{\pgfqpoint{5.594028in}{4.996671in}}%
\pgfpathlineto{\pgfqpoint{5.594565in}{4.968109in}}%
\pgfpathlineto{\pgfqpoint{5.596711in}{4.909630in}}%
\pgfpathlineto{\pgfqpoint{5.597247in}{4.891147in}}%
\pgfpathlineto{\pgfqpoint{5.597783in}{4.978610in}}%
\pgfpathlineto{\pgfqpoint{5.598320in}{4.921174in}}%
\pgfpathlineto{\pgfqpoint{5.599393in}{4.993316in}}%
\pgfpathlineto{\pgfqpoint{5.601002in}{4.818404in}}%
\pgfpathlineto{\pgfqpoint{5.602075in}{4.990822in}}%
\pgfpathlineto{\pgfqpoint{5.602611in}{4.987869in}}%
\pgfpathlineto{\pgfqpoint{5.603684in}{4.829101in}}%
\pgfpathlineto{\pgfqpoint{5.604221in}{4.872777in}}%
\pgfpathlineto{\pgfqpoint{5.605293in}{4.998487in}}%
\pgfpathlineto{\pgfqpoint{5.605830in}{4.944672in}}%
\pgfpathlineto{\pgfqpoint{5.606903in}{4.780740in}}%
\pgfpathlineto{\pgfqpoint{5.607976in}{4.995171in}}%
\pgfpathlineto{\pgfqpoint{5.608512in}{4.971825in}}%
\pgfpathlineto{\pgfqpoint{5.610121in}{4.840254in}}%
\pgfpathlineto{\pgfqpoint{5.610658in}{4.804532in}}%
\pgfpathlineto{\pgfqpoint{5.611194in}{4.964079in}}%
\pgfpathlineto{\pgfqpoint{5.611731in}{4.929245in}}%
\pgfpathlineto{\pgfqpoint{5.612267in}{4.862942in}}%
\pgfpathlineto{\pgfqpoint{5.612804in}{4.870784in}}%
\pgfpathlineto{\pgfqpoint{5.613340in}{4.876304in}}%
\pgfpathlineto{\pgfqpoint{5.614413in}{4.754467in}}%
\pgfpathlineto{\pgfqpoint{5.615486in}{4.916739in}}%
\pgfpathlineto{\pgfqpoint{5.616022in}{4.916655in}}%
\pgfpathlineto{\pgfqpoint{5.617095in}{4.740228in}}%
\pgfpathlineto{\pgfqpoint{5.617632in}{4.760009in}}%
\pgfpathlineto{\pgfqpoint{5.618704in}{5.020748in}}%
\pgfpathlineto{\pgfqpoint{5.619777in}{4.728575in}}%
\pgfpathlineto{\pgfqpoint{5.620314in}{4.741438in}}%
\pgfpathlineto{\pgfqpoint{5.621923in}{4.979576in}}%
\pgfpathlineto{\pgfqpoint{5.622459in}{4.856099in}}%
\pgfpathlineto{\pgfqpoint{5.622996in}{4.437595in}}%
\pgfpathlineto{\pgfqpoint{5.623532in}{4.663839in}}%
\pgfpathlineto{\pgfqpoint{5.624605in}{4.928136in}}%
\pgfpathlineto{\pgfqpoint{5.625142in}{4.905891in}}%
\pgfpathlineto{\pgfqpoint{5.626751in}{4.667106in}}%
\pgfpathlineto{\pgfqpoint{5.627287in}{4.903555in}}%
\pgfpathlineto{\pgfqpoint{5.627824in}{4.821006in}}%
\pgfpathlineto{\pgfqpoint{5.628360in}{4.766283in}}%
\pgfpathlineto{\pgfqpoint{5.628897in}{4.769578in}}%
\pgfpathlineto{\pgfqpoint{5.629433in}{4.894769in}}%
\pgfpathlineto{\pgfqpoint{5.629970in}{4.827365in}}%
\pgfpathlineto{\pgfqpoint{5.631042in}{4.845615in}}%
\pgfpathlineto{\pgfqpoint{5.631579in}{4.713719in}}%
\pgfpathlineto{\pgfqpoint{5.632115in}{4.913476in}}%
\pgfpathlineto{\pgfqpoint{5.632652in}{4.793331in}}%
\pgfpathlineto{\pgfqpoint{5.633188in}{4.605005in}}%
\pgfpathlineto{\pgfqpoint{5.633725in}{4.700759in}}%
\pgfpathlineto{\pgfqpoint{5.634797in}{5.005437in}}%
\pgfpathlineto{\pgfqpoint{5.636407in}{4.348079in}}%
\pgfpathlineto{\pgfqpoint{5.638016in}{4.998420in}}%
\pgfpathlineto{\pgfqpoint{5.638553in}{4.849042in}}%
\pgfpathlineto{\pgfqpoint{5.639089in}{4.253328in}}%
\pgfpathlineto{\pgfqpoint{5.639625in}{4.536207in}}%
\pgfpathlineto{\pgfqpoint{5.640162in}{4.582499in}}%
\pgfpathlineto{\pgfqpoint{5.640698in}{4.938939in}}%
\pgfpathlineto{\pgfqpoint{5.641235in}{4.906565in}}%
\pgfpathlineto{\pgfqpoint{5.642308in}{4.552797in}}%
\pgfpathlineto{\pgfqpoint{5.643380in}{4.908802in}}%
\pgfpathlineto{\pgfqpoint{5.644453in}{4.883962in}}%
\pgfpathlineto{\pgfqpoint{5.645526in}{4.607093in}}%
\pgfpathlineto{\pgfqpoint{5.646599in}{4.845750in}}%
\pgfpathlineto{\pgfqpoint{5.647136in}{4.748031in}}%
\pgfpathlineto{\pgfqpoint{5.647672in}{4.646735in}}%
\pgfpathlineto{\pgfqpoint{5.648208in}{5.060360in}}%
\pgfpathlineto{\pgfqpoint{5.648745in}{4.682794in}}%
\pgfpathlineto{\pgfqpoint{5.649281in}{4.646989in}}%
\pgfpathlineto{\pgfqpoint{5.651427in}{4.903102in}}%
\pgfpathlineto{\pgfqpoint{5.651963in}{4.868673in}}%
\pgfpathlineto{\pgfqpoint{5.652500in}{4.266690in}}%
\pgfpathlineto{\pgfqpoint{5.653036in}{4.643946in}}%
\pgfpathlineto{\pgfqpoint{5.654109in}{5.005936in}}%
\pgfpathlineto{\pgfqpoint{5.655718in}{4.342539in}}%
\pgfpathlineto{\pgfqpoint{5.656255in}{4.501560in}}%
\pgfpathlineto{\pgfqpoint{5.656791in}{5.061782in}}%
\pgfpathlineto{\pgfqpoint{5.657328in}{4.789799in}}%
\pgfpathlineto{\pgfqpoint{5.658401in}{4.014875in}}%
\pgfpathlineto{\pgfqpoint{5.660010in}{4.958599in}}%
\pgfpathlineto{\pgfqpoint{5.662156in}{4.566158in}}%
\pgfpathlineto{\pgfqpoint{5.662692in}{4.808577in}}%
\pgfpathlineto{\pgfqpoint{5.663229in}{4.742693in}}%
\pgfpathlineto{\pgfqpoint{5.663765in}{4.755489in}}%
\pgfpathlineto{\pgfqpoint{5.664301in}{4.808245in}}%
\pgfpathlineto{\pgfqpoint{5.664838in}{4.723007in}}%
\pgfpathlineto{\pgfqpoint{5.665374in}{4.865284in}}%
\pgfpathlineto{\pgfqpoint{5.665911in}{4.460281in}}%
\pgfpathlineto{\pgfqpoint{5.666447in}{4.814471in}}%
\pgfpathlineto{\pgfqpoint{5.667520in}{4.998114in}}%
\pgfpathlineto{\pgfqpoint{5.668593in}{4.346896in}}%
\pgfpathlineto{\pgfqpoint{5.670202in}{5.070955in}}%
\pgfpathlineto{\pgfqpoint{5.670739in}{4.835787in}}%
\pgfpathlineto{\pgfqpoint{5.671812in}{3.907404in}}%
\pgfpathlineto{\pgfqpoint{5.672884in}{4.908184in}}%
\pgfpathlineto{\pgfqpoint{5.673421in}{4.877445in}}%
\pgfpathlineto{\pgfqpoint{5.674494in}{4.557194in}}%
\pgfpathlineto{\pgfqpoint{5.676103in}{4.983261in}}%
\pgfpathlineto{\pgfqpoint{5.676639in}{4.530169in}}%
\pgfpathlineto{\pgfqpoint{5.677176in}{4.793872in}}%
\pgfpathlineto{\pgfqpoint{5.677712in}{4.773892in}}%
\pgfpathlineto{\pgfqpoint{5.678785in}{4.909578in}}%
\pgfpathlineto{\pgfqpoint{5.680395in}{4.670763in}}%
\pgfpathlineto{\pgfqpoint{5.680931in}{4.897198in}}%
\pgfpathlineto{\pgfqpoint{5.680931in}{4.897198in}}%
\pgfpathlineto{\pgfqpoint{5.680931in}{4.897198in}}%
\pgfpathlineto{\pgfqpoint{5.681467in}{4.660331in}}%
\pgfpathlineto{\pgfqpoint{5.682004in}{4.776741in}}%
\pgfpathlineto{\pgfqpoint{5.682540in}{4.722978in}}%
\pgfpathlineto{\pgfqpoint{5.683077in}{4.984447in}}%
\pgfpathlineto{\pgfqpoint{5.683613in}{4.924226in}}%
\pgfpathlineto{\pgfqpoint{5.685222in}{4.629286in}}%
\pgfpathlineto{\pgfqpoint{5.685759in}{5.009236in}}%
\pgfpathlineto{\pgfqpoint{5.686295in}{4.834820in}}%
\pgfpathlineto{\pgfqpoint{5.686832in}{4.792695in}}%
\pgfpathlineto{\pgfqpoint{5.687368in}{4.444463in}}%
\pgfpathlineto{\pgfqpoint{5.687905in}{4.748999in}}%
\pgfpathlineto{\pgfqpoint{5.688978in}{5.025797in}}%
\pgfpathlineto{\pgfqpoint{5.689514in}{4.986012in}}%
\pgfpathlineto{\pgfqpoint{5.690050in}{4.491050in}}%
\pgfpathlineto{\pgfqpoint{5.690587in}{4.742114in}}%
\pgfpathlineto{\pgfqpoint{5.691123in}{4.780143in}}%
\pgfpathlineto{\pgfqpoint{5.691660in}{4.979362in}}%
\pgfpathlineto{\pgfqpoint{5.692196in}{4.862608in}}%
\pgfpathlineto{\pgfqpoint{5.692733in}{4.830509in}}%
\pgfpathlineto{\pgfqpoint{5.693269in}{4.703367in}}%
\pgfpathlineto{\pgfqpoint{5.693805in}{4.773390in}}%
\pgfpathlineto{\pgfqpoint{5.694342in}{4.928418in}}%
\pgfpathlineto{\pgfqpoint{5.694878in}{4.856242in}}%
\pgfpathlineto{\pgfqpoint{5.695415in}{4.872382in}}%
\pgfpathlineto{\pgfqpoint{5.695951in}{4.843382in}}%
\pgfpathlineto{\pgfqpoint{5.696488in}{5.000258in}}%
\pgfpathlineto{\pgfqpoint{5.697024in}{4.749420in}}%
\pgfpathlineto{\pgfqpoint{5.698097in}{4.754443in}}%
\pgfpathlineto{\pgfqpoint{5.699170in}{4.948613in}}%
\pgfpathlineto{\pgfqpoint{5.699706in}{4.885589in}}%
\pgfpathlineto{\pgfqpoint{5.700243in}{4.846811in}}%
\pgfpathlineto{\pgfqpoint{5.700779in}{4.699862in}}%
\pgfpathlineto{\pgfqpoint{5.702388in}{5.017561in}}%
\pgfpathlineto{\pgfqpoint{5.703461in}{4.769313in}}%
\pgfpathlineto{\pgfqpoint{5.703998in}{4.811286in}}%
\pgfpathlineto{\pgfqpoint{5.705071in}{4.959848in}}%
\pgfpathlineto{\pgfqpoint{5.705607in}{4.800799in}}%
\pgfpathlineto{\pgfqpoint{5.706143in}{4.887774in}}%
\pgfpathlineto{\pgfqpoint{5.706680in}{4.819958in}}%
\pgfpathlineto{\pgfqpoint{5.707216in}{5.012910in}}%
\pgfpathlineto{\pgfqpoint{5.707753in}{4.972771in}}%
\pgfpathlineto{\pgfqpoint{5.708826in}{4.820612in}}%
\pgfpathlineto{\pgfqpoint{5.709899in}{4.941877in}}%
\pgfpathlineto{\pgfqpoint{5.711508in}{4.866201in}}%
\pgfpathlineto{\pgfqpoint{5.713117in}{4.984508in}}%
\pgfpathlineto{\pgfqpoint{5.713654in}{4.896797in}}%
\pgfpathlineto{\pgfqpoint{5.714190in}{4.912450in}}%
\pgfpathlineto{\pgfqpoint{5.714726in}{4.989794in}}%
\pgfpathlineto{\pgfqpoint{5.715263in}{4.972255in}}%
\pgfpathlineto{\pgfqpoint{5.716336in}{4.853843in}}%
\pgfpathlineto{\pgfqpoint{5.717945in}{4.988252in}}%
\pgfpathlineto{\pgfqpoint{5.718482in}{4.890249in}}%
\pgfpathlineto{\pgfqpoint{5.719018in}{4.937330in}}%
\pgfpathlineto{\pgfqpoint{5.719554in}{4.909477in}}%
\pgfpathlineto{\pgfqpoint{5.720091in}{5.041892in}}%
\pgfpathlineto{\pgfqpoint{5.720627in}{5.033867in}}%
\pgfpathlineto{\pgfqpoint{5.722237in}{4.873594in}}%
\pgfpathlineto{\pgfqpoint{5.723846in}{4.998244in}}%
\pgfpathlineto{\pgfqpoint{5.724382in}{4.924244in}}%
\pgfpathlineto{\pgfqpoint{5.724919in}{4.942458in}}%
\pgfpathlineto{\pgfqpoint{5.725455in}{4.960376in}}%
\pgfpathlineto{\pgfqpoint{5.725992in}{5.036303in}}%
\pgfpathlineto{\pgfqpoint{5.726528in}{4.985517in}}%
\pgfpathlineto{\pgfqpoint{5.727064in}{4.968488in}}%
\pgfpathlineto{\pgfqpoint{5.727601in}{4.998955in}}%
\pgfpathlineto{\pgfqpoint{5.729210in}{4.924604in}}%
\pgfpathlineto{\pgfqpoint{5.730283in}{5.034590in}}%
\pgfpathlineto{\pgfqpoint{5.730820in}{5.018108in}}%
\pgfpathlineto{\pgfqpoint{5.731892in}{4.959626in}}%
\pgfpathlineto{\pgfqpoint{5.732429in}{4.968918in}}%
\pgfpathlineto{\pgfqpoint{5.733502in}{5.053057in}}%
\pgfpathlineto{\pgfqpoint{5.735111in}{4.928922in}}%
\pgfpathlineto{\pgfqpoint{5.736720in}{5.029652in}}%
\pgfpathlineto{\pgfqpoint{5.737793in}{5.003551in}}%
\pgfpathlineto{\pgfqpoint{5.738866in}{5.060222in}}%
\pgfpathlineto{\pgfqpoint{5.739403in}{5.023437in}}%
\pgfpathlineto{\pgfqpoint{5.739939in}{4.983308in}}%
\pgfpathlineto{\pgfqpoint{5.740475in}{5.008796in}}%
\pgfpathlineto{\pgfqpoint{5.741012in}{5.009630in}}%
\pgfpathlineto{\pgfqpoint{5.741548in}{5.044017in}}%
\pgfpathlineto{\pgfqpoint{5.741548in}{5.044017in}}%
\pgfpathlineto{\pgfqpoint{5.741548in}{5.044017in}}%
\pgfpathlineto{\pgfqpoint{5.742085in}{5.009471in}}%
\pgfpathlineto{\pgfqpoint{5.742621in}{5.022333in}}%
\pgfpathlineto{\pgfqpoint{5.743158in}{5.044616in}}%
\pgfpathlineto{\pgfqpoint{5.743694in}{5.015879in}}%
\pgfpathlineto{\pgfqpoint{5.743694in}{5.015879in}}%
\pgfpathlineto{\pgfqpoint{5.743694in}{5.015879in}}%
\pgfpathlineto{\pgfqpoint{5.744230in}{5.046628in}}%
\pgfpathlineto{\pgfqpoint{5.744230in}{5.046628in}}%
\pgfpathlineto{\pgfqpoint{5.744230in}{5.046628in}}%
\pgfpathlineto{\pgfqpoint{5.744767in}{5.014170in}}%
\pgfpathlineto{\pgfqpoint{5.745303in}{5.036016in}}%
\pgfpathlineto{\pgfqpoint{5.745840in}{5.068121in}}%
\pgfpathlineto{\pgfqpoint{5.746376in}{5.048633in}}%
\pgfpathlineto{\pgfqpoint{5.746913in}{5.055950in}}%
\pgfpathlineto{\pgfqpoint{5.747449in}{5.005951in}}%
\pgfpathlineto{\pgfqpoint{5.747985in}{5.015540in}}%
\pgfpathlineto{\pgfqpoint{5.748522in}{5.072963in}}%
\pgfpathlineto{\pgfqpoint{5.749058in}{5.066850in}}%
\pgfpathlineto{\pgfqpoint{5.750131in}{5.044827in}}%
\pgfpathlineto{\pgfqpoint{5.751741in}{5.084585in}}%
\pgfpathlineto{\pgfqpoint{5.752813in}{5.038830in}}%
\pgfpathlineto{\pgfqpoint{5.754423in}{5.083085in}}%
\pgfpathlineto{\pgfqpoint{5.756032in}{5.043505in}}%
\pgfpathlineto{\pgfqpoint{5.757105in}{5.104373in}}%
\pgfpathlineto{\pgfqpoint{5.758178in}{5.090623in}}%
\pgfpathlineto{\pgfqpoint{5.759251in}{5.057045in}}%
\pgfpathlineto{\pgfqpoint{5.759787in}{5.073668in}}%
\pgfpathlineto{\pgfqpoint{5.760324in}{5.055841in}}%
\pgfpathlineto{\pgfqpoint{5.761933in}{5.110014in}}%
\pgfpathlineto{\pgfqpoint{5.762469in}{5.080995in}}%
\pgfpathlineto{\pgfqpoint{5.763006in}{5.092251in}}%
\pgfpathlineto{\pgfqpoint{5.763542in}{5.085248in}}%
\pgfpathlineto{\pgfqpoint{5.764615in}{5.105159in}}%
\pgfpathlineto{\pgfqpoint{5.765688in}{5.076529in}}%
\pgfpathlineto{\pgfqpoint{5.766224in}{5.081289in}}%
\pgfpathlineto{\pgfqpoint{5.766761in}{5.121783in}}%
\pgfpathlineto{\pgfqpoint{5.767297in}{5.108250in}}%
\pgfpathlineto{\pgfqpoint{5.767834in}{5.076372in}}%
\pgfpathlineto{\pgfqpoint{5.768370in}{5.086880in}}%
\pgfpathlineto{\pgfqpoint{5.769979in}{5.132354in}}%
\pgfpathlineto{\pgfqpoint{5.770516in}{5.090428in}}%
\pgfpathlineto{\pgfqpoint{5.771052in}{5.106999in}}%
\pgfpathlineto{\pgfqpoint{5.771589in}{5.097251in}}%
\pgfpathlineto{\pgfqpoint{5.772125in}{5.097785in}}%
\pgfpathlineto{\pgfqpoint{5.774271in}{5.124415in}}%
\pgfpathlineto{\pgfqpoint{5.774807in}{5.123582in}}%
\pgfpathlineto{\pgfqpoint{5.775344in}{5.112622in}}%
\pgfpathlineto{\pgfqpoint{5.775880in}{5.128510in}}%
\pgfpathlineto{\pgfqpoint{5.776953in}{5.127509in}}%
\pgfpathlineto{\pgfqpoint{5.777489in}{5.115274in}}%
\pgfpathlineto{\pgfqpoint{5.778026in}{5.135064in}}%
\pgfpathlineto{\pgfqpoint{5.778562in}{5.123415in}}%
\pgfpathlineto{\pgfqpoint{5.779635in}{5.131006in}}%
\pgfpathlineto{\pgfqpoint{5.780708in}{5.110169in}}%
\pgfpathlineto{\pgfqpoint{5.782854in}{5.156300in}}%
\pgfpathlineto{\pgfqpoint{5.783390in}{5.113954in}}%
\pgfpathlineto{\pgfqpoint{5.783927in}{5.126565in}}%
\pgfpathlineto{\pgfqpoint{5.784463in}{5.124376in}}%
\pgfpathlineto{\pgfqpoint{5.786072in}{5.137908in}}%
\pgfpathlineto{\pgfqpoint{5.786609in}{5.127561in}}%
\pgfpathlineto{\pgfqpoint{5.788218in}{5.149915in}}%
\pgfpathlineto{\pgfqpoint{5.788755in}{5.137342in}}%
\pgfpathlineto{\pgfqpoint{5.788755in}{5.137342in}}%
\pgfpathlineto{\pgfqpoint{5.788755in}{5.137342in}}%
\pgfpathlineto{\pgfqpoint{5.790900in}{5.165422in}}%
\pgfpathlineto{\pgfqpoint{5.791437in}{5.154141in}}%
\pgfpathlineto{\pgfqpoint{5.791973in}{5.157146in}}%
\pgfpathlineto{\pgfqpoint{5.793046in}{5.157899in}}%
\pgfpathlineto{\pgfqpoint{5.793583in}{5.125929in}}%
\pgfpathlineto{\pgfqpoint{5.795192in}{5.176415in}}%
\pgfpathlineto{\pgfqpoint{5.796801in}{5.146905in}}%
\pgfpathlineto{\pgfqpoint{5.797874in}{5.164045in}}%
\pgfpathlineto{\pgfqpoint{5.798410in}{5.159823in}}%
\pgfpathlineto{\pgfqpoint{5.798947in}{5.161410in}}%
\pgfpathlineto{\pgfqpoint{5.799483in}{5.157404in}}%
\pgfpathlineto{\pgfqpoint{5.800020in}{5.168312in}}%
\pgfpathlineto{\pgfqpoint{5.800556in}{5.164434in}}%
\pgfpathlineto{\pgfqpoint{5.801093in}{5.160708in}}%
\pgfpathlineto{\pgfqpoint{5.801629in}{5.163198in}}%
\pgfpathlineto{\pgfqpoint{5.802166in}{5.183506in}}%
\pgfpathlineto{\pgfqpoint{5.802702in}{5.165759in}}%
\pgfpathlineto{\pgfqpoint{5.804311in}{5.181373in}}%
\pgfpathlineto{\pgfqpoint{5.804848in}{5.165079in}}%
\pgfpathlineto{\pgfqpoint{5.805384in}{5.185718in}}%
\pgfpathlineto{\pgfqpoint{5.805921in}{5.182611in}}%
\pgfpathlineto{\pgfqpoint{5.806457in}{5.163567in}}%
\pgfpathlineto{\pgfqpoint{5.806993in}{5.181986in}}%
\pgfpathlineto{\pgfqpoint{5.807530in}{5.183058in}}%
\pgfpathlineto{\pgfqpoint{5.808066in}{5.191048in}}%
\pgfpathlineto{\pgfqpoint{5.809139in}{5.173826in}}%
\pgfpathlineto{\pgfqpoint{5.809676in}{5.182321in}}%
\pgfpathlineto{\pgfqpoint{5.810212in}{5.180106in}}%
\pgfpathlineto{\pgfqpoint{5.810749in}{5.169340in}}%
\pgfpathlineto{\pgfqpoint{5.811285in}{5.174384in}}%
\pgfpathlineto{\pgfqpoint{5.811821in}{5.185860in}}%
\pgfpathlineto{\pgfqpoint{5.812358in}{5.185351in}}%
\pgfpathlineto{\pgfqpoint{5.813967in}{5.171281in}}%
\pgfpathlineto{\pgfqpoint{5.815576in}{5.205820in}}%
\pgfpathlineto{\pgfqpoint{5.817186in}{5.189967in}}%
\pgfpathlineto{\pgfqpoint{5.817722in}{5.190553in}}%
\pgfpathlineto{\pgfqpoint{5.818259in}{5.199333in}}%
\pgfpathlineto{\pgfqpoint{5.819332in}{5.198995in}}%
\pgfpathlineto{\pgfqpoint{5.819868in}{5.201052in}}%
\pgfpathlineto{\pgfqpoint{5.822014in}{5.194242in}}%
\pgfpathlineto{\pgfqpoint{5.822550in}{5.209150in}}%
\pgfpathlineto{\pgfqpoint{5.823087in}{5.199513in}}%
\pgfpathlineto{\pgfqpoint{5.824159in}{5.190612in}}%
\pgfpathlineto{\pgfqpoint{5.824696in}{5.207590in}}%
\pgfpathlineto{\pgfqpoint{5.825232in}{5.200562in}}%
\pgfpathlineto{\pgfqpoint{5.826305in}{5.199365in}}%
\pgfpathlineto{\pgfqpoint{5.826842in}{5.191494in}}%
\pgfpathlineto{\pgfqpoint{5.828451in}{5.213974in}}%
\pgfpathlineto{\pgfqpoint{5.830060in}{5.201901in}}%
\pgfpathlineto{\pgfqpoint{5.830597in}{5.207008in}}%
\pgfpathlineto{\pgfqpoint{5.831133in}{5.220078in}}%
\pgfpathlineto{\pgfqpoint{5.831670in}{5.219165in}}%
\pgfpathlineto{\pgfqpoint{5.832206in}{5.213181in}}%
\pgfpathlineto{\pgfqpoint{5.832742in}{5.220811in}}%
\pgfpathlineto{\pgfqpoint{5.834352in}{5.208242in}}%
\pgfpathlineto{\pgfqpoint{5.835425in}{5.213250in}}%
\pgfpathlineto{\pgfqpoint{5.835961in}{5.221844in}}%
\pgfpathlineto{\pgfqpoint{5.836497in}{5.207762in}}%
\pgfpathlineto{\pgfqpoint{5.837034in}{5.208784in}}%
\pgfpathlineto{\pgfqpoint{5.837570in}{5.216326in}}%
\pgfpathlineto{\pgfqpoint{5.837570in}{5.216326in}}%
\pgfpathlineto{\pgfqpoint{5.837570in}{5.216326in}}%
\pgfpathlineto{\pgfqpoint{5.838643in}{5.207207in}}%
\pgfpathlineto{\pgfqpoint{5.840253in}{5.219013in}}%
\pgfpathlineto{\pgfqpoint{5.840789in}{5.216289in}}%
\pgfpathlineto{\pgfqpoint{5.841325in}{5.209628in}}%
\pgfpathlineto{\pgfqpoint{5.842398in}{5.221386in}}%
\pgfpathlineto{\pgfqpoint{5.842935in}{5.219400in}}%
\pgfpathlineto{\pgfqpoint{5.843471in}{5.210980in}}%
\pgfpathlineto{\pgfqpoint{5.844008in}{5.224228in}}%
\pgfpathlineto{\pgfqpoint{5.844544in}{5.222675in}}%
\pgfpathlineto{\pgfqpoint{5.845080in}{5.218507in}}%
\pgfpathlineto{\pgfqpoint{5.846153in}{5.230296in}}%
\pgfpathlineto{\pgfqpoint{5.847763in}{5.211801in}}%
\pgfpathlineto{\pgfqpoint{5.848835in}{5.230601in}}%
\pgfpathlineto{\pgfqpoint{5.849908in}{5.215274in}}%
\pgfpathlineto{\pgfqpoint{5.850445in}{5.225400in}}%
\pgfpathlineto{\pgfqpoint{5.850981in}{5.214471in}}%
\pgfpathlineto{\pgfqpoint{5.851518in}{5.220218in}}%
\pgfpathlineto{\pgfqpoint{5.852054in}{5.216591in}}%
\pgfpathlineto{\pgfqpoint{5.852591in}{5.228918in}}%
\pgfpathlineto{\pgfqpoint{5.853127in}{5.221479in}}%
\pgfpathlineto{\pgfqpoint{5.853663in}{5.219403in}}%
\pgfpathlineto{\pgfqpoint{5.855273in}{5.232859in}}%
\pgfpathlineto{\pgfqpoint{5.855809in}{5.219957in}}%
\pgfpathlineto{\pgfqpoint{5.856346in}{5.224656in}}%
\pgfpathlineto{\pgfqpoint{5.857418in}{5.236217in}}%
\pgfpathlineto{\pgfqpoint{5.857955in}{5.220138in}}%
\pgfpathlineto{\pgfqpoint{5.858491in}{5.232375in}}%
\pgfpathlineto{\pgfqpoint{5.860101in}{5.224866in}}%
\pgfpathlineto{\pgfqpoint{5.860637in}{5.225074in}}%
\pgfpathlineto{\pgfqpoint{5.861174in}{5.236132in}}%
\pgfpathlineto{\pgfqpoint{5.861710in}{5.225997in}}%
\pgfpathlineto{\pgfqpoint{5.862783in}{5.226705in}}%
\pgfpathlineto{\pgfqpoint{5.863319in}{5.235791in}}%
\pgfpathlineto{\pgfqpoint{5.863856in}{5.220758in}}%
\pgfpathlineto{\pgfqpoint{5.864392in}{5.229109in}}%
\pgfpathlineto{\pgfqpoint{5.864929in}{5.225236in}}%
\pgfpathlineto{\pgfqpoint{5.865465in}{5.238030in}}%
\pgfpathlineto{\pgfqpoint{5.866001in}{5.230982in}}%
\pgfpathlineto{\pgfqpoint{5.866538in}{5.226814in}}%
\pgfpathlineto{\pgfqpoint{5.867074in}{5.237583in}}%
\pgfpathlineto{\pgfqpoint{5.867611in}{5.223158in}}%
\pgfpathlineto{\pgfqpoint{5.867611in}{5.223158in}}%
\pgfpathlineto{\pgfqpoint{5.867611in}{5.223158in}}%
\pgfpathlineto{\pgfqpoint{5.868147in}{5.238546in}}%
\pgfpathlineto{\pgfqpoint{5.868684in}{5.221594in}}%
\pgfpathlineto{\pgfqpoint{5.869220in}{5.236050in}}%
\pgfpathlineto{\pgfqpoint{5.869757in}{5.237974in}}%
\pgfpathlineto{\pgfqpoint{5.870293in}{5.232738in}}%
\pgfpathlineto{\pgfqpoint{5.871366in}{5.233150in}}%
\pgfpathlineto{\pgfqpoint{5.871902in}{5.236655in}}%
\pgfpathlineto{\pgfqpoint{5.872439in}{5.227092in}}%
\pgfpathlineto{\pgfqpoint{5.872975in}{5.235941in}}%
\pgfpathlineto{\pgfqpoint{5.874584in}{5.224248in}}%
\pgfpathlineto{\pgfqpoint{5.876194in}{5.237320in}}%
\pgfpathlineto{\pgfqpoint{5.877803in}{5.229635in}}%
\pgfpathlineto{\pgfqpoint{5.878339in}{5.230559in}}%
\pgfpathlineto{\pgfqpoint{5.878876in}{5.235152in}}%
\pgfpathlineto{\pgfqpoint{5.879412in}{5.226622in}}%
\pgfpathlineto{\pgfqpoint{5.879949in}{5.239083in}}%
\pgfpathlineto{\pgfqpoint{5.880485in}{5.223725in}}%
\pgfpathlineto{\pgfqpoint{5.880485in}{5.223725in}}%
\pgfpathlineto{\pgfqpoint{5.880485in}{5.223725in}}%
\pgfpathlineto{\pgfqpoint{5.881022in}{5.239496in}}%
\pgfpathlineto{\pgfqpoint{5.881558in}{5.229242in}}%
\pgfpathlineto{\pgfqpoint{5.882095in}{5.237007in}}%
\pgfpathlineto{\pgfqpoint{5.882631in}{5.236842in}}%
\pgfpathlineto{\pgfqpoint{5.883167in}{5.234656in}}%
\pgfpathlineto{\pgfqpoint{5.883704in}{5.243877in}}%
\pgfpathlineto{\pgfqpoint{5.884240in}{5.223669in}}%
\pgfpathlineto{\pgfqpoint{5.884777in}{5.231841in}}%
\pgfpathlineto{\pgfqpoint{5.885850in}{5.235113in}}%
\pgfpathlineto{\pgfqpoint{5.886386in}{5.226578in}}%
\pgfpathlineto{\pgfqpoint{5.886922in}{5.236946in}}%
\pgfpathlineto{\pgfqpoint{5.887459in}{5.234227in}}%
\pgfpathlineto{\pgfqpoint{5.887995in}{5.231782in}}%
\pgfpathlineto{\pgfqpoint{5.888532in}{5.222226in}}%
\pgfpathlineto{\pgfqpoint{5.889068in}{5.236227in}}%
\pgfpathlineto{\pgfqpoint{5.889605in}{5.235289in}}%
\pgfpathlineto{\pgfqpoint{5.891214in}{5.223716in}}%
\pgfpathlineto{\pgfqpoint{5.891750in}{5.236117in}}%
\pgfpathlineto{\pgfqpoint{5.892287in}{5.229205in}}%
\pgfpathlineto{\pgfqpoint{5.892823in}{5.234326in}}%
\pgfpathlineto{\pgfqpoint{5.893360in}{5.228646in}}%
\pgfpathlineto{\pgfqpoint{5.893896in}{5.234885in}}%
\pgfpathlineto{\pgfqpoint{5.894433in}{5.226133in}}%
\pgfpathlineto{\pgfqpoint{5.894969in}{5.226767in}}%
\pgfpathlineto{\pgfqpoint{5.896042in}{5.237474in}}%
\pgfpathlineto{\pgfqpoint{5.896578in}{5.233928in}}%
\pgfpathlineto{\pgfqpoint{5.897115in}{5.216948in}}%
\pgfpathlineto{\pgfqpoint{5.897651in}{5.228613in}}%
\pgfpathlineto{\pgfqpoint{5.898724in}{5.227157in}}%
\pgfpathlineto{\pgfqpoint{5.899260in}{5.221222in}}%
\pgfpathlineto{\pgfqpoint{5.900333in}{5.235780in}}%
\pgfpathlineto{\pgfqpoint{5.901406in}{5.221757in}}%
\pgfpathlineto{\pgfqpoint{5.901943in}{5.232578in}}%
\pgfpathlineto{\pgfqpoint{5.902479in}{5.231473in}}%
\pgfpathlineto{\pgfqpoint{5.903552in}{5.224351in}}%
\pgfpathlineto{\pgfqpoint{5.904088in}{5.229664in}}%
\pgfpathlineto{\pgfqpoint{5.904625in}{5.226373in}}%
\pgfpathlineto{\pgfqpoint{5.905161in}{5.224572in}}%
\pgfpathlineto{\pgfqpoint{5.905698in}{5.236403in}}%
\pgfpathlineto{\pgfqpoint{5.906234in}{5.229561in}}%
\pgfpathlineto{\pgfqpoint{5.906771in}{5.231558in}}%
\pgfpathlineto{\pgfqpoint{5.907843in}{5.217422in}}%
\pgfpathlineto{\pgfqpoint{5.908916in}{5.229338in}}%
\pgfpathlineto{\pgfqpoint{5.909453in}{5.222870in}}%
\pgfpathlineto{\pgfqpoint{5.909989in}{5.210199in}}%
\pgfpathlineto{\pgfqpoint{5.910526in}{5.222585in}}%
\pgfpathlineto{\pgfqpoint{5.911062in}{5.221542in}}%
\pgfpathlineto{\pgfqpoint{5.911599in}{5.210265in}}%
\pgfpathlineto{\pgfqpoint{5.912135in}{5.220062in}}%
\pgfpathlineto{\pgfqpoint{5.912671in}{5.228264in}}%
\pgfpathlineto{\pgfqpoint{5.913744in}{5.210491in}}%
\pgfpathlineto{\pgfqpoint{5.914281in}{5.216813in}}%
\pgfpathlineto{\pgfqpoint{5.914817in}{5.224622in}}%
\pgfpathlineto{\pgfqpoint{5.915354in}{5.210875in}}%
\pgfpathlineto{\pgfqpoint{5.915890in}{5.218004in}}%
\pgfpathlineto{\pgfqpoint{5.916426in}{5.222729in}}%
\pgfpathlineto{\pgfqpoint{5.917499in}{5.211445in}}%
\pgfpathlineto{\pgfqpoint{5.918572in}{5.223871in}}%
\pgfpathlineto{\pgfqpoint{5.919109in}{5.220948in}}%
\pgfpathlineto{\pgfqpoint{5.920182in}{5.209406in}}%
\pgfpathlineto{\pgfqpoint{5.921254in}{5.211486in}}%
\pgfpathlineto{\pgfqpoint{5.921791in}{5.220174in}}%
\pgfpathlineto{\pgfqpoint{5.922327in}{5.218696in}}%
\pgfpathlineto{\pgfqpoint{5.924473in}{5.204983in}}%
\pgfpathlineto{\pgfqpoint{5.925546in}{5.218391in}}%
\pgfpathlineto{\pgfqpoint{5.926619in}{5.201366in}}%
\pgfpathlineto{\pgfqpoint{5.927692in}{5.212035in}}%
\pgfpathlineto{\pgfqpoint{5.928228in}{5.193097in}}%
\pgfpathlineto{\pgfqpoint{5.928764in}{5.210655in}}%
\pgfpathlineto{\pgfqpoint{5.929837in}{5.197649in}}%
\pgfpathlineto{\pgfqpoint{5.930374in}{5.199510in}}%
\pgfpathlineto{\pgfqpoint{5.931447in}{5.205802in}}%
\pgfpathlineto{\pgfqpoint{5.931983in}{5.191080in}}%
\pgfpathlineto{\pgfqpoint{5.932520in}{5.199755in}}%
\pgfpathlineto{\pgfqpoint{5.933592in}{5.191743in}}%
\pgfpathlineto{\pgfqpoint{5.934665in}{5.206444in}}%
\pgfpathlineto{\pgfqpoint{5.935738in}{5.204357in}}%
\pgfpathlineto{\pgfqpoint{5.936811in}{5.191820in}}%
\pgfpathlineto{\pgfqpoint{5.937884in}{5.201104in}}%
\pgfpathlineto{\pgfqpoint{5.938957in}{5.190262in}}%
\pgfpathlineto{\pgfqpoint{5.939493in}{5.204838in}}%
\pgfpathlineto{\pgfqpoint{5.940030in}{5.187704in}}%
\pgfpathlineto{\pgfqpoint{5.940566in}{5.193063in}}%
\pgfpathlineto{\pgfqpoint{5.941103in}{5.191217in}}%
\pgfpathlineto{\pgfqpoint{5.941639in}{5.197003in}}%
\pgfpathlineto{\pgfqpoint{5.943248in}{5.173644in}}%
\pgfpathlineto{\pgfqpoint{5.944321in}{5.191860in}}%
\pgfpathlineto{\pgfqpoint{5.946467in}{5.170881in}}%
\pgfpathlineto{\pgfqpoint{5.947003in}{5.171562in}}%
\pgfpathlineto{\pgfqpoint{5.947540in}{5.187167in}}%
\pgfpathlineto{\pgfqpoint{5.948076in}{5.182855in}}%
\pgfpathlineto{\pgfqpoint{5.949685in}{5.170220in}}%
\pgfpathlineto{\pgfqpoint{5.951295in}{5.182119in}}%
\pgfpathlineto{\pgfqpoint{5.951831in}{5.167474in}}%
\pgfpathlineto{\pgfqpoint{5.952368in}{5.175559in}}%
\pgfpathlineto{\pgfqpoint{5.952904in}{5.172381in}}%
\pgfpathlineto{\pgfqpoint{5.953441in}{5.184820in}}%
\pgfpathlineto{\pgfqpoint{5.953977in}{5.179971in}}%
\pgfpathlineto{\pgfqpoint{5.954513in}{5.175002in}}%
\pgfpathlineto{\pgfqpoint{5.955050in}{5.155893in}}%
\pgfpathlineto{\pgfqpoint{5.955586in}{5.162872in}}%
\pgfpathlineto{\pgfqpoint{5.956659in}{5.182120in}}%
\pgfpathlineto{\pgfqpoint{5.958268in}{5.161525in}}%
\pgfpathlineto{\pgfqpoint{5.959878in}{5.152697in}}%
\pgfpathlineto{\pgfqpoint{5.960414in}{5.156090in}}%
\pgfpathlineto{\pgfqpoint{5.960951in}{5.168682in}}%
\pgfpathlineto{\pgfqpoint{5.961487in}{5.147238in}}%
\pgfpathlineto{\pgfqpoint{5.962024in}{5.161354in}}%
\pgfpathlineto{\pgfqpoint{5.963633in}{5.135319in}}%
\pgfpathlineto{\pgfqpoint{5.964706in}{5.156060in}}%
\pgfpathlineto{\pgfqpoint{5.966851in}{5.128508in}}%
\pgfpathlineto{\pgfqpoint{5.967388in}{5.156083in}}%
\pgfpathlineto{\pgfqpoint{5.967924in}{5.147635in}}%
\pgfpathlineto{\pgfqpoint{5.968461in}{5.147091in}}%
\pgfpathlineto{\pgfqpoint{5.968997in}{5.137578in}}%
\pgfpathlineto{\pgfqpoint{5.969534in}{5.139759in}}%
\pgfpathlineto{\pgfqpoint{5.970606in}{5.155678in}}%
\pgfpathlineto{\pgfqpoint{5.971143in}{5.152131in}}%
\pgfpathlineto{\pgfqpoint{5.971679in}{5.128622in}}%
\pgfpathlineto{\pgfqpoint{5.972216in}{5.129868in}}%
\pgfpathlineto{\pgfqpoint{5.973289in}{5.147374in}}%
\pgfpathlineto{\pgfqpoint{5.974898in}{5.123168in}}%
\pgfpathlineto{\pgfqpoint{5.975434in}{5.139812in}}%
\pgfpathlineto{\pgfqpoint{5.975971in}{5.132395in}}%
\pgfpathlineto{\pgfqpoint{5.977044in}{5.120389in}}%
\pgfpathlineto{\pgfqpoint{5.977580in}{5.128698in}}%
\pgfpathlineto{\pgfqpoint{5.978117in}{5.105347in}}%
\pgfpathlineto{\pgfqpoint{5.978653in}{5.128184in}}%
\pgfpathlineto{\pgfqpoint{5.979189in}{5.132400in}}%
\pgfpathlineto{\pgfqpoint{5.980262in}{5.102884in}}%
\pgfpathlineto{\pgfqpoint{5.980799in}{5.103916in}}%
\pgfpathlineto{\pgfqpoint{5.981872in}{5.130400in}}%
\pgfpathlineto{\pgfqpoint{5.982408in}{5.122954in}}%
\pgfpathlineto{\pgfqpoint{5.982945in}{5.117260in}}%
\pgfpathlineto{\pgfqpoint{5.983481in}{5.073329in}}%
\pgfpathlineto{\pgfqpoint{5.984017in}{5.126199in}}%
\pgfpathlineto{\pgfqpoint{5.984554in}{5.118451in}}%
\pgfpathlineto{\pgfqpoint{5.985627in}{5.109755in}}%
\pgfpathlineto{\pgfqpoint{5.986163in}{5.110680in}}%
\pgfpathlineto{\pgfqpoint{5.987236in}{5.120750in}}%
\pgfpathlineto{\pgfqpoint{5.988845in}{5.079051in}}%
\pgfpathlineto{\pgfqpoint{5.989382in}{5.104893in}}%
\pgfpathlineto{\pgfqpoint{5.989918in}{5.087965in}}%
\pgfpathlineto{\pgfqpoint{5.991528in}{5.102112in}}%
\pgfpathlineto{\pgfqpoint{5.992600in}{5.091394in}}%
\pgfpathlineto{\pgfqpoint{5.993137in}{5.073913in}}%
\pgfpathlineto{\pgfqpoint{5.993673in}{5.097421in}}%
\pgfpathlineto{\pgfqpoint{5.994746in}{5.057464in}}%
\pgfpathlineto{\pgfqpoint{5.995819in}{5.106960in}}%
\pgfpathlineto{\pgfqpoint{5.996355in}{5.089521in}}%
\pgfpathlineto{\pgfqpoint{5.997428in}{5.042065in}}%
\pgfpathlineto{\pgfqpoint{5.999038in}{5.092300in}}%
\pgfpathlineto{\pgfqpoint{6.000110in}{5.043353in}}%
\pgfpathlineto{\pgfqpoint{6.000647in}{5.093182in}}%
\pgfpathlineto{\pgfqpoint{6.001183in}{5.092342in}}%
\pgfpathlineto{\pgfqpoint{6.002793in}{5.053745in}}%
\pgfpathlineto{\pgfqpoint{6.003866in}{5.077613in}}%
\pgfpathlineto{\pgfqpoint{6.005475in}{5.028443in}}%
\pgfpathlineto{\pgfqpoint{6.006011in}{5.038487in}}%
\pgfpathlineto{\pgfqpoint{6.006548in}{5.039419in}}%
\pgfpathlineto{\pgfqpoint{6.007621in}{5.067193in}}%
\pgfpathlineto{\pgfqpoint{6.008693in}{5.017840in}}%
\pgfpathlineto{\pgfqpoint{6.010303in}{5.084229in}}%
\pgfpathlineto{\pgfqpoint{6.011376in}{4.997626in}}%
\pgfpathlineto{\pgfqpoint{6.012449in}{5.065728in}}%
\pgfpathlineto{\pgfqpoint{6.013521in}{5.050576in}}%
\pgfpathlineto{\pgfqpoint{6.014058in}{5.017439in}}%
\pgfpathlineto{\pgfqpoint{6.014594in}{5.021171in}}%
\pgfpathlineto{\pgfqpoint{6.015131in}{5.022829in}}%
\pgfpathlineto{\pgfqpoint{6.015667in}{5.030572in}}%
\pgfpathlineto{\pgfqpoint{6.016740in}{5.003056in}}%
\pgfpathlineto{\pgfqpoint{6.017276in}{5.010245in}}%
\pgfpathlineto{\pgfqpoint{6.017813in}{5.012035in}}%
\pgfpathlineto{\pgfqpoint{6.018349in}{5.022609in}}%
\pgfpathlineto{\pgfqpoint{6.018886in}{5.063365in}}%
\pgfpathlineto{\pgfqpoint{6.019959in}{4.976998in}}%
\pgfpathlineto{\pgfqpoint{6.020495in}{5.009602in}}%
\pgfpathlineto{\pgfqpoint{6.021568in}{5.049681in}}%
\pgfpathlineto{\pgfqpoint{6.022641in}{4.944962in}}%
\pgfpathlineto{\pgfqpoint{6.023714in}{5.015746in}}%
\pgfpathlineto{\pgfqpoint{6.024250in}{5.007096in}}%
\pgfpathlineto{\pgfqpoint{6.024787in}{4.953906in}}%
\pgfpathlineto{\pgfqpoint{6.025323in}{4.959012in}}%
\pgfpathlineto{\pgfqpoint{6.025859in}{5.033363in}}%
\pgfpathlineto{\pgfqpoint{6.026396in}{5.013148in}}%
\pgfpathlineto{\pgfqpoint{6.026932in}{4.997674in}}%
\pgfpathlineto{\pgfqpoint{6.027469in}{4.935935in}}%
\pgfpathlineto{\pgfqpoint{6.028005in}{4.956460in}}%
\pgfpathlineto{\pgfqpoint{6.028542in}{5.016176in}}%
\pgfpathlineto{\pgfqpoint{6.029078in}{4.999820in}}%
\pgfpathlineto{\pgfqpoint{6.029614in}{4.949657in}}%
\pgfpathlineto{\pgfqpoint{6.030151in}{5.008707in}}%
\pgfpathlineto{\pgfqpoint{6.030687in}{4.981759in}}%
\pgfpathlineto{\pgfqpoint{6.031224in}{4.949535in}}%
\pgfpathlineto{\pgfqpoint{6.031760in}{4.956747in}}%
\pgfpathlineto{\pgfqpoint{6.032297in}{4.983732in}}%
\pgfpathlineto{\pgfqpoint{6.032833in}{4.975280in}}%
\pgfpathlineto{\pgfqpoint{6.033906in}{4.923069in}}%
\pgfpathlineto{\pgfqpoint{6.034979in}{5.002154in}}%
\pgfpathlineto{\pgfqpoint{6.035515in}{4.999705in}}%
\pgfpathlineto{\pgfqpoint{6.036588in}{4.902120in}}%
\pgfpathlineto{\pgfqpoint{6.037661in}{4.982785in}}%
\pgfpathlineto{\pgfqpoint{6.038734in}{4.850315in}}%
\pgfpathlineto{\pgfqpoint{6.039270in}{4.886666in}}%
\pgfpathlineto{\pgfqpoint{6.040343in}{4.960229in}}%
\pgfpathlineto{\pgfqpoint{6.041953in}{4.925507in}}%
\pgfpathlineto{\pgfqpoint{6.042489in}{4.965095in}}%
\pgfpathlineto{\pgfqpoint{6.043562in}{4.963302in}}%
\pgfpathlineto{\pgfqpoint{6.044635in}{4.862880in}}%
\pgfpathlineto{\pgfqpoint{6.045171in}{4.885735in}}%
\pgfpathlineto{\pgfqpoint{6.046244in}{4.962474in}}%
\pgfpathlineto{\pgfqpoint{6.046780in}{4.952733in}}%
\pgfpathlineto{\pgfqpoint{6.047317in}{4.815351in}}%
\pgfpathlineto{\pgfqpoint{6.047853in}{4.838818in}}%
\pgfpathlineto{\pgfqpoint{6.048390in}{5.021998in}}%
\pgfpathlineto{\pgfqpoint{6.048926in}{4.973407in}}%
\pgfpathlineto{\pgfqpoint{6.049999in}{4.763291in}}%
\pgfpathlineto{\pgfqpoint{6.050535in}{4.856144in}}%
\pgfpathlineto{\pgfqpoint{6.051072in}{4.979758in}}%
\pgfpathlineto{\pgfqpoint{6.051608in}{4.941323in}}%
\pgfpathlineto{\pgfqpoint{6.052145in}{4.768124in}}%
\pgfpathlineto{\pgfqpoint{6.052681in}{4.867199in}}%
\pgfpathlineto{\pgfqpoint{6.053218in}{4.877389in}}%
\pgfpathlineto{\pgfqpoint{6.053754in}{4.990754in}}%
\pgfpathlineto{\pgfqpoint{6.054291in}{4.898026in}}%
\pgfpathlineto{\pgfqpoint{6.054827in}{4.822151in}}%
\pgfpathlineto{\pgfqpoint{6.055363in}{4.828654in}}%
\pgfpathlineto{\pgfqpoint{6.055900in}{4.907281in}}%
\pgfpathlineto{\pgfqpoint{6.056436in}{4.842337in}}%
\pgfpathlineto{\pgfqpoint{6.057509in}{4.924412in}}%
\pgfpathlineto{\pgfqpoint{6.058046in}{4.883601in}}%
\pgfpathlineto{\pgfqpoint{6.058582in}{4.826688in}}%
\pgfpathlineto{\pgfqpoint{6.059118in}{4.831882in}}%
\pgfpathlineto{\pgfqpoint{6.059655in}{4.933598in}}%
\pgfpathlineto{\pgfqpoint{6.060191in}{4.922375in}}%
\pgfpathlineto{\pgfqpoint{6.061264in}{4.687037in}}%
\pgfpathlineto{\pgfqpoint{6.062337in}{4.923972in}}%
\pgfpathlineto{\pgfqpoint{6.063946in}{4.720754in}}%
\pgfpathlineto{\pgfqpoint{6.065019in}{4.987212in}}%
\pgfpathlineto{\pgfqpoint{6.066092in}{4.710289in}}%
\pgfpathlineto{\pgfqpoint{6.066629in}{4.787216in}}%
\pgfpathlineto{\pgfqpoint{6.067701in}{4.869870in}}%
\pgfpathlineto{\pgfqpoint{6.068238in}{4.705163in}}%
\pgfpathlineto{\pgfqpoint{6.068774in}{4.802004in}}%
\pgfpathlineto{\pgfqpoint{6.069847in}{4.770681in}}%
\pgfpathlineto{\pgfqpoint{6.070920in}{5.012421in}}%
\pgfpathlineto{\pgfqpoint{6.072529in}{4.658323in}}%
\pgfpathlineto{\pgfqpoint{6.073602in}{4.953614in}}%
\pgfpathlineto{\pgfqpoint{6.074675in}{4.486223in}}%
\pgfpathlineto{\pgfqpoint{6.076284in}{4.957976in}}%
\pgfpathlineto{\pgfqpoint{6.077357in}{4.478081in}}%
\pgfpathlineto{\pgfqpoint{6.077894in}{4.929922in}}%
\pgfpathlineto{\pgfqpoint{6.078430in}{4.906798in}}%
\pgfpathlineto{\pgfqpoint{6.080039in}{4.619559in}}%
\pgfpathlineto{\pgfqpoint{6.080576in}{4.875934in}}%
\pgfpathlineto{\pgfqpoint{6.081112in}{4.818711in}}%
\pgfpathlineto{\pgfqpoint{6.081649in}{4.729272in}}%
\pgfpathlineto{\pgfqpoint{6.082185in}{4.839444in}}%
\pgfpathlineto{\pgfqpoint{6.082722in}{4.707538in}}%
\pgfpathlineto{\pgfqpoint{6.083258in}{4.798345in}}%
\pgfpathlineto{\pgfqpoint{6.084331in}{4.871593in}}%
\pgfpathlineto{\pgfqpoint{6.085404in}{4.626360in}}%
\pgfpathlineto{\pgfqpoint{6.085940in}{4.229952in}}%
\pgfpathlineto{\pgfqpoint{6.087013in}{4.959290in}}%
\pgfpathlineto{\pgfqpoint{6.088086in}{4.063505in}}%
\pgfpathlineto{\pgfqpoint{6.088622in}{4.584276in}}%
\pgfpathlineto{\pgfqpoint{6.089695in}{4.981671in}}%
\pgfpathlineto{\pgfqpoint{6.090232in}{4.329966in}}%
\pgfpathlineto{\pgfqpoint{6.090768in}{4.428258in}}%
\pgfpathlineto{\pgfqpoint{6.091841in}{4.868615in}}%
\pgfpathlineto{\pgfqpoint{6.093450in}{4.589488in}}%
\pgfpathlineto{\pgfqpoint{6.093987in}{4.912835in}}%
\pgfpathlineto{\pgfqpoint{6.094523in}{4.682994in}}%
\pgfpathlineto{\pgfqpoint{6.095060in}{4.620803in}}%
\pgfpathlineto{\pgfqpoint{6.095596in}{4.826378in}}%
\pgfpathlineto{\pgfqpoint{6.095596in}{4.826378in}}%
\pgfpathlineto{\pgfqpoint{6.095596in}{4.826378in}}%
\pgfpathlineto{\pgfqpoint{6.096133in}{4.614954in}}%
\pgfpathlineto{\pgfqpoint{6.096669in}{4.764705in}}%
\pgfpathlineto{\pgfqpoint{6.097205in}{4.710611in}}%
\pgfpathlineto{\pgfqpoint{6.097742in}{4.750264in}}%
\pgfpathlineto{\pgfqpoint{6.098278in}{4.874969in}}%
\pgfpathlineto{\pgfqpoint{6.098815in}{4.299739in}}%
\pgfpathlineto{\pgfqpoint{6.099351in}{4.364217in}}%
\pgfpathlineto{\pgfqpoint{6.100424in}{5.061783in}}%
\pgfpathlineto{\pgfqpoint{6.101497in}{4.154282in}}%
\pgfpathlineto{\pgfqpoint{6.102570in}{5.022408in}}%
\pgfpathlineto{\pgfqpoint{6.103106in}{4.819349in}}%
\pgfpathlineto{\pgfqpoint{6.103643in}{3.984524in}}%
\pgfpathlineto{\pgfqpoint{6.104179in}{4.583683in}}%
\pgfpathlineto{\pgfqpoint{6.105252in}{4.878279in}}%
\pgfpathlineto{\pgfqpoint{6.105788in}{4.591498in}}%
\pgfpathlineto{\pgfqpoint{6.106861in}{4.610451in}}%
\pgfpathlineto{\pgfqpoint{6.107398in}{4.930052in}}%
\pgfpathlineto{\pgfqpoint{6.107934in}{4.422378in}}%
\pgfpathlineto{\pgfqpoint{6.108471in}{4.664937in}}%
\pgfpathlineto{\pgfqpoint{6.109007in}{4.876213in}}%
\pgfpathlineto{\pgfqpoint{6.109543in}{4.721950in}}%
\pgfpathlineto{\pgfqpoint{6.110080in}{4.601565in}}%
\pgfpathlineto{\pgfqpoint{6.110616in}{4.613372in}}%
\pgfpathlineto{\pgfqpoint{6.111153in}{4.983170in}}%
\pgfpathlineto{\pgfqpoint{6.111689in}{4.867647in}}%
\pgfpathlineto{\pgfqpoint{6.112226in}{4.089641in}}%
\pgfpathlineto{\pgfqpoint{6.112762in}{4.384766in}}%
\pgfpathlineto{\pgfqpoint{6.113299in}{5.051277in}}%
\pgfpathlineto{\pgfqpoint{6.113835in}{4.877929in}}%
\pgfpathlineto{\pgfqpoint{6.114908in}{4.187975in}}%
\pgfpathlineto{\pgfqpoint{6.115981in}{4.990448in}}%
\pgfpathlineto{\pgfqpoint{6.117054in}{4.472097in}}%
\pgfpathlineto{\pgfqpoint{6.118126in}{4.948385in}}%
\pgfpathlineto{\pgfqpoint{6.118663in}{4.843452in}}%
\pgfpathlineto{\pgfqpoint{6.119199in}{4.608039in}}%
\pgfpathlineto{\pgfqpoint{6.119736in}{4.685886in}}%
\pgfpathlineto{\pgfqpoint{6.120272in}{4.947084in}}%
\pgfpathlineto{\pgfqpoint{6.120809in}{4.742966in}}%
\pgfpathlineto{\pgfqpoint{6.121345in}{4.531616in}}%
\pgfpathlineto{\pgfqpoint{6.122418in}{4.888297in}}%
\pgfpathlineto{\pgfqpoint{6.123491in}{4.334697in}}%
\pgfpathlineto{\pgfqpoint{6.124564in}{5.017896in}}%
\pgfpathlineto{\pgfqpoint{6.125637in}{4.134225in}}%
\pgfpathlineto{\pgfqpoint{6.126709in}{4.926835in}}%
\pgfpathlineto{\pgfqpoint{6.127246in}{4.901285in}}%
\pgfpathlineto{\pgfqpoint{6.127782in}{3.921941in}}%
\pgfpathlineto{\pgfqpoint{6.128319in}{4.221612in}}%
\pgfpathlineto{\pgfqpoint{6.129392in}{4.983159in}}%
\pgfpathlineto{\pgfqpoint{6.129928in}{4.522062in}}%
\pgfpathlineto{\pgfqpoint{6.130464in}{4.686100in}}%
\pgfpathlineto{\pgfqpoint{6.131537in}{5.001176in}}%
\pgfpathlineto{\pgfqpoint{6.132074in}{4.677743in}}%
\pgfpathlineto{\pgfqpoint{6.132610in}{4.699383in}}%
\pgfpathlineto{\pgfqpoint{6.133147in}{4.950222in}}%
\pgfpathlineto{\pgfqpoint{6.133683in}{4.823302in}}%
\pgfpathlineto{\pgfqpoint{6.134220in}{4.603471in}}%
\pgfpathlineto{\pgfqpoint{6.134756in}{4.739955in}}%
\pgfpathlineto{\pgfqpoint{6.135292in}{4.946422in}}%
\pgfpathlineto{\pgfqpoint{6.135829in}{4.825691in}}%
\pgfpathlineto{\pgfqpoint{6.136365in}{4.562038in}}%
\pgfpathlineto{\pgfqpoint{6.136902in}{4.697899in}}%
\pgfpathlineto{\pgfqpoint{6.137975in}{4.945941in}}%
\pgfpathlineto{\pgfqpoint{6.138511in}{4.403799in}}%
\pgfpathlineto{\pgfqpoint{6.139047in}{4.446693in}}%
\pgfpathlineto{\pgfqpoint{6.140120in}{5.062047in}}%
\pgfpathlineto{\pgfqpoint{6.141193in}{4.445451in}}%
\pgfpathlineto{\pgfqpoint{6.142266in}{5.033566in}}%
\pgfpathlineto{\pgfqpoint{6.143339in}{4.620566in}}%
\pgfpathlineto{\pgfqpoint{6.144412in}{4.961821in}}%
\pgfpathlineto{\pgfqpoint{6.145485in}{4.560988in}}%
\pgfpathlineto{\pgfqpoint{6.146021in}{4.978426in}}%
\pgfpathlineto{\pgfqpoint{6.146558in}{4.866962in}}%
\pgfpathlineto{\pgfqpoint{6.147630in}{4.669690in}}%
\pgfpathlineto{\pgfqpoint{6.148703in}{4.977770in}}%
\pgfpathlineto{\pgfqpoint{6.149776in}{4.549653in}}%
\pgfpathlineto{\pgfqpoint{6.150849in}{5.085835in}}%
\pgfpathlineto{\pgfqpoint{6.151922in}{4.598703in}}%
\pgfpathlineto{\pgfqpoint{6.152995in}{5.071109in}}%
\pgfpathlineto{\pgfqpoint{6.153531in}{4.912814in}}%
\pgfpathlineto{\pgfqpoint{6.154068in}{4.609572in}}%
\pgfpathlineto{\pgfqpoint{6.154604in}{4.907552in}}%
\pgfpathlineto{\pgfqpoint{6.155141in}{5.063530in}}%
\pgfpathlineto{\pgfqpoint{6.155677in}{4.925584in}}%
\pgfpathlineto{\pgfqpoint{6.156213in}{4.541502in}}%
\pgfpathlineto{\pgfqpoint{6.156750in}{4.898753in}}%
\pgfpathlineto{\pgfqpoint{6.157286in}{4.976672in}}%
\pgfpathlineto{\pgfqpoint{6.158359in}{4.659731in}}%
\pgfpathlineto{\pgfqpoint{6.159432in}{4.979540in}}%
\pgfpathlineto{\pgfqpoint{6.160505in}{4.682499in}}%
\pgfpathlineto{\pgfqpoint{6.161578in}{5.086275in}}%
\pgfpathlineto{\pgfqpoint{6.162651in}{4.715189in}}%
\pgfpathlineto{\pgfqpoint{6.163724in}{5.089240in}}%
\pgfpathlineto{\pgfqpoint{6.164260in}{4.966777in}}%
\pgfpathlineto{\pgfqpoint{6.164796in}{4.697767in}}%
\pgfpathlineto{\pgfqpoint{6.165333in}{4.928471in}}%
\pgfpathlineto{\pgfqpoint{6.165869in}{5.073615in}}%
\pgfpathlineto{\pgfqpoint{6.166406in}{5.000146in}}%
\pgfpathlineto{\pgfqpoint{6.166942in}{4.670494in}}%
\pgfpathlineto{\pgfqpoint{6.167479in}{4.904929in}}%
\pgfpathlineto{\pgfqpoint{6.168015in}{5.052010in}}%
\pgfpathlineto{\pgfqpoint{6.168551in}{4.932465in}}%
\pgfpathlineto{\pgfqpoint{6.169088in}{4.751397in}}%
\pgfpathlineto{\pgfqpoint{6.169624in}{4.901385in}}%
\pgfpathlineto{\pgfqpoint{6.170161in}{5.037378in}}%
\pgfpathlineto{\pgfqpoint{6.170697in}{4.942110in}}%
\pgfpathlineto{\pgfqpoint{6.171234in}{4.758112in}}%
\pgfpathlineto{\pgfqpoint{6.171234in}{4.758112in}}%
\pgfpathlineto{\pgfqpoint{6.171234in}{4.758112in}}%
\pgfpathlineto{\pgfqpoint{6.172306in}{5.077139in}}%
\pgfpathlineto{\pgfqpoint{6.172843in}{4.942646in}}%
\pgfpathlineto{\pgfqpoint{6.173379in}{4.816771in}}%
\pgfpathlineto{\pgfqpoint{6.174452in}{5.093738in}}%
\pgfpathlineto{\pgfqpoint{6.175525in}{4.777295in}}%
\pgfpathlineto{\pgfqpoint{6.176598in}{5.099070in}}%
\pgfpathlineto{\pgfqpoint{6.177134in}{5.041882in}}%
\pgfpathlineto{\pgfqpoint{6.177671in}{4.796204in}}%
\pgfpathlineto{\pgfqpoint{6.178207in}{4.959196in}}%
\pgfpathlineto{\pgfqpoint{6.178744in}{5.103137in}}%
\pgfpathlineto{\pgfqpoint{6.179280in}{5.000849in}}%
\pgfpathlineto{\pgfqpoint{6.179817in}{4.821370in}}%
\pgfpathlineto{\pgfqpoint{6.180353in}{4.950051in}}%
\pgfpathlineto{\pgfqpoint{6.180889in}{5.078277in}}%
\pgfpathlineto{\pgfqpoint{6.181426in}{4.983382in}}%
\pgfpathlineto{\pgfqpoint{6.181962in}{4.857175in}}%
\pgfpathlineto{\pgfqpoint{6.182499in}{4.957371in}}%
\pgfpathlineto{\pgfqpoint{6.183035in}{5.059660in}}%
\pgfpathlineto{\pgfqpoint{6.183572in}{4.982163in}}%
\pgfpathlineto{\pgfqpoint{6.184108in}{4.828198in}}%
\pgfpathlineto{\pgfqpoint{6.185181in}{5.102936in}}%
\pgfpathlineto{\pgfqpoint{6.186254in}{4.923422in}}%
\pgfpathlineto{\pgfqpoint{6.187327in}{5.119614in}}%
\pgfpathlineto{\pgfqpoint{6.188400in}{4.927899in}}%
\pgfpathlineto{\pgfqpoint{6.188936in}{5.016745in}}%
\pgfpathlineto{\pgfqpoint{6.189472in}{5.138076in}}%
\pgfpathlineto{\pgfqpoint{6.189472in}{5.138076in}}%
\pgfpathlineto{\pgfqpoint{6.189472in}{5.138076in}}%
\pgfpathlineto{\pgfqpoint{6.190545in}{4.866200in}}%
\pgfpathlineto{\pgfqpoint{6.191618in}{5.104491in}}%
\pgfpathlineto{\pgfqpoint{6.192691in}{4.915146in}}%
\pgfpathlineto{\pgfqpoint{6.193764in}{5.081159in}}%
\pgfpathlineto{\pgfqpoint{6.194300in}{5.038400in}}%
\pgfpathlineto{\pgfqpoint{6.194837in}{4.887034in}}%
\pgfpathlineto{\pgfqpoint{6.195910in}{5.106743in}}%
\pgfpathlineto{\pgfqpoint{6.196983in}{4.958427in}}%
\pgfpathlineto{\pgfqpoint{6.198055in}{5.129566in}}%
\pgfpathlineto{\pgfqpoint{6.198592in}{5.007530in}}%
\pgfpathlineto{\pgfqpoint{6.199128in}{5.022324in}}%
\pgfpathlineto{\pgfqpoint{6.200201in}{5.165353in}}%
\pgfpathlineto{\pgfqpoint{6.201274in}{4.942788in}}%
\pgfpathlineto{\pgfqpoint{6.201810in}{5.107201in}}%
\pgfpathlineto{\pgfqpoint{6.202347in}{5.103954in}}%
\pgfpathlineto{\pgfqpoint{6.203420in}{4.953419in}}%
\pgfpathlineto{\pgfqpoint{6.204493in}{5.099122in}}%
\pgfpathlineto{\pgfqpoint{6.205566in}{4.992676in}}%
\pgfpathlineto{\pgfqpoint{6.206638in}{5.139096in}}%
\pgfpathlineto{\pgfqpoint{6.207175in}{5.015352in}}%
\pgfpathlineto{\pgfqpoint{6.207711in}{5.031715in}}%
\pgfpathlineto{\pgfqpoint{6.208784in}{5.141340in}}%
\pgfpathlineto{\pgfqpoint{6.209321in}{5.012535in}}%
\pgfpathlineto{\pgfqpoint{6.209857in}{5.039359in}}%
\pgfpathlineto{\pgfqpoint{6.210393in}{5.150623in}}%
\pgfpathlineto{\pgfqpoint{6.210930in}{5.120037in}}%
\pgfpathlineto{\pgfqpoint{6.211466in}{5.035253in}}%
\pgfpathlineto{\pgfqpoint{6.212003in}{5.061592in}}%
\pgfpathlineto{\pgfqpoint{6.212539in}{5.164542in}}%
\pgfpathlineto{\pgfqpoint{6.213076in}{5.134230in}}%
\pgfpathlineto{\pgfqpoint{6.214149in}{5.019542in}}%
\pgfpathlineto{\pgfqpoint{6.214685in}{5.158564in}}%
\pgfpathlineto{\pgfqpoint{6.215221in}{5.123143in}}%
\pgfpathlineto{\pgfqpoint{6.215758in}{5.021468in}}%
\pgfpathlineto{\pgfqpoint{6.216294in}{5.081046in}}%
\pgfpathlineto{\pgfqpoint{6.217367in}{5.130660in}}%
\pgfpathlineto{\pgfqpoint{6.217904in}{5.029639in}}%
\pgfpathlineto{\pgfqpoint{6.218440in}{5.046152in}}%
\pgfpathlineto{\pgfqpoint{6.218976in}{5.133704in}}%
\pgfpathlineto{\pgfqpoint{6.219513in}{5.132060in}}%
\pgfpathlineto{\pgfqpoint{6.220049in}{5.037713in}}%
\pgfpathlineto{\pgfqpoint{6.220586in}{5.075718in}}%
\pgfpathlineto{\pgfqpoint{6.221122in}{5.205333in}}%
\pgfpathlineto{\pgfqpoint{6.221659in}{5.100356in}}%
\pgfpathlineto{\pgfqpoint{6.222195in}{5.073072in}}%
\pgfpathlineto{\pgfqpoint{6.223268in}{5.156752in}}%
\pgfpathlineto{\pgfqpoint{6.223804in}{5.139237in}}%
\pgfpathlineto{\pgfqpoint{6.224341in}{5.056940in}}%
\pgfpathlineto{\pgfqpoint{6.224341in}{5.056940in}}%
\pgfpathlineto{\pgfqpoint{6.224341in}{5.056940in}}%
\pgfpathlineto{\pgfqpoint{6.225414in}{5.175544in}}%
\pgfpathlineto{\pgfqpoint{6.226487in}{5.051899in}}%
\pgfpathlineto{\pgfqpoint{6.227023in}{5.125114in}}%
\pgfpathlineto{\pgfqpoint{6.227559in}{5.166025in}}%
\pgfpathlineto{\pgfqpoint{6.228632in}{5.082386in}}%
\pgfpathlineto{\pgfqpoint{6.229705in}{5.147164in}}%
\pgfpathlineto{\pgfqpoint{6.230778in}{5.091008in}}%
\pgfpathlineto{\pgfqpoint{6.231851in}{5.158483in}}%
\pgfpathlineto{\pgfqpoint{6.232924in}{5.091590in}}%
\pgfpathlineto{\pgfqpoint{6.233997in}{5.185308in}}%
\pgfpathlineto{\pgfqpoint{6.234533in}{5.106594in}}%
\pgfpathlineto{\pgfqpoint{6.235070in}{5.115294in}}%
\pgfpathlineto{\pgfqpoint{6.235606in}{5.195928in}}%
\pgfpathlineto{\pgfqpoint{6.236142in}{5.154129in}}%
\pgfpathlineto{\pgfqpoint{6.236679in}{5.113264in}}%
\pgfpathlineto{\pgfqpoint{6.237215in}{5.139552in}}%
\pgfpathlineto{\pgfqpoint{6.238288in}{5.188165in}}%
\pgfpathlineto{\pgfqpoint{6.238825in}{5.108819in}}%
\pgfpathlineto{\pgfqpoint{6.239361in}{5.127138in}}%
\pgfpathlineto{\pgfqpoint{6.239897in}{5.202666in}}%
\pgfpathlineto{\pgfqpoint{6.240434in}{5.149211in}}%
\pgfpathlineto{\pgfqpoint{6.240970in}{5.107545in}}%
\pgfpathlineto{\pgfqpoint{6.240970in}{5.107545in}}%
\pgfpathlineto{\pgfqpoint{6.240970in}{5.107545in}}%
\pgfpathlineto{\pgfqpoint{6.242043in}{5.164773in}}%
\pgfpathlineto{\pgfqpoint{6.242580in}{5.141364in}}%
\pgfpathlineto{\pgfqpoint{6.243116in}{5.146469in}}%
\pgfpathlineto{\pgfqpoint{6.243652in}{5.151028in}}%
\pgfpathlineto{\pgfqpoint{6.244725in}{5.176785in}}%
\pgfpathlineto{\pgfqpoint{6.245262in}{5.110994in}}%
\pgfpathlineto{\pgfqpoint{6.245798in}{5.175171in}}%
\pgfpathlineto{\pgfqpoint{6.246335in}{5.199159in}}%
\pgfpathlineto{\pgfqpoint{6.246871in}{5.142186in}}%
\pgfpathlineto{\pgfqpoint{6.247408in}{5.160989in}}%
\pgfpathlineto{\pgfqpoint{6.248480in}{5.190482in}}%
\pgfpathlineto{\pgfqpoint{6.249017in}{5.148055in}}%
\pgfpathlineto{\pgfqpoint{6.249553in}{5.150928in}}%
\pgfpathlineto{\pgfqpoint{6.250626in}{5.185230in}}%
\pgfpathlineto{\pgfqpoint{6.251163in}{5.150754in}}%
\pgfpathlineto{\pgfqpoint{6.251699in}{5.154623in}}%
\pgfpathlineto{\pgfqpoint{6.252235in}{5.201347in}}%
\pgfpathlineto{\pgfqpoint{6.252772in}{5.187680in}}%
\pgfpathlineto{\pgfqpoint{6.253308in}{5.161017in}}%
\pgfpathlineto{\pgfqpoint{6.253845in}{5.164385in}}%
\pgfpathlineto{\pgfqpoint{6.254381in}{5.194008in}}%
\pgfpathlineto{\pgfqpoint{6.254918in}{5.174345in}}%
\pgfpathlineto{\pgfqpoint{6.255454in}{5.128715in}}%
\pgfpathlineto{\pgfqpoint{6.256527in}{5.189023in}}%
\pgfpathlineto{\pgfqpoint{6.257063in}{5.156552in}}%
\pgfpathlineto{\pgfqpoint{6.258136in}{5.156762in}}%
\pgfpathlineto{\pgfqpoint{6.258673in}{5.199691in}}%
\pgfpathlineto{\pgfqpoint{6.259209in}{5.189189in}}%
\pgfpathlineto{\pgfqpoint{6.259746in}{5.154055in}}%
\pgfpathlineto{\pgfqpoint{6.260282in}{5.186726in}}%
\pgfpathlineto{\pgfqpoint{6.260818in}{5.226779in}}%
\pgfpathlineto{\pgfqpoint{6.261355in}{5.176199in}}%
\pgfpathlineto{\pgfqpoint{6.261891in}{5.186644in}}%
\pgfpathlineto{\pgfqpoint{6.262428in}{5.208084in}}%
\pgfpathlineto{\pgfqpoint{6.264037in}{5.172208in}}%
\pgfpathlineto{\pgfqpoint{6.264574in}{5.179694in}}%
\pgfpathlineto{\pgfqpoint{6.265110in}{5.213235in}}%
\pgfpathlineto{\pgfqpoint{6.265646in}{5.181943in}}%
\pgfpathlineto{\pgfqpoint{6.266183in}{5.174800in}}%
\pgfpathlineto{\pgfqpoint{6.266719in}{5.206649in}}%
\pgfpathlineto{\pgfqpoint{6.267256in}{5.183652in}}%
\pgfpathlineto{\pgfqpoint{6.267792in}{5.180311in}}%
\pgfpathlineto{\pgfqpoint{6.268865in}{5.207040in}}%
\pgfpathlineto{\pgfqpoint{6.269401in}{5.197882in}}%
\pgfpathlineto{\pgfqpoint{6.269938in}{5.185079in}}%
\pgfpathlineto{\pgfqpoint{6.270474in}{5.196099in}}%
\pgfpathlineto{\pgfqpoint{6.271011in}{5.206080in}}%
\pgfpathlineto{\pgfqpoint{6.272620in}{5.182097in}}%
\pgfpathlineto{\pgfqpoint{6.273693in}{5.205073in}}%
\pgfpathlineto{\pgfqpoint{6.274229in}{5.178635in}}%
\pgfpathlineto{\pgfqpoint{6.275302in}{5.231368in}}%
\pgfpathlineto{\pgfqpoint{6.275839in}{5.191399in}}%
\pgfpathlineto{\pgfqpoint{6.276375in}{5.201817in}}%
\pgfpathlineto{\pgfqpoint{6.277984in}{5.221488in}}%
\pgfpathlineto{\pgfqpoint{6.279057in}{5.197201in}}%
\pgfpathlineto{\pgfqpoint{6.279594in}{5.223578in}}%
\pgfpathlineto{\pgfqpoint{6.280130in}{5.197384in}}%
\pgfpathlineto{\pgfqpoint{6.280667in}{5.190542in}}%
\pgfpathlineto{\pgfqpoint{6.281203in}{5.219541in}}%
\pgfpathlineto{\pgfqpoint{6.281739in}{5.184259in}}%
\pgfpathlineto{\pgfqpoint{6.282276in}{5.185210in}}%
\pgfpathlineto{\pgfqpoint{6.283349in}{5.218291in}}%
\pgfpathlineto{\pgfqpoint{6.283885in}{5.216465in}}%
\pgfpathlineto{\pgfqpoint{6.284422in}{5.209076in}}%
\pgfpathlineto{\pgfqpoint{6.284958in}{5.214951in}}%
\pgfpathlineto{\pgfqpoint{6.285495in}{5.221377in}}%
\pgfpathlineto{\pgfqpoint{6.286031in}{5.192346in}}%
\pgfpathlineto{\pgfqpoint{6.286567in}{5.206073in}}%
\pgfpathlineto{\pgfqpoint{6.287104in}{5.218085in}}%
\pgfpathlineto{\pgfqpoint{6.287640in}{5.217119in}}%
\pgfpathlineto{\pgfqpoint{6.288713in}{5.192646in}}%
\pgfpathlineto{\pgfqpoint{6.289786in}{5.226495in}}%
\pgfpathlineto{\pgfqpoint{6.290859in}{5.206765in}}%
\pgfpathlineto{\pgfqpoint{6.292468in}{5.232812in}}%
\pgfpathlineto{\pgfqpoint{6.293005in}{5.239036in}}%
\pgfpathlineto{\pgfqpoint{6.295150in}{5.196395in}}%
\pgfpathlineto{\pgfqpoint{6.295687in}{5.225448in}}%
\pgfpathlineto{\pgfqpoint{6.296223in}{5.211385in}}%
\pgfpathlineto{\pgfqpoint{6.296760in}{5.221417in}}%
\pgfpathlineto{\pgfqpoint{6.297296in}{5.218069in}}%
\pgfpathlineto{\pgfqpoint{6.298369in}{5.209130in}}%
\pgfpathlineto{\pgfqpoint{6.298905in}{5.231588in}}%
\pgfpathlineto{\pgfqpoint{6.299442in}{5.228180in}}%
\pgfpathlineto{\pgfqpoint{6.300515in}{5.213258in}}%
\pgfpathlineto{\pgfqpoint{6.301588in}{5.233109in}}%
\pgfpathlineto{\pgfqpoint{6.302660in}{5.214240in}}%
\pgfpathlineto{\pgfqpoint{6.303197in}{5.214666in}}%
\pgfpathlineto{\pgfqpoint{6.304270in}{5.212186in}}%
\pgfpathlineto{\pgfqpoint{6.304806in}{5.218351in}}%
\pgfpathlineto{\pgfqpoint{6.305343in}{5.213243in}}%
\pgfpathlineto{\pgfqpoint{6.305879in}{5.244756in}}%
\pgfpathlineto{\pgfqpoint{6.305879in}{5.244756in}}%
\pgfpathlineto{\pgfqpoint{6.305879in}{5.244756in}}%
\pgfpathlineto{\pgfqpoint{6.306416in}{5.211817in}}%
\pgfpathlineto{\pgfqpoint{6.306952in}{5.233011in}}%
\pgfpathlineto{\pgfqpoint{6.307488in}{5.228736in}}%
\pgfpathlineto{\pgfqpoint{6.308025in}{5.230591in}}%
\pgfpathlineto{\pgfqpoint{6.308561in}{5.236794in}}%
\pgfpathlineto{\pgfqpoint{6.309098in}{5.229638in}}%
\pgfpathlineto{\pgfqpoint{6.309634in}{5.204644in}}%
\pgfpathlineto{\pgfqpoint{6.310171in}{5.227942in}}%
\pgfpathlineto{\pgfqpoint{6.311243in}{5.224608in}}%
\pgfpathlineto{\pgfqpoint{6.311780in}{5.227948in}}%
\pgfpathlineto{\pgfqpoint{6.312316in}{5.210997in}}%
\pgfpathlineto{\pgfqpoint{6.312853in}{5.213591in}}%
\pgfpathlineto{\pgfqpoint{6.313926in}{5.230421in}}%
\pgfpathlineto{\pgfqpoint{6.314462in}{5.222798in}}%
\pgfpathlineto{\pgfqpoint{6.316071in}{5.244047in}}%
\pgfpathlineto{\pgfqpoint{6.316608in}{5.216105in}}%
\pgfpathlineto{\pgfqpoint{6.317144in}{5.225638in}}%
\pgfpathlineto{\pgfqpoint{6.318754in}{5.217567in}}%
\pgfpathlineto{\pgfqpoint{6.319826in}{5.222138in}}%
\pgfpathlineto{\pgfqpoint{6.320363in}{5.241722in}}%
\pgfpathlineto{\pgfqpoint{6.320899in}{5.227336in}}%
\pgfpathlineto{\pgfqpoint{6.321972in}{5.227791in}}%
\pgfpathlineto{\pgfqpoint{6.322509in}{5.223107in}}%
\pgfpathlineto{\pgfqpoint{6.323045in}{5.247857in}}%
\pgfpathlineto{\pgfqpoint{6.323581in}{5.233340in}}%
\pgfpathlineto{\pgfqpoint{6.324118in}{5.215929in}}%
\pgfpathlineto{\pgfqpoint{6.324654in}{5.226868in}}%
\pgfpathlineto{\pgfqpoint{6.325191in}{5.234259in}}%
\pgfpathlineto{\pgfqpoint{6.326800in}{5.205488in}}%
\pgfpathlineto{\pgfqpoint{6.327337in}{5.226156in}}%
\pgfpathlineto{\pgfqpoint{6.327873in}{5.222111in}}%
\pgfpathlineto{\pgfqpoint{6.328409in}{5.224476in}}%
\pgfpathlineto{\pgfqpoint{6.328946in}{5.216869in}}%
\pgfpathlineto{\pgfqpoint{6.329482in}{5.240194in}}%
\pgfpathlineto{\pgfqpoint{6.330019in}{5.234705in}}%
\pgfpathlineto{\pgfqpoint{6.330555in}{5.229904in}}%
\pgfpathlineto{\pgfqpoint{6.331092in}{5.211494in}}%
\pgfpathlineto{\pgfqpoint{6.331628in}{5.223407in}}%
\pgfpathlineto{\pgfqpoint{6.332164in}{5.230103in}}%
\pgfpathlineto{\pgfqpoint{6.333774in}{5.210388in}}%
\pgfpathlineto{\pgfqpoint{6.334847in}{5.237272in}}%
\pgfpathlineto{\pgfqpoint{6.335383in}{5.218450in}}%
\pgfpathlineto{\pgfqpoint{6.335920in}{5.221435in}}%
\pgfpathlineto{\pgfqpoint{6.337529in}{5.229521in}}%
\pgfpathlineto{\pgfqpoint{6.338065in}{5.218060in}}%
\pgfpathlineto{\pgfqpoint{6.338602in}{5.220917in}}%
\pgfpathlineto{\pgfqpoint{6.339675in}{5.228287in}}%
\pgfpathlineto{\pgfqpoint{6.340747in}{5.208671in}}%
\pgfpathlineto{\pgfqpoint{6.342893in}{5.228350in}}%
\pgfpathlineto{\pgfqpoint{6.343430in}{5.227184in}}%
\pgfpathlineto{\pgfqpoint{6.344502in}{5.224363in}}%
\pgfpathlineto{\pgfqpoint{6.345039in}{5.226893in}}%
\pgfpathlineto{\pgfqpoint{6.345575in}{5.214427in}}%
\pgfpathlineto{\pgfqpoint{6.346112in}{5.230004in}}%
\pgfpathlineto{\pgfqpoint{6.346648in}{5.212484in}}%
\pgfpathlineto{\pgfqpoint{6.347185in}{5.214433in}}%
\pgfpathlineto{\pgfqpoint{6.347721in}{5.215790in}}%
\pgfpathlineto{\pgfqpoint{6.348258in}{5.223055in}}%
\pgfpathlineto{\pgfqpoint{6.349867in}{5.211855in}}%
\pgfpathlineto{\pgfqpoint{6.352013in}{5.233536in}}%
\pgfpathlineto{\pgfqpoint{6.353622in}{5.203745in}}%
\pgfpathlineto{\pgfqpoint{6.354158in}{5.217244in}}%
\pgfpathlineto{\pgfqpoint{6.354695in}{5.213854in}}%
\pgfpathlineto{\pgfqpoint{6.355768in}{5.199680in}}%
\pgfpathlineto{\pgfqpoint{6.357377in}{5.226605in}}%
\pgfpathlineto{\pgfqpoint{6.358986in}{5.204436in}}%
\pgfpathlineto{\pgfqpoint{6.359523in}{5.220339in}}%
\pgfpathlineto{\pgfqpoint{6.360059in}{5.198558in}}%
\pgfpathlineto{\pgfqpoint{6.360596in}{5.216863in}}%
\pgfpathlineto{\pgfqpoint{6.361132in}{5.213042in}}%
\pgfpathlineto{\pgfqpoint{6.361668in}{5.221305in}}%
\pgfpathlineto{\pgfqpoint{6.362741in}{5.200550in}}%
\pgfpathlineto{\pgfqpoint{6.363814in}{5.223739in}}%
\pgfpathlineto{\pgfqpoint{6.364887in}{5.195035in}}%
\pgfpathlineto{\pgfqpoint{6.365424in}{5.208029in}}%
\pgfpathlineto{\pgfqpoint{6.366496in}{5.218101in}}%
\pgfpathlineto{\pgfqpoint{6.368106in}{5.196167in}}%
\pgfpathlineto{\pgfqpoint{6.368642in}{5.203207in}}%
\pgfpathlineto{\pgfqpoint{6.369179in}{5.187768in}}%
\pgfpathlineto{\pgfqpoint{6.370788in}{5.215039in}}%
\pgfpathlineto{\pgfqpoint{6.371324in}{5.193775in}}%
\pgfpathlineto{\pgfqpoint{6.371861in}{5.198934in}}%
\pgfpathlineto{\pgfqpoint{6.372397in}{5.197376in}}%
\pgfpathlineto{\pgfqpoint{6.372934in}{5.216662in}}%
\pgfpathlineto{\pgfqpoint{6.373470in}{5.206541in}}%
\pgfpathlineto{\pgfqpoint{6.374543in}{5.184305in}}%
\pgfpathlineto{\pgfqpoint{6.375616in}{5.208189in}}%
\pgfpathlineto{\pgfqpoint{6.376689in}{5.167587in}}%
\pgfpathlineto{\pgfqpoint{6.377762in}{5.198355in}}%
\pgfpathlineto{\pgfqpoint{6.378834in}{5.191423in}}%
\pgfpathlineto{\pgfqpoint{6.379371in}{5.198258in}}%
\pgfpathlineto{\pgfqpoint{6.379907in}{5.195469in}}%
\pgfpathlineto{\pgfqpoint{6.380444in}{5.186725in}}%
\pgfpathlineto{\pgfqpoint{6.380980in}{5.205526in}}%
\pgfpathlineto{\pgfqpoint{6.381517in}{5.193792in}}%
\pgfpathlineto{\pgfqpoint{6.382053in}{5.197981in}}%
\pgfpathlineto{\pgfqpoint{6.382589in}{5.177915in}}%
\pgfpathlineto{\pgfqpoint{6.383126in}{5.193768in}}%
\pgfpathlineto{\pgfqpoint{6.384199in}{5.176336in}}%
\pgfpathlineto{\pgfqpoint{6.385272in}{5.189550in}}%
\pgfpathlineto{\pgfqpoint{6.385808in}{5.175979in}}%
\pgfpathlineto{\pgfqpoint{6.386345in}{5.179679in}}%
\pgfpathlineto{\pgfqpoint{6.387417in}{5.193139in}}%
\pgfpathlineto{\pgfqpoint{6.387954in}{5.163976in}}%
\pgfpathlineto{\pgfqpoint{6.388490in}{5.166824in}}%
\pgfpathlineto{\pgfqpoint{6.389027in}{5.170488in}}%
\pgfpathlineto{\pgfqpoint{6.389563in}{5.189139in}}%
\pgfpathlineto{\pgfqpoint{6.390100in}{5.171275in}}%
\pgfpathlineto{\pgfqpoint{6.391172in}{5.188819in}}%
\pgfpathlineto{\pgfqpoint{6.391709in}{5.171757in}}%
\pgfpathlineto{\pgfqpoint{6.392245in}{5.178499in}}%
\pgfpathlineto{\pgfqpoint{6.392782in}{5.184491in}}%
\pgfpathlineto{\pgfqpoint{6.393318in}{5.181608in}}%
\pgfpathlineto{\pgfqpoint{6.394391in}{5.158096in}}%
\pgfpathlineto{\pgfqpoint{6.394927in}{5.174597in}}%
\pgfpathlineto{\pgfqpoint{6.395464in}{5.161835in}}%
\pgfpathlineto{\pgfqpoint{6.396000in}{5.169816in}}%
\pgfpathlineto{\pgfqpoint{6.396537in}{5.160527in}}%
\pgfpathlineto{\pgfqpoint{6.397073in}{5.161552in}}%
\pgfpathlineto{\pgfqpoint{6.397610in}{5.163145in}}%
\pgfpathlineto{\pgfqpoint{6.398146in}{5.157010in}}%
\pgfpathlineto{\pgfqpoint{6.398683in}{5.180702in}}%
\pgfpathlineto{\pgfqpoint{6.399219in}{5.177261in}}%
\pgfpathlineto{\pgfqpoint{6.400292in}{5.165160in}}%
\pgfpathlineto{\pgfqpoint{6.400828in}{5.182327in}}%
\pgfpathlineto{\pgfqpoint{6.402438in}{5.149247in}}%
\pgfpathlineto{\pgfqpoint{6.402974in}{5.147941in}}%
\pgfpathlineto{\pgfqpoint{6.403510in}{5.136146in}}%
\pgfpathlineto{\pgfqpoint{6.404047in}{5.143769in}}%
\pgfpathlineto{\pgfqpoint{6.404583in}{5.167253in}}%
\pgfpathlineto{\pgfqpoint{6.405120in}{5.159078in}}%
\pgfpathlineto{\pgfqpoint{6.405656in}{5.127078in}}%
\pgfpathlineto{\pgfqpoint{6.406193in}{5.156224in}}%
\pgfpathlineto{\pgfqpoint{6.406729in}{5.158761in}}%
\pgfpathlineto{\pgfqpoint{6.407266in}{5.153745in}}%
\pgfpathlineto{\pgfqpoint{6.407802in}{5.134551in}}%
\pgfpathlineto{\pgfqpoint{6.407802in}{5.134551in}}%
\pgfpathlineto{\pgfqpoint{6.407802in}{5.134551in}}%
\pgfpathlineto{\pgfqpoint{6.408338in}{5.154201in}}%
\pgfpathlineto{\pgfqpoint{6.408875in}{5.149981in}}%
\pgfpathlineto{\pgfqpoint{6.409411in}{5.138987in}}%
\pgfpathlineto{\pgfqpoint{6.409948in}{5.149108in}}%
\pgfpathlineto{\pgfqpoint{6.410484in}{5.152335in}}%
\pgfpathlineto{\pgfqpoint{6.411557in}{5.117425in}}%
\pgfpathlineto{\pgfqpoint{6.412093in}{5.131722in}}%
\pgfpathlineto{\pgfqpoint{6.412630in}{5.153599in}}%
\pgfpathlineto{\pgfqpoint{6.413703in}{5.116969in}}%
\pgfpathlineto{\pgfqpoint{6.414239in}{5.142913in}}%
\pgfpathlineto{\pgfqpoint{6.414776in}{5.118751in}}%
\pgfpathlineto{\pgfqpoint{6.415312in}{5.119602in}}%
\pgfpathlineto{\pgfqpoint{6.416385in}{5.157101in}}%
\pgfpathlineto{\pgfqpoint{6.416921in}{5.108150in}}%
\pgfpathlineto{\pgfqpoint{6.417458in}{5.133599in}}%
\pgfpathlineto{\pgfqpoint{6.417994in}{5.118941in}}%
\pgfpathlineto{\pgfqpoint{6.418531in}{5.154328in}}%
\pgfpathlineto{\pgfqpoint{6.419067in}{5.121426in}}%
\pgfpathlineto{\pgfqpoint{6.419604in}{5.111715in}}%
\pgfpathlineto{\pgfqpoint{6.420140in}{5.144304in}}%
\pgfpathlineto{\pgfqpoint{6.420676in}{5.120448in}}%
\pgfpathlineto{\pgfqpoint{6.421213in}{5.097107in}}%
\pgfpathlineto{\pgfqpoint{6.421749in}{5.130364in}}%
\pgfpathlineto{\pgfqpoint{6.422286in}{5.116926in}}%
\pgfpathlineto{\pgfqpoint{6.423359in}{5.081113in}}%
\pgfpathlineto{\pgfqpoint{6.424431in}{5.124372in}}%
\pgfpathlineto{\pgfqpoint{6.424968in}{5.112111in}}%
\pgfpathlineto{\pgfqpoint{6.425504in}{5.119814in}}%
\pgfpathlineto{\pgfqpoint{6.427114in}{5.108982in}}%
\pgfpathlineto{\pgfqpoint{6.427650in}{5.129354in}}%
\pgfpathlineto{\pgfqpoint{6.427650in}{5.129354in}}%
\pgfpathlineto{\pgfqpoint{6.427650in}{5.129354in}}%
\pgfpathlineto{\pgfqpoint{6.428723in}{5.058340in}}%
\pgfpathlineto{\pgfqpoint{6.429259in}{5.087472in}}%
\pgfpathlineto{\pgfqpoint{6.429796in}{5.129988in}}%
\pgfpathlineto{\pgfqpoint{6.430332in}{5.094592in}}%
\pgfpathlineto{\pgfqpoint{6.430869in}{5.075666in}}%
\pgfpathlineto{\pgfqpoint{6.431405in}{5.083931in}}%
\pgfpathlineto{\pgfqpoint{6.431942in}{5.094196in}}%
\pgfpathlineto{\pgfqpoint{6.432478in}{5.064310in}}%
\pgfpathlineto{\pgfqpoint{6.433014in}{5.082042in}}%
\pgfpathlineto{\pgfqpoint{6.434087in}{5.115215in}}%
\pgfpathlineto{\pgfqpoint{6.434624in}{5.083256in}}%
\pgfpathlineto{\pgfqpoint{6.435160in}{5.092390in}}%
\pgfpathlineto{\pgfqpoint{6.435697in}{5.114802in}}%
\pgfpathlineto{\pgfqpoint{6.436233in}{5.059236in}}%
\pgfpathlineto{\pgfqpoint{6.436770in}{5.084292in}}%
\pgfpathlineto{\pgfqpoint{6.437306in}{5.084661in}}%
\pgfpathlineto{\pgfqpoint{6.437842in}{5.046447in}}%
\pgfpathlineto{\pgfqpoint{6.438379in}{5.052824in}}%
\pgfpathlineto{\pgfqpoint{6.439452in}{5.101183in}}%
\pgfpathlineto{\pgfqpoint{6.440525in}{5.044542in}}%
\pgfpathlineto{\pgfqpoint{6.441061in}{5.091546in}}%
\pgfpathlineto{\pgfqpoint{6.441597in}{5.086920in}}%
\pgfpathlineto{\pgfqpoint{6.442134in}{5.027006in}}%
\pgfpathlineto{\pgfqpoint{6.442670in}{5.052098in}}%
\pgfpathlineto{\pgfqpoint{6.443207in}{5.082669in}}%
\pgfpathlineto{\pgfqpoint{6.443743in}{5.066465in}}%
\pgfpathlineto{\pgfqpoint{6.444816in}{5.087382in}}%
\pgfpathlineto{\pgfqpoint{6.445352in}{5.080673in}}%
\pgfpathlineto{\pgfqpoint{6.445889in}{5.019605in}}%
\pgfpathlineto{\pgfqpoint{6.446425in}{5.037115in}}%
\pgfpathlineto{\pgfqpoint{6.446962in}{5.059785in}}%
\pgfpathlineto{\pgfqpoint{6.446962in}{5.059785in}}%
\pgfpathlineto{\pgfqpoint{6.446962in}{5.059785in}}%
\pgfpathlineto{\pgfqpoint{6.447498in}{5.036112in}}%
\pgfpathlineto{\pgfqpoint{6.448035in}{5.051436in}}%
\pgfpathlineto{\pgfqpoint{6.448571in}{5.052760in}}%
\pgfpathlineto{\pgfqpoint{6.449644in}{5.019805in}}%
\pgfpathlineto{\pgfqpoint{6.450717in}{5.081985in}}%
\pgfpathlineto{\pgfqpoint{6.451790in}{4.977184in}}%
\pgfpathlineto{\pgfqpoint{6.452863in}{5.054111in}}%
\pgfpathlineto{\pgfqpoint{6.453935in}{5.021154in}}%
\pgfpathlineto{\pgfqpoint{6.454472in}{5.064552in}}%
\pgfpathlineto{\pgfqpoint{6.455008in}{5.052252in}}%
\pgfpathlineto{\pgfqpoint{6.455545in}{5.003907in}}%
\pgfpathlineto{\pgfqpoint{6.456618in}{5.006050in}}%
\pgfpathlineto{\pgfqpoint{6.457154in}{5.007601in}}%
\pgfpathlineto{\pgfqpoint{6.457691in}{4.984498in}}%
\pgfpathlineto{\pgfqpoint{6.458227in}{5.035798in}}%
\pgfpathlineto{\pgfqpoint{6.458763in}{5.025966in}}%
\pgfpathlineto{\pgfqpoint{6.459300in}{4.999930in}}%
\pgfpathlineto{\pgfqpoint{6.459836in}{5.047057in}}%
\pgfpathlineto{\pgfqpoint{6.460909in}{4.967914in}}%
\pgfpathlineto{\pgfqpoint{6.462518in}{5.037691in}}%
\pgfpathlineto{\pgfqpoint{6.463055in}{4.935769in}}%
\pgfpathlineto{\pgfqpoint{6.463591in}{5.011298in}}%
\pgfpathlineto{\pgfqpoint{6.464128in}{5.051328in}}%
\pgfpathlineto{\pgfqpoint{6.465201in}{4.957295in}}%
\pgfpathlineto{\pgfqpoint{6.465737in}{4.957371in}}%
\pgfpathlineto{\pgfqpoint{6.466274in}{4.987908in}}%
\pgfpathlineto{\pgfqpoint{6.466810in}{4.935672in}}%
\pgfpathlineto{\pgfqpoint{6.467346in}{4.937597in}}%
\pgfpathlineto{\pgfqpoint{6.467883in}{4.991445in}}%
\pgfpathlineto{\pgfqpoint{6.468419in}{4.976746in}}%
\pgfpathlineto{\pgfqpoint{6.468956in}{4.967874in}}%
\pgfpathlineto{\pgfqpoint{6.469492in}{4.972421in}}%
\pgfpathlineto{\pgfqpoint{6.470029in}{4.974691in}}%
\pgfpathlineto{\pgfqpoint{6.470565in}{4.937627in}}%
\pgfpathlineto{\pgfqpoint{6.470565in}{4.937627in}}%
\pgfpathlineto{\pgfqpoint{6.470565in}{4.937627in}}%
\pgfpathlineto{\pgfqpoint{6.471638in}{5.004771in}}%
\pgfpathlineto{\pgfqpoint{6.472174in}{4.892312in}}%
\pgfpathlineto{\pgfqpoint{6.472711in}{4.983672in}}%
\pgfpathlineto{\pgfqpoint{6.473247in}{5.022863in}}%
\pgfpathlineto{\pgfqpoint{6.473784in}{4.986894in}}%
\pgfpathlineto{\pgfqpoint{6.474856in}{4.883916in}}%
\pgfpathlineto{\pgfqpoint{6.475393in}{4.995546in}}%
\pgfpathlineto{\pgfqpoint{6.475929in}{4.935675in}}%
\pgfpathlineto{\pgfqpoint{6.476466in}{4.831208in}}%
\pgfpathlineto{\pgfqpoint{6.477002in}{4.963727in}}%
\pgfpathlineto{\pgfqpoint{6.477539in}{4.959800in}}%
\pgfpathlineto{\pgfqpoint{6.478075in}{4.908129in}}%
\pgfpathlineto{\pgfqpoint{6.478612in}{4.948624in}}%
\pgfpathlineto{\pgfqpoint{6.479148in}{4.951576in}}%
\pgfpathlineto{\pgfqpoint{6.480221in}{4.915565in}}%
\pgfpathlineto{\pgfqpoint{6.480757in}{4.961479in}}%
\pgfpathlineto{\pgfqpoint{6.481294in}{4.874491in}}%
\pgfpathlineto{\pgfqpoint{6.481830in}{4.926532in}}%
\pgfpathlineto{\pgfqpoint{6.482367in}{4.976679in}}%
\pgfpathlineto{\pgfqpoint{6.482903in}{4.956040in}}%
\pgfpathlineto{\pgfqpoint{6.483976in}{4.851574in}}%
\pgfpathlineto{\pgfqpoint{6.484512in}{4.967445in}}%
\pgfpathlineto{\pgfqpoint{6.485049in}{4.916810in}}%
\pgfpathlineto{\pgfqpoint{6.485585in}{4.735836in}}%
\pgfpathlineto{\pgfqpoint{6.486122in}{4.903063in}}%
\pgfpathlineto{\pgfqpoint{6.486658in}{4.916915in}}%
\pgfpathlineto{\pgfqpoint{6.487731in}{4.862043in}}%
\pgfpathlineto{\pgfqpoint{6.488267in}{4.886334in}}%
\pgfpathlineto{\pgfqpoint{6.488804in}{4.910518in}}%
\pgfpathlineto{\pgfqpoint{6.489340in}{4.883325in}}%
\pgfpathlineto{\pgfqpoint{6.489877in}{4.913409in}}%
\pgfpathlineto{\pgfqpoint{6.490413in}{4.832853in}}%
\pgfpathlineto{\pgfqpoint{6.490950in}{4.886940in}}%
\pgfpathlineto{\pgfqpoint{6.491486in}{4.912120in}}%
\pgfpathlineto{\pgfqpoint{6.493095in}{4.829204in}}%
\pgfpathlineto{\pgfqpoint{6.493632in}{4.924708in}}%
\pgfpathlineto{\pgfqpoint{6.494168in}{4.913938in}}%
\pgfpathlineto{\pgfqpoint{6.494705in}{4.722433in}}%
\pgfpathlineto{\pgfqpoint{6.495241in}{4.749861in}}%
\pgfpathlineto{\pgfqpoint{6.495777in}{4.878456in}}%
\pgfpathlineto{\pgfqpoint{6.496314in}{4.842790in}}%
\pgfpathlineto{\pgfqpoint{6.496850in}{4.750803in}}%
\pgfpathlineto{\pgfqpoint{6.497387in}{4.812208in}}%
\pgfpathlineto{\pgfqpoint{6.497923in}{4.893387in}}%
\pgfpathlineto{\pgfqpoint{6.498460in}{4.860561in}}%
\pgfpathlineto{\pgfqpoint{6.499533in}{4.770876in}}%
\pgfpathlineto{\pgfqpoint{6.500605in}{4.881995in}}%
\pgfpathlineto{\pgfqpoint{6.501678in}{4.776561in}}%
\pgfpathlineto{\pgfqpoint{6.502215in}{4.836989in}}%
\pgfpathlineto{\pgfqpoint{6.502751in}{4.832202in}}%
\pgfpathlineto{\pgfqpoint{6.503288in}{4.888447in}}%
\pgfpathlineto{\pgfqpoint{6.503824in}{4.648968in}}%
\pgfpathlineto{\pgfqpoint{6.504360in}{4.718293in}}%
\pgfpathlineto{\pgfqpoint{6.504897in}{4.880281in}}%
\pgfpathlineto{\pgfqpoint{6.505433in}{4.731743in}}%
\pgfpathlineto{\pgfqpoint{6.505970in}{4.626650in}}%
\pgfpathlineto{\pgfqpoint{6.506506in}{4.667844in}}%
\pgfpathlineto{\pgfqpoint{6.507043in}{4.847469in}}%
\pgfpathlineto{\pgfqpoint{6.507579in}{4.796059in}}%
\pgfpathlineto{\pgfqpoint{6.508116in}{4.702111in}}%
\pgfpathlineto{\pgfqpoint{6.508652in}{4.714329in}}%
\pgfpathlineto{\pgfqpoint{6.509725in}{4.839878in}}%
\pgfpathlineto{\pgfqpoint{6.510261in}{4.631645in}}%
\pgfpathlineto{\pgfqpoint{6.510798in}{4.737821in}}%
\pgfpathlineto{\pgfqpoint{6.511871in}{4.847637in}}%
\pgfpathlineto{\pgfqpoint{6.512407in}{4.798726in}}%
\pgfpathlineto{\pgfqpoint{6.512943in}{4.638696in}}%
\pgfpathlineto{\pgfqpoint{6.513480in}{4.709194in}}%
\pgfpathlineto{\pgfqpoint{6.514016in}{4.749744in}}%
\pgfpathlineto{\pgfqpoint{6.514553in}{4.712885in}}%
\pgfpathlineto{\pgfqpoint{6.515089in}{4.443954in}}%
\pgfpathlineto{\pgfqpoint{6.515626in}{4.679332in}}%
\pgfpathlineto{\pgfqpoint{6.516162in}{4.746360in}}%
\pgfpathlineto{\pgfqpoint{6.516698in}{4.719098in}}%
\pgfpathlineto{\pgfqpoint{6.517235in}{4.494596in}}%
\pgfpathlineto{\pgfqpoint{6.517771in}{4.604604in}}%
\pgfpathlineto{\pgfqpoint{6.518308in}{4.852226in}}%
\pgfpathlineto{\pgfqpoint{6.518844in}{4.706003in}}%
\pgfpathlineto{\pgfqpoint{6.519381in}{4.602211in}}%
\pgfpathlineto{\pgfqpoint{6.519917in}{4.654051in}}%
\pgfpathlineto{\pgfqpoint{6.520454in}{4.823728in}}%
\pgfpathlineto{\pgfqpoint{6.520990in}{4.713875in}}%
\pgfpathlineto{\pgfqpoint{6.521526in}{4.639768in}}%
\pgfpathlineto{\pgfqpoint{6.522063in}{4.665726in}}%
\pgfpathlineto{\pgfqpoint{6.523136in}{4.763469in}}%
\pgfpathlineto{\pgfqpoint{6.524745in}{4.483249in}}%
\pgfpathlineto{\pgfqpoint{6.525281in}{4.800659in}}%
\pgfpathlineto{\pgfqpoint{6.525818in}{4.588405in}}%
\pgfpathlineto{\pgfqpoint{6.526354in}{4.251209in}}%
\pgfpathlineto{\pgfqpoint{6.527964in}{4.678436in}}%
\pgfpathlineto{\pgfqpoint{6.528500in}{4.141776in}}%
\pgfpathlineto{\pgfqpoint{6.529037in}{4.662130in}}%
\pgfpathlineto{\pgfqpoint{6.529573in}{4.777592in}}%
\pgfpathlineto{\pgfqpoint{6.530646in}{4.535933in}}%
\pgfpathlineto{\pgfqpoint{6.531182in}{4.570974in}}%
\pgfpathlineto{\pgfqpoint{6.531719in}{4.734449in}}%
\pgfpathlineto{\pgfqpoint{6.532255in}{4.594818in}}%
\pgfpathlineto{\pgfqpoint{6.532792in}{4.461146in}}%
\pgfpathlineto{\pgfqpoint{6.533328in}{4.484258in}}%
\pgfpathlineto{\pgfqpoint{6.533864in}{4.703876in}}%
\pgfpathlineto{\pgfqpoint{6.534401in}{4.678842in}}%
\pgfpathlineto{\pgfqpoint{6.535474in}{4.262424in}}%
\pgfpathlineto{\pgfqpoint{6.536547in}{4.743069in}}%
\pgfpathlineto{\pgfqpoint{6.537083in}{4.072726in}}%
\pgfpathlineto{\pgfqpoint{6.537620in}{4.357611in}}%
\pgfpathlineto{\pgfqpoint{6.538156in}{4.429301in}}%
\pgfpathlineto{\pgfqpoint{6.538692in}{4.692624in}}%
\pgfpathlineto{\pgfqpoint{6.539765in}{4.022209in}}%
\pgfpathlineto{\pgfqpoint{6.540302in}{4.732817in}}%
\pgfpathlineto{\pgfqpoint{6.540838in}{4.542550in}}%
\pgfpathlineto{\pgfqpoint{6.541375in}{4.583121in}}%
\pgfpathlineto{\pgfqpoint{6.541911in}{3.839068in}}%
\pgfpathlineto{\pgfqpoint{6.542984in}{4.642478in}}%
\pgfpathlineto{\pgfqpoint{6.543520in}{4.512971in}}%
\pgfpathlineto{\pgfqpoint{6.544057in}{4.083787in}}%
\pgfpathlineto{\pgfqpoint{6.544057in}{4.083787in}}%
\pgfpathlineto{\pgfqpoint{6.544057in}{4.083787in}}%
\pgfpathlineto{\pgfqpoint{6.545130in}{4.603482in}}%
\pgfpathlineto{\pgfqpoint{6.546202in}{3.954651in}}%
\pgfpathlineto{\pgfqpoint{6.546739in}{4.010074in}}%
\pgfpathlineto{\pgfqpoint{6.547275in}{4.703836in}}%
\pgfpathlineto{\pgfqpoint{6.547812in}{4.395266in}}%
\pgfpathlineto{\pgfqpoint{6.548348in}{3.625723in}}%
\pgfpathlineto{\pgfqpoint{6.548885in}{4.114457in}}%
\pgfpathlineto{\pgfqpoint{6.549958in}{4.622275in}}%
\pgfpathlineto{\pgfqpoint{6.550494in}{3.590004in}}%
\pgfpathlineto{\pgfqpoint{6.551030in}{4.202381in}}%
\pgfpathlineto{\pgfqpoint{6.552103in}{4.548932in}}%
\pgfpathlineto{\pgfqpoint{6.552640in}{4.182272in}}%
\pgfpathlineto{\pgfqpoint{6.553176in}{4.446994in}}%
\pgfpathlineto{\pgfqpoint{6.553713in}{4.483257in}}%
\pgfpathlineto{\pgfqpoint{6.554249in}{4.433652in}}%
\pgfpathlineto{\pgfqpoint{6.555322in}{4.121890in}}%
\pgfpathlineto{\pgfqpoint{6.555858in}{4.512816in}}%
\pgfpathlineto{\pgfqpoint{6.556395in}{4.393765in}}%
\pgfpathlineto{\pgfqpoint{6.557468in}{3.696279in}}%
\pgfpathlineto{\pgfqpoint{6.558541in}{4.495945in}}%
\pgfpathlineto{\pgfqpoint{6.559613in}{3.733629in}}%
\pgfpathlineto{\pgfqpoint{6.560686in}{4.518383in}}%
\pgfpathlineto{\pgfqpoint{6.561759in}{3.318647in}}%
\pgfpathlineto{\pgfqpoint{6.562832in}{4.427722in}}%
\pgfpathlineto{\pgfqpoint{6.563368in}{4.300362in}}%
\pgfpathlineto{\pgfqpoint{6.563905in}{3.743604in}}%
\pgfpathlineto{\pgfqpoint{6.564441in}{4.388017in}}%
\pgfpathlineto{\pgfqpoint{6.565514in}{4.372943in}}%
\pgfpathlineto{\pgfqpoint{6.566051in}{3.918141in}}%
\pgfpathlineto{\pgfqpoint{6.566587in}{4.360772in}}%
\pgfpathlineto{\pgfqpoint{6.567660in}{4.119501in}}%
\pgfpathlineto{\pgfqpoint{6.568196in}{3.497213in}}%
\pgfpathlineto{\pgfqpoint{6.568733in}{4.020817in}}%
\pgfpathlineto{\pgfqpoint{6.569269in}{4.367887in}}%
\pgfpathlineto{\pgfqpoint{6.570342in}{3.637276in}}%
\pgfpathlineto{\pgfqpoint{6.571415in}{4.358739in}}%
\pgfpathlineto{\pgfqpoint{6.571951in}{4.248870in}}%
\pgfpathlineto{\pgfqpoint{6.572488in}{3.031635in}}%
\pgfpathlineto{\pgfqpoint{6.573024in}{3.841163in}}%
\pgfpathlineto{\pgfqpoint{6.573561in}{4.378425in}}%
\pgfpathlineto{\pgfqpoint{6.574097in}{4.294287in}}%
\pgfpathlineto{\pgfqpoint{6.574634in}{3.347557in}}%
\pgfpathlineto{\pgfqpoint{6.575170in}{4.058488in}}%
\pgfpathlineto{\pgfqpoint{6.576243in}{4.295254in}}%
\pgfpathlineto{\pgfqpoint{6.576779in}{3.493480in}}%
\pgfpathlineto{\pgfqpoint{6.577316in}{4.089134in}}%
\pgfpathlineto{\pgfqpoint{6.577852in}{4.072202in}}%
\pgfpathlineto{\pgfqpoint{6.578389in}{4.215845in}}%
\pgfpathlineto{\pgfqpoint{6.578925in}{3.805918in}}%
\pgfpathlineto{\pgfqpoint{6.579462in}{3.875579in}}%
\pgfpathlineto{\pgfqpoint{6.579998in}{4.135090in}}%
\pgfpathlineto{\pgfqpoint{6.580534in}{3.974023in}}%
\pgfpathlineto{\pgfqpoint{6.581607in}{3.451882in}}%
\pgfpathlineto{\pgfqpoint{6.582680in}{4.132143in}}%
\pgfpathlineto{\pgfqpoint{6.583753in}{3.505757in}}%
\pgfpathlineto{\pgfqpoint{6.584826in}{4.220760in}}%
\pgfpathlineto{\pgfqpoint{6.585362in}{3.403554in}}%
\pgfpathlineto{\pgfqpoint{6.585899in}{3.649576in}}%
\pgfpathlineto{\pgfqpoint{6.586435in}{4.090533in}}%
\pgfpathlineto{\pgfqpoint{6.586972in}{4.064934in}}%
\pgfpathlineto{\pgfqpoint{6.588045in}{3.777312in}}%
\pgfpathlineto{\pgfqpoint{6.589117in}{4.107173in}}%
\pgfpathlineto{\pgfqpoint{6.589654in}{3.673603in}}%
\pgfpathlineto{\pgfqpoint{6.590190in}{3.693437in}}%
\pgfpathlineto{\pgfqpoint{6.590727in}{3.917612in}}%
\pgfpathlineto{\pgfqpoint{6.591263in}{3.823954in}}%
\pgfpathlineto{\pgfqpoint{6.591800in}{3.900748in}}%
\pgfpathlineto{\pgfqpoint{6.592336in}{3.595430in}}%
\pgfpathlineto{\pgfqpoint{6.592872in}{3.889714in}}%
\pgfpathlineto{\pgfqpoint{6.593409in}{3.929915in}}%
\pgfpathlineto{\pgfqpoint{6.594482in}{3.232933in}}%
\pgfpathlineto{\pgfqpoint{6.595555in}{4.001273in}}%
\pgfpathlineto{\pgfqpoint{6.596627in}{3.490213in}}%
\pgfpathlineto{\pgfqpoint{6.597700in}{3.959985in}}%
\pgfpathlineto{\pgfqpoint{6.598773in}{3.408071in}}%
\pgfpathlineto{\pgfqpoint{6.599310in}{3.821799in}}%
\pgfpathlineto{\pgfqpoint{6.599846in}{3.773701in}}%
\pgfpathlineto{\pgfqpoint{6.600919in}{3.561366in}}%
\pgfpathlineto{\pgfqpoint{6.601455in}{3.750023in}}%
\pgfpathlineto{\pgfqpoint{6.601992in}{3.744576in}}%
\pgfpathlineto{\pgfqpoint{6.602528in}{3.709039in}}%
\pgfpathlineto{\pgfqpoint{6.603065in}{3.523247in}}%
\pgfpathlineto{\pgfqpoint{6.603601in}{3.703253in}}%
\pgfpathlineto{\pgfqpoint{6.604138in}{3.657398in}}%
\pgfpathlineto{\pgfqpoint{6.604674in}{3.762728in}}%
\pgfpathlineto{\pgfqpoint{6.605210in}{3.230951in}}%
\pgfpathlineto{\pgfqpoint{6.605747in}{3.695861in}}%
\pgfpathlineto{\pgfqpoint{6.606283in}{3.769159in}}%
\pgfpathlineto{\pgfqpoint{6.607356in}{3.150959in}}%
\pgfpathlineto{\pgfqpoint{6.608429in}{3.775106in}}%
\pgfpathlineto{\pgfqpoint{6.609502in}{3.230963in}}%
\pgfpathlineto{\pgfqpoint{6.610038in}{3.747393in}}%
\pgfpathlineto{\pgfqpoint{6.610575in}{3.690438in}}%
\pgfpathlineto{\pgfqpoint{6.611648in}{3.278926in}}%
\pgfpathlineto{\pgfqpoint{6.612184in}{3.596912in}}%
\pgfpathlineto{\pgfqpoint{6.612721in}{3.511978in}}%
\pgfpathlineto{\pgfqpoint{6.613257in}{3.508336in}}%
\pgfpathlineto{\pgfqpoint{6.613793in}{3.328189in}}%
\pgfpathlineto{\pgfqpoint{6.614330in}{3.478164in}}%
\pgfpathlineto{\pgfqpoint{6.614866in}{3.530921in}}%
\pgfpathlineto{\pgfqpoint{6.616476in}{3.340706in}}%
\pgfpathlineto{\pgfqpoint{6.617012in}{3.560366in}}%
\pgfpathlineto{\pgfqpoint{6.617548in}{3.498788in}}%
\pgfpathlineto{\pgfqpoint{6.618085in}{2.949134in}}%
\pgfpathlineto{\pgfqpoint{6.618621in}{3.394494in}}%
\pgfpathlineto{\pgfqpoint{6.619158in}{3.659397in}}%
\pgfpathlineto{\pgfqpoint{6.620231in}{3.169608in}}%
\pgfpathlineto{\pgfqpoint{6.620767in}{3.465129in}}%
\pgfpathlineto{\pgfqpoint{6.621304in}{3.439507in}}%
\pgfpathlineto{\pgfqpoint{6.622376in}{3.031046in}}%
\pgfpathlineto{\pgfqpoint{6.622913in}{3.429820in}}%
\pgfpathlineto{\pgfqpoint{6.623449in}{3.324781in}}%
\pgfpathlineto{\pgfqpoint{6.624522in}{3.194198in}}%
\pgfpathlineto{\pgfqpoint{6.625059in}{3.421236in}}%
\pgfpathlineto{\pgfqpoint{6.625595in}{3.163072in}}%
\pgfpathlineto{\pgfqpoint{6.626131in}{3.307226in}}%
\pgfpathlineto{\pgfqpoint{6.626668in}{3.280733in}}%
\pgfpathlineto{\pgfqpoint{6.627204in}{3.125108in}}%
\pgfpathlineto{\pgfqpoint{6.627741in}{3.354025in}}%
\pgfpathlineto{\pgfqpoint{6.628277in}{3.281476in}}%
\pgfpathlineto{\pgfqpoint{6.628814in}{3.152366in}}%
\pgfpathlineto{\pgfqpoint{6.629350in}{3.189806in}}%
\pgfpathlineto{\pgfqpoint{6.629887in}{3.277534in}}%
\pgfpathlineto{\pgfqpoint{6.630959in}{2.933188in}}%
\pgfpathlineto{\pgfqpoint{6.632032in}{3.296439in}}%
\pgfpathlineto{\pgfqpoint{6.633105in}{2.808966in}}%
\pgfpathlineto{\pgfqpoint{6.633642in}{3.247485in}}%
\pgfpathlineto{\pgfqpoint{6.634178in}{3.245467in}}%
\pgfpathlineto{\pgfqpoint{6.634714in}{2.842266in}}%
\pgfpathlineto{\pgfqpoint{6.635251in}{3.057839in}}%
\pgfpathlineto{\pgfqpoint{6.635787in}{3.205280in}}%
\pgfpathlineto{\pgfqpoint{6.636860in}{3.014546in}}%
\pgfpathlineto{\pgfqpoint{6.637397in}{3.140434in}}%
\pgfpathlineto{\pgfqpoint{6.637933in}{3.137253in}}%
\pgfpathlineto{\pgfqpoint{6.639006in}{2.946406in}}%
\pgfpathlineto{\pgfqpoint{6.639542in}{2.958470in}}%
\pgfpathlineto{\pgfqpoint{6.640079in}{2.926129in}}%
\pgfpathlineto{\pgfqpoint{6.640615in}{2.985622in}}%
\pgfpathlineto{\pgfqpoint{6.641152in}{2.963675in}}%
\pgfpathlineto{\pgfqpoint{6.641688in}{2.695874in}}%
\pgfpathlineto{\pgfqpoint{6.642225in}{2.928662in}}%
\pgfpathlineto{\pgfqpoint{6.642761in}{3.076517in}}%
\pgfpathlineto{\pgfqpoint{6.643834in}{2.753062in}}%
\pgfpathlineto{\pgfqpoint{6.644907in}{2.991077in}}%
\pgfpathlineto{\pgfqpoint{6.645980in}{2.701300in}}%
\pgfpathlineto{\pgfqpoint{6.647052in}{3.009767in}}%
\pgfpathlineto{\pgfqpoint{6.647589in}{2.616869in}}%
\pgfpathlineto{\pgfqpoint{6.648125in}{2.819676in}}%
\pgfpathlineto{\pgfqpoint{6.648662in}{2.941848in}}%
\pgfpathlineto{\pgfqpoint{6.649198in}{2.566231in}}%
\pgfpathlineto{\pgfqpoint{6.649735in}{2.652787in}}%
\pgfpathlineto{\pgfqpoint{6.650271in}{2.865681in}}%
\pgfpathlineto{\pgfqpoint{6.650808in}{2.637322in}}%
\pgfpathlineto{\pgfqpoint{6.651344in}{2.657414in}}%
\pgfpathlineto{\pgfqpoint{6.651880in}{2.767523in}}%
\pgfpathlineto{\pgfqpoint{6.652417in}{2.767084in}}%
\pgfpathlineto{\pgfqpoint{6.652953in}{2.666789in}}%
\pgfpathlineto{\pgfqpoint{6.653490in}{2.789647in}}%
\pgfpathlineto{\pgfqpoint{6.654026in}{2.669785in}}%
\pgfpathlineto{\pgfqpoint{6.654563in}{2.617828in}}%
\pgfpathlineto{\pgfqpoint{6.655099in}{2.626573in}}%
\pgfpathlineto{\pgfqpoint{6.655635in}{2.728114in}}%
\pgfpathlineto{\pgfqpoint{6.656708in}{2.495830in}}%
\pgfpathlineto{\pgfqpoint{6.657781in}{2.640258in}}%
\pgfpathlineto{\pgfqpoint{6.658318in}{2.424451in}}%
\pgfpathlineto{\pgfqpoint{6.658854in}{2.473901in}}%
\pgfpathlineto{\pgfqpoint{6.659391in}{2.635374in}}%
\pgfpathlineto{\pgfqpoint{6.659927in}{2.494868in}}%
\pgfpathlineto{\pgfqpoint{6.660463in}{2.444671in}}%
\pgfpathlineto{\pgfqpoint{6.661000in}{2.698502in}}%
\pgfpathlineto{\pgfqpoint{6.661536in}{2.602809in}}%
\pgfpathlineto{\pgfqpoint{6.662073in}{2.292411in}}%
\pgfpathlineto{\pgfqpoint{6.662609in}{2.397569in}}%
\pgfpathlineto{\pgfqpoint{6.663146in}{2.602946in}}%
\pgfpathlineto{\pgfqpoint{6.663682in}{2.399762in}}%
\pgfpathlineto{\pgfqpoint{6.664218in}{2.128165in}}%
\pgfpathlineto{\pgfqpoint{6.664218in}{2.128165in}}%
\pgfpathlineto{\pgfqpoint{6.664218in}{2.128165in}}%
\pgfpathlineto{\pgfqpoint{6.665291in}{2.451305in}}%
\pgfpathlineto{\pgfqpoint{6.665828in}{2.318381in}}%
\pgfpathlineto{\pgfqpoint{6.666364in}{2.438073in}}%
\pgfpathlineto{\pgfqpoint{6.667437in}{2.165058in}}%
\pgfpathlineto{\pgfqpoint{6.667973in}{2.395812in}}%
\pgfpathlineto{\pgfqpoint{6.668510in}{2.392021in}}%
\pgfpathlineto{\pgfqpoint{6.669583in}{2.177788in}}%
\pgfpathlineto{\pgfqpoint{6.670656in}{2.394941in}}%
\pgfpathlineto{\pgfqpoint{6.671192in}{2.143552in}}%
\pgfpathlineto{\pgfqpoint{6.671729in}{2.248677in}}%
\pgfpathlineto{\pgfqpoint{6.672265in}{2.200982in}}%
\pgfpathlineto{\pgfqpoint{6.672801in}{2.035743in}}%
\pgfpathlineto{\pgfqpoint{6.673338in}{2.084940in}}%
\pgfpathlineto{\pgfqpoint{6.673874in}{2.353232in}}%
\pgfpathlineto{\pgfqpoint{6.674947in}{2.003697in}}%
\pgfpathlineto{\pgfqpoint{6.676020in}{2.308032in}}%
\pgfpathlineto{\pgfqpoint{6.677093in}{1.868473in}}%
\pgfpathlineto{\pgfqpoint{6.678166in}{2.157408in}}%
\pgfpathlineto{\pgfqpoint{6.678702in}{1.964448in}}%
\pgfpathlineto{\pgfqpoint{6.679239in}{2.143029in}}%
\pgfpathlineto{\pgfqpoint{6.680848in}{1.920484in}}%
\pgfpathlineto{\pgfqpoint{6.681384in}{2.007255in}}%
\pgfpathlineto{\pgfqpoint{6.682457in}{1.637042in}}%
\pgfpathlineto{\pgfqpoint{6.682994in}{2.066278in}}%
\pgfpathlineto{\pgfqpoint{6.683530in}{1.896833in}}%
\pgfpathlineto{\pgfqpoint{6.684067in}{1.773902in}}%
\pgfpathlineto{\pgfqpoint{6.684603in}{1.975023in}}%
\pgfpathlineto{\pgfqpoint{6.685139in}{1.883098in}}%
\pgfpathlineto{\pgfqpoint{6.685676in}{1.694587in}}%
\pgfpathlineto{\pgfqpoint{6.686212in}{1.844742in}}%
\pgfpathlineto{\pgfqpoint{6.686749in}{1.885467in}}%
\pgfpathlineto{\pgfqpoint{6.687822in}{1.557922in}}%
\pgfpathlineto{\pgfqpoint{6.688358in}{1.918469in}}%
\pgfpathlineto{\pgfqpoint{6.688895in}{1.884843in}}%
\pgfpathlineto{\pgfqpoint{6.689431in}{1.426004in}}%
\pgfpathlineto{\pgfqpoint{6.689967in}{1.466493in}}%
\pgfpathlineto{\pgfqpoint{6.691040in}{1.648687in}}%
\pgfpathlineto{\pgfqpoint{6.691577in}{1.488340in}}%
\pgfpathlineto{\pgfqpoint{6.692113in}{1.736702in}}%
\pgfpathlineto{\pgfqpoint{6.692650in}{1.495949in}}%
\pgfpathlineto{\pgfqpoint{6.694259in}{1.736995in}}%
\pgfpathlineto{\pgfqpoint{6.694795in}{1.316734in}}%
\pgfpathlineto{\pgfqpoint{6.695332in}{1.413939in}}%
\pgfpathlineto{\pgfqpoint{6.696405in}{1.562300in}}%
\pgfpathlineto{\pgfqpoint{6.696941in}{1.089668in}}%
\pgfpathlineto{\pgfqpoint{6.697477in}{1.447382in}}%
\pgfpathlineto{\pgfqpoint{6.698014in}{1.491172in}}%
\pgfpathlineto{\pgfqpoint{6.699087in}{1.322633in}}%
\pgfpathlineto{\pgfqpoint{6.699623in}{1.392281in}}%
\pgfpathlineto{\pgfqpoint{6.700160in}{1.045768in}}%
\pgfpathlineto{\pgfqpoint{6.701233in}{1.563148in}}%
\pgfpathlineto{\pgfqpoint{6.702305in}{0.934739in}}%
\pgfpathlineto{\pgfqpoint{6.702842in}{1.150204in}}%
\pgfpathlineto{\pgfqpoint{6.703378in}{1.389301in}}%
\pgfpathlineto{\pgfqpoint{6.703915in}{1.171522in}}%
\pgfpathlineto{\pgfqpoint{6.704451in}{0.599744in}}%
\pgfpathlineto{\pgfqpoint{6.704988in}{1.074985in}}%
\pgfpathlineto{\pgfqpoint{6.705524in}{0.972251in}}%
\pgfpathlineto{\pgfqpoint{6.706060in}{0.985398in}}%
\pgfpathlineto{\pgfqpoint{6.706597in}{1.355458in}}%
\pgfpathlineto{\pgfqpoint{6.707133in}{1.106382in}}%
\pgfpathlineto{\pgfqpoint{6.708206in}{0.939103in}}%
\pgfpathlineto{\pgfqpoint{6.708743in}{1.107437in}}%
\pgfpathlineto{\pgfqpoint{6.709279in}{0.943520in}}%
\pgfpathlineto{\pgfqpoint{6.709816in}{0.640145in}}%
\pgfpathlineto{\pgfqpoint{6.710352in}{0.999774in}}%
\pgfpathlineto{\pgfqpoint{6.710888in}{0.973845in}}%
\pgfpathlineto{\pgfqpoint{6.712498in}{0.725155in}}%
\pgfpathlineto{\pgfqpoint{6.714107in}{0.850144in}}%
\pgfpathlineto{\pgfqpoint{6.714684in}{0.590000in}}%
\pgfpathmoveto{\pgfqpoint{6.715388in}{0.590000in}}%
\pgfpathlineto{\pgfqpoint{6.715716in}{0.869511in}}%
\pgfpathlineto{\pgfqpoint{6.715716in}{0.869511in}}%
\pgfpathlineto{\pgfqpoint{6.715716in}{0.869511in}}%
\pgfpathlineto{\pgfqpoint{6.716253in}{0.838697in}}%
\pgfpathlineto{\pgfqpoint{6.716721in}{0.590000in}}%
\pgfpathmoveto{\pgfqpoint{6.719346in}{0.590000in}}%
\pgfpathlineto{\pgfqpoint{6.719471in}{0.659736in}}%
\pgfpathlineto{\pgfqpoint{6.719471in}{0.659736in}}%
\pgfpathlineto{\pgfqpoint{6.719471in}{0.659736in}}%
\pgfpathlineto{\pgfqpoint{6.719668in}{0.590000in}}%
\pgfpathmoveto{\pgfqpoint{6.830304in}{0.590000in}}%
\pgfpathlineto{\pgfqpoint{6.830514in}{0.717432in}}%
\pgfpathlineto{\pgfqpoint{6.830514in}{0.717432in}}%
\pgfpathlineto{\pgfqpoint{6.830514in}{0.717432in}}%
\pgfpathlineto{\pgfqpoint{6.830727in}{0.590000in}}%
\pgfpathmoveto{\pgfqpoint{6.834113in}{0.590000in}}%
\pgfpathlineto{\pgfqpoint{6.834269in}{0.728224in}}%
\pgfpathlineto{\pgfqpoint{6.834805in}{0.802241in}}%
\pgfpathlineto{\pgfqpoint{6.834805in}{0.802241in}}%
\pgfpathlineto{\pgfqpoint{6.835145in}{0.590000in}}%
\pgfpathmoveto{\pgfqpoint{6.835583in}{0.590000in}}%
\pgfpathlineto{\pgfqpoint{6.835878in}{0.739550in}}%
\pgfpathlineto{\pgfqpoint{6.835878in}{0.739550in}}%
\pgfpathlineto{\pgfqpoint{6.835878in}{0.739550in}}%
\pgfpathlineto{\pgfqpoint{6.836156in}{0.590000in}}%
\pgfpathmoveto{\pgfqpoint{6.837128in}{0.590000in}}%
\pgfpathlineto{\pgfqpoint{6.837487in}{0.711970in}}%
\pgfpathlineto{\pgfqpoint{6.837487in}{0.711970in}}%
\pgfpathlineto{\pgfqpoint{6.837487in}{0.711970in}}%
\pgfpathlineto{\pgfqpoint{6.838024in}{0.696937in}}%
\pgfpathlineto{\pgfqpoint{6.838560in}{0.698159in}}%
\pgfpathlineto{\pgfqpoint{6.839097in}{0.725816in}}%
\pgfpathlineto{\pgfqpoint{6.839633in}{1.042189in}}%
\pgfpathlineto{\pgfqpoint{6.840190in}{0.590000in}}%
\pgfpathmoveto{\pgfqpoint{6.841043in}{0.590000in}}%
\pgfpathlineto{\pgfqpoint{6.841242in}{0.906772in}}%
\pgfpathlineto{\pgfqpoint{6.841242in}{0.906772in}}%
\pgfpathlineto{\pgfqpoint{6.841242in}{0.906772in}}%
\pgfpathlineto{\pgfqpoint{6.842572in}{0.590000in}}%
\pgfpathmoveto{\pgfqpoint{6.842911in}{0.590000in}}%
\pgfpathlineto{\pgfqpoint{6.843388in}{0.903650in}}%
\pgfpathlineto{\pgfqpoint{6.843925in}{0.994246in}}%
\pgfpathlineto{\pgfqpoint{6.844461in}{0.844349in}}%
\pgfpathlineto{\pgfqpoint{6.844997in}{1.019736in}}%
\pgfpathlineto{\pgfqpoint{6.845534in}{0.648593in}}%
\pgfpathlineto{\pgfqpoint{6.846070in}{0.727326in}}%
\pgfpathlineto{\pgfqpoint{6.846607in}{1.012759in}}%
\pgfpathlineto{\pgfqpoint{6.847143in}{0.947204in}}%
\pgfpathlineto{\pgfqpoint{6.847680in}{0.714968in}}%
\pgfpathlineto{\pgfqpoint{6.848216in}{0.825878in}}%
\pgfpathlineto{\pgfqpoint{6.848752in}{1.067899in}}%
\pgfpathlineto{\pgfqpoint{6.849289in}{0.948541in}}%
\pgfpathlineto{\pgfqpoint{6.849485in}{0.590000in}}%
\pgfpathmoveto{\pgfqpoint{6.850124in}{0.590000in}}%
\pgfpathlineto{\pgfqpoint{6.850362in}{1.084013in}}%
\pgfpathlineto{\pgfqpoint{6.850362in}{1.084013in}}%
\pgfpathlineto{\pgfqpoint{6.850362in}{1.084013in}}%
\pgfpathlineto{\pgfqpoint{6.852508in}{0.865799in}}%
\pgfpathlineto{\pgfqpoint{6.854117in}{1.230710in}}%
\pgfpathlineto{\pgfqpoint{6.855190in}{0.835182in}}%
\pgfpathlineto{\pgfqpoint{6.855726in}{1.290187in}}%
\pgfpathlineto{\pgfqpoint{6.856263in}{1.111943in}}%
\pgfpathlineto{\pgfqpoint{6.857335in}{0.635874in}}%
\pgfpathlineto{\pgfqpoint{6.857872in}{1.250298in}}%
\pgfpathlineto{\pgfqpoint{6.858408in}{1.223570in}}%
\pgfpathlineto{\pgfqpoint{6.858945in}{0.707928in}}%
\pgfpathlineto{\pgfqpoint{6.859481in}{1.347264in}}%
\pgfpathlineto{\pgfqpoint{6.860018in}{1.038074in}}%
\pgfpathlineto{\pgfqpoint{6.861091in}{1.163136in}}%
\pgfpathlineto{\pgfqpoint{6.861627in}{1.303717in}}%
\pgfpathlineto{\pgfqpoint{6.862163in}{1.100771in}}%
\pgfpathlineto{\pgfqpoint{6.862700in}{1.296975in}}%
\pgfpathlineto{\pgfqpoint{6.863236in}{1.466745in}}%
\pgfpathlineto{\pgfqpoint{6.864309in}{0.769138in}}%
\pgfpathlineto{\pgfqpoint{6.864846in}{1.440499in}}%
\pgfpathlineto{\pgfqpoint{6.865382in}{1.307502in}}%
\pgfpathlineto{\pgfqpoint{6.866455in}{0.893759in}}%
\pgfpathlineto{\pgfqpoint{6.867528in}{1.434551in}}%
\pgfpathlineto{\pgfqpoint{6.868064in}{1.047542in}}%
\pgfpathlineto{\pgfqpoint{6.868064in}{1.047542in}}%
\pgfpathlineto{\pgfqpoint{6.868064in}{1.047542in}}%
\pgfpathlineto{\pgfqpoint{6.868601in}{1.441989in}}%
\pgfpathlineto{\pgfqpoint{6.869137in}{1.073888in}}%
\pgfpathlineto{\pgfqpoint{6.869673in}{1.084764in}}%
\pgfpathlineto{\pgfqpoint{6.870746in}{1.430048in}}%
\pgfpathlineto{\pgfqpoint{6.871283in}{1.289321in}}%
\pgfpathlineto{\pgfqpoint{6.872356in}{1.595792in}}%
\pgfpathlineto{\pgfqpoint{6.872892in}{1.353712in}}%
\pgfpathlineto{\pgfqpoint{6.873427in}{0.590000in}}%
\pgfpathmoveto{\pgfqpoint{6.873430in}{0.590000in}}%
\pgfpathlineto{\pgfqpoint{6.873965in}{1.559670in}}%
\pgfpathlineto{\pgfqpoint{6.873965in}{1.559670in}}%
\pgfpathlineto{\pgfqpoint{6.873965in}{1.559670in}}%
\pgfpathlineto{\pgfqpoint{6.875574in}{1.163512in}}%
\pgfpathlineto{\pgfqpoint{6.876111in}{1.526298in}}%
\pgfpathlineto{\pgfqpoint{6.876647in}{1.471950in}}%
\pgfpathlineto{\pgfqpoint{6.877184in}{1.251677in}}%
\pgfpathlineto{\pgfqpoint{6.877720in}{1.513788in}}%
\pgfpathlineto{\pgfqpoint{6.878256in}{1.338791in}}%
\pgfpathlineto{\pgfqpoint{6.878793in}{1.318505in}}%
\pgfpathlineto{\pgfqpoint{6.879329in}{1.711403in}}%
\pgfpathlineto{\pgfqpoint{6.879866in}{1.630622in}}%
\pgfpathlineto{\pgfqpoint{6.880402in}{1.151303in}}%
\pgfpathlineto{\pgfqpoint{6.880939in}{1.450755in}}%
\pgfpathlineto{\pgfqpoint{6.881475in}{1.682061in}}%
\pgfpathlineto{\pgfqpoint{6.882012in}{1.672528in}}%
\pgfpathlineto{\pgfqpoint{6.882548in}{1.021575in}}%
\pgfpathlineto{\pgfqpoint{6.883084in}{1.726078in}}%
\pgfpathlineto{\pgfqpoint{6.883621in}{1.604627in}}%
\pgfpathlineto{\pgfqpoint{6.884694in}{1.268843in}}%
\pgfpathlineto{\pgfqpoint{6.885767in}{1.667560in}}%
\pgfpathlineto{\pgfqpoint{6.886303in}{1.542514in}}%
\pgfpathlineto{\pgfqpoint{6.886839in}{1.668511in}}%
\pgfpathlineto{\pgfqpoint{6.887376in}{1.652179in}}%
\pgfpathlineto{\pgfqpoint{6.887912in}{1.379269in}}%
\pgfpathlineto{\pgfqpoint{6.888449in}{1.840508in}}%
\pgfpathlineto{\pgfqpoint{6.888985in}{1.665203in}}%
\pgfpathlineto{\pgfqpoint{6.889522in}{1.371413in}}%
\pgfpathlineto{\pgfqpoint{6.890058in}{1.385347in}}%
\pgfpathlineto{\pgfqpoint{6.890594in}{1.788400in}}%
\pgfpathlineto{\pgfqpoint{6.891131in}{1.788218in}}%
\pgfpathlineto{\pgfqpoint{6.891667in}{1.002229in}}%
\pgfpathlineto{\pgfqpoint{6.891667in}{1.002229in}}%
\pgfpathlineto{\pgfqpoint{6.891667in}{1.002229in}}%
\pgfpathlineto{\pgfqpoint{6.892204in}{1.819085in}}%
\pgfpathlineto{\pgfqpoint{6.892740in}{1.567504in}}%
\pgfpathlineto{\pgfqpoint{6.893277in}{1.534058in}}%
\pgfpathlineto{\pgfqpoint{6.893813in}{1.355422in}}%
\pgfpathlineto{\pgfqpoint{6.894350in}{1.452592in}}%
\pgfpathlineto{\pgfqpoint{6.895959in}{1.807819in}}%
\pgfpathlineto{\pgfqpoint{6.897032in}{1.632024in}}%
\pgfpathlineto{\pgfqpoint{6.897568in}{1.865594in}}%
\pgfpathlineto{\pgfqpoint{6.897568in}{1.865594in}}%
\pgfpathlineto{\pgfqpoint{6.897568in}{1.865594in}}%
\pgfpathlineto{\pgfqpoint{6.899177in}{1.449716in}}%
\pgfpathlineto{\pgfqpoint{6.899714in}{1.917291in}}%
\pgfpathlineto{\pgfqpoint{6.900250in}{1.800752in}}%
\pgfpathlineto{\pgfqpoint{6.900787in}{1.352102in}}%
\pgfpathlineto{\pgfqpoint{6.900787in}{1.352102in}}%
\pgfpathlineto{\pgfqpoint{6.900787in}{1.352102in}}%
\pgfpathlineto{\pgfqpoint{6.901323in}{1.831325in}}%
\pgfpathlineto{\pgfqpoint{6.901860in}{1.589102in}}%
\pgfpathlineto{\pgfqpoint{6.902396in}{1.643945in}}%
\pgfpathlineto{\pgfqpoint{6.902933in}{1.627015in}}%
\pgfpathlineto{\pgfqpoint{6.903469in}{1.547951in}}%
\pgfpathlineto{\pgfqpoint{6.904005in}{1.589932in}}%
\pgfpathlineto{\pgfqpoint{6.904542in}{1.869364in}}%
\pgfpathlineto{\pgfqpoint{6.905078in}{1.782454in}}%
\pgfpathlineto{\pgfqpoint{6.905615in}{1.607134in}}%
\pgfpathlineto{\pgfqpoint{6.906151in}{1.628900in}}%
\pgfpathlineto{\pgfqpoint{6.906688in}{1.991104in}}%
\pgfpathlineto{\pgfqpoint{6.907224in}{1.678283in}}%
\pgfpathlineto{\pgfqpoint{6.907760in}{1.711364in}}%
\pgfpathlineto{\pgfqpoint{6.908297in}{1.681222in}}%
\pgfpathlineto{\pgfqpoint{6.908833in}{2.008505in}}%
\pgfpathlineto{\pgfqpoint{6.909370in}{1.744585in}}%
\pgfpathlineto{\pgfqpoint{6.909906in}{1.737305in}}%
\pgfpathlineto{\pgfqpoint{6.910443in}{1.925642in}}%
\pgfpathlineto{\pgfqpoint{6.910979in}{1.799291in}}%
\pgfpathlineto{\pgfqpoint{6.911516in}{1.735424in}}%
\pgfpathlineto{\pgfqpoint{6.912052in}{1.931162in}}%
\pgfpathlineto{\pgfqpoint{6.912588in}{1.557259in}}%
\pgfpathlineto{\pgfqpoint{6.913125in}{1.580936in}}%
\pgfpathlineto{\pgfqpoint{6.913661in}{2.116678in}}%
\pgfpathlineto{\pgfqpoint{6.914198in}{1.826955in}}%
\pgfpathlineto{\pgfqpoint{6.914734in}{1.720342in}}%
\pgfpathlineto{\pgfqpoint{6.915271in}{1.725756in}}%
\pgfpathlineto{\pgfqpoint{6.915807in}{2.123965in}}%
\pgfpathlineto{\pgfqpoint{6.916343in}{1.791334in}}%
\pgfpathlineto{\pgfqpoint{6.916880in}{1.594405in}}%
\pgfpathlineto{\pgfqpoint{6.917953in}{2.065727in}}%
\pgfpathlineto{\pgfqpoint{6.919026in}{1.629867in}}%
\pgfpathlineto{\pgfqpoint{6.919562in}{2.029660in}}%
\pgfpathlineto{\pgfqpoint{6.920098in}{1.884080in}}%
\pgfpathlineto{\pgfqpoint{6.920635in}{1.693831in}}%
\pgfpathlineto{\pgfqpoint{6.921171in}{2.114805in}}%
\pgfpathlineto{\pgfqpoint{6.921708in}{1.332621in}}%
\pgfpathlineto{\pgfqpoint{6.922244in}{1.731652in}}%
\pgfpathlineto{\pgfqpoint{6.922781in}{2.069973in}}%
\pgfpathlineto{\pgfqpoint{6.923317in}{1.908318in}}%
\pgfpathlineto{\pgfqpoint{6.924390in}{1.439898in}}%
\pgfpathlineto{\pgfqpoint{6.924926in}{2.162290in}}%
\pgfpathlineto{\pgfqpoint{6.925463in}{1.771888in}}%
\pgfpathlineto{\pgfqpoint{6.925999in}{1.565362in}}%
\pgfpathlineto{\pgfqpoint{6.927072in}{2.211909in}}%
\pgfpathlineto{\pgfqpoint{6.927609in}{1.892634in}}%
\pgfpathlineto{\pgfqpoint{6.928145in}{1.577575in}}%
\pgfpathlineto{\pgfqpoint{6.929218in}{2.062990in}}%
\pgfpathlineto{\pgfqpoint{6.931364in}{1.710806in}}%
\pgfpathlineto{\pgfqpoint{6.931900in}{2.079730in}}%
\pgfpathlineto{\pgfqpoint{6.932437in}{1.882727in}}%
\pgfpathlineto{\pgfqpoint{6.932973in}{1.606720in}}%
\pgfpathlineto{\pgfqpoint{6.933509in}{1.810006in}}%
\pgfpathlineto{\pgfqpoint{6.934046in}{2.087907in}}%
\pgfpathlineto{\pgfqpoint{6.934582in}{1.953307in}}%
\pgfpathlineto{\pgfqpoint{6.935119in}{1.503562in}}%
\pgfpathlineto{\pgfqpoint{6.935655in}{1.952389in}}%
\pgfpathlineto{\pgfqpoint{6.936192in}{2.185932in}}%
\pgfpathlineto{\pgfqpoint{6.936728in}{1.925384in}}%
\pgfpathlineto{\pgfqpoint{6.937264in}{2.028050in}}%
\pgfpathlineto{\pgfqpoint{6.937801in}{1.965909in}}%
\pgfpathlineto{\pgfqpoint{6.938337in}{2.178015in}}%
\pgfpathlineto{\pgfqpoint{6.938874in}{1.982302in}}%
\pgfpathlineto{\pgfqpoint{6.939410in}{2.122945in}}%
\pgfpathlineto{\pgfqpoint{6.940483in}{1.651907in}}%
\pgfpathlineto{\pgfqpoint{6.941019in}{2.310918in}}%
\pgfpathlineto{\pgfqpoint{6.941556in}{1.798763in}}%
\pgfpathlineto{\pgfqpoint{6.942092in}{1.460979in}}%
\pgfpathlineto{\pgfqpoint{6.942629in}{1.756548in}}%
\pgfpathlineto{\pgfqpoint{6.943165in}{2.204152in}}%
\pgfpathlineto{\pgfqpoint{6.943702in}{1.766532in}}%
\pgfpathlineto{\pgfqpoint{6.944238in}{1.286717in}}%
\pgfpathlineto{\pgfqpoint{6.944775in}{2.193652in}}%
\pgfpathlineto{\pgfqpoint{6.945311in}{2.118499in}}%
\pgfpathlineto{\pgfqpoint{6.945847in}{1.879523in}}%
\pgfpathlineto{\pgfqpoint{6.946384in}{1.918299in}}%
\pgfpathlineto{\pgfqpoint{6.947457in}{2.124179in}}%
\pgfpathlineto{\pgfqpoint{6.947993in}{2.045311in}}%
\pgfpathlineto{\pgfqpoint{6.948530in}{2.055846in}}%
\pgfpathlineto{\pgfqpoint{6.949602in}{1.809025in}}%
\pgfpathlineto{\pgfqpoint{6.950139in}{2.136012in}}%
\pgfpathlineto{\pgfqpoint{6.950675in}{1.928437in}}%
\pgfpathlineto{\pgfqpoint{6.951212in}{1.631664in}}%
\pgfpathlineto{\pgfqpoint{6.952285in}{2.176823in}}%
\pgfpathlineto{\pgfqpoint{6.952821in}{1.511074in}}%
\pgfpathlineto{\pgfqpoint{6.953358in}{1.677155in}}%
\pgfpathlineto{\pgfqpoint{6.954430in}{2.141330in}}%
\pgfpathlineto{\pgfqpoint{6.954967in}{1.733084in}}%
\pgfpathlineto{\pgfqpoint{6.955503in}{1.890679in}}%
\pgfpathlineto{\pgfqpoint{6.956576in}{2.219371in}}%
\pgfpathlineto{\pgfqpoint{6.957113in}{2.038564in}}%
\pgfpathlineto{\pgfqpoint{6.957649in}{1.607576in}}%
\pgfpathlineto{\pgfqpoint{6.959258in}{2.185470in}}%
\pgfpathlineto{\pgfqpoint{6.959795in}{1.820999in}}%
\pgfpathlineto{\pgfqpoint{6.960331in}{1.982387in}}%
\pgfpathlineto{\pgfqpoint{6.961404in}{2.162535in}}%
\pgfpathlineto{\pgfqpoint{6.961941in}{1.499087in}}%
\pgfpathlineto{\pgfqpoint{6.962477in}{1.955547in}}%
\pgfpathlineto{\pgfqpoint{6.963013in}{2.254953in}}%
\pgfpathlineto{\pgfqpoint{6.964086in}{1.670526in}}%
\pgfpathlineto{\pgfqpoint{6.964623in}{1.729365in}}%
\pgfpathlineto{\pgfqpoint{6.965696in}{2.187665in}}%
\pgfpathlineto{\pgfqpoint{6.966768in}{1.612536in}}%
\pgfpathlineto{\pgfqpoint{6.968378in}{2.187711in}}%
\pgfpathlineto{\pgfqpoint{6.968914in}{1.362396in}}%
\pgfpathlineto{\pgfqpoint{6.969451in}{2.047755in}}%
\pgfpathlineto{\pgfqpoint{6.969987in}{2.198483in}}%
\pgfpathlineto{\pgfqpoint{6.970523in}{2.160782in}}%
\pgfpathlineto{\pgfqpoint{6.971060in}{1.619599in}}%
\pgfpathlineto{\pgfqpoint{6.971596in}{2.103787in}}%
\pgfpathlineto{\pgfqpoint{6.972133in}{2.303471in}}%
\pgfpathlineto{\pgfqpoint{6.973206in}{0.734717in}}%
\pgfpathlineto{\pgfqpoint{6.974815in}{2.185089in}}%
\pgfpathlineto{\pgfqpoint{6.975888in}{1.146792in}}%
\pgfpathlineto{\pgfqpoint{6.976424in}{2.279982in}}%
\pgfpathlineto{\pgfqpoint{6.976961in}{2.033837in}}%
\pgfpathlineto{\pgfqpoint{6.977497in}{1.987525in}}%
\pgfpathlineto{\pgfqpoint{6.978034in}{1.606729in}}%
\pgfpathlineto{\pgfqpoint{6.978570in}{2.130233in}}%
\pgfpathlineto{\pgfqpoint{6.979106in}{2.103938in}}%
\pgfpathlineto{\pgfqpoint{6.980179in}{1.676172in}}%
\pgfpathlineto{\pgfqpoint{6.980716in}{2.230511in}}%
\pgfpathlineto{\pgfqpoint{6.981252in}{2.124424in}}%
\pgfpathlineto{\pgfqpoint{6.981789in}{1.694282in}}%
\pgfpathlineto{\pgfqpoint{6.982325in}{1.856339in}}%
\pgfpathlineto{\pgfqpoint{6.983398in}{2.091172in}}%
\pgfpathlineto{\pgfqpoint{6.984471in}{1.581276in}}%
\pgfpathlineto{\pgfqpoint{6.985544in}{2.219482in}}%
\pgfpathlineto{\pgfqpoint{6.986080in}{1.972955in}}%
\pgfpathlineto{\pgfqpoint{6.986617in}{1.685332in}}%
\pgfpathlineto{\pgfqpoint{6.987153in}{1.878406in}}%
\pgfpathlineto{\pgfqpoint{6.987689in}{2.159347in}}%
\pgfpathlineto{\pgfqpoint{6.988226in}{2.066683in}}%
\pgfpathlineto{\pgfqpoint{6.988762in}{1.588209in}}%
\pgfpathlineto{\pgfqpoint{6.989299in}{1.875333in}}%
\pgfpathlineto{\pgfqpoint{6.989835in}{2.234765in}}%
\pgfpathlineto{\pgfqpoint{6.990372in}{2.116117in}}%
\pgfpathlineto{\pgfqpoint{6.990908in}{1.381716in}}%
\pgfpathlineto{\pgfqpoint{6.991444in}{2.206991in}}%
\pgfpathlineto{\pgfqpoint{6.991981in}{2.144852in}}%
\pgfpathlineto{\pgfqpoint{6.992517in}{2.016232in}}%
\pgfpathlineto{\pgfqpoint{6.993054in}{1.487909in}}%
\pgfpathlineto{\pgfqpoint{6.993590in}{1.916287in}}%
\pgfpathlineto{\pgfqpoint{6.994127in}{2.053817in}}%
\pgfpathlineto{\pgfqpoint{6.994663in}{1.985004in}}%
\pgfpathlineto{\pgfqpoint{6.995736in}{1.352677in}}%
\pgfpathlineto{\pgfqpoint{6.996272in}{2.082859in}}%
\pgfpathlineto{\pgfqpoint{6.996809in}{2.056010in}}%
\pgfpathlineto{\pgfqpoint{6.997882in}{1.348399in}}%
\pgfpathlineto{\pgfqpoint{6.998418in}{2.206445in}}%
\pgfpathlineto{\pgfqpoint{6.998955in}{2.011226in}}%
\pgfpathlineto{\pgfqpoint{7.000027in}{1.463269in}}%
\pgfpathlineto{\pgfqpoint{7.000564in}{2.215886in}}%
\pgfpathlineto{\pgfqpoint{7.001100in}{2.174643in}}%
\pgfpathlineto{\pgfqpoint{7.002173in}{1.732264in}}%
\pgfpathlineto{\pgfqpoint{7.002710in}{2.208117in}}%
\pgfpathlineto{\pgfqpoint{7.003246in}{1.919489in}}%
\pgfpathlineto{\pgfqpoint{7.004855in}{1.088836in}}%
\pgfpathlineto{\pgfqpoint{7.005392in}{2.030354in}}%
\pgfpathlineto{\pgfqpoint{7.005928in}{1.562222in}}%
\pgfpathlineto{\pgfqpoint{7.006465in}{0.977132in}}%
\pgfpathlineto{\pgfqpoint{7.007538in}{1.993218in}}%
\pgfpathlineto{\pgfqpoint{7.008610in}{0.749453in}}%
\pgfpathlineto{\pgfqpoint{7.009683in}{2.244171in}}%
\pgfpathlineto{\pgfqpoint{7.010220in}{1.878704in}}%
\pgfpathlineto{\pgfqpoint{7.010756in}{1.326597in}}%
\pgfpathlineto{\pgfqpoint{7.011293in}{2.171390in}}%
\pgfpathlineto{\pgfqpoint{7.011829in}{2.117757in}}%
\pgfpathlineto{\pgfqpoint{7.012366in}{2.046270in}}%
\pgfpathlineto{\pgfqpoint{7.012902in}{1.290337in}}%
\pgfpathlineto{\pgfqpoint{7.013438in}{1.990161in}}%
\pgfpathlineto{\pgfqpoint{7.013975in}{1.683455in}}%
\pgfpathlineto{\pgfqpoint{7.014511in}{2.033502in}}%
\pgfpathlineto{\pgfqpoint{7.015048in}{1.471078in}}%
\pgfpathlineto{\pgfqpoint{7.015584in}{1.730724in}}%
\pgfpathlineto{\pgfqpoint{7.016121in}{1.841360in}}%
\pgfpathlineto{\pgfqpoint{7.016657in}{1.714906in}}%
\pgfpathlineto{\pgfqpoint{7.017193in}{1.148134in}}%
\pgfpathlineto{\pgfqpoint{7.017730in}{1.180851in}}%
\pgfpathlineto{\pgfqpoint{7.018803in}{2.038527in}}%
\pgfpathlineto{\pgfqpoint{7.019339in}{1.392182in}}%
\pgfpathlineto{\pgfqpoint{7.019876in}{1.546190in}}%
\pgfpathlineto{\pgfqpoint{7.020412in}{2.085351in}}%
\pgfpathlineto{\pgfqpoint{7.020948in}{2.049994in}}%
\pgfpathlineto{\pgfqpoint{7.021485in}{1.588871in}}%
\pgfpathlineto{\pgfqpoint{7.022021in}{1.687881in}}%
\pgfpathlineto{\pgfqpoint{7.022558in}{2.075298in}}%
\pgfpathlineto{\pgfqpoint{7.023094in}{1.708352in}}%
\pgfpathlineto{\pgfqpoint{7.024167in}{2.003250in}}%
\pgfpathlineto{\pgfqpoint{7.025776in}{1.425699in}}%
\pgfpathlineto{\pgfqpoint{7.026849in}{1.472286in}}%
\pgfpathlineto{\pgfqpoint{7.027386in}{1.427720in}}%
\pgfpathlineto{\pgfqpoint{7.027922in}{1.503650in}}%
\pgfpathlineto{\pgfqpoint{7.028459in}{1.499147in}}%
\pgfpathlineto{\pgfqpoint{7.029531in}{1.923930in}}%
\pgfpathlineto{\pgfqpoint{7.030604in}{0.834766in}}%
\pgfpathlineto{\pgfqpoint{7.031677in}{2.054458in}}%
\pgfpathlineto{\pgfqpoint{7.032214in}{1.289148in}}%
\pgfpathlineto{\pgfqpoint{7.032750in}{1.565088in}}%
\pgfpathlineto{\pgfqpoint{7.033287in}{2.008065in}}%
\pgfpathlineto{\pgfqpoint{7.033823in}{1.773350in}}%
\pgfpathlineto{\pgfqpoint{7.034896in}{1.533777in}}%
\pgfpathlineto{\pgfqpoint{7.035432in}{1.949062in}}%
\pgfpathlineto{\pgfqpoint{7.035969in}{1.291855in}}%
\pgfpathlineto{\pgfqpoint{7.036505in}{1.362259in}}%
\pgfpathlineto{\pgfqpoint{7.037042in}{1.860836in}}%
\pgfpathlineto{\pgfqpoint{7.037578in}{1.593499in}}%
\pgfpathlineto{\pgfqpoint{7.038114in}{1.808843in}}%
\pgfpathlineto{\pgfqpoint{7.038651in}{1.346027in}}%
\pgfpathlineto{\pgfqpoint{7.039187in}{1.380803in}}%
\pgfpathlineto{\pgfqpoint{7.039724in}{1.633563in}}%
\pgfpathlineto{\pgfqpoint{7.040260in}{1.563142in}}%
\pgfpathlineto{\pgfqpoint{7.040797in}{1.572601in}}%
\pgfpathlineto{\pgfqpoint{7.041333in}{0.754109in}}%
\pgfpathlineto{\pgfqpoint{7.042406in}{1.855081in}}%
\pgfpathlineto{\pgfqpoint{7.042942in}{1.650092in}}%
\pgfpathlineto{\pgfqpoint{7.043479in}{0.903879in}}%
\pgfpathlineto{\pgfqpoint{7.044015in}{1.864921in}}%
\pgfpathlineto{\pgfqpoint{7.044552in}{1.847119in}}%
\pgfpathlineto{\pgfqpoint{7.045088in}{0.646646in}}%
\pgfpathlineto{\pgfqpoint{7.045625in}{1.531450in}}%
\pgfpathlineto{\pgfqpoint{7.046161in}{1.890771in}}%
\pgfpathlineto{\pgfqpoint{7.046161in}{1.890771in}}%
\pgfpathlineto{\pgfqpoint{7.046161in}{1.890771in}}%
\pgfpathlineto{\pgfqpoint{7.047770in}{1.186837in}}%
\pgfpathlineto{\pgfqpoint{7.048307in}{1.712397in}}%
\pgfpathlineto{\pgfqpoint{7.048843in}{1.482095in}}%
\pgfpathlineto{\pgfqpoint{7.049916in}{1.362285in}}%
\pgfpathlineto{\pgfqpoint{7.050989in}{1.660876in}}%
\pgfpathlineto{\pgfqpoint{7.051525in}{1.548279in}}%
\pgfpathlineto{\pgfqpoint{7.052062in}{1.194418in}}%
\pgfpathlineto{\pgfqpoint{7.052598in}{1.211718in}}%
\pgfpathlineto{\pgfqpoint{7.053135in}{1.695225in}}%
\pgfpathlineto{\pgfqpoint{7.053671in}{1.663968in}}%
\pgfpathlineto{\pgfqpoint{7.054141in}{0.590000in}}%
\pgfpathmoveto{\pgfqpoint{7.054277in}{0.590000in}}%
\pgfpathlineto{\pgfqpoint{7.054744in}{1.596644in}}%
\pgfpathlineto{\pgfqpoint{7.055280in}{1.841446in}}%
\pgfpathlineto{\pgfqpoint{7.056353in}{1.288993in}}%
\pgfpathlineto{\pgfqpoint{7.056890in}{1.548505in}}%
\pgfpathlineto{\pgfqpoint{7.057426in}{1.502379in}}%
\pgfpathlineto{\pgfqpoint{7.058499in}{1.191926in}}%
\pgfpathlineto{\pgfqpoint{7.059035in}{1.473308in}}%
\pgfpathlineto{\pgfqpoint{7.059572in}{1.133951in}}%
\pgfpathlineto{\pgfqpoint{7.060108in}{1.442018in}}%
\pgfpathlineto{\pgfqpoint{7.060645in}{1.346574in}}%
\pgfpathlineto{\pgfqpoint{7.061181in}{1.617956in}}%
\pgfpathlineto{\pgfqpoint{7.061609in}{0.590000in}}%
\pgfpathmoveto{\pgfqpoint{7.061849in}{0.590000in}}%
\pgfpathlineto{\pgfqpoint{7.062254in}{1.392241in}}%
\pgfpathlineto{\pgfqpoint{7.062791in}{1.488578in}}%
\pgfpathlineto{\pgfqpoint{7.063327in}{1.256992in}}%
\pgfpathlineto{\pgfqpoint{7.063863in}{1.434040in}}%
\pgfpathlineto{\pgfqpoint{7.064400in}{1.538376in}}%
\pgfpathlineto{\pgfqpoint{7.064936in}{1.234354in}}%
\pgfpathlineto{\pgfqpoint{7.064936in}{1.234354in}}%
\pgfpathlineto{\pgfqpoint{7.064936in}{1.234354in}}%
\pgfpathlineto{\pgfqpoint{7.065473in}{1.545375in}}%
\pgfpathlineto{\pgfqpoint{7.066009in}{1.474573in}}%
\pgfpathlineto{\pgfqpoint{7.066546in}{1.253831in}}%
\pgfpathlineto{\pgfqpoint{7.066789in}{0.590000in}}%
\pgfpathmoveto{\pgfqpoint{7.067354in}{0.590000in}}%
\pgfpathlineto{\pgfqpoint{7.067618in}{1.364573in}}%
\pgfpathlineto{\pgfqpoint{7.068155in}{1.598546in}}%
\pgfpathlineto{\pgfqpoint{7.069228in}{0.798238in}}%
\pgfpathlineto{\pgfqpoint{7.069764in}{1.362302in}}%
\pgfpathlineto{\pgfqpoint{7.070301in}{1.316110in}}%
\pgfpathlineto{\pgfqpoint{7.070837in}{0.995119in}}%
\pgfpathlineto{\pgfqpoint{7.071373in}{1.299914in}}%
\pgfpathlineto{\pgfqpoint{7.071910in}{1.271281in}}%
\pgfpathlineto{\pgfqpoint{7.072446in}{0.715883in}}%
\pgfpathlineto{\pgfqpoint{7.072983in}{1.259400in}}%
\pgfpathlineto{\pgfqpoint{7.073519in}{1.497229in}}%
\pgfpathlineto{\pgfqpoint{7.074056in}{1.359509in}}%
\pgfpathlineto{\pgfqpoint{7.075129in}{1.129034in}}%
\pgfpathlineto{\pgfqpoint{7.075665in}{1.332978in}}%
\pgfpathlineto{\pgfqpoint{7.076201in}{1.135187in}}%
\pgfpathlineto{\pgfqpoint{7.076738in}{1.297591in}}%
\pgfpathlineto{\pgfqpoint{7.076738in}{1.297591in}}%
\pgfpathlineto{\pgfqpoint{7.076738in}{1.297591in}}%
\pgfpathlineto{\pgfqpoint{7.077811in}{0.639487in}}%
\pgfpathlineto{\pgfqpoint{7.078884in}{1.304071in}}%
\pgfpathlineto{\pgfqpoint{7.079420in}{1.257881in}}%
\pgfpathlineto{\pgfqpoint{7.079939in}{0.590000in}}%
\pgfpathmoveto{\pgfqpoint{7.079972in}{0.590000in}}%
\pgfpathlineto{\pgfqpoint{7.080493in}{1.319516in}}%
\pgfpathlineto{\pgfqpoint{7.081029in}{1.350630in}}%
\pgfpathlineto{\pgfqpoint{7.081941in}{0.590000in}}%
\pgfpathmoveto{\pgfqpoint{7.082238in}{0.590000in}}%
\pgfpathlineto{\pgfqpoint{7.082639in}{1.168885in}}%
\pgfpathlineto{\pgfqpoint{7.082639in}{1.168885in}}%
\pgfpathlineto{\pgfqpoint{7.082639in}{1.168885in}}%
\pgfpathlineto{\pgfqpoint{7.083175in}{1.159412in}}%
\pgfpathlineto{\pgfqpoint{7.083712in}{0.856163in}}%
\pgfpathlineto{\pgfqpoint{7.084248in}{1.108073in}}%
\pgfpathlineto{\pgfqpoint{7.085069in}{0.590000in}}%
\pgfpathmoveto{\pgfqpoint{7.085511in}{0.590000in}}%
\pgfpathlineto{\pgfqpoint{7.085857in}{1.173856in}}%
\pgfpathlineto{\pgfqpoint{7.086394in}{1.347625in}}%
\pgfpathlineto{\pgfqpoint{7.086792in}{0.590000in}}%
\pgfpathmoveto{\pgfqpoint{7.087139in}{0.590000in}}%
\pgfpathlineto{\pgfqpoint{7.087467in}{1.001388in}}%
\pgfpathlineto{\pgfqpoint{7.088539in}{1.145965in}}%
\pgfpathlineto{\pgfqpoint{7.089076in}{1.153057in}}%
\pgfpathlineto{\pgfqpoint{7.089612in}{1.032511in}}%
\pgfpathlineto{\pgfqpoint{7.090043in}{0.590000in}}%
\pgfpathmoveto{\pgfqpoint{7.090287in}{0.590000in}}%
\pgfpathlineto{\pgfqpoint{7.090685in}{0.903119in}}%
\pgfpathlineto{\pgfqpoint{7.091758in}{0.998947in}}%
\pgfpathlineto{\pgfqpoint{7.092294in}{0.919676in}}%
\pgfpathlineto{\pgfqpoint{7.092643in}{0.590000in}}%
\pgfpathmoveto{\pgfqpoint{7.093028in}{0.590000in}}%
\pgfpathlineto{\pgfqpoint{7.093367in}{0.894836in}}%
\pgfpathlineto{\pgfqpoint{7.093904in}{1.148752in}}%
\pgfpathlineto{\pgfqpoint{7.094363in}{0.590000in}}%
\pgfpathmoveto{\pgfqpoint{7.094603in}{0.590000in}}%
\pgfpathlineto{\pgfqpoint{7.094977in}{0.806219in}}%
\pgfpathlineto{\pgfqpoint{7.095513in}{1.017895in}}%
\pgfpathlineto{\pgfqpoint{7.096050in}{0.757777in}}%
\pgfpathlineto{\pgfqpoint{7.096586in}{0.774212in}}%
\pgfpathlineto{\pgfqpoint{7.097122in}{1.110593in}}%
\pgfpathlineto{\pgfqpoint{7.097659in}{0.934268in}}%
\pgfpathlineto{\pgfqpoint{7.097987in}{0.590000in}}%
\pgfpathmoveto{\pgfqpoint{7.098412in}{0.590000in}}%
\pgfpathlineto{\pgfqpoint{7.098732in}{0.913497in}}%
\pgfpathlineto{\pgfqpoint{7.098732in}{0.913497in}}%
\pgfpathlineto{\pgfqpoint{7.098732in}{0.913497in}}%
\pgfpathlineto{\pgfqpoint{7.099268in}{0.910343in}}%
\pgfpathlineto{\pgfqpoint{7.099699in}{0.590000in}}%
\pgfpathmoveto{\pgfqpoint{7.100658in}{0.590000in}}%
\pgfpathlineto{\pgfqpoint{7.100877in}{0.868067in}}%
\pgfpathlineto{\pgfqpoint{7.100877in}{0.868067in}}%
\pgfpathlineto{\pgfqpoint{7.100877in}{0.868067in}}%
\pgfpathlineto{\pgfqpoint{7.101950in}{0.722871in}}%
\pgfpathlineto{\pgfqpoint{7.102487in}{0.809349in}}%
\pgfpathlineto{\pgfqpoint{7.103023in}{0.805307in}}%
\pgfpathlineto{\pgfqpoint{7.103560in}{0.823190in}}%
\pgfpathlineto{\pgfqpoint{7.104096in}{0.942038in}}%
\pgfpathlineto{\pgfqpoint{7.104740in}{0.590000in}}%
\pgfpathmoveto{\pgfqpoint{7.105855in}{0.590000in}}%
\pgfpathlineto{\pgfqpoint{7.106242in}{0.914419in}}%
\pgfpathlineto{\pgfqpoint{7.106242in}{0.914419in}}%
\pgfpathlineto{\pgfqpoint{7.106242in}{0.914419in}}%
\pgfpathlineto{\pgfqpoint{7.106778in}{0.852585in}}%
\pgfpathlineto{\pgfqpoint{7.107228in}{0.590000in}}%
\pgfpathmoveto{\pgfqpoint{7.108162in}{0.590000in}}%
\pgfpathlineto{\pgfqpoint{7.108388in}{0.655502in}}%
\pgfpathlineto{\pgfqpoint{7.108388in}{0.655502in}}%
\pgfpathlineto{\pgfqpoint{7.108388in}{0.655502in}}%
\pgfpathlineto{\pgfqpoint{7.108673in}{0.590000in}}%
\pgfpathmoveto{\pgfqpoint{7.109715in}{0.590000in}}%
\pgfpathlineto{\pgfqpoint{7.109997in}{0.767403in}}%
\pgfpathlineto{\pgfqpoint{7.109997in}{0.767403in}}%
\pgfpathlineto{\pgfqpoint{7.109997in}{0.767403in}}%
\pgfpathlineto{\pgfqpoint{7.110161in}{0.590000in}}%
\pgfpathmoveto{\pgfqpoint{7.111426in}{0.590000in}}%
\pgfpathlineto{\pgfqpoint{7.111606in}{0.682978in}}%
\pgfpathlineto{\pgfqpoint{7.112143in}{0.767335in}}%
\pgfpathlineto{\pgfqpoint{7.112143in}{0.767335in}}%
\pgfpathlineto{\pgfqpoint{7.112386in}{0.590000in}}%
\pgfpathmoveto{\pgfqpoint{7.113688in}{0.590000in}}%
\pgfpathlineto{\pgfqpoint{7.114288in}{0.680958in}}%
\pgfpathlineto{\pgfqpoint{7.114288in}{0.680958in}}%
\pgfpathlineto{\pgfqpoint{7.114288in}{0.680958in}}%
\pgfpathlineto{\pgfqpoint{7.114342in}{0.590000in}}%
\pgfpathmoveto{\pgfqpoint{7.115316in}{0.590000in}}%
\pgfpathlineto{\pgfqpoint{7.115361in}{0.665850in}}%
\pgfpathlineto{\pgfqpoint{7.115898in}{0.704710in}}%
\pgfpathlineto{\pgfqpoint{7.115898in}{0.704710in}}%
\pgfpathlineto{\pgfqpoint{7.116057in}{0.590000in}}%
\pgfpathmoveto{\pgfqpoint{7.118995in}{0.590000in}}%
\pgfpathlineto{\pgfqpoint{7.119116in}{0.701043in}}%
\pgfpathlineto{\pgfqpoint{7.119116in}{0.701043in}}%
\pgfpathlineto{\pgfqpoint{7.119116in}{0.701043in}}%
\pgfpathlineto{\pgfqpoint{7.119214in}{0.590000in}}%
\pgfpathlineto{\pgfqpoint{7.119214in}{0.590000in}}%
\pgfusepath{stroke}%
\end{pgfscope}%
\begin{pgfscope}%
\pgfsetrectcap%
\pgfsetmiterjoin%
\pgfsetlinewidth{1.003750pt}%
\definecolor{currentstroke}{rgb}{0.000000,0.000000,0.000000}%
\pgfsetstrokecolor{currentstroke}%
\pgfsetdash{}{0pt}%
\pgfpathmoveto{\pgfqpoint{1.000000in}{5.400000in}}%
\pgfpathlineto{\pgfqpoint{7.200000in}{5.400000in}}%
\pgfusepath{stroke}%
\end{pgfscope}%
\begin{pgfscope}%
\pgfsetrectcap%
\pgfsetmiterjoin%
\pgfsetlinewidth{1.003750pt}%
\definecolor{currentstroke}{rgb}{0.000000,0.000000,0.000000}%
\pgfsetstrokecolor{currentstroke}%
\pgfsetdash{}{0pt}%
\pgfpathmoveto{\pgfqpoint{7.200000in}{0.600000in}}%
\pgfpathlineto{\pgfqpoint{7.200000in}{5.400000in}}%
\pgfusepath{stroke}%
\end{pgfscope}%
\begin{pgfscope}%
\pgfsetrectcap%
\pgfsetmiterjoin%
\pgfsetlinewidth{1.003750pt}%
\definecolor{currentstroke}{rgb}{0.000000,0.000000,0.000000}%
\pgfsetstrokecolor{currentstroke}%
\pgfsetdash{}{0pt}%
\pgfpathmoveto{\pgfqpoint{1.000000in}{0.600000in}}%
\pgfpathlineto{\pgfqpoint{7.200000in}{0.600000in}}%
\pgfusepath{stroke}%
\end{pgfscope}%
\begin{pgfscope}%
\pgfsetrectcap%
\pgfsetmiterjoin%
\pgfsetlinewidth{1.003750pt}%
\definecolor{currentstroke}{rgb}{0.000000,0.000000,0.000000}%
\pgfsetstrokecolor{currentstroke}%
\pgfsetdash{}{0pt}%
\pgfpathmoveto{\pgfqpoint{1.000000in}{0.600000in}}%
\pgfpathlineto{\pgfqpoint{1.000000in}{5.400000in}}%
\pgfusepath{stroke}%
\end{pgfscope}%
\begin{pgfscope}%
\pgfpathrectangle{\pgfqpoint{1.000000in}{0.600000in}}{\pgfqpoint{6.200000in}{4.800000in}} %
\pgfusepath{clip}%
\pgfsetbuttcap%
\pgfsetroundjoin%
\pgfsetlinewidth{0.501875pt}%
\definecolor{currentstroke}{rgb}{0.000000,0.000000,0.000000}%
\pgfsetstrokecolor{currentstroke}%
\pgfsetdash{{1.000000pt}{3.000000pt}}{0.000000pt}%
\pgfpathmoveto{\pgfqpoint{1.524351in}{0.600000in}}%
\pgfpathlineto{\pgfqpoint{1.524351in}{5.400000in}}%
\pgfusepath{stroke}%
\end{pgfscope}%
\begin{pgfscope}%
\pgfsetbuttcap%
\pgfsetroundjoin%
\definecolor{currentfill}{rgb}{0.000000,0.000000,0.000000}%
\pgfsetfillcolor{currentfill}%
\pgfsetlinewidth{0.501875pt}%
\definecolor{currentstroke}{rgb}{0.000000,0.000000,0.000000}%
\pgfsetstrokecolor{currentstroke}%
\pgfsetdash{}{0pt}%
\pgfsys@defobject{currentmarker}{\pgfqpoint{0.000000in}{0.000000in}}{\pgfqpoint{0.000000in}{0.055556in}}{%
\pgfpathmoveto{\pgfqpoint{0.000000in}{0.000000in}}%
\pgfpathlineto{\pgfqpoint{0.000000in}{0.055556in}}%
\pgfusepath{stroke,fill}%
}%
\begin{pgfscope}%
\pgfsys@transformshift{1.524351in}{0.600000in}%
\pgfsys@useobject{currentmarker}{}%
\end{pgfscope}%
\end{pgfscope}%
\begin{pgfscope}%
\pgfsetbuttcap%
\pgfsetroundjoin%
\definecolor{currentfill}{rgb}{0.000000,0.000000,0.000000}%
\pgfsetfillcolor{currentfill}%
\pgfsetlinewidth{0.501875pt}%
\definecolor{currentstroke}{rgb}{0.000000,0.000000,0.000000}%
\pgfsetstrokecolor{currentstroke}%
\pgfsetdash{}{0pt}%
\pgfsys@defobject{currentmarker}{\pgfqpoint{0.000000in}{-0.055556in}}{\pgfqpoint{0.000000in}{0.000000in}}{%
\pgfpathmoveto{\pgfqpoint{0.000000in}{0.000000in}}%
\pgfpathlineto{\pgfqpoint{0.000000in}{-0.055556in}}%
\pgfusepath{stroke,fill}%
}%
\begin{pgfscope}%
\pgfsys@transformshift{1.524351in}{5.400000in}%
\pgfsys@useobject{currentmarker}{}%
\end{pgfscope}%
\end{pgfscope}%
\begin{pgfscope}%
\pgftext[x=1.524351in,y=0.544444in,,top]{\sffamily\fontsize{12.000000}{14.400000}\selectfont −40000}%
\end{pgfscope}%
\begin{pgfscope}%
\pgfpathrectangle{\pgfqpoint{1.000000in}{0.600000in}}{\pgfqpoint{6.200000in}{4.800000in}} %
\pgfusepath{clip}%
\pgfsetbuttcap%
\pgfsetroundjoin%
\pgfsetlinewidth{0.501875pt}%
\definecolor{currentstroke}{rgb}{0.000000,0.000000,0.000000}%
\pgfsetstrokecolor{currentstroke}%
\pgfsetdash{{1.000000pt}{3.000000pt}}{0.000000pt}%
\pgfpathmoveto{\pgfqpoint{2.812176in}{0.600000in}}%
\pgfpathlineto{\pgfqpoint{2.812176in}{5.400000in}}%
\pgfusepath{stroke}%
\end{pgfscope}%
\begin{pgfscope}%
\pgfsetbuttcap%
\pgfsetroundjoin%
\definecolor{currentfill}{rgb}{0.000000,0.000000,0.000000}%
\pgfsetfillcolor{currentfill}%
\pgfsetlinewidth{0.501875pt}%
\definecolor{currentstroke}{rgb}{0.000000,0.000000,0.000000}%
\pgfsetstrokecolor{currentstroke}%
\pgfsetdash{}{0pt}%
\pgfsys@defobject{currentmarker}{\pgfqpoint{0.000000in}{0.000000in}}{\pgfqpoint{0.000000in}{0.055556in}}{%
\pgfpathmoveto{\pgfqpoint{0.000000in}{0.000000in}}%
\pgfpathlineto{\pgfqpoint{0.000000in}{0.055556in}}%
\pgfusepath{stroke,fill}%
}%
\begin{pgfscope}%
\pgfsys@transformshift{2.812176in}{0.600000in}%
\pgfsys@useobject{currentmarker}{}%
\end{pgfscope}%
\end{pgfscope}%
\begin{pgfscope}%
\pgfsetbuttcap%
\pgfsetroundjoin%
\definecolor{currentfill}{rgb}{0.000000,0.000000,0.000000}%
\pgfsetfillcolor{currentfill}%
\pgfsetlinewidth{0.501875pt}%
\definecolor{currentstroke}{rgb}{0.000000,0.000000,0.000000}%
\pgfsetstrokecolor{currentstroke}%
\pgfsetdash{}{0pt}%
\pgfsys@defobject{currentmarker}{\pgfqpoint{0.000000in}{-0.055556in}}{\pgfqpoint{0.000000in}{0.000000in}}{%
\pgfpathmoveto{\pgfqpoint{0.000000in}{0.000000in}}%
\pgfpathlineto{\pgfqpoint{0.000000in}{-0.055556in}}%
\pgfusepath{stroke,fill}%
}%
\begin{pgfscope}%
\pgfsys@transformshift{2.812176in}{5.400000in}%
\pgfsys@useobject{currentmarker}{}%
\end{pgfscope}%
\end{pgfscope}%
\begin{pgfscope}%
\pgftext[x=2.812176in,y=0.544444in,,top]{\sffamily\fontsize{12.000000}{14.400000}\selectfont −20000}%
\end{pgfscope}%
\begin{pgfscope}%
\pgfpathrectangle{\pgfqpoint{1.000000in}{0.600000in}}{\pgfqpoint{6.200000in}{4.800000in}} %
\pgfusepath{clip}%
\pgfsetbuttcap%
\pgfsetroundjoin%
\pgfsetlinewidth{0.501875pt}%
\definecolor{currentstroke}{rgb}{0.000000,0.000000,0.000000}%
\pgfsetstrokecolor{currentstroke}%
\pgfsetdash{{1.000000pt}{3.000000pt}}{0.000000pt}%
\pgfpathmoveto{\pgfqpoint{4.100000in}{0.600000in}}%
\pgfpathlineto{\pgfqpoint{4.100000in}{5.400000in}}%
\pgfusepath{stroke}%
\end{pgfscope}%
\begin{pgfscope}%
\pgfsetbuttcap%
\pgfsetroundjoin%
\definecolor{currentfill}{rgb}{0.000000,0.000000,0.000000}%
\pgfsetfillcolor{currentfill}%
\pgfsetlinewidth{0.501875pt}%
\definecolor{currentstroke}{rgb}{0.000000,0.000000,0.000000}%
\pgfsetstrokecolor{currentstroke}%
\pgfsetdash{}{0pt}%
\pgfsys@defobject{currentmarker}{\pgfqpoint{0.000000in}{0.000000in}}{\pgfqpoint{0.000000in}{0.055556in}}{%
\pgfpathmoveto{\pgfqpoint{0.000000in}{0.000000in}}%
\pgfpathlineto{\pgfqpoint{0.000000in}{0.055556in}}%
\pgfusepath{stroke,fill}%
}%
\begin{pgfscope}%
\pgfsys@transformshift{4.100000in}{0.600000in}%
\pgfsys@useobject{currentmarker}{}%
\end{pgfscope}%
\end{pgfscope}%
\begin{pgfscope}%
\pgfsetbuttcap%
\pgfsetroundjoin%
\definecolor{currentfill}{rgb}{0.000000,0.000000,0.000000}%
\pgfsetfillcolor{currentfill}%
\pgfsetlinewidth{0.501875pt}%
\definecolor{currentstroke}{rgb}{0.000000,0.000000,0.000000}%
\pgfsetstrokecolor{currentstroke}%
\pgfsetdash{}{0pt}%
\pgfsys@defobject{currentmarker}{\pgfqpoint{0.000000in}{-0.055556in}}{\pgfqpoint{0.000000in}{0.000000in}}{%
\pgfpathmoveto{\pgfqpoint{0.000000in}{0.000000in}}%
\pgfpathlineto{\pgfqpoint{0.000000in}{-0.055556in}}%
\pgfusepath{stroke,fill}%
}%
\begin{pgfscope}%
\pgfsys@transformshift{4.100000in}{5.400000in}%
\pgfsys@useobject{currentmarker}{}%
\end{pgfscope}%
\end{pgfscope}%
\begin{pgfscope}%
\pgftext[x=4.100000in,y=0.544444in,,top]{\sffamily\fontsize{12.000000}{14.400000}\selectfont 0}%
\end{pgfscope}%
\begin{pgfscope}%
\pgfpathrectangle{\pgfqpoint{1.000000in}{0.600000in}}{\pgfqpoint{6.200000in}{4.800000in}} %
\pgfusepath{clip}%
\pgfsetbuttcap%
\pgfsetroundjoin%
\pgfsetlinewidth{0.501875pt}%
\definecolor{currentstroke}{rgb}{0.000000,0.000000,0.000000}%
\pgfsetstrokecolor{currentstroke}%
\pgfsetdash{{1.000000pt}{3.000000pt}}{0.000000pt}%
\pgfpathmoveto{\pgfqpoint{5.387824in}{0.600000in}}%
\pgfpathlineto{\pgfqpoint{5.387824in}{5.400000in}}%
\pgfusepath{stroke}%
\end{pgfscope}%
\begin{pgfscope}%
\pgfsetbuttcap%
\pgfsetroundjoin%
\definecolor{currentfill}{rgb}{0.000000,0.000000,0.000000}%
\pgfsetfillcolor{currentfill}%
\pgfsetlinewidth{0.501875pt}%
\definecolor{currentstroke}{rgb}{0.000000,0.000000,0.000000}%
\pgfsetstrokecolor{currentstroke}%
\pgfsetdash{}{0pt}%
\pgfsys@defobject{currentmarker}{\pgfqpoint{0.000000in}{0.000000in}}{\pgfqpoint{0.000000in}{0.055556in}}{%
\pgfpathmoveto{\pgfqpoint{0.000000in}{0.000000in}}%
\pgfpathlineto{\pgfqpoint{0.000000in}{0.055556in}}%
\pgfusepath{stroke,fill}%
}%
\begin{pgfscope}%
\pgfsys@transformshift{5.387824in}{0.600000in}%
\pgfsys@useobject{currentmarker}{}%
\end{pgfscope}%
\end{pgfscope}%
\begin{pgfscope}%
\pgfsetbuttcap%
\pgfsetroundjoin%
\definecolor{currentfill}{rgb}{0.000000,0.000000,0.000000}%
\pgfsetfillcolor{currentfill}%
\pgfsetlinewidth{0.501875pt}%
\definecolor{currentstroke}{rgb}{0.000000,0.000000,0.000000}%
\pgfsetstrokecolor{currentstroke}%
\pgfsetdash{}{0pt}%
\pgfsys@defobject{currentmarker}{\pgfqpoint{0.000000in}{-0.055556in}}{\pgfqpoint{0.000000in}{0.000000in}}{%
\pgfpathmoveto{\pgfqpoint{0.000000in}{0.000000in}}%
\pgfpathlineto{\pgfqpoint{0.000000in}{-0.055556in}}%
\pgfusepath{stroke,fill}%
}%
\begin{pgfscope}%
\pgfsys@transformshift{5.387824in}{5.400000in}%
\pgfsys@useobject{currentmarker}{}%
\end{pgfscope}%
\end{pgfscope}%
\begin{pgfscope}%
\pgftext[x=5.387824in,y=0.544444in,,top]{\sffamily\fontsize{12.000000}{14.400000}\selectfont 20000}%
\end{pgfscope}%
\begin{pgfscope}%
\pgfpathrectangle{\pgfqpoint{1.000000in}{0.600000in}}{\pgfqpoint{6.200000in}{4.800000in}} %
\pgfusepath{clip}%
\pgfsetbuttcap%
\pgfsetroundjoin%
\pgfsetlinewidth{0.501875pt}%
\definecolor{currentstroke}{rgb}{0.000000,0.000000,0.000000}%
\pgfsetstrokecolor{currentstroke}%
\pgfsetdash{{1.000000pt}{3.000000pt}}{0.000000pt}%
\pgfpathmoveto{\pgfqpoint{6.675649in}{0.600000in}}%
\pgfpathlineto{\pgfqpoint{6.675649in}{5.400000in}}%
\pgfusepath{stroke}%
\end{pgfscope}%
\begin{pgfscope}%
\pgfsetbuttcap%
\pgfsetroundjoin%
\definecolor{currentfill}{rgb}{0.000000,0.000000,0.000000}%
\pgfsetfillcolor{currentfill}%
\pgfsetlinewidth{0.501875pt}%
\definecolor{currentstroke}{rgb}{0.000000,0.000000,0.000000}%
\pgfsetstrokecolor{currentstroke}%
\pgfsetdash{}{0pt}%
\pgfsys@defobject{currentmarker}{\pgfqpoint{0.000000in}{0.000000in}}{\pgfqpoint{0.000000in}{0.055556in}}{%
\pgfpathmoveto{\pgfqpoint{0.000000in}{0.000000in}}%
\pgfpathlineto{\pgfqpoint{0.000000in}{0.055556in}}%
\pgfusepath{stroke,fill}%
}%
\begin{pgfscope}%
\pgfsys@transformshift{6.675649in}{0.600000in}%
\pgfsys@useobject{currentmarker}{}%
\end{pgfscope}%
\end{pgfscope}%
\begin{pgfscope}%
\pgfsetbuttcap%
\pgfsetroundjoin%
\definecolor{currentfill}{rgb}{0.000000,0.000000,0.000000}%
\pgfsetfillcolor{currentfill}%
\pgfsetlinewidth{0.501875pt}%
\definecolor{currentstroke}{rgb}{0.000000,0.000000,0.000000}%
\pgfsetstrokecolor{currentstroke}%
\pgfsetdash{}{0pt}%
\pgfsys@defobject{currentmarker}{\pgfqpoint{0.000000in}{-0.055556in}}{\pgfqpoint{0.000000in}{0.000000in}}{%
\pgfpathmoveto{\pgfqpoint{0.000000in}{0.000000in}}%
\pgfpathlineto{\pgfqpoint{0.000000in}{-0.055556in}}%
\pgfusepath{stroke,fill}%
}%
\begin{pgfscope}%
\pgfsys@transformshift{6.675649in}{5.400000in}%
\pgfsys@useobject{currentmarker}{}%
\end{pgfscope}%
\end{pgfscope}%
\begin{pgfscope}%
\pgftext[x=6.675649in,y=0.544444in,,top]{\sffamily\fontsize{12.000000}{14.400000}\selectfont 40000}%
\end{pgfscope}%
\begin{pgfscope}%
\pgftext[x=4.100000in,y=0.313705in,,top]{\sffamily\fontsize{12.000000}{14.400000}\selectfont Doppler frequency (Hz)}%
\end{pgfscope}%
\begin{pgfscope}%
\pgfpathrectangle{\pgfqpoint{1.000000in}{0.600000in}}{\pgfqpoint{6.200000in}{4.800000in}} %
\pgfusepath{clip}%
\pgfsetbuttcap%
\pgfsetroundjoin%
\pgfsetlinewidth{0.501875pt}%
\definecolor{currentstroke}{rgb}{0.000000,0.000000,0.000000}%
\pgfsetstrokecolor{currentstroke}%
\pgfsetdash{{1.000000pt}{3.000000pt}}{0.000000pt}%
\pgfpathmoveto{\pgfqpoint{1.000000in}{0.617052in}}%
\pgfpathlineto{\pgfqpoint{7.200000in}{0.617052in}}%
\pgfusepath{stroke}%
\end{pgfscope}%
\begin{pgfscope}%
\pgfsetbuttcap%
\pgfsetroundjoin%
\definecolor{currentfill}{rgb}{0.000000,0.000000,0.000000}%
\pgfsetfillcolor{currentfill}%
\pgfsetlinewidth{0.501875pt}%
\definecolor{currentstroke}{rgb}{0.000000,0.000000,0.000000}%
\pgfsetstrokecolor{currentstroke}%
\pgfsetdash{}{0pt}%
\pgfsys@defobject{currentmarker}{\pgfqpoint{0.000000in}{0.000000in}}{\pgfqpoint{0.055556in}{0.000000in}}{%
\pgfpathmoveto{\pgfqpoint{0.000000in}{0.000000in}}%
\pgfpathlineto{\pgfqpoint{0.055556in}{0.000000in}}%
\pgfusepath{stroke,fill}%
}%
\begin{pgfscope}%
\pgfsys@transformshift{1.000000in}{0.617052in}%
\pgfsys@useobject{currentmarker}{}%
\end{pgfscope}%
\end{pgfscope}%
\begin{pgfscope}%
\pgfsetbuttcap%
\pgfsetroundjoin%
\definecolor{currentfill}{rgb}{0.000000,0.000000,0.000000}%
\pgfsetfillcolor{currentfill}%
\pgfsetlinewidth{0.501875pt}%
\definecolor{currentstroke}{rgb}{0.000000,0.000000,0.000000}%
\pgfsetstrokecolor{currentstroke}%
\pgfsetdash{}{0pt}%
\pgfsys@defobject{currentmarker}{\pgfqpoint{-0.055556in}{0.000000in}}{\pgfqpoint{0.000000in}{0.000000in}}{%
\pgfpathmoveto{\pgfqpoint{0.000000in}{0.000000in}}%
\pgfpathlineto{\pgfqpoint{-0.055556in}{0.000000in}}%
\pgfusepath{stroke,fill}%
}%
\begin{pgfscope}%
\pgfsys@transformshift{7.200000in}{0.617052in}%
\pgfsys@useobject{currentmarker}{}%
\end{pgfscope}%
\end{pgfscope}%
\begin{pgfscope}%
\pgftext[x=0.944444in,y=0.617052in,right,]{\sffamily\fontsize{12.000000}{14.400000}\selectfont −30}%
\end{pgfscope}%
\begin{pgfscope}%
\pgfpathrectangle{\pgfqpoint{1.000000in}{0.600000in}}{\pgfqpoint{6.200000in}{4.800000in}} %
\pgfusepath{clip}%
\pgfsetbuttcap%
\pgfsetroundjoin%
\pgfsetlinewidth{0.501875pt}%
\definecolor{currentstroke}{rgb}{0.000000,0.000000,0.000000}%
\pgfsetstrokecolor{currentstroke}%
\pgfsetdash{{1.000000pt}{3.000000pt}}{0.000000pt}%
\pgfpathmoveto{\pgfqpoint{1.000000in}{1.266447in}}%
\pgfpathlineto{\pgfqpoint{7.200000in}{1.266447in}}%
\pgfusepath{stroke}%
\end{pgfscope}%
\begin{pgfscope}%
\pgfsetbuttcap%
\pgfsetroundjoin%
\definecolor{currentfill}{rgb}{0.000000,0.000000,0.000000}%
\pgfsetfillcolor{currentfill}%
\pgfsetlinewidth{0.501875pt}%
\definecolor{currentstroke}{rgb}{0.000000,0.000000,0.000000}%
\pgfsetstrokecolor{currentstroke}%
\pgfsetdash{}{0pt}%
\pgfsys@defobject{currentmarker}{\pgfqpoint{0.000000in}{0.000000in}}{\pgfqpoint{0.055556in}{0.000000in}}{%
\pgfpathmoveto{\pgfqpoint{0.000000in}{0.000000in}}%
\pgfpathlineto{\pgfqpoint{0.055556in}{0.000000in}}%
\pgfusepath{stroke,fill}%
}%
\begin{pgfscope}%
\pgfsys@transformshift{1.000000in}{1.266447in}%
\pgfsys@useobject{currentmarker}{}%
\end{pgfscope}%
\end{pgfscope}%
\begin{pgfscope}%
\pgfsetbuttcap%
\pgfsetroundjoin%
\definecolor{currentfill}{rgb}{0.000000,0.000000,0.000000}%
\pgfsetfillcolor{currentfill}%
\pgfsetlinewidth{0.501875pt}%
\definecolor{currentstroke}{rgb}{0.000000,0.000000,0.000000}%
\pgfsetstrokecolor{currentstroke}%
\pgfsetdash{}{0pt}%
\pgfsys@defobject{currentmarker}{\pgfqpoint{-0.055556in}{0.000000in}}{\pgfqpoint{0.000000in}{0.000000in}}{%
\pgfpathmoveto{\pgfqpoint{0.000000in}{0.000000in}}%
\pgfpathlineto{\pgfqpoint{-0.055556in}{0.000000in}}%
\pgfusepath{stroke,fill}%
}%
\begin{pgfscope}%
\pgfsys@transformshift{7.200000in}{1.266447in}%
\pgfsys@useobject{currentmarker}{}%
\end{pgfscope}%
\end{pgfscope}%
\begin{pgfscope}%
\pgftext[x=0.944444in,y=1.266447in,right,]{\sffamily\fontsize{12.000000}{14.400000}\selectfont −20}%
\end{pgfscope}%
\begin{pgfscope}%
\pgfpathrectangle{\pgfqpoint{1.000000in}{0.600000in}}{\pgfqpoint{6.200000in}{4.800000in}} %
\pgfusepath{clip}%
\pgfsetbuttcap%
\pgfsetroundjoin%
\pgfsetlinewidth{0.501875pt}%
\definecolor{currentstroke}{rgb}{0.000000,0.000000,0.000000}%
\pgfsetstrokecolor{currentstroke}%
\pgfsetdash{{1.000000pt}{3.000000pt}}{0.000000pt}%
\pgfpathmoveto{\pgfqpoint{1.000000in}{1.915843in}}%
\pgfpathlineto{\pgfqpoint{7.200000in}{1.915843in}}%
\pgfusepath{stroke}%
\end{pgfscope}%
\begin{pgfscope}%
\pgfsetbuttcap%
\pgfsetroundjoin%
\definecolor{currentfill}{rgb}{0.000000,0.000000,0.000000}%
\pgfsetfillcolor{currentfill}%
\pgfsetlinewidth{0.501875pt}%
\definecolor{currentstroke}{rgb}{0.000000,0.000000,0.000000}%
\pgfsetstrokecolor{currentstroke}%
\pgfsetdash{}{0pt}%
\pgfsys@defobject{currentmarker}{\pgfqpoint{0.000000in}{0.000000in}}{\pgfqpoint{0.055556in}{0.000000in}}{%
\pgfpathmoveto{\pgfqpoint{0.000000in}{0.000000in}}%
\pgfpathlineto{\pgfqpoint{0.055556in}{0.000000in}}%
\pgfusepath{stroke,fill}%
}%
\begin{pgfscope}%
\pgfsys@transformshift{1.000000in}{1.915843in}%
\pgfsys@useobject{currentmarker}{}%
\end{pgfscope}%
\end{pgfscope}%
\begin{pgfscope}%
\pgfsetbuttcap%
\pgfsetroundjoin%
\definecolor{currentfill}{rgb}{0.000000,0.000000,0.000000}%
\pgfsetfillcolor{currentfill}%
\pgfsetlinewidth{0.501875pt}%
\definecolor{currentstroke}{rgb}{0.000000,0.000000,0.000000}%
\pgfsetstrokecolor{currentstroke}%
\pgfsetdash{}{0pt}%
\pgfsys@defobject{currentmarker}{\pgfqpoint{-0.055556in}{0.000000in}}{\pgfqpoint{0.000000in}{0.000000in}}{%
\pgfpathmoveto{\pgfqpoint{0.000000in}{0.000000in}}%
\pgfpathlineto{\pgfqpoint{-0.055556in}{0.000000in}}%
\pgfusepath{stroke,fill}%
}%
\begin{pgfscope}%
\pgfsys@transformshift{7.200000in}{1.915843in}%
\pgfsys@useobject{currentmarker}{}%
\end{pgfscope}%
\end{pgfscope}%
\begin{pgfscope}%
\pgftext[x=0.944444in,y=1.915843in,right,]{\sffamily\fontsize{12.000000}{14.400000}\selectfont −10}%
\end{pgfscope}%
\begin{pgfscope}%
\pgfpathrectangle{\pgfqpoint{1.000000in}{0.600000in}}{\pgfqpoint{6.200000in}{4.800000in}} %
\pgfusepath{clip}%
\pgfsetbuttcap%
\pgfsetroundjoin%
\pgfsetlinewidth{0.501875pt}%
\definecolor{currentstroke}{rgb}{0.000000,0.000000,0.000000}%
\pgfsetstrokecolor{currentstroke}%
\pgfsetdash{{1.000000pt}{3.000000pt}}{0.000000pt}%
\pgfpathmoveto{\pgfqpoint{1.000000in}{2.565238in}}%
\pgfpathlineto{\pgfqpoint{7.200000in}{2.565238in}}%
\pgfusepath{stroke}%
\end{pgfscope}%
\begin{pgfscope}%
\pgfsetbuttcap%
\pgfsetroundjoin%
\definecolor{currentfill}{rgb}{0.000000,0.000000,0.000000}%
\pgfsetfillcolor{currentfill}%
\pgfsetlinewidth{0.501875pt}%
\definecolor{currentstroke}{rgb}{0.000000,0.000000,0.000000}%
\pgfsetstrokecolor{currentstroke}%
\pgfsetdash{}{0pt}%
\pgfsys@defobject{currentmarker}{\pgfqpoint{0.000000in}{0.000000in}}{\pgfqpoint{0.055556in}{0.000000in}}{%
\pgfpathmoveto{\pgfqpoint{0.000000in}{0.000000in}}%
\pgfpathlineto{\pgfqpoint{0.055556in}{0.000000in}}%
\pgfusepath{stroke,fill}%
}%
\begin{pgfscope}%
\pgfsys@transformshift{1.000000in}{2.565238in}%
\pgfsys@useobject{currentmarker}{}%
\end{pgfscope}%
\end{pgfscope}%
\begin{pgfscope}%
\pgfsetbuttcap%
\pgfsetroundjoin%
\definecolor{currentfill}{rgb}{0.000000,0.000000,0.000000}%
\pgfsetfillcolor{currentfill}%
\pgfsetlinewidth{0.501875pt}%
\definecolor{currentstroke}{rgb}{0.000000,0.000000,0.000000}%
\pgfsetstrokecolor{currentstroke}%
\pgfsetdash{}{0pt}%
\pgfsys@defobject{currentmarker}{\pgfqpoint{-0.055556in}{0.000000in}}{\pgfqpoint{0.000000in}{0.000000in}}{%
\pgfpathmoveto{\pgfqpoint{0.000000in}{0.000000in}}%
\pgfpathlineto{\pgfqpoint{-0.055556in}{0.000000in}}%
\pgfusepath{stroke,fill}%
}%
\begin{pgfscope}%
\pgfsys@transformshift{7.200000in}{2.565238in}%
\pgfsys@useobject{currentmarker}{}%
\end{pgfscope}%
\end{pgfscope}%
\begin{pgfscope}%
\pgftext[x=0.944444in,y=2.565238in,right,]{\sffamily\fontsize{12.000000}{14.400000}\selectfont 0}%
\end{pgfscope}%
\begin{pgfscope}%
\pgfpathrectangle{\pgfqpoint{1.000000in}{0.600000in}}{\pgfqpoint{6.200000in}{4.800000in}} %
\pgfusepath{clip}%
\pgfsetbuttcap%
\pgfsetroundjoin%
\pgfsetlinewidth{0.501875pt}%
\definecolor{currentstroke}{rgb}{0.000000,0.000000,0.000000}%
\pgfsetstrokecolor{currentstroke}%
\pgfsetdash{{1.000000pt}{3.000000pt}}{0.000000pt}%
\pgfpathmoveto{\pgfqpoint{1.000000in}{3.214633in}}%
\pgfpathlineto{\pgfqpoint{7.200000in}{3.214633in}}%
\pgfusepath{stroke}%
\end{pgfscope}%
\begin{pgfscope}%
\pgfsetbuttcap%
\pgfsetroundjoin%
\definecolor{currentfill}{rgb}{0.000000,0.000000,0.000000}%
\pgfsetfillcolor{currentfill}%
\pgfsetlinewidth{0.501875pt}%
\definecolor{currentstroke}{rgb}{0.000000,0.000000,0.000000}%
\pgfsetstrokecolor{currentstroke}%
\pgfsetdash{}{0pt}%
\pgfsys@defobject{currentmarker}{\pgfqpoint{0.000000in}{0.000000in}}{\pgfqpoint{0.055556in}{0.000000in}}{%
\pgfpathmoveto{\pgfqpoint{0.000000in}{0.000000in}}%
\pgfpathlineto{\pgfqpoint{0.055556in}{0.000000in}}%
\pgfusepath{stroke,fill}%
}%
\begin{pgfscope}%
\pgfsys@transformshift{1.000000in}{3.214633in}%
\pgfsys@useobject{currentmarker}{}%
\end{pgfscope}%
\end{pgfscope}%
\begin{pgfscope}%
\pgfsetbuttcap%
\pgfsetroundjoin%
\definecolor{currentfill}{rgb}{0.000000,0.000000,0.000000}%
\pgfsetfillcolor{currentfill}%
\pgfsetlinewidth{0.501875pt}%
\definecolor{currentstroke}{rgb}{0.000000,0.000000,0.000000}%
\pgfsetstrokecolor{currentstroke}%
\pgfsetdash{}{0pt}%
\pgfsys@defobject{currentmarker}{\pgfqpoint{-0.055556in}{0.000000in}}{\pgfqpoint{0.000000in}{0.000000in}}{%
\pgfpathmoveto{\pgfqpoint{0.000000in}{0.000000in}}%
\pgfpathlineto{\pgfqpoint{-0.055556in}{0.000000in}}%
\pgfusepath{stroke,fill}%
}%
\begin{pgfscope}%
\pgfsys@transformshift{7.200000in}{3.214633in}%
\pgfsys@useobject{currentmarker}{}%
\end{pgfscope}%
\end{pgfscope}%
\begin{pgfscope}%
\pgftext[x=0.944444in,y=3.214633in,right,]{\sffamily\fontsize{12.000000}{14.400000}\selectfont 10}%
\end{pgfscope}%
\begin{pgfscope}%
\pgfpathrectangle{\pgfqpoint{1.000000in}{0.600000in}}{\pgfqpoint{6.200000in}{4.800000in}} %
\pgfusepath{clip}%
\pgfsetbuttcap%
\pgfsetroundjoin%
\pgfsetlinewidth{0.501875pt}%
\definecolor{currentstroke}{rgb}{0.000000,0.000000,0.000000}%
\pgfsetstrokecolor{currentstroke}%
\pgfsetdash{{1.000000pt}{3.000000pt}}{0.000000pt}%
\pgfpathmoveto{\pgfqpoint{1.000000in}{3.864028in}}%
\pgfpathlineto{\pgfqpoint{7.200000in}{3.864028in}}%
\pgfusepath{stroke}%
\end{pgfscope}%
\begin{pgfscope}%
\pgfsetbuttcap%
\pgfsetroundjoin%
\definecolor{currentfill}{rgb}{0.000000,0.000000,0.000000}%
\pgfsetfillcolor{currentfill}%
\pgfsetlinewidth{0.501875pt}%
\definecolor{currentstroke}{rgb}{0.000000,0.000000,0.000000}%
\pgfsetstrokecolor{currentstroke}%
\pgfsetdash{}{0pt}%
\pgfsys@defobject{currentmarker}{\pgfqpoint{0.000000in}{0.000000in}}{\pgfqpoint{0.055556in}{0.000000in}}{%
\pgfpathmoveto{\pgfqpoint{0.000000in}{0.000000in}}%
\pgfpathlineto{\pgfqpoint{0.055556in}{0.000000in}}%
\pgfusepath{stroke,fill}%
}%
\begin{pgfscope}%
\pgfsys@transformshift{1.000000in}{3.864028in}%
\pgfsys@useobject{currentmarker}{}%
\end{pgfscope}%
\end{pgfscope}%
\begin{pgfscope}%
\pgfsetbuttcap%
\pgfsetroundjoin%
\definecolor{currentfill}{rgb}{0.000000,0.000000,0.000000}%
\pgfsetfillcolor{currentfill}%
\pgfsetlinewidth{0.501875pt}%
\definecolor{currentstroke}{rgb}{0.000000,0.000000,0.000000}%
\pgfsetstrokecolor{currentstroke}%
\pgfsetdash{}{0pt}%
\pgfsys@defobject{currentmarker}{\pgfqpoint{-0.055556in}{0.000000in}}{\pgfqpoint{0.000000in}{0.000000in}}{%
\pgfpathmoveto{\pgfqpoint{0.000000in}{0.000000in}}%
\pgfpathlineto{\pgfqpoint{-0.055556in}{0.000000in}}%
\pgfusepath{stroke,fill}%
}%
\begin{pgfscope}%
\pgfsys@transformshift{7.200000in}{3.864028in}%
\pgfsys@useobject{currentmarker}{}%
\end{pgfscope}%
\end{pgfscope}%
\begin{pgfscope}%
\pgftext[x=0.944444in,y=3.864028in,right,]{\sffamily\fontsize{12.000000}{14.400000}\selectfont 20}%
\end{pgfscope}%
\begin{pgfscope}%
\pgfpathrectangle{\pgfqpoint{1.000000in}{0.600000in}}{\pgfqpoint{6.200000in}{4.800000in}} %
\pgfusepath{clip}%
\pgfsetbuttcap%
\pgfsetroundjoin%
\pgfsetlinewidth{0.501875pt}%
\definecolor{currentstroke}{rgb}{0.000000,0.000000,0.000000}%
\pgfsetstrokecolor{currentstroke}%
\pgfsetdash{{1.000000pt}{3.000000pt}}{0.000000pt}%
\pgfpathmoveto{\pgfqpoint{1.000000in}{4.513423in}}%
\pgfpathlineto{\pgfqpoint{7.200000in}{4.513423in}}%
\pgfusepath{stroke}%
\end{pgfscope}%
\begin{pgfscope}%
\pgfsetbuttcap%
\pgfsetroundjoin%
\definecolor{currentfill}{rgb}{0.000000,0.000000,0.000000}%
\pgfsetfillcolor{currentfill}%
\pgfsetlinewidth{0.501875pt}%
\definecolor{currentstroke}{rgb}{0.000000,0.000000,0.000000}%
\pgfsetstrokecolor{currentstroke}%
\pgfsetdash{}{0pt}%
\pgfsys@defobject{currentmarker}{\pgfqpoint{0.000000in}{0.000000in}}{\pgfqpoint{0.055556in}{0.000000in}}{%
\pgfpathmoveto{\pgfqpoint{0.000000in}{0.000000in}}%
\pgfpathlineto{\pgfqpoint{0.055556in}{0.000000in}}%
\pgfusepath{stroke,fill}%
}%
\begin{pgfscope}%
\pgfsys@transformshift{1.000000in}{4.513423in}%
\pgfsys@useobject{currentmarker}{}%
\end{pgfscope}%
\end{pgfscope}%
\begin{pgfscope}%
\pgfsetbuttcap%
\pgfsetroundjoin%
\definecolor{currentfill}{rgb}{0.000000,0.000000,0.000000}%
\pgfsetfillcolor{currentfill}%
\pgfsetlinewidth{0.501875pt}%
\definecolor{currentstroke}{rgb}{0.000000,0.000000,0.000000}%
\pgfsetstrokecolor{currentstroke}%
\pgfsetdash{}{0pt}%
\pgfsys@defobject{currentmarker}{\pgfqpoint{-0.055556in}{0.000000in}}{\pgfqpoint{0.000000in}{0.000000in}}{%
\pgfpathmoveto{\pgfqpoint{0.000000in}{0.000000in}}%
\pgfpathlineto{\pgfqpoint{-0.055556in}{0.000000in}}%
\pgfusepath{stroke,fill}%
}%
\begin{pgfscope}%
\pgfsys@transformshift{7.200000in}{4.513423in}%
\pgfsys@useobject{currentmarker}{}%
\end{pgfscope}%
\end{pgfscope}%
\begin{pgfscope}%
\pgftext[x=0.944444in,y=4.513423in,right,]{\sffamily\fontsize{12.000000}{14.400000}\selectfont 30}%
\end{pgfscope}%
\begin{pgfscope}%
\pgfpathrectangle{\pgfqpoint{1.000000in}{0.600000in}}{\pgfqpoint{6.200000in}{4.800000in}} %
\pgfusepath{clip}%
\pgfsetbuttcap%
\pgfsetroundjoin%
\pgfsetlinewidth{0.501875pt}%
\definecolor{currentstroke}{rgb}{0.000000,0.000000,0.000000}%
\pgfsetstrokecolor{currentstroke}%
\pgfsetdash{{1.000000pt}{3.000000pt}}{0.000000pt}%
\pgfpathmoveto{\pgfqpoint{1.000000in}{5.162819in}}%
\pgfpathlineto{\pgfqpoint{7.200000in}{5.162819in}}%
\pgfusepath{stroke}%
\end{pgfscope}%
\begin{pgfscope}%
\pgfsetbuttcap%
\pgfsetroundjoin%
\definecolor{currentfill}{rgb}{0.000000,0.000000,0.000000}%
\pgfsetfillcolor{currentfill}%
\pgfsetlinewidth{0.501875pt}%
\definecolor{currentstroke}{rgb}{0.000000,0.000000,0.000000}%
\pgfsetstrokecolor{currentstroke}%
\pgfsetdash{}{0pt}%
\pgfsys@defobject{currentmarker}{\pgfqpoint{0.000000in}{0.000000in}}{\pgfqpoint{0.055556in}{0.000000in}}{%
\pgfpathmoveto{\pgfqpoint{0.000000in}{0.000000in}}%
\pgfpathlineto{\pgfqpoint{0.055556in}{0.000000in}}%
\pgfusepath{stroke,fill}%
}%
\begin{pgfscope}%
\pgfsys@transformshift{1.000000in}{5.162819in}%
\pgfsys@useobject{currentmarker}{}%
\end{pgfscope}%
\end{pgfscope}%
\begin{pgfscope}%
\pgfsetbuttcap%
\pgfsetroundjoin%
\definecolor{currentfill}{rgb}{0.000000,0.000000,0.000000}%
\pgfsetfillcolor{currentfill}%
\pgfsetlinewidth{0.501875pt}%
\definecolor{currentstroke}{rgb}{0.000000,0.000000,0.000000}%
\pgfsetstrokecolor{currentstroke}%
\pgfsetdash{}{0pt}%
\pgfsys@defobject{currentmarker}{\pgfqpoint{-0.055556in}{0.000000in}}{\pgfqpoint{0.000000in}{0.000000in}}{%
\pgfpathmoveto{\pgfqpoint{0.000000in}{0.000000in}}%
\pgfpathlineto{\pgfqpoint{-0.055556in}{0.000000in}}%
\pgfusepath{stroke,fill}%
}%
\begin{pgfscope}%
\pgfsys@transformshift{7.200000in}{5.162819in}%
\pgfsys@useobject{currentmarker}{}%
\end{pgfscope}%
\end{pgfscope}%
\begin{pgfscope}%
\pgftext[x=0.944444in,y=5.162819in,right,]{\sffamily\fontsize{12.000000}{14.400000}\selectfont 40}%
\end{pgfscope}%
\begin{pgfscope}%
\pgftext[x=0.523275in,y=3.000000in,,bottom,rotate=90.000000]{\sffamily\fontsize{12.000000}{14.400000}\selectfont Reconstructed signal (dB)}%
\end{pgfscope}%
\end{pgfpicture}%
\makeatother%
\endgroup%
}
 \caption{Response of the reconstructed signal in the Doppler domain.}
 \label{fg:reconstructed}
 \end{center}
\end{figure}
The simulator then azimuth compresses this reconstructed signal with a computed chirp and transforms the data back into the spatial domain to produce the Point Spread Function (PSF) shown in \fgref{fg:psf}. This figure illustrates that the system achieves the desired resolution with the response width less than 0.1 m at the -5 dB level. While the figure also shows a second set of sidelobes at -15 dB, it is assumed that these can be reduced by an appropriate choice of weighting on the antenna patterns. 
\begin{figure}[h!]
\begin{center}
 \resizebox{\columnwidth}{!}{%% Creator: Matplotlib, PGF backend
%%
%% To include the figure in your LaTeX document, write
%%   \input{<filename>.pgf}
%%
%% Make sure the required packages are loaded in your preamble
%%   \usepackage{pgf}
%%
%% Figures using additional raster images can only be included by \input if
%% they are in the same directory as the main LaTeX file. For loading figures
%% from other directories you can use the `import` package
%%   \usepackage{import}
%% and then include the figures with
%%   \import{<path to file>}{<filename>.pgf}
%%
%% Matplotlib used the following preamble
%%   \usepackage{fontspec}
%%   \setmainfont{Bitstream Vera Serif}
%%   \setsansfont{Bitstream Vera Sans}
%%   \setmonofont{Bitstream Vera Sans Mono}
%%
\begingroup%
\makeatletter%
\begin{pgfpicture}%
\pgfpathrectangle{\pgfpointorigin}{\pgfqpoint{8.000000in}{6.000000in}}%
\pgfusepath{use as bounding box, clip}%
\begin{pgfscope}%
\pgfsetbuttcap%
\pgfsetmiterjoin%
\definecolor{currentfill}{rgb}{1.000000,1.000000,1.000000}%
\pgfsetfillcolor{currentfill}%
\pgfsetlinewidth{0.000000pt}%
\definecolor{currentstroke}{rgb}{1.000000,1.000000,1.000000}%
\pgfsetstrokecolor{currentstroke}%
\pgfsetdash{}{0pt}%
\pgfpathmoveto{\pgfqpoint{0.000000in}{0.000000in}}%
\pgfpathlineto{\pgfqpoint{8.000000in}{0.000000in}}%
\pgfpathlineto{\pgfqpoint{8.000000in}{6.000000in}}%
\pgfpathlineto{\pgfqpoint{0.000000in}{6.000000in}}%
\pgfpathclose%
\pgfusepath{fill}%
\end{pgfscope}%
\begin{pgfscope}%
\pgfsetbuttcap%
\pgfsetmiterjoin%
\definecolor{currentfill}{rgb}{1.000000,1.000000,1.000000}%
\pgfsetfillcolor{currentfill}%
\pgfsetlinewidth{0.000000pt}%
\definecolor{currentstroke}{rgb}{0.000000,0.000000,0.000000}%
\pgfsetstrokecolor{currentstroke}%
\pgfsetstrokeopacity{0.000000}%
\pgfsetdash{}{0pt}%
\pgfpathmoveto{\pgfqpoint{1.000000in}{0.600000in}}%
\pgfpathlineto{\pgfqpoint{7.200000in}{0.600000in}}%
\pgfpathlineto{\pgfqpoint{7.200000in}{5.400000in}}%
\pgfpathlineto{\pgfqpoint{1.000000in}{5.400000in}}%
\pgfpathclose%
\pgfusepath{fill}%
\end{pgfscope}%
\begin{pgfscope}%
\pgfpathrectangle{\pgfqpoint{1.000000in}{0.600000in}}{\pgfqpoint{6.200000in}{4.800000in}} %
\pgfusepath{clip}%
\pgfsetrectcap%
\pgfsetroundjoin%
\pgfsetlinewidth{1.003750pt}%
\definecolor{currentstroke}{rgb}{0.000000,0.000000,1.000000}%
\pgfsetstrokecolor{currentstroke}%
\pgfsetdash{}{0pt}%
\pgfpathmoveto{\pgfqpoint{1.605699in}{0.590000in}}%
\pgfpathlineto{\pgfqpoint{1.613792in}{0.767007in}}%
\pgfpathlineto{\pgfqpoint{1.674902in}{2.337208in}}%
\pgfpathlineto{\pgfqpoint{1.736012in}{3.021159in}}%
\pgfpathlineto{\pgfqpoint{1.797122in}{3.218588in}}%
\pgfpathlineto{\pgfqpoint{1.858233in}{2.946978in}}%
\pgfpathlineto{\pgfqpoint{1.919343in}{1.912511in}}%
\pgfpathlineto{\pgfqpoint{1.963082in}{0.590000in}}%
\pgfpathmoveto{\pgfqpoint{1.998807in}{0.590000in}}%
\pgfpathlineto{\pgfqpoint{2.041563in}{1.813461in}}%
\pgfpathlineto{\pgfqpoint{2.102673in}{1.529751in}}%
\pgfpathlineto{\pgfqpoint{2.111518in}{0.590000in}}%
\pgfpathmoveto{\pgfqpoint{2.217419in}{0.590000in}}%
\pgfpathlineto{\pgfqpoint{2.224893in}{1.363833in}}%
\pgfpathlineto{\pgfqpoint{2.286003in}{1.496767in}}%
\pgfpathlineto{\pgfqpoint{2.315893in}{0.590000in}}%
\pgfpathmoveto{\pgfqpoint{2.389828in}{0.590000in}}%
\pgfpathlineto{\pgfqpoint{2.408223in}{0.997892in}}%
\pgfpathlineto{\pgfqpoint{2.469334in}{1.524317in}}%
\pgfpathlineto{\pgfqpoint{2.523901in}{0.590000in}}%
\pgfpathmoveto{\pgfqpoint{2.592413in}{0.590000in}}%
\pgfpathlineto{\pgfqpoint{2.652664in}{1.562893in}}%
\pgfpathlineto{\pgfqpoint{2.713774in}{0.986686in}}%
\pgfpathlineto{\pgfqpoint{2.737210in}{0.590000in}}%
\pgfpathmoveto{\pgfqpoint{2.798427in}{0.590000in}}%
\pgfpathlineto{\pgfqpoint{2.835994in}{1.607564in}}%
\pgfpathlineto{\pgfqpoint{2.897104in}{1.387944in}}%
\pgfpathlineto{\pgfqpoint{2.913760in}{0.590000in}}%
\pgfpathmoveto{\pgfqpoint{2.998860in}{0.590000in}}%
\pgfpathlineto{\pgfqpoint{3.019324in}{1.662320in}}%
\pgfpathlineto{\pgfqpoint{3.080434in}{1.761003in}}%
\pgfpathlineto{\pgfqpoint{3.107588in}{0.590000in}}%
\pgfpathmoveto{\pgfqpoint{3.175893in}{0.590000in}}%
\pgfpathlineto{\pgfqpoint{3.202655in}{1.730885in}}%
\pgfpathlineto{\pgfqpoint{3.263765in}{2.153909in}}%
\pgfpathlineto{\pgfqpoint{3.324875in}{0.767018in}}%
\pgfpathlineto{\pgfqpoint{3.385985in}{1.812709in}}%
\pgfpathlineto{\pgfqpoint{3.447095in}{2.614036in}}%
\pgfpathlineto{\pgfqpoint{3.508205in}{1.862276in}}%
\pgfpathlineto{\pgfqpoint{3.569315in}{1.909735in}}%
\pgfpathlineto{\pgfqpoint{3.630425in}{3.236603in}}%
\pgfpathlineto{\pgfqpoint{3.691535in}{3.006035in}}%
\pgfpathlineto{\pgfqpoint{3.752645in}{2.090406in}}%
\pgfpathlineto{\pgfqpoint{3.813756in}{4.428767in}}%
\pgfpathlineto{\pgfqpoint{3.874866in}{5.184629in}}%
\pgfpathlineto{\pgfqpoint{3.935976in}{5.400000in}}%
\pgfpathlineto{\pgfqpoint{3.997086in}{5.184626in}}%
\pgfpathlineto{\pgfqpoint{4.058196in}{4.428762in}}%
\pgfpathlineto{\pgfqpoint{4.119306in}{2.090423in}}%
\pgfpathlineto{\pgfqpoint{4.180416in}{3.005994in}}%
\pgfpathlineto{\pgfqpoint{4.241526in}{3.236551in}}%
\pgfpathlineto{\pgfqpoint{4.302636in}{1.909609in}}%
\pgfpathlineto{\pgfqpoint{4.363746in}{1.862335in}}%
\pgfpathlineto{\pgfqpoint{4.424857in}{2.614027in}}%
\pgfpathlineto{\pgfqpoint{4.485967in}{1.812668in}}%
\pgfpathlineto{\pgfqpoint{4.547077in}{0.766990in}}%
\pgfpathlineto{\pgfqpoint{4.608187in}{2.153817in}}%
\pgfpathlineto{\pgfqpoint{4.669297in}{1.730652in}}%
\pgfpathlineto{\pgfqpoint{4.696075in}{0.590000in}}%
\pgfpathmoveto{\pgfqpoint{4.764341in}{0.590000in}}%
\pgfpathlineto{\pgfqpoint{4.791517in}{1.761217in}}%
\pgfpathlineto{\pgfqpoint{4.852627in}{1.662408in}}%
\pgfpathlineto{\pgfqpoint{4.873088in}{0.590000in}}%
\pgfpathmoveto{\pgfqpoint{4.958193in}{0.590000in}}%
\pgfpathlineto{\pgfqpoint{4.974847in}{1.388137in}}%
\pgfpathlineto{\pgfqpoint{5.035957in}{1.607698in}}%
\pgfpathlineto{\pgfqpoint{5.073527in}{0.590000in}}%
\pgfpathmoveto{\pgfqpoint{5.134733in}{0.590000in}}%
\pgfpathlineto{\pgfqpoint{5.158178in}{0.986916in}}%
\pgfpathlineto{\pgfqpoint{5.219288in}{1.563125in}}%
\pgfpathlineto{\pgfqpoint{5.279565in}{0.590000in}}%
\pgfpathmoveto{\pgfqpoint{5.348060in}{0.590000in}}%
\pgfpathlineto{\pgfqpoint{5.402618in}{1.524382in}}%
\pgfpathlineto{\pgfqpoint{5.463728in}{0.998184in}}%
\pgfpathlineto{\pgfqpoint{5.482118in}{0.590000in}}%
\pgfpathmoveto{\pgfqpoint{5.556082in}{0.590000in}}%
\pgfpathlineto{\pgfqpoint{5.585948in}{1.496450in}}%
\pgfpathlineto{\pgfqpoint{5.647058in}{1.363474in}}%
\pgfpathlineto{\pgfqpoint{5.654577in}{0.590000in}}%
\pgfpathmoveto{\pgfqpoint{5.760377in}{0.590000in}}%
\pgfpathlineto{\pgfqpoint{5.769279in}{1.530052in}}%
\pgfpathlineto{\pgfqpoint{5.830389in}{1.813663in}}%
\pgfpathlineto{\pgfqpoint{5.873157in}{0.590000in}}%
\pgfpathmoveto{\pgfqpoint{5.908859in}{0.590000in}}%
\pgfpathlineto{\pgfqpoint{5.952609in}{1.912504in}}%
\pgfpathlineto{\pgfqpoint{6.013719in}{2.947010in}}%
\pgfpathlineto{\pgfqpoint{6.074829in}{3.218619in}}%
\pgfpathlineto{\pgfqpoint{6.135939in}{3.021167in}}%
\pgfpathlineto{\pgfqpoint{6.197049in}{2.337129in}}%
\pgfpathlineto{\pgfqpoint{6.258159in}{0.766405in}}%
\pgfpathlineto{\pgfqpoint{6.266242in}{0.590000in}}%
\pgfusepath{stroke}%
\end{pgfscope}%
\begin{pgfscope}%
\pgfsetrectcap%
\pgfsetmiterjoin%
\pgfsetlinewidth{1.003750pt}%
\definecolor{currentstroke}{rgb}{0.000000,0.000000,0.000000}%
\pgfsetstrokecolor{currentstroke}%
\pgfsetdash{}{0pt}%
\pgfpathmoveto{\pgfqpoint{1.000000in}{5.400000in}}%
\pgfpathlineto{\pgfqpoint{7.200000in}{5.400000in}}%
\pgfusepath{stroke}%
\end{pgfscope}%
\begin{pgfscope}%
\pgfsetrectcap%
\pgfsetmiterjoin%
\pgfsetlinewidth{1.003750pt}%
\definecolor{currentstroke}{rgb}{0.000000,0.000000,0.000000}%
\pgfsetstrokecolor{currentstroke}%
\pgfsetdash{}{0pt}%
\pgfpathmoveto{\pgfqpoint{7.200000in}{0.600000in}}%
\pgfpathlineto{\pgfqpoint{7.200000in}{5.400000in}}%
\pgfusepath{stroke}%
\end{pgfscope}%
\begin{pgfscope}%
\pgfsetrectcap%
\pgfsetmiterjoin%
\pgfsetlinewidth{1.003750pt}%
\definecolor{currentstroke}{rgb}{0.000000,0.000000,0.000000}%
\pgfsetstrokecolor{currentstroke}%
\pgfsetdash{}{0pt}%
\pgfpathmoveto{\pgfqpoint{1.000000in}{0.600000in}}%
\pgfpathlineto{\pgfqpoint{7.200000in}{0.600000in}}%
\pgfusepath{stroke}%
\end{pgfscope}%
\begin{pgfscope}%
\pgfsetrectcap%
\pgfsetmiterjoin%
\pgfsetlinewidth{1.003750pt}%
\definecolor{currentstroke}{rgb}{0.000000,0.000000,0.000000}%
\pgfsetstrokecolor{currentstroke}%
\pgfsetdash{}{0pt}%
\pgfpathmoveto{\pgfqpoint{1.000000in}{0.600000in}}%
\pgfpathlineto{\pgfqpoint{1.000000in}{5.400000in}}%
\pgfusepath{stroke}%
\end{pgfscope}%
\begin{pgfscope}%
\pgfpathrectangle{\pgfqpoint{1.000000in}{0.600000in}}{\pgfqpoint{6.200000in}{4.800000in}} %
\pgfusepath{clip}%
\pgfsetbuttcap%
\pgfsetroundjoin%
\pgfsetlinewidth{0.501875pt}%
\definecolor{currentstroke}{rgb}{0.000000,0.000000,0.000000}%
\pgfsetstrokecolor{currentstroke}%
\pgfsetdash{{1.000000pt}{3.000000pt}}{0.000000pt}%
\pgfpathmoveto{\pgfqpoint{1.966782in}{0.600000in}}%
\pgfpathlineto{\pgfqpoint{1.966782in}{5.400000in}}%
\pgfusepath{stroke}%
\end{pgfscope}%
\begin{pgfscope}%
\pgfsetbuttcap%
\pgfsetroundjoin%
\definecolor{currentfill}{rgb}{0.000000,0.000000,0.000000}%
\pgfsetfillcolor{currentfill}%
\pgfsetlinewidth{0.501875pt}%
\definecolor{currentstroke}{rgb}{0.000000,0.000000,0.000000}%
\pgfsetstrokecolor{currentstroke}%
\pgfsetdash{}{0pt}%
\pgfsys@defobject{currentmarker}{\pgfqpoint{0.000000in}{0.000000in}}{\pgfqpoint{0.000000in}{0.055556in}}{%
\pgfpathmoveto{\pgfqpoint{0.000000in}{0.000000in}}%
\pgfpathlineto{\pgfqpoint{0.000000in}{0.055556in}}%
\pgfusepath{stroke,fill}%
}%
\begin{pgfscope}%
\pgfsys@transformshift{1.966782in}{0.600000in}%
\pgfsys@useobject{currentmarker}{}%
\end{pgfscope}%
\end{pgfscope}%
\begin{pgfscope}%
\pgfsetbuttcap%
\pgfsetroundjoin%
\definecolor{currentfill}{rgb}{0.000000,0.000000,0.000000}%
\pgfsetfillcolor{currentfill}%
\pgfsetlinewidth{0.501875pt}%
\definecolor{currentstroke}{rgb}{0.000000,0.000000,0.000000}%
\pgfsetstrokecolor{currentstroke}%
\pgfsetdash{}{0pt}%
\pgfsys@defobject{currentmarker}{\pgfqpoint{0.000000in}{-0.055556in}}{\pgfqpoint{0.000000in}{0.000000in}}{%
\pgfpathmoveto{\pgfqpoint{0.000000in}{0.000000in}}%
\pgfpathlineto{\pgfqpoint{0.000000in}{-0.055556in}}%
\pgfusepath{stroke,fill}%
}%
\begin{pgfscope}%
\pgfsys@transformshift{1.966782in}{5.400000in}%
\pgfsys@useobject{currentmarker}{}%
\end{pgfscope}%
\end{pgfscope}%
\begin{pgfscope}%
\pgftext[x=1.966782in,y=0.544444in,,top]{\sffamily\fontsize{12.000000}{14.400000}\selectfont −1.0}%
\end{pgfscope}%
\begin{pgfscope}%
\pgfpathrectangle{\pgfqpoint{1.000000in}{0.600000in}}{\pgfqpoint{6.200000in}{4.800000in}} %
\pgfusepath{clip}%
\pgfsetbuttcap%
\pgfsetroundjoin%
\pgfsetlinewidth{0.501875pt}%
\definecolor{currentstroke}{rgb}{0.000000,0.000000,0.000000}%
\pgfsetstrokecolor{currentstroke}%
\pgfsetdash{{1.000000pt}{3.000000pt}}{0.000000pt}%
\pgfpathmoveto{\pgfqpoint{2.951379in}{0.600000in}}%
\pgfpathlineto{\pgfqpoint{2.951379in}{5.400000in}}%
\pgfusepath{stroke}%
\end{pgfscope}%
\begin{pgfscope}%
\pgfsetbuttcap%
\pgfsetroundjoin%
\definecolor{currentfill}{rgb}{0.000000,0.000000,0.000000}%
\pgfsetfillcolor{currentfill}%
\pgfsetlinewidth{0.501875pt}%
\definecolor{currentstroke}{rgb}{0.000000,0.000000,0.000000}%
\pgfsetstrokecolor{currentstroke}%
\pgfsetdash{}{0pt}%
\pgfsys@defobject{currentmarker}{\pgfqpoint{0.000000in}{0.000000in}}{\pgfqpoint{0.000000in}{0.055556in}}{%
\pgfpathmoveto{\pgfqpoint{0.000000in}{0.000000in}}%
\pgfpathlineto{\pgfqpoint{0.000000in}{0.055556in}}%
\pgfusepath{stroke,fill}%
}%
\begin{pgfscope}%
\pgfsys@transformshift{2.951379in}{0.600000in}%
\pgfsys@useobject{currentmarker}{}%
\end{pgfscope}%
\end{pgfscope}%
\begin{pgfscope}%
\pgfsetbuttcap%
\pgfsetroundjoin%
\definecolor{currentfill}{rgb}{0.000000,0.000000,0.000000}%
\pgfsetfillcolor{currentfill}%
\pgfsetlinewidth{0.501875pt}%
\definecolor{currentstroke}{rgb}{0.000000,0.000000,0.000000}%
\pgfsetstrokecolor{currentstroke}%
\pgfsetdash{}{0pt}%
\pgfsys@defobject{currentmarker}{\pgfqpoint{0.000000in}{-0.055556in}}{\pgfqpoint{0.000000in}{0.000000in}}{%
\pgfpathmoveto{\pgfqpoint{0.000000in}{0.000000in}}%
\pgfpathlineto{\pgfqpoint{0.000000in}{-0.055556in}}%
\pgfusepath{stroke,fill}%
}%
\begin{pgfscope}%
\pgfsys@transformshift{2.951379in}{5.400000in}%
\pgfsys@useobject{currentmarker}{}%
\end{pgfscope}%
\end{pgfscope}%
\begin{pgfscope}%
\pgftext[x=2.951379in,y=0.544444in,,top]{\sffamily\fontsize{12.000000}{14.400000}\selectfont −0.5}%
\end{pgfscope}%
\begin{pgfscope}%
\pgfpathrectangle{\pgfqpoint{1.000000in}{0.600000in}}{\pgfqpoint{6.200000in}{4.800000in}} %
\pgfusepath{clip}%
\pgfsetbuttcap%
\pgfsetroundjoin%
\pgfsetlinewidth{0.501875pt}%
\definecolor{currentstroke}{rgb}{0.000000,0.000000,0.000000}%
\pgfsetstrokecolor{currentstroke}%
\pgfsetdash{{1.000000pt}{3.000000pt}}{0.000000pt}%
\pgfpathmoveto{\pgfqpoint{3.935976in}{0.600000in}}%
\pgfpathlineto{\pgfqpoint{3.935976in}{5.400000in}}%
\pgfusepath{stroke}%
\end{pgfscope}%
\begin{pgfscope}%
\pgfsetbuttcap%
\pgfsetroundjoin%
\definecolor{currentfill}{rgb}{0.000000,0.000000,0.000000}%
\pgfsetfillcolor{currentfill}%
\pgfsetlinewidth{0.501875pt}%
\definecolor{currentstroke}{rgb}{0.000000,0.000000,0.000000}%
\pgfsetstrokecolor{currentstroke}%
\pgfsetdash{}{0pt}%
\pgfsys@defobject{currentmarker}{\pgfqpoint{0.000000in}{0.000000in}}{\pgfqpoint{0.000000in}{0.055556in}}{%
\pgfpathmoveto{\pgfqpoint{0.000000in}{0.000000in}}%
\pgfpathlineto{\pgfqpoint{0.000000in}{0.055556in}}%
\pgfusepath{stroke,fill}%
}%
\begin{pgfscope}%
\pgfsys@transformshift{3.935976in}{0.600000in}%
\pgfsys@useobject{currentmarker}{}%
\end{pgfscope}%
\end{pgfscope}%
\begin{pgfscope}%
\pgfsetbuttcap%
\pgfsetroundjoin%
\definecolor{currentfill}{rgb}{0.000000,0.000000,0.000000}%
\pgfsetfillcolor{currentfill}%
\pgfsetlinewidth{0.501875pt}%
\definecolor{currentstroke}{rgb}{0.000000,0.000000,0.000000}%
\pgfsetstrokecolor{currentstroke}%
\pgfsetdash{}{0pt}%
\pgfsys@defobject{currentmarker}{\pgfqpoint{0.000000in}{-0.055556in}}{\pgfqpoint{0.000000in}{0.000000in}}{%
\pgfpathmoveto{\pgfqpoint{0.000000in}{0.000000in}}%
\pgfpathlineto{\pgfqpoint{0.000000in}{-0.055556in}}%
\pgfusepath{stroke,fill}%
}%
\begin{pgfscope}%
\pgfsys@transformshift{3.935976in}{5.400000in}%
\pgfsys@useobject{currentmarker}{}%
\end{pgfscope}%
\end{pgfscope}%
\begin{pgfscope}%
\pgftext[x=3.935976in,y=0.544444in,,top]{\sffamily\fontsize{12.000000}{14.400000}\selectfont 0.0}%
\end{pgfscope}%
\begin{pgfscope}%
\pgfpathrectangle{\pgfqpoint{1.000000in}{0.600000in}}{\pgfqpoint{6.200000in}{4.800000in}} %
\pgfusepath{clip}%
\pgfsetbuttcap%
\pgfsetroundjoin%
\pgfsetlinewidth{0.501875pt}%
\definecolor{currentstroke}{rgb}{0.000000,0.000000,0.000000}%
\pgfsetstrokecolor{currentstroke}%
\pgfsetdash{{1.000000pt}{3.000000pt}}{0.000000pt}%
\pgfpathmoveto{\pgfqpoint{4.920573in}{0.600000in}}%
\pgfpathlineto{\pgfqpoint{4.920573in}{5.400000in}}%
\pgfusepath{stroke}%
\end{pgfscope}%
\begin{pgfscope}%
\pgfsetbuttcap%
\pgfsetroundjoin%
\definecolor{currentfill}{rgb}{0.000000,0.000000,0.000000}%
\pgfsetfillcolor{currentfill}%
\pgfsetlinewidth{0.501875pt}%
\definecolor{currentstroke}{rgb}{0.000000,0.000000,0.000000}%
\pgfsetstrokecolor{currentstroke}%
\pgfsetdash{}{0pt}%
\pgfsys@defobject{currentmarker}{\pgfqpoint{0.000000in}{0.000000in}}{\pgfqpoint{0.000000in}{0.055556in}}{%
\pgfpathmoveto{\pgfqpoint{0.000000in}{0.000000in}}%
\pgfpathlineto{\pgfqpoint{0.000000in}{0.055556in}}%
\pgfusepath{stroke,fill}%
}%
\begin{pgfscope}%
\pgfsys@transformshift{4.920573in}{0.600000in}%
\pgfsys@useobject{currentmarker}{}%
\end{pgfscope}%
\end{pgfscope}%
\begin{pgfscope}%
\pgfsetbuttcap%
\pgfsetroundjoin%
\definecolor{currentfill}{rgb}{0.000000,0.000000,0.000000}%
\pgfsetfillcolor{currentfill}%
\pgfsetlinewidth{0.501875pt}%
\definecolor{currentstroke}{rgb}{0.000000,0.000000,0.000000}%
\pgfsetstrokecolor{currentstroke}%
\pgfsetdash{}{0pt}%
\pgfsys@defobject{currentmarker}{\pgfqpoint{0.000000in}{-0.055556in}}{\pgfqpoint{0.000000in}{0.000000in}}{%
\pgfpathmoveto{\pgfqpoint{0.000000in}{0.000000in}}%
\pgfpathlineto{\pgfqpoint{0.000000in}{-0.055556in}}%
\pgfusepath{stroke,fill}%
}%
\begin{pgfscope}%
\pgfsys@transformshift{4.920573in}{5.400000in}%
\pgfsys@useobject{currentmarker}{}%
\end{pgfscope}%
\end{pgfscope}%
\begin{pgfscope}%
\pgftext[x=4.920573in,y=0.544444in,,top]{\sffamily\fontsize{12.000000}{14.400000}\selectfont 0.5}%
\end{pgfscope}%
\begin{pgfscope}%
\pgfpathrectangle{\pgfqpoint{1.000000in}{0.600000in}}{\pgfqpoint{6.200000in}{4.800000in}} %
\pgfusepath{clip}%
\pgfsetbuttcap%
\pgfsetroundjoin%
\pgfsetlinewidth{0.501875pt}%
\definecolor{currentstroke}{rgb}{0.000000,0.000000,0.000000}%
\pgfsetstrokecolor{currentstroke}%
\pgfsetdash{{1.000000pt}{3.000000pt}}{0.000000pt}%
\pgfpathmoveto{\pgfqpoint{5.905169in}{0.600000in}}%
\pgfpathlineto{\pgfqpoint{5.905169in}{5.400000in}}%
\pgfusepath{stroke}%
\end{pgfscope}%
\begin{pgfscope}%
\pgfsetbuttcap%
\pgfsetroundjoin%
\definecolor{currentfill}{rgb}{0.000000,0.000000,0.000000}%
\pgfsetfillcolor{currentfill}%
\pgfsetlinewidth{0.501875pt}%
\definecolor{currentstroke}{rgb}{0.000000,0.000000,0.000000}%
\pgfsetstrokecolor{currentstroke}%
\pgfsetdash{}{0pt}%
\pgfsys@defobject{currentmarker}{\pgfqpoint{0.000000in}{0.000000in}}{\pgfqpoint{0.000000in}{0.055556in}}{%
\pgfpathmoveto{\pgfqpoint{0.000000in}{0.000000in}}%
\pgfpathlineto{\pgfqpoint{0.000000in}{0.055556in}}%
\pgfusepath{stroke,fill}%
}%
\begin{pgfscope}%
\pgfsys@transformshift{5.905169in}{0.600000in}%
\pgfsys@useobject{currentmarker}{}%
\end{pgfscope}%
\end{pgfscope}%
\begin{pgfscope}%
\pgfsetbuttcap%
\pgfsetroundjoin%
\definecolor{currentfill}{rgb}{0.000000,0.000000,0.000000}%
\pgfsetfillcolor{currentfill}%
\pgfsetlinewidth{0.501875pt}%
\definecolor{currentstroke}{rgb}{0.000000,0.000000,0.000000}%
\pgfsetstrokecolor{currentstroke}%
\pgfsetdash{}{0pt}%
\pgfsys@defobject{currentmarker}{\pgfqpoint{0.000000in}{-0.055556in}}{\pgfqpoint{0.000000in}{0.000000in}}{%
\pgfpathmoveto{\pgfqpoint{0.000000in}{0.000000in}}%
\pgfpathlineto{\pgfqpoint{0.000000in}{-0.055556in}}%
\pgfusepath{stroke,fill}%
}%
\begin{pgfscope}%
\pgfsys@transformshift{5.905169in}{5.400000in}%
\pgfsys@useobject{currentmarker}{}%
\end{pgfscope}%
\end{pgfscope}%
\begin{pgfscope}%
\pgftext[x=5.905169in,y=0.544444in,,top]{\sffamily\fontsize{12.000000}{14.400000}\selectfont 1.0}%
\end{pgfscope}%
\begin{pgfscope}%
\pgfpathrectangle{\pgfqpoint{1.000000in}{0.600000in}}{\pgfqpoint{6.200000in}{4.800000in}} %
\pgfusepath{clip}%
\pgfsetbuttcap%
\pgfsetroundjoin%
\pgfsetlinewidth{0.501875pt}%
\definecolor{currentstroke}{rgb}{0.000000,0.000000,0.000000}%
\pgfsetstrokecolor{currentstroke}%
\pgfsetdash{{1.000000pt}{3.000000pt}}{0.000000pt}%
\pgfpathmoveto{\pgfqpoint{6.889766in}{0.600000in}}%
\pgfpathlineto{\pgfqpoint{6.889766in}{5.400000in}}%
\pgfusepath{stroke}%
\end{pgfscope}%
\begin{pgfscope}%
\pgfsetbuttcap%
\pgfsetroundjoin%
\definecolor{currentfill}{rgb}{0.000000,0.000000,0.000000}%
\pgfsetfillcolor{currentfill}%
\pgfsetlinewidth{0.501875pt}%
\definecolor{currentstroke}{rgb}{0.000000,0.000000,0.000000}%
\pgfsetstrokecolor{currentstroke}%
\pgfsetdash{}{0pt}%
\pgfsys@defobject{currentmarker}{\pgfqpoint{0.000000in}{0.000000in}}{\pgfqpoint{0.000000in}{0.055556in}}{%
\pgfpathmoveto{\pgfqpoint{0.000000in}{0.000000in}}%
\pgfpathlineto{\pgfqpoint{0.000000in}{0.055556in}}%
\pgfusepath{stroke,fill}%
}%
\begin{pgfscope}%
\pgfsys@transformshift{6.889766in}{0.600000in}%
\pgfsys@useobject{currentmarker}{}%
\end{pgfscope}%
\end{pgfscope}%
\begin{pgfscope}%
\pgfsetbuttcap%
\pgfsetroundjoin%
\definecolor{currentfill}{rgb}{0.000000,0.000000,0.000000}%
\pgfsetfillcolor{currentfill}%
\pgfsetlinewidth{0.501875pt}%
\definecolor{currentstroke}{rgb}{0.000000,0.000000,0.000000}%
\pgfsetstrokecolor{currentstroke}%
\pgfsetdash{}{0pt}%
\pgfsys@defobject{currentmarker}{\pgfqpoint{0.000000in}{-0.055556in}}{\pgfqpoint{0.000000in}{0.000000in}}{%
\pgfpathmoveto{\pgfqpoint{0.000000in}{0.000000in}}%
\pgfpathlineto{\pgfqpoint{0.000000in}{-0.055556in}}%
\pgfusepath{stroke,fill}%
}%
\begin{pgfscope}%
\pgfsys@transformshift{6.889766in}{5.400000in}%
\pgfsys@useobject{currentmarker}{}%
\end{pgfscope}%
\end{pgfscope}%
\begin{pgfscope}%
\pgftext[x=6.889766in,y=0.544444in,,top]{\sffamily\fontsize{12.000000}{14.400000}\selectfont 1.5}%
\end{pgfscope}%
\begin{pgfscope}%
\pgftext[x=4.100000in,y=0.313705in,,top]{\sffamily\fontsize{12.000000}{14.400000}\selectfont xPos (m)}%
\end{pgfscope}%
\begin{pgfscope}%
\pgfpathrectangle{\pgfqpoint{1.000000in}{0.600000in}}{\pgfqpoint{6.200000in}{4.800000in}} %
\pgfusepath{clip}%
\pgfsetbuttcap%
\pgfsetroundjoin%
\pgfsetlinewidth{0.501875pt}%
\definecolor{currentstroke}{rgb}{0.000000,0.000000,0.000000}%
\pgfsetstrokecolor{currentstroke}%
\pgfsetdash{{1.000000pt}{3.000000pt}}{0.000000pt}%
\pgfpathmoveto{\pgfqpoint{1.000000in}{0.781499in}}%
\pgfpathlineto{\pgfqpoint{7.200000in}{0.781499in}}%
\pgfusepath{stroke}%
\end{pgfscope}%
\begin{pgfscope}%
\pgfsetbuttcap%
\pgfsetroundjoin%
\definecolor{currentfill}{rgb}{0.000000,0.000000,0.000000}%
\pgfsetfillcolor{currentfill}%
\pgfsetlinewidth{0.501875pt}%
\definecolor{currentstroke}{rgb}{0.000000,0.000000,0.000000}%
\pgfsetstrokecolor{currentstroke}%
\pgfsetdash{}{0pt}%
\pgfsys@defobject{currentmarker}{\pgfqpoint{0.000000in}{0.000000in}}{\pgfqpoint{0.055556in}{0.000000in}}{%
\pgfpathmoveto{\pgfqpoint{0.000000in}{0.000000in}}%
\pgfpathlineto{\pgfqpoint{0.055556in}{0.000000in}}%
\pgfusepath{stroke,fill}%
}%
\begin{pgfscope}%
\pgfsys@transformshift{1.000000in}{0.781499in}%
\pgfsys@useobject{currentmarker}{}%
\end{pgfscope}%
\end{pgfscope}%
\begin{pgfscope}%
\pgfsetbuttcap%
\pgfsetroundjoin%
\definecolor{currentfill}{rgb}{0.000000,0.000000,0.000000}%
\pgfsetfillcolor{currentfill}%
\pgfsetlinewidth{0.501875pt}%
\definecolor{currentstroke}{rgb}{0.000000,0.000000,0.000000}%
\pgfsetstrokecolor{currentstroke}%
\pgfsetdash{}{0pt}%
\pgfsys@defobject{currentmarker}{\pgfqpoint{-0.055556in}{0.000000in}}{\pgfqpoint{0.000000in}{0.000000in}}{%
\pgfpathmoveto{\pgfqpoint{0.000000in}{0.000000in}}%
\pgfpathlineto{\pgfqpoint{-0.055556in}{0.000000in}}%
\pgfusepath{stroke,fill}%
}%
\begin{pgfscope}%
\pgfsys@transformshift{7.200000in}{0.781499in}%
\pgfsys@useobject{currentmarker}{}%
\end{pgfscope}%
\end{pgfscope}%
\begin{pgfscope}%
\pgftext[x=0.944444in,y=0.781499in,right,]{\sffamily\fontsize{12.000000}{14.400000}\selectfont −30}%
\end{pgfscope}%
\begin{pgfscope}%
\pgfpathrectangle{\pgfqpoint{1.000000in}{0.600000in}}{\pgfqpoint{6.200000in}{4.800000in}} %
\pgfusepath{clip}%
\pgfsetbuttcap%
\pgfsetroundjoin%
\pgfsetlinewidth{0.501875pt}%
\definecolor{currentstroke}{rgb}{0.000000,0.000000,0.000000}%
\pgfsetstrokecolor{currentstroke}%
\pgfsetdash{{1.000000pt}{3.000000pt}}{0.000000pt}%
\pgfpathmoveto{\pgfqpoint{1.000000in}{1.551249in}}%
\pgfpathlineto{\pgfqpoint{7.200000in}{1.551249in}}%
\pgfusepath{stroke}%
\end{pgfscope}%
\begin{pgfscope}%
\pgfsetbuttcap%
\pgfsetroundjoin%
\definecolor{currentfill}{rgb}{0.000000,0.000000,0.000000}%
\pgfsetfillcolor{currentfill}%
\pgfsetlinewidth{0.501875pt}%
\definecolor{currentstroke}{rgb}{0.000000,0.000000,0.000000}%
\pgfsetstrokecolor{currentstroke}%
\pgfsetdash{}{0pt}%
\pgfsys@defobject{currentmarker}{\pgfqpoint{0.000000in}{0.000000in}}{\pgfqpoint{0.055556in}{0.000000in}}{%
\pgfpathmoveto{\pgfqpoint{0.000000in}{0.000000in}}%
\pgfpathlineto{\pgfqpoint{0.055556in}{0.000000in}}%
\pgfusepath{stroke,fill}%
}%
\begin{pgfscope}%
\pgfsys@transformshift{1.000000in}{1.551249in}%
\pgfsys@useobject{currentmarker}{}%
\end{pgfscope}%
\end{pgfscope}%
\begin{pgfscope}%
\pgfsetbuttcap%
\pgfsetroundjoin%
\definecolor{currentfill}{rgb}{0.000000,0.000000,0.000000}%
\pgfsetfillcolor{currentfill}%
\pgfsetlinewidth{0.501875pt}%
\definecolor{currentstroke}{rgb}{0.000000,0.000000,0.000000}%
\pgfsetstrokecolor{currentstroke}%
\pgfsetdash{}{0pt}%
\pgfsys@defobject{currentmarker}{\pgfqpoint{-0.055556in}{0.000000in}}{\pgfqpoint{0.000000in}{0.000000in}}{%
\pgfpathmoveto{\pgfqpoint{0.000000in}{0.000000in}}%
\pgfpathlineto{\pgfqpoint{-0.055556in}{0.000000in}}%
\pgfusepath{stroke,fill}%
}%
\begin{pgfscope}%
\pgfsys@transformshift{7.200000in}{1.551249in}%
\pgfsys@useobject{currentmarker}{}%
\end{pgfscope}%
\end{pgfscope}%
\begin{pgfscope}%
\pgftext[x=0.944444in,y=1.551249in,right,]{\sffamily\fontsize{12.000000}{14.400000}\selectfont −25}%
\end{pgfscope}%
\begin{pgfscope}%
\pgfpathrectangle{\pgfqpoint{1.000000in}{0.600000in}}{\pgfqpoint{6.200000in}{4.800000in}} %
\pgfusepath{clip}%
\pgfsetbuttcap%
\pgfsetroundjoin%
\pgfsetlinewidth{0.501875pt}%
\definecolor{currentstroke}{rgb}{0.000000,0.000000,0.000000}%
\pgfsetstrokecolor{currentstroke}%
\pgfsetdash{{1.000000pt}{3.000000pt}}{0.000000pt}%
\pgfpathmoveto{\pgfqpoint{1.000000in}{2.320999in}}%
\pgfpathlineto{\pgfqpoint{7.200000in}{2.320999in}}%
\pgfusepath{stroke}%
\end{pgfscope}%
\begin{pgfscope}%
\pgfsetbuttcap%
\pgfsetroundjoin%
\definecolor{currentfill}{rgb}{0.000000,0.000000,0.000000}%
\pgfsetfillcolor{currentfill}%
\pgfsetlinewidth{0.501875pt}%
\definecolor{currentstroke}{rgb}{0.000000,0.000000,0.000000}%
\pgfsetstrokecolor{currentstroke}%
\pgfsetdash{}{0pt}%
\pgfsys@defobject{currentmarker}{\pgfqpoint{0.000000in}{0.000000in}}{\pgfqpoint{0.055556in}{0.000000in}}{%
\pgfpathmoveto{\pgfqpoint{0.000000in}{0.000000in}}%
\pgfpathlineto{\pgfqpoint{0.055556in}{0.000000in}}%
\pgfusepath{stroke,fill}%
}%
\begin{pgfscope}%
\pgfsys@transformshift{1.000000in}{2.320999in}%
\pgfsys@useobject{currentmarker}{}%
\end{pgfscope}%
\end{pgfscope}%
\begin{pgfscope}%
\pgfsetbuttcap%
\pgfsetroundjoin%
\definecolor{currentfill}{rgb}{0.000000,0.000000,0.000000}%
\pgfsetfillcolor{currentfill}%
\pgfsetlinewidth{0.501875pt}%
\definecolor{currentstroke}{rgb}{0.000000,0.000000,0.000000}%
\pgfsetstrokecolor{currentstroke}%
\pgfsetdash{}{0pt}%
\pgfsys@defobject{currentmarker}{\pgfqpoint{-0.055556in}{0.000000in}}{\pgfqpoint{0.000000in}{0.000000in}}{%
\pgfpathmoveto{\pgfqpoint{0.000000in}{0.000000in}}%
\pgfpathlineto{\pgfqpoint{-0.055556in}{0.000000in}}%
\pgfusepath{stroke,fill}%
}%
\begin{pgfscope}%
\pgfsys@transformshift{7.200000in}{2.320999in}%
\pgfsys@useobject{currentmarker}{}%
\end{pgfscope}%
\end{pgfscope}%
\begin{pgfscope}%
\pgftext[x=0.944444in,y=2.320999in,right,]{\sffamily\fontsize{12.000000}{14.400000}\selectfont −20}%
\end{pgfscope}%
\begin{pgfscope}%
\pgfpathrectangle{\pgfqpoint{1.000000in}{0.600000in}}{\pgfqpoint{6.200000in}{4.800000in}} %
\pgfusepath{clip}%
\pgfsetbuttcap%
\pgfsetroundjoin%
\pgfsetlinewidth{0.501875pt}%
\definecolor{currentstroke}{rgb}{0.000000,0.000000,0.000000}%
\pgfsetstrokecolor{currentstroke}%
\pgfsetdash{{1.000000pt}{3.000000pt}}{0.000000pt}%
\pgfpathmoveto{\pgfqpoint{1.000000in}{3.090749in}}%
\pgfpathlineto{\pgfqpoint{7.200000in}{3.090749in}}%
\pgfusepath{stroke}%
\end{pgfscope}%
\begin{pgfscope}%
\pgfsetbuttcap%
\pgfsetroundjoin%
\definecolor{currentfill}{rgb}{0.000000,0.000000,0.000000}%
\pgfsetfillcolor{currentfill}%
\pgfsetlinewidth{0.501875pt}%
\definecolor{currentstroke}{rgb}{0.000000,0.000000,0.000000}%
\pgfsetstrokecolor{currentstroke}%
\pgfsetdash{}{0pt}%
\pgfsys@defobject{currentmarker}{\pgfqpoint{0.000000in}{0.000000in}}{\pgfqpoint{0.055556in}{0.000000in}}{%
\pgfpathmoveto{\pgfqpoint{0.000000in}{0.000000in}}%
\pgfpathlineto{\pgfqpoint{0.055556in}{0.000000in}}%
\pgfusepath{stroke,fill}%
}%
\begin{pgfscope}%
\pgfsys@transformshift{1.000000in}{3.090749in}%
\pgfsys@useobject{currentmarker}{}%
\end{pgfscope}%
\end{pgfscope}%
\begin{pgfscope}%
\pgfsetbuttcap%
\pgfsetroundjoin%
\definecolor{currentfill}{rgb}{0.000000,0.000000,0.000000}%
\pgfsetfillcolor{currentfill}%
\pgfsetlinewidth{0.501875pt}%
\definecolor{currentstroke}{rgb}{0.000000,0.000000,0.000000}%
\pgfsetstrokecolor{currentstroke}%
\pgfsetdash{}{0pt}%
\pgfsys@defobject{currentmarker}{\pgfqpoint{-0.055556in}{0.000000in}}{\pgfqpoint{0.000000in}{0.000000in}}{%
\pgfpathmoveto{\pgfqpoint{0.000000in}{0.000000in}}%
\pgfpathlineto{\pgfqpoint{-0.055556in}{0.000000in}}%
\pgfusepath{stroke,fill}%
}%
\begin{pgfscope}%
\pgfsys@transformshift{7.200000in}{3.090749in}%
\pgfsys@useobject{currentmarker}{}%
\end{pgfscope}%
\end{pgfscope}%
\begin{pgfscope}%
\pgftext[x=0.944444in,y=3.090749in,right,]{\sffamily\fontsize{12.000000}{14.400000}\selectfont −15}%
\end{pgfscope}%
\begin{pgfscope}%
\pgfpathrectangle{\pgfqpoint{1.000000in}{0.600000in}}{\pgfqpoint{6.200000in}{4.800000in}} %
\pgfusepath{clip}%
\pgfsetbuttcap%
\pgfsetroundjoin%
\pgfsetlinewidth{0.501875pt}%
\definecolor{currentstroke}{rgb}{0.000000,0.000000,0.000000}%
\pgfsetstrokecolor{currentstroke}%
\pgfsetdash{{1.000000pt}{3.000000pt}}{0.000000pt}%
\pgfpathmoveto{\pgfqpoint{1.000000in}{3.860500in}}%
\pgfpathlineto{\pgfqpoint{7.200000in}{3.860500in}}%
\pgfusepath{stroke}%
\end{pgfscope}%
\begin{pgfscope}%
\pgfsetbuttcap%
\pgfsetroundjoin%
\definecolor{currentfill}{rgb}{0.000000,0.000000,0.000000}%
\pgfsetfillcolor{currentfill}%
\pgfsetlinewidth{0.501875pt}%
\definecolor{currentstroke}{rgb}{0.000000,0.000000,0.000000}%
\pgfsetstrokecolor{currentstroke}%
\pgfsetdash{}{0pt}%
\pgfsys@defobject{currentmarker}{\pgfqpoint{0.000000in}{0.000000in}}{\pgfqpoint{0.055556in}{0.000000in}}{%
\pgfpathmoveto{\pgfqpoint{0.000000in}{0.000000in}}%
\pgfpathlineto{\pgfqpoint{0.055556in}{0.000000in}}%
\pgfusepath{stroke,fill}%
}%
\begin{pgfscope}%
\pgfsys@transformshift{1.000000in}{3.860500in}%
\pgfsys@useobject{currentmarker}{}%
\end{pgfscope}%
\end{pgfscope}%
\begin{pgfscope}%
\pgfsetbuttcap%
\pgfsetroundjoin%
\definecolor{currentfill}{rgb}{0.000000,0.000000,0.000000}%
\pgfsetfillcolor{currentfill}%
\pgfsetlinewidth{0.501875pt}%
\definecolor{currentstroke}{rgb}{0.000000,0.000000,0.000000}%
\pgfsetstrokecolor{currentstroke}%
\pgfsetdash{}{0pt}%
\pgfsys@defobject{currentmarker}{\pgfqpoint{-0.055556in}{0.000000in}}{\pgfqpoint{0.000000in}{0.000000in}}{%
\pgfpathmoveto{\pgfqpoint{0.000000in}{0.000000in}}%
\pgfpathlineto{\pgfqpoint{-0.055556in}{0.000000in}}%
\pgfusepath{stroke,fill}%
}%
\begin{pgfscope}%
\pgfsys@transformshift{7.200000in}{3.860500in}%
\pgfsys@useobject{currentmarker}{}%
\end{pgfscope}%
\end{pgfscope}%
\begin{pgfscope}%
\pgftext[x=0.944444in,y=3.860500in,right,]{\sffamily\fontsize{12.000000}{14.400000}\selectfont −10}%
\end{pgfscope}%
\begin{pgfscope}%
\pgfpathrectangle{\pgfqpoint{1.000000in}{0.600000in}}{\pgfqpoint{6.200000in}{4.800000in}} %
\pgfusepath{clip}%
\pgfsetbuttcap%
\pgfsetroundjoin%
\pgfsetlinewidth{0.501875pt}%
\definecolor{currentstroke}{rgb}{0.000000,0.000000,0.000000}%
\pgfsetstrokecolor{currentstroke}%
\pgfsetdash{{1.000000pt}{3.000000pt}}{0.000000pt}%
\pgfpathmoveto{\pgfqpoint{1.000000in}{4.630250in}}%
\pgfpathlineto{\pgfqpoint{7.200000in}{4.630250in}}%
\pgfusepath{stroke}%
\end{pgfscope}%
\begin{pgfscope}%
\pgfsetbuttcap%
\pgfsetroundjoin%
\definecolor{currentfill}{rgb}{0.000000,0.000000,0.000000}%
\pgfsetfillcolor{currentfill}%
\pgfsetlinewidth{0.501875pt}%
\definecolor{currentstroke}{rgb}{0.000000,0.000000,0.000000}%
\pgfsetstrokecolor{currentstroke}%
\pgfsetdash{}{0pt}%
\pgfsys@defobject{currentmarker}{\pgfqpoint{0.000000in}{0.000000in}}{\pgfqpoint{0.055556in}{0.000000in}}{%
\pgfpathmoveto{\pgfqpoint{0.000000in}{0.000000in}}%
\pgfpathlineto{\pgfqpoint{0.055556in}{0.000000in}}%
\pgfusepath{stroke,fill}%
}%
\begin{pgfscope}%
\pgfsys@transformshift{1.000000in}{4.630250in}%
\pgfsys@useobject{currentmarker}{}%
\end{pgfscope}%
\end{pgfscope}%
\begin{pgfscope}%
\pgfsetbuttcap%
\pgfsetroundjoin%
\definecolor{currentfill}{rgb}{0.000000,0.000000,0.000000}%
\pgfsetfillcolor{currentfill}%
\pgfsetlinewidth{0.501875pt}%
\definecolor{currentstroke}{rgb}{0.000000,0.000000,0.000000}%
\pgfsetstrokecolor{currentstroke}%
\pgfsetdash{}{0pt}%
\pgfsys@defobject{currentmarker}{\pgfqpoint{-0.055556in}{0.000000in}}{\pgfqpoint{0.000000in}{0.000000in}}{%
\pgfpathmoveto{\pgfqpoint{0.000000in}{0.000000in}}%
\pgfpathlineto{\pgfqpoint{-0.055556in}{0.000000in}}%
\pgfusepath{stroke,fill}%
}%
\begin{pgfscope}%
\pgfsys@transformshift{7.200000in}{4.630250in}%
\pgfsys@useobject{currentmarker}{}%
\end{pgfscope}%
\end{pgfscope}%
\begin{pgfscope}%
\pgftext[x=0.944444in,y=4.630250in,right,]{\sffamily\fontsize{12.000000}{14.400000}\selectfont −5}%
\end{pgfscope}%
\begin{pgfscope}%
\pgfpathrectangle{\pgfqpoint{1.000000in}{0.600000in}}{\pgfqpoint{6.200000in}{4.800000in}} %
\pgfusepath{clip}%
\pgfsetbuttcap%
\pgfsetroundjoin%
\pgfsetlinewidth{0.501875pt}%
\definecolor{currentstroke}{rgb}{0.000000,0.000000,0.000000}%
\pgfsetstrokecolor{currentstroke}%
\pgfsetdash{{1.000000pt}{3.000000pt}}{0.000000pt}%
\pgfpathmoveto{\pgfqpoint{1.000000in}{5.400000in}}%
\pgfpathlineto{\pgfqpoint{7.200000in}{5.400000in}}%
\pgfusepath{stroke}%
\end{pgfscope}%
\begin{pgfscope}%
\pgfsetbuttcap%
\pgfsetroundjoin%
\definecolor{currentfill}{rgb}{0.000000,0.000000,0.000000}%
\pgfsetfillcolor{currentfill}%
\pgfsetlinewidth{0.501875pt}%
\definecolor{currentstroke}{rgb}{0.000000,0.000000,0.000000}%
\pgfsetstrokecolor{currentstroke}%
\pgfsetdash{}{0pt}%
\pgfsys@defobject{currentmarker}{\pgfqpoint{0.000000in}{0.000000in}}{\pgfqpoint{0.055556in}{0.000000in}}{%
\pgfpathmoveto{\pgfqpoint{0.000000in}{0.000000in}}%
\pgfpathlineto{\pgfqpoint{0.055556in}{0.000000in}}%
\pgfusepath{stroke,fill}%
}%
\begin{pgfscope}%
\pgfsys@transformshift{1.000000in}{5.400000in}%
\pgfsys@useobject{currentmarker}{}%
\end{pgfscope}%
\end{pgfscope}%
\begin{pgfscope}%
\pgfsetbuttcap%
\pgfsetroundjoin%
\definecolor{currentfill}{rgb}{0.000000,0.000000,0.000000}%
\pgfsetfillcolor{currentfill}%
\pgfsetlinewidth{0.501875pt}%
\definecolor{currentstroke}{rgb}{0.000000,0.000000,0.000000}%
\pgfsetstrokecolor{currentstroke}%
\pgfsetdash{}{0pt}%
\pgfsys@defobject{currentmarker}{\pgfqpoint{-0.055556in}{0.000000in}}{\pgfqpoint{0.000000in}{0.000000in}}{%
\pgfpathmoveto{\pgfqpoint{0.000000in}{0.000000in}}%
\pgfpathlineto{\pgfqpoint{-0.055556in}{0.000000in}}%
\pgfusepath{stroke,fill}%
}%
\begin{pgfscope}%
\pgfsys@transformshift{7.200000in}{5.400000in}%
\pgfsys@useobject{currentmarker}{}%
\end{pgfscope}%
\end{pgfscope}%
\begin{pgfscope}%
\pgftext[x=0.944444in,y=5.400000in,right,]{\sffamily\fontsize{12.000000}{14.400000}\selectfont 0}%
\end{pgfscope}%
\begin{pgfscope}%
\pgftext[x=0.523275in,y=3.000000in,,bottom,rotate=90.000000]{\sffamily\fontsize{12.000000}{14.400000}\selectfont Response (dB)}%
\end{pgfscope}%
\end{pgfpicture}%
\makeatother%
\endgroup%
}
 \caption{Point Spread Function using the described processing method.}
 \label{fg:psf}
 \end{center}
\end{figure}
\par
A test noise signal is also generated by the simulator. The SNR prior to filtering is compared with the SNR after filtering (but before azimuth compression) giving a change of about -0.4 dB. This shows that with the simulated parameters, the SNR does not change significantly.
\par
In summary, the simulation demonstrates the suitability of the proposed signal processing algorithm and also shows how the generated PSF contains extra sidelobes that most likely result from the different shape of the signal response in the Doppler domain. If these sidelobes are intolerable, they can possibly be removed by modifying the phased-array beam tables; however, this is a topic for further research.